\documentclass[a4paper,11pt]{article}
\usepackage[utf8]{inputenc}
\usepackage[margin=1in]{geometry}
\usepackage{pdfpages}
\usepackage{mathrsfs}
\usepackage{amsfonts}
\usepackage{amsmath}
\DeclareMathOperator\arctanh{arctanh}
\usepackage{amssymb}
\usepackage{bbm}
\usepackage{amsthm}
\usepackage{graphicx}
\usepackage{centernot}
\usepackage{caption}
\usepackage{subcaption}
\usepackage{braket}
\usepackage{pgfplots}
\usepackage{lastpage}
\usepackage{enumitem}
\usepackage{setspace}
\usepackage{xcolor}
\usepackage{cancel}
\usepackage[english]{babel} 

\usepackage[square,sort,comma,numbers]{natbib}
\usepackage[colorlinks=true,linkcolor=blue]{hyperref}

\usepackage{fancyhdr}
\newcommand{\euler}[1]{\text{e}^{#1}}
\newcommand{\Real}{\text{Re}}
\newcommand{\Imag}{\text{Im}}
\newcommand{\supp}{\text{supp}}
\newcommand{\pare}[1]{\left( #1 \right)}
\newcommand{\norm}[1]{\left\lVert #1 \right\rVert}
\newcommand{\abs}[1]{\left\lvert #1 \right\rvert}
\newcommand{\floor}[1]{\left\lfloor #1 \right\rfloor}
\newcommand{\Span}[1]{\text{span}\left(#1\right)}
\newcommand{\dom}[1]{\mathscr D\left(#1\right)}
\newcommand{\Ran}[1]{\text{Ran}\left(#1\right)}
\newcommand{\conv}[1]{\text{co}\left\{#1\right\}}
\newcommand{\Ext}[1]{\text{Ext}\left\{#1\right\}}
\newcommand{\vin}{\rotatebox[origin=c]{-90}{$\in$}}
\newcommand{\interior}[1]{%
	{\kern0pt#1}^{\mathrm{o}}%
}
\newcommand*\diff{\mathop{}\!\mathrm{d}}
\newcommand{\ie}{\emph{i.e.} }
\newcommand{\eg}{\emph{e.g.} }
\newcommand{\dd}{\partial }
\newcommand{\N}{\mathbb{N}}
\newcommand{\Z}{\mathbb{Z}}
\newcommand{\R}{\mathbb{R}}
\newcommand{\C}{\mathbb{C}}
\newcommand{\w}{\mathsf{w}}

\newcommand{\Gliminf}{\Gamma\text{-}\liminf}
\newcommand{\Glimsup}{\Gamma\text{-}\limsup}
\newcommand{\Glim}{\Gamma\text{-}\lim}
\newcommand{\pipe}{\ \vert \ }

\newtheorem{theorem}{Theorem}
\newtheorem{definition}{Definition}
\newtheorem{proposition}{Proposition}
\newtheorem{lemma}{Lemma}
\newtheorem{corollary}{Corollary}

\numberwithin{equation}{section}
\linespread{1.3}

\pagestyle{fancy}
\fancyhf{}
\rhead{CoCo - Assignment 4}
\lhead{}
\rfoot{\thepage}
\lfoot{Dated: \today}
\author{Johannes Agerskov, Mads Friis Frand-Madsen, and Sifan Huang}
\date{Dated: \today}
\title{CoCo - Assignment 4}
\begin{document}

\maketitle
\section*{6.23}
Let $ t $ be the computable function that, in a TM describtion interchange the accept and reject states. We give a concrete example of a fixed point of $ t $. We notice that the proof fixed point version of the recursion theorem (Theorem 6.8) gives a construction of a fixed point, $ F $, which obtain its own describtion and then simulate the TM described by $ t(\braket{F}) $. However, in this specific example of $ t $, migth actually conclude that $ F $ is the Turring machine that neither accepts nor rejects any string. This is easily seen by the fact that if $ w $ is string such that $ F $ accepts $ w $, then the TM, $ G $, described by $ T(\braket{F}) $ will reject $ w $. But since $ F $ simulates exactly $ G $, we conclude that $ F $ also rejects $ w $ which is a contradiction. Thus $ F $ cannot accept any string. Swapping \emph{accept} and \emph{reject} in the previous argument shows the $ F $ also cannot reject any string. Conversely, if $ F $ is a TM such that it neither accept nor reject any state, $ t(\braket{text}) $ obviously describes an equivalent TM. Thus any TM, $ F $, which never halts on any string is a fixed point of $ t $.\\ 
Therefore we describe $ F $ by $ F=(Q,\Sigma,\Gamma,\delta,q_0,q_{\text{accepts}},q_{\text{reject}}) $, where $ Q=\{q_0,q_{\text{accepts}},q_{\text{reject}}\} $. $ \Sigma=\{1\} $, $ \Gamma=\{\sqcup,1\} $, $ \delta: Q\times \Gamma\to Q\times\Gamma\times\{L,R\} $ is defined by $ \delta(q_0,1)=(q_0,1,R) $ and $ \delta(q_0,\sqcup)=(q_0,\sqcup,L) $, and $ \delta $ need not be defined, or rather could be defined to be anything, on the accept and reject state, since the machine halts there anyway. Thus, $ F $ loops forever by going to the last $ 1 $ in any input string and then going back and forth between $ \sqcup $ and $ 1 $ at the end of the string. On the empty string it loops by trying to move left on the tape, but stays at the first tape position. This TM is a fixed point of $ t $ be the above argument.

\section*{6.6}
For each $ m>1 $ let $ \mathcal{Z}_m=\{0,1,2,...,m-1\} $, and let $ \mathcal{F}_m=(\mathcal{Z}_m,+,\times) $ be the model whose universe is $ \mathcal{Z}_m $ and that has relations corresponding to the $ + $ and $ \times $ relations modulo $ m $. We show in the following that $ \text{Th}(\mathcal{F}_m) $ is decidable. \begin{proof}
	We construct an $ TM $ that decides $ \text{Th}(\mathcal{F}_m) $. We know from Theorem 6.12 that $ \text{Th}(\mathcal{N},+) $ is decidable. Let $ S $ be the TM that decides $ \text{Th}(\mathcal{N},+) $. We can clearly use this TM to show that $ \text{Th}(\mathcal{Z}_m,+) $ is decidable. We can design, $ M_1  $, to be a TM that has $ S $ as a subroutine, such that $ M_1 $ can add modulo $ m $. For any formula $ \phi $, $ M_1 $ simply does the following:
	\begin{enumerate}
		\item identify all substrings of $ \phi $ of the form $ x_1+...+x_n=y $.
		\item Generate a new substring $ \tilde{\phi} $ by replacing every substring of $ \phi $ of the form $ x_1+...+x_n=y $, by $ \bigvee_{i=1}^n(x_1+...+x_n=y+\underbrace{m+m+...+m}_{i\text{ terms}}) $, where $ \bigvee_{i=1}^{n}\varphi(i)=\varphi(1)\vee...\vee\varphi(n) $, for any formula $ \varphi(i) $ depending on $ i $.
		\item Run $ S $ on $ \tilde{\phi} $, if true \emph{accept}. If false \emph{reject}
	\end{enumerate}
	Thus $M_1 $ accepts a formula if and only if it is true modulo $ m $. Clearly we can never have $ z-m\geq x_1+...+x_n\geq z+\underbrace{m+m+...+m}_{n\text{ terms}} $ for $ x,y\in \mathcal{Z}_m  $, so this TM desides $ \text{Th}(\mathcal{Z}_m,+) $. Thus we have shown that $ \text{Th}(\mathcal{Z}_m,+)$ is reducible to $ \text{Th}(\mathcal{\mathcal{N}},+) $, so from Theorem 6.12 it follows that $ \text{Th}(\mathcal{Z}_m,+) $ is decidable. Now that $ \text{Th}(\mathcal{Z}_m,+,\times) $ is decidable can be shown by the fact that $ \text{Th}(\mathcal{Z}_m,+,\times)$ reduces to  $\text{Th}(\mathcal{Z}_m,+) $. Given a decider $ M_1 $ for $ \text{Th}(\mathcal{Z}_m,+) $, we design the TM, $ M $ that desides $ \text{Th}(\mathcal{Z}_m,+,\times) $ it can be described as: On formula $ \varphi $ follows\begin{enumerate}
		\item Identify all substrings of $ \varphi $ of the form, $ x\times y $ where $ y $ is a free variable, and rewrite the substring to $ \underbrace{x+x+x+...}_{y\text{ times}} $.
		\item Indentify any formula that has a substring of the form, $ x\times y $ where $ y $ is not a free variable. Then we consider two cases:\begin{enumerate}
			\item $\varphi $ contains a substring of the form $ \forall y $, \ie can be written $ \forall y[\phi(y)]\Phi $, for some formula $ \phi(y) $ depending on $ y $ and some string $ \Phi $, which is just the rest of $ \varphi $. In this case $ M $ discards the quantifier of $ y $ and generate the formula \begin{equation}
			\bigwedge_{i=0}^{m-1}\phi(i)\Phi,
			\end{equation}.
			where $ \bigwedge_{i=1}^N \phi(i)=\phi(1)\wedge\phi(2)\wedge...\wedge\phi(N) $
			\item $\varphi $ contains a substring of the form $ \exists y $, \ie can be written $ \exists y[\phi(y)]\Phi $, for some formula $ \phi(y) $ depending on $ y $ and some string $ \Phi $, which is just the rest of $ \varphi $, in which case $ M $ discards the quantifier of $ y $ and generate the formula \begin{equation}
			\bigvee_{i=0}^{m-1}\phi(i)\Phi,
			\end{equation}.
			where $ \bigvee_{i=1}^N \phi(i)=\phi(1)\vee\phi(2)\vee...\vee\phi(N) $
		\end{enumerate}
		this process is repeated iteratively til no products are left.
		\item Run $ M_1 $ on the generated formula, if true \emph{accept}, else \emph{reject.}
	\end{enumerate}
	Now since $ \mathcal{Z}_m $ is finite, there are always only finitely many terms in the above description. Thus $ M $ will always halt, making it a decider. Furhtermore, it is obvious by construction, that $ M $ accepts any formula if and only if it is true in the usual model of $ \mathcal{Z}_m $ with addition and multiplication modulo $ m $.
\end{proof}



\begin{proof}
    From Theorem 6.12 we may infer the existence of a Turing machine $T$ deciding $\operatorname{Th}(\mathcal{N},+)$. Define a Turing machine $T_{m,+}$ for deciding $\operatorname{Th}(\mathcal{Z}_m,+)$ as follows:
    \begin{enumerate}[label=\arabic*)]
        \item Given a formula $\psi = Q_1x_1\ldots Q_kx_kR_1y_1\ldots R_ly_l[x_1+\ldots+x_n = y_1+\ldots y_l]$ construct the formula
        \begin{align*}
            \widetilde{\phi} = &\pare{\bigvee_{i=0}^{m-1}\pare{x_1+\ldots+x_n + \underbrace{m + \ldots + m}_{i \text{ terms}} = y_1+\ldots y_l}} \\ 
            &\quad\quad\quad \lor 
            \pare{\bigvee_{i=0}^{m-1}\pare{x_1+\ldots+x_n = y_1+\ldots y_l + \underbrace{m + \ldots + m}_{i \text{ terms}}}}
        \end{align*}
        \item Run $T$ on $Q_1x_1\ldots Q_kx_kR_1y_1\ldots R_ly_l[\widetilde{\phi}]$ and return the outcome of $T$.
    \end{enumerate}
    Evidently, the Turing machine above decides $\operatorname{Th}(\mathcal{Z}_m,+)$, and so we move on to constructing a decider for $\operatorname{Th}(\mathcal{Z}_m,+,\times)$. 
    \begin{enumerate}[label=\arabic*)]
        \item Given a formula $Q_1x_1\ldots Q_kx_kR_1y_1\ldots R_ly_l[\psi]$, identify a substring of the form $x_i\times y_j$. If there are no such substrings, go to 4)
        
        \item If $Q_i = \forall$, replace $x_i\times y_j$ by
        $$ \bigwedge_{i=0}^{m-1} \underbrace{\pare{y_j + \ldots y_j}}_{i \text{ terms}}, $$
        and return to 1).
        
        \item If $Q_i = \exists$, replace $x_i\times y_j$ by
        $$ \bigvee_{i=0}^{m-1} \underbrace{\pare{y_j + \ldots y_j}}_{i \text{ terms}}, $$
        and return to 1).
        
        \item Run $T_{m,+}$ on $R_1y_1\ldots R_ly_l[\psi]$ and return the outome of $T_{m,+}$.
    \end{enumerate}
    It is clear the Turing machine $T_m$ halts, and likewise it is constructed to accept exactly $\operatorname{Th}(\mathcal{Z}_m,+,\times)$.
\end{proof}


\section*{6.7}
Let $ A $ and $ B $ be any two languages, we show that then there exist a language $ J $ such that $ A\leq_T J $ and $ B\leq_T J $.
\begin{proof}
	Let $ \aleph $ be a special symbol that is not in the alphabet of $ A $ or $ B $. Construct the the language $ \aleph A $ \ie all string of the form $ \aleph w $ for $ w\in A $. Define the language $ J=\aleph A\cup B $. We then construct an oracle TM, $ M^J_A $ with an oracle for $ J $ that desides $ A $. $ M_A^J $ acts as follows in input $ w $. \begin{enumerate}
		\item Construct the string $ \aleph w $.
		\item Query the oracle for $ J $ if $ \aleph w\in J $, if true, \emph{accept}, if false, \emph{reject}.
	\end{enumerate}
	Clearly $ M_A^J $ accepts $ w $ if and only if $ \aleph w\in J $ but since $ \aleph w\notin B $, we conclude that $ \aleph w\in J $ if and only if $ \aleph w\in \aleph A $ which can be true if and only if $ w\in A $. Thus $ M_A^J $ decides $ A $, and $ A $ is reducible relative to $ J $, \ie $ A\leq_T J $.\\
	We now proceed by also constructing and oracle TM, $ M_B^J $, that decides $ B $. $ M_B^J $ acts as follows on input $ w=w_1... $. \begin{enumerate}
	    \item Check if $w_1=\aleph$, if true, \emph{reject}.
		\item Query the oracle for $ J $ if $ w\in J $, if true, \emph{accept}, if false, \emph{reject}.
	\end{enumerate}
	Clearly $ M_B^J $ accept if and only if $ w\in J $ but since $ w\notin \aleph A $ we conclude that $ w\in J $ if and only if $ w\in B $. So $ M_B^J $ decides $ B $, and $ B $ is reducible relative to $ J $, \ie $ B\leq_T J $. This concludes the proof.
\end{proof}


\section*{6.11} We show that the complement of $ EQ_{TM} $ is recognizable by a TM with an oracle for $ A_{TM} $. 
\begin{proof}
	First we define the oracle TM, $ S_{\braket{G,F}}^{A_{TM}} $, that, given two $ TM $s $ G $ and $ F $, decides the language $ L(G)\Delta L(F)=(L(G)\cap  L(F)^\complement)\cup (L(G)^\complement\cap L(F))  $, \ie the symmetric difference of the languages. Let $ S_{\braket{G,F}}^{A_{TM}} $ be the oracle TM that acts a follows: On input $ w $ \begin{enumerate}
		\item Query the oracle of $ A_{TM} $ if $ G $ accepts $ w $, and store the result.
		\item Query the oracle of $ A_{TM} $ if $ F $ accepts $ w $, and store the result.
		\item If exactly one of step 1 and step 2 returned True, \emph{accept}. If both or neither of step 1 and step 2 returned True, \emph{reject}.
	\end{enumerate}
	Clearly $ S_{\braket{G,F}}^{A_{TM}} $ is a decider, and accept $ w $ if and only if $ w\in L(G)\Delta L(F) $.
	Now we define the oracle TM $ M^{A_{TM}} $ which recognizes $ EQ_{TM} $. On input $ \braket{G,F} $, where $ G $ and $ F $ are TMs, $ M^{A_{TM}} $ acts as follows:\begin{enumerate}
		\item Check that $ F $ and $ G $ have the same input alpabet alphabet, $ \Sigma $, if not \emph{accept}.
		\item Construct the oracle TM $ S^{A_{TM}}_{\braket{G,F}} $ using the oracle of $ A_{TM} $, as described above.
		\item Run $ S^{A_{TM}}_{\braket{G,F}} $ on $ w $ for each string in $ w\in \Sigma^\ast $, say in lexicographical ordering. If for some string $ \tilde{w}\in\Sigma^* $, $ S^{A_{TM}}_{\braket{F,G}} $ accepts, \emph{accept}.
	\end{enumerate}
	It is easily seen that $ M^{A_{TM}} $ accepts if and only if $ L(G) $ and $ L(F) $ does not have the same alphabet, \emph{or} they have the same alphabet $ \Sigma $ and there exist a string $ w $ in $ \Sigma^* $ such that $ w\in L(G)\Delta(F) $ which is again true if and only if $ L(G)\neq L(F) $. Thus $ M^{A_{TM}} $ accepts input $ \braket{G,F} $ if and only if $ L(G)\neq L(F) $ which is equaivalent to $ L(M^{A_{TM}})=EQ_{TM}^\complement $, as desired.
\end{proof}

\begin{proof}
    We construct a Turing machine $T$ with access to an oracle deciding $A_{TM}$ as follows.
    \begin{enumerate}[label=\arabic*)]
        \item On input $\langle M_1,M_2\rangle$, accept if the input alphabets are distinct. 
        
        \item Query the oracle for $A_{TM}$ with $\langle M_1\rangle w$ and $\langle M_2\rangle w$ for all strings $w\in\Sigma^*$ in lexicographical order.
        \item If the query results for some string $w$ are distinct, then accept. 
    \end{enumerate}
\end{proof}

\section*{6.14}
Given a string $ x $, we show how to compute the descriptive complexity $ K(x) $, given an oracle for $ A_{TM} $. Given a string, $ x $, over the alphabet $ \Sigma=\{0,1\} $ we simply check all strings in $ \Sigma^* $ one by one, in lexocographical ordering. Given a string $ v\in\Sigma^* $ we split it in all possible ways $ v=v_1v_2 $ and query the oracle if $ v_1 $ as a descriptions of a TM accepts $ v_2 $, if so, we run the TM that $ v_1 $ decribes on $ v_2 $ and read of the tape after acceptance. We then check wether the read off tape matches $ x $ if so, $ K(x)=\abs{v} $, where $ v $ is the first string in lexicographilcal ordering that succesfully could be split in a description of a TM and an input such that the TM run on the input printed $ x $. Clearly this can be done, since we never encounter any loops, so we are guarenteed to finish this proceedure after finitely many steps, especially beacuse of Theorem 6.24 that bounds $ K(x)\leq \abs{x}+b $.
\\
Notice that there is a subtlety in the above argument, namely that in a general encoding $ \braket{M}w $ of $ x $, $ M $ need not \emph{accept} $ w $, but only \emph{halt}. However, if there is a TM $ M $ that \emph{rejects} $ w $ with $ x $ on its tape, we can describe the correpsonding TM, $ \bar{M} $ with $ q_{\text{accepts}} $ and $ q_{\text{reject}} $ by a string $ \braket{\bar{M}} $ such that $ \abs{\braket{\bar{M}}}=\abs{\braket{M}} $, and thus we can restrict to encodings TMs that accept the input, without increasing $ K(x) $. It is clear by definition of $ K(x) $ that we are also \emph{not} decreasing $ K(x) $ by restricting the TM. \\
Alternatively one migth argue that an oracel for $ A_{TM} $ gives us an oracle for $ HALT_{TM} $, by reducibility, and then use this oracle instead.

\section*{$ PCP $ and $ E_{TM} $}
We show that $ PCP\leq_T E_{TM} $ and that $ E_{TM}\leq_T PCP$. We use in the followsing the proof of Theorem 5.15 (that $ PCP $ is undecidable) and Example 6.19 i the book. By the proof of Theorem 5.15, we have that $ A_{TM}\leq_m MPCP\leq_m PCP $, (as also noted in example 5.25) which implies $ A_{TM}\leq_T PCP $ by transitivity of $ \leq_m $ and by the fact that $ A\leq_m B $ implies $ A\leq_T B $. By Example 6.19 we have that $ E_{TM}\leq_T A_{TM} $. By transitivity of $ \leq_T $ we thus only need to show $ A_{TM}\leq_T PCP $ and $ E_{TM}\leq_T A_{TM} $...\\
\\
\\
We show that $ PCP\leq_T E_{TM} $ and that $ E_{TM}\leq_T PCP$.
\begin{proof}
	We start by showing that $ PCP\leq_T E_{TM} $. Given an oracle for $ E_{TM} $ and an instance of the $ PCP $, we simply construct a TM, $ S $, such that $ S $ has non-empty language only if there is a match in the given instance of the $ PCP $. Let $ P= \left\{\left[\frac{t_1}{b_1}\right],\left[\frac{t_2}{b_2}\right],...,\left[\frac{t_k}{b_k}\right]\right\} $ be an instance of the $ PCP $. We design $ S_P $ to have an input alphabet $ a_1,...,a_k $. $ S_P $ also contain the $ PCP $ instance, $ P $, in the sense, that it associates to every $ a_i $ the domino $ \left[\frac{t_i}{b_i}\right] $. We may decribe $ S_P $ in the following way: On input $ w $, $ S_P $ does the following \begin{enumerate}
		\item Check that $ w\in \{a_1,a_2,...,a_k\}^+ $ \ie $ w=a_{\sigma(1)}...a_{\sigma(m)} $ for some function $ \sigma:\{1,...,m\} \to\{1,...,k\} $ and some $ m\geq 1 $. If not, $ S_P $ \emph{rejects}.
		\item Rewrite $ a_i $ on the tape as $ \left[t_i/b_i\right] $, for every $ i=1,...,k $, where $ [,\ /,\text{ and } ] $ are tape letters. (This will require to rearrange the tape in order to make room for the extra letters).
		\item Check if $ t_{\sigma(1)}...t_{\sigma(m)}=b_{\sigma(1)}...b_{\sigma(m)} $, if true, $ S_P $ \emph{accepts}, if false, $ S_P $ \emph{rejects}.
	\end{enumerate}
Clearly $ S $ accepts $ w $ if and only if $ w $ corresponds via the bijective map $ a_i\mapsto \left[\frac{t_i}{b_i}\right] $, to a match of $ P $.\\
Now given an oracle for $ E_{TM} $ is clear that we can decide $ PCP $ defined by\\ $ PCP=\{\braket{P} \ \vert\  P\text{ is an instance of }PCP\text{ with a match}\} $ with the following oracle TM, $ M^{E_{TM}} $.
On input $ \braket{P} $, $ M^{E_{TM}} $ does the following \begin{enumerate}
	\item Check that $ \braket{P} $ encodes an instance of the PCP.
	\item Construct $ S_P $ as described above.
	\item Query the oracle for $ E_{TM} $, wether $ L(S_P)=\emptyset $, if true $ M^{E_{TM}} $ \emph{rejects}, if false, $ M^{E_{TM}} $ \emph{accepts}.
\end{enumerate} 
Thus $ M^{E_{TM}} $ accepts input $ \braket{P} $ if and only if $ P $ is an instance of $ PCP $ and $ S_P $ has non-empty language, which is again true if and only if $ P $ has a match. This shows that $ PCP $ is Turing reducible to $ E_{TM} $ \ie $ PCP\leq_T E_{TM} $. 
\\
\\
We show now that $ E_{TM}\leq_T PCP $, this can be seen in the following way. We take inspiration from the proof of Theorem 5.15, and modify it slightly. Given a TM describtion $ \braket{M} $, we construct an instance of the $ PCP $, $ P_M $, such that $ P_M $ has a match if and only if $ \braket{M}\in E_{TM} $. 
Let $ P_M $ have all the same dominos as $ P $ in the proof of theorem 5.15, except for the first domino. instead we include the dominos $ \left[\frac{\ast\ast\aleph}{\ast\ast\aleph\ast\ast\aleph\ast\ast}\right],\ \left[\frac{\ast\ast\aleph}{\aleph\ast\ast\aleph\ast\ast}\right],\ \left[\frac{\ast\ast\aleph}{a\ast}\right],\ \left[\frac{\ast\ast\aleph\ast}{\#\ast}\right] $ for every $ a $ in the alpabet of $ M $. It is clear that any match must start with $ \left[\frac{\ast\ast\aleph\ast}{\ast\ast\aleph\ast\ast\aleph\ast\ast}\right] $, since this is the only domino that has upper string matches the first part of the lower string. Also since we have more $ \aleph $s in the lower string of the first domino by construction, the upper string can catch up in two ways: It can catch up by simply using $ \left[\frac{\ast\ast\aleph\ast}{\#\ast}\right]  $ at which case we have $ \left[\frac{\ast\ast\aleph\ast\ast\aleph\ast\quad }{\ast\ast\aleph\ast\ast\aleph\ast\#\ast}\right] $, thus we are set to simulate the empty string, by use of the dominos from the proof of Theorem 5.15.
Alternatively it catches up by first using dominos of the form $ \left[\frac{\ast\ast\aleph}{\aleph\ast\ast\aleph\ast\ast}\right] $ \ie it actually falls behind in the upper string, but then catches back up by using first dominos of the form $ \left[\frac{\ast\ast\aleph}{a\ast}\right] $ until it is forced to use the domino $ \left[\frac{\ast\ast\aleph\ast}{\#\ast}\right] $, as an example we might have $ \left[\frac{\ast\ast\aleph\ast}{\#\ast}\right]  $ at which case we have $ \left[\frac{\ast\ast\aleph\ast\ast\aleph\ast\ast\aleph\ast\ast\aleph\ast\ast\aleph\ast\ast\aleph\ast\quad\quad \qquad }{\ast\ast\aleph\ast\ast\aleph\ast\ast\aleph\ast\ast\aleph\ast\ast\aleph\ast\ast\aleph\ast\ast a_1\ast a_2\ast\#\ast}\right] $, where we used domino $ \left[\frac{\ast\ast\aleph}{\aleph\ast\ast\aleph\ast\ast}\right] $ twice, followed by $ \left[\frac{\ast\ast\aleph}{a_1\ast}\right] $, $ \left[\frac{\ast\ast\aleph}{a_2\ast}\right] $ and $ \left[\frac{\ast\ast\aleph\ast}{\#\ast}\right] $. It is clear that by doing this we have set up the $ P_M $ instance to simulate $ M $ on input $ a_1a_2 $ by constructing any match, if it exists. By this last method we can clearly generate strings of arbitrary length. Thus we see that the constructed $ PCP $ instance, $ P_M $, will have a match if and  only if there \emph{exist} a string $ w $ such that $ M $ accepts $ w $ or equivalently $ P_M $ has a match if and only if $ L(M) $ is non-empty.  
Thus we see that we may construct an oracle TM $ M^{PCP} $ with an oracle for $ PCP $ such that $ M^{PCP} $ decides $ E_{TM} $. $ M^{PCP} $ acts as follows on input $ \braket{K} $ which is a TM describtion:\begin{enumerate}
	\item Construct $ P_K $ as decribed above.
	\item Query the oracle for $ PCP $, if $ P_K\in PCP $, if true \emph{reject}, if false, \emph{accept}.
\end{enumerate}
Thus by the arguments above we clearly have that $ M^{PCP} $ accepts $ \braket{K} $ if and only if $ L(K)=\emptyset $. This shows that $ E_{TM}\leq_T PCP $.
\end{proof}
\end{document}