\documentclass[a4paper,11pt]{article}
\usepackage[utf8]{inputenc}
\usepackage[margin=1in]{geometry}
\usepackage{pdfpages}
\usepackage{mathrsfs}
\usepackage{amsfonts}
\usepackage{amsmath}
\DeclareMathOperator\arctanh{arctanh}
\usepackage{amssymb}
\usepackage{bbm}
\usepackage{amsthm}
\usepackage{graphicx}
\usepackage{centernot}
\usepackage{caption}
\usepackage{subcaption}
\usepackage{braket}
\usepackage{pgfplots}
\usepackage{lastpage}
\usepackage{enumitem}
\usepackage{setspace}
\usepackage{xcolor}
\usepackage{cancel}
\usepackage{scrextend}
\usepackage[english]{babel} 

\usepackage[square,sort,comma,numbers]{natbib}
\usepackage[colorlinks=true,linkcolor=blue]{hyperref}

\usepackage{fancyhdr}
\newcommand{\euler}[1]{\text{e}^{#1}}
\newcommand{\Real}{\text{Re}}
\newcommand{\Imag}{\text{Im}}
\newcommand{\supp}{\text{supp}}
\newcommand{\pare}[1]{\left( #1 \right)}
\newcommand{\norm}[1]{\left\lVert #1 \right\rVert}
\newcommand{\abs}[1]{\left\lvert #1 \right\rvert}
\newcommand{\floor}[1]{\left\lfloor #1 \right\rfloor}
\newcommand{\Span}[1]{\text{span}\left(#1\right)}
\newcommand{\dom}[1]{\mathscr D\left(#1\right)}
\newcommand{\Ran}[1]{\text{Ran}\left(#1\right)}
\newcommand{\conv}[1]{\text{co}\left\{#1\right\}}
\newcommand{\Ext}[1]{\text{Ext}\left\{#1\right\}}
\newcommand{\vin}{\rotatebox[origin=c]{-90}{$\in$}}
\newcommand{\vnotin}{\rotatebox[origin=c]{-90}{$\notin$}}
\newcommand{\interior}[1]{%
	{\kern0pt#1}^{\mathrm{o}}%
}
\newcommand*\diff{\mathop{}\!\mathrm{d}}
\newcommand{\ie}{\emph{i.e.} }
\newcommand{\eg}{\emph{e.g.} }
\newcommand{\dd}{\partial }
\newcommand{\N}{\mathbb{N}}
\newcommand{\Z}{\mathbb{Z}}
\newcommand{\R}{\mathbb{R}}
\newcommand{\C}{\mathbb{C}}
\newcommand{\w}{\mathsf{w}}

\newcommand{\Gliminf}{\Gamma\text{-}\liminf}
\newcommand{\Glimsup}{\Gamma\text{-}\limsup}
\newcommand{\Glim}{\Gamma\text{-}\lim}
\newcommand{\pipe}{\ \vert \ }

\newtheorem{theorem}{Theorem}
\newtheorem{definition}{Definition}
\newtheorem{proposition}{Proposition}
\newtheorem{lemma}{Lemma}
\newtheorem{corollary}{Corollary}

\numberwithin{equation}{section}
\linespread{1.3}

\pagestyle{fancy}
\fancyhf{}
\rhead{CoCo - Assignment 7}
\lhead{}
\rfoot{\thepage}
\lfoot{Dated: \today}
\author{Johannes Agerskov, Mads Friis Frand-Madsen, and Sifan Huang}
\date{Dated: \today}
\title{CoCo - Assignment 7 (Exam 2020)}
\begin{document}
	
	\maketitle
	\section*{Question 1}
	\subsection*{Part 1.1}
	We prove that the following languages over the alphabet $ \{a,b\} $ are regular:\begin{equation}
		\begin{aligned}
		L_i&=\{a^nb^n\pipe1\leq n\leq 10^n\},\\
		L_{ii}&=\{aa\}\cup\{b^n\pipe n\geq 0\},\\
		L_{iii}&=\{a^nb^n\pipe n\geq1\}\cap\{(ab)^n\pipe n\geq 1\},\\
		L_{iv}&=\{a^{2n+1}\pipe n\geq0\},\\
		\end{aligned}
	\end{equation}
	and for any fixed $ k,d\geq1 $:
	\begin{equation*}
		L_{v}=\{a^{kn+d}\pipe n\geq0\}.
	\end{equation*}
	\begin{proof}In the following we will write equality between languages and regular expressions, if and only if the regular expression generates the laguage.
		$ L_i=\cup_{n=1}^{10^{10}}a^nb^n $ which is a regular expression. Therefore, the language $ L_i $ is regular. $ L_{ii}=aa\cup b^* $, which is a regular expression, therefore it is regular. $ L_{iii}=ab $ since the $ \{a^nb^n\pipe n\geq1\} $ intersects $ \{(ab)^n\pipe n\geq 1\} $ only at $ ab $. This is a regular expression, therefore $ L_{iii} $ is regular. $ L_{iv}=a(a^2)^* $ which is a regular expression, therefore it is regular. And finally $ L_v=a^d(a^k)^* $ which is a regular expression, therefore it is regular.
	\end{proof}
	
	
	\subsection*{Part 1.2}
	Consider the language $ L_1=\{a^{2n}b^n\pipe n\geq1\} $. We show that $ L_1 $ is not regular, but it is context free.\begin{proof}
		Clearly, if $ L_1 $ was regular it would have a pumping length, $ p $. Thus assume $ L_1 $ is regular, and let $ p $ denote its pumping length as given by the pumping lemma. Consider then the string $ w=a^{2p}b^p\in L_1 $. According to the pumping lemma, we can split $ w $ in $ w=xyz $ such that $ \abs{xy}\leq p $ and $ \abs{y}>0 $, and such that $ xy^iz\in L_1 $ for any $ i\geq 0 $. Clearly in this case we have $ y=a^k $ for some $ 0<k\leq p $ and thus $ xy^2z=a^{2p+k}y^p $ which is clearly not in $ L_1 $ contradicting the pumping lemma. Hence, $ L_1 $ is \emph{not} regular.\\
		That $ L_1 $ \emph{is} a CFL, can be seen by the fact that it is generated by the CFG\begin{equation}
		 \begin{aligned}
		 S&\to a^2Xb,\\
		 X&\to a^2Xb\pipe \epsilon.
		 \end{aligned}
		\end{equation}
		This concludes the proof.
	\end{proof}
	\subsection*{1.3}
	Let $ L $ be an arbitrary language that does not contain the empty string, $ \epsilon $. Define \begin{equation}
	OTHER(L)=\{x_1x_2x_3...x_k\pipe x_1\in L,\ x_2\notin L,\ x_3\in L,\ x_4\notin L,...x_k\in L,\ k\in \mathbb{N}_{\text{odd}}  \},
	\end{equation}
	where $ \N_{\text{odd}} $ are the odd integers. We show that if $ L $ is regular, then $ OTHER(L) $ is regular. We also show that if $ L $ is in NL then $ OTHER(L) $ is in NL.\\
	Notice there is something not well defined in this question, since in the explanation in the problem it is said that if $ L=\{aa\} $ then $ aaaaaa\notin OTHER(L) $, but clearly $$ aaaaaa=\underbrace{aa}_{\in L}\underbrace{\epsilon}_{\notin L} \underbrace{aa}_{\in L}\underbrace{\epsilon}_{\notin L} \underbrace{aa}_{\in L}\in OTHER(L). $$
	Thus if this explanation is true, we must redefine $ OTHER(L) $ under the assumption that $ x_i $ are all not the empty string in the definition of $ OTHER(L) $, in which case $ aaaaaa\notin L $.
	
	\begin{proof}
		If $ L $ is regular, then there is a regular expression, $ R $ such that $ L(R)=L $. But then $ OTHER(L)=L(R(\overline{R}R)^*) $, where $ \overline{R} $ denotes the regular expression for $ \overline{L}\setminus\{\epsilon\}=\overline{L\cup \{\epsilon\}} $ (regular languages are closed under complementation). $ R(\overline{R}R)^* $ is a regular expression, and therefore $ OTHER(L) $ is regular.\\
		If $ L $ is in NL, then there is a log-space TM, $ M $, that desides it. By theorem 8.27 we know that coNL$ = $NL and therefore there is also a log-space TM, $ \overline{M} $, that desides $ \overline{L} $. Consider now the log-space TM $ M_{OTHER}= $"On input $ w=w_1....w_n $
		\begin{enumerate}
			\item Store the length of the input, $ n $, on the worktape. If $ n=0 $ \emph{reject}.
			\item Non-deterministically choose $ i=1 $ to $ n $, and run $ M $ on $w_1...w_i $. If $ i=n $ and $ M $ accepts, \emph{accept}. If $ i<n $ and $ M $ rejects, skip step 3.
			\item \begin{addmargin}[2em]{0em}
				Non-deterministically choose $ j=i+1 $ to $ n $, and run $ M $ on the $ w_j...w_n $ and $ \overline{M_{OTHER}} $ on $ w_{i+1}...w_j $. If for some $ j $ $ M $ and $ \overline{M_{OTHER}} $ accept, \emph{accept}.
			\end{addmargin} 
		\end{enumerate}
		where $ \overline{M_{OTHER}} $ is the same as $ M_{OTHER} $ but with all $ M $ replaced by $ \overline{M} $ and with  $ \overline{M_{OTHER}} $ in step $ 3 $ replaced by $ M_{OTHER} $ and also with new counters $ \overline{i} $ and $ \overline{j} $. Thus $ M_{OTHER} $ calls $ \overline{ M_{OTHER} } $ and vice versa. Hence this uses the recursion theorem.
		Evidently, this machine recursively check that the input $ w $ can be split in $ w=x_1w_1z_1 $ and further that $ w_i=x_{i+1}w_{i+1}z_{i+1} $ where $ x_i,z_i\in L $ for $ i $ odd and $ x_i,z_i\notin L  $ for $ i $ even, until some point where $ w_n\in L $ and the machine accepts. If this is not the case then clearly $ w\notin OTHER(L) $. Clearly, $ M_{OTHER} $ stores only the four counters counters $ i,j,\overline{i},\overline{j} $ which are alternatingly overwritten during the recursion, and simulates log-space TMs on smaller inputs than the original input, and therefore, $ M_{OTHER} $ runs itself in log-space.\\
		Alternatively, we might reduce to PATH which is known to be in NL. This done by the following log-space transducer, $ T $="On input $ w=w_1...w_n $\begin{enumerate}
			\item Produce graph $ G $ which have two vertices for each $ i=1,...,n $, $ v_i $, $ \tilde{v}_i $ on the output tape.
			\item Add one edge $ (v_i,\tilde{v_j}) $ for every $ i<j $ such that $ w_{i+1}...w_j\in L $ and one edge $ (\tilde{v_i},v_j) $ for every $ i<j $ such that $ w_i+1...w_j\notin L $ (i.e. $ w_i...w_j\in \overline{L} $) .
			\item Add vertices $ s,t $ and edges $ (s,\tilde{v_j}) $ if $ w_1...w_j\in L $ and $ (v_j,t) $ if $ w_j...w_n\in L $.
		\end{enumerate}
		Clearly $ G $ has a directed path from $ s $ to $ t $ if and only if $ w=x_1x_2....x_k $ such that $ x_1\in L $, $ x_2\notin L $,...,$ x_k\in L $. Furthermore, $ T $ is a log-space transducer as all step requires only $ G $ to store simple counters. Thus we see that we have constructed a log-space tranducer such that $ T(w)\in PATH $ if and only if $ w\in OTHER(L) $. Hence $ OTHER(L)\leq_L PATH $, and since $ PATH $ is in NL, we conclude that also $ OTHER(L) $ is in NL.
	\end{proof}
	
	
	\section*{Question 2}
	\subsection*{Part 2.1}
	Let \begin{equation*}
	\begin{aligned}
	MAXCELL_{TM}=\{\braket{M,w,k}\pipe M\text{ is a determinitstic TM, }w\text{ is a string, and }k\in \N_0\\ \text{ such that for every tape cell }c\text{ of }M,\ M\text{ writes to }c\text{ at most }k\\\text{ times during its execution on input }w\}.
	\end{aligned}
	\end{equation*}
	We show that there is a constant $ k\in\N_0 $ and a TM, $ N $, which on input $ \braket{M,w} $ simulates $ M $ on $ w $ such that $ N $ writes to every tape cell at most $ k $ times. Simply construct the TM $ N $ as the universal turing machine, but with the extra criterion, that it marks the last character on the tape. Then every time it simulates a single step of the computation of $ M $ on $ w $, it copies the entire tape, to the right of the marked character on the tape, and updates it simultaneously according to the single computation step. Clearly this TM uses a lot of space, but writes at most once to every cell. 
	
	\subsection*{Part 2.2}
	We now show that $ MAXCELL_{TM} $ is not Turing recognizable. \begin{proof}
		This follows if we can reduce $ \overline{A_{TM}} $ to $ MAXCELL_{TM} $, as it is known that $ \overline{A_{TM}} $ is not Turing recognizable. Consider the following reduction. Given $ \braket{M,w} $, construct $ \braket{\tilde{N},\braket{M,w},1} $, where $ \tilde{N} $ is the TM from part 2.1  but with the additional property, that it overwrites the entire tape with blanks if it reaches the accept state of $ M $. Thus if $ M $ accepts $ w $ $ \braket{\tilde{N},\braket{M,w},1}\notin MAXCELL_{TM} $ and if $ M $ does not accept $ w $, $ \braket{\tilde{N},\braket{M,w},1}\in MAXCELL_{TM} $. Clearly, there exist a TM that given input $ \braket{M,w} $ halts with $ \braket{\tilde{N},\braket{M,w},1}  $ on the tape. Hence the construction above constitutes a Turing reduction from $ \overline{A_{TM}} $ to $ MAXCELL_{TM}  $ and we have $ \overline{A_{TM}}\leq_T MAXCELL_{TM}  $, from which it follows that $ MAXCELL_{TM} $ is not Turing recognizable.
	\end{proof}
	
	\subsection*{Part 2.3}
	Let $ MAXCELL_{LBA} $ be like $ MAXCELL_{TM} $ with with TM $ M $ replaces by LBA $ M $. We show that $ MAXCELL_{LBA} $ is decidable\begin{proof}
		This follows easily by noticing that an LBA has only just enough tape to contain the input. Thus we may design a TM, $ N $, that simulates LBA, $ M $ on the input $ w $ and simultaneously keeps a count of how many times $ M $ writes to any tape slot. Thus consider the decider $ D= $" On input $ \braket{M,w,k} $, 
		\begin{enumerate}
			\item Run $ N $ on $ \braket{M,w} $ one step at a time, and update counters described above at each computation step.
			\item If a counter exeeds $ k $, \emph{reject}.
			\item If $ N $ halts, and all counters are below $ k $, \emph{accept}.
		\end{enumerate} 
		Clearly this is a decider, since if $ w $ has lenght $ n $ there can be at most $ nk $ computation steps before a counter exeeds $ k $. Furthermore, we see by straightforward inspection that it this decider exactly accepts $ MAXCELL_{LBA} $.
	\end{proof}
	
	
	\section*{Question 3}
	\subsection*{Part 3.1}
	We formulate $ WEAK-HAMPATH $ as the laguage\begin{equation*}
	\begin{aligned}
		k-WEAK-HAMPATH=\{\braket{G,s,t}\pipe G\text{ is a directed graph, and there exist a simple path}\\\text{$ \gamma $ in $ G $ from $ s $ to $ t $ such that }\abs{\gamma}\geq\abs{V(G)}-k\}
	\end{aligned}
	\end{equation*}
	where for a path, $ \gamma $, we denote by $ \abs{\gamma} $ the number of vertices the path visits. Next we show that $ k-WEAK-HAMPATH $ is in NP\begin{proof}
		Consider the polynomial time verifier $ V= $"On input $ (\braket{G,s,t},c) $\begin{enumerate}
			\item Check that $ G $ is a directed graph.
			\item Check that $ c $ is a simple path in $ G $ that starts at $ s $ and ends at $ t $.
			\item Check that the number of vertices in $ c $ is greater than or equal to $ \abs{V(G)}-k $
		\end{enumerate}
		Each step requires trivially only polynomial time. Furhermore, it is evident that $ \braket{G,s,t} $ is in $ k-WEAK-HAMPATH $ if and ony if there exist a $ c $ such that $ (\braket{G,s,t},c) $ is accepted by $ V $. This shows that $ k-WEAK-HAMPATH $ is verified in polynomial time, and we conclude that $ k-WEAK-HAMPATH $ is in NP.
	\end{proof}
	
	\subsection*{Part 3.2}
	We show that $ k-WEAK-HAMPATH $ is NP-complete by reducing from $ HAMPATH $.
	\begin{proof}
		Consider the following reduction: Given $ \braket{G,s,t} $ produce $ \braket{\tilde{G},s,t} $ where $ \tilde{G} $ is just $ G $ with $ k $ new vertices that are not connected to anything. Clearly if $ G $ has a Hamiltonian path from $ s $ to $ t $ then $ \braket{\tilde{G},s,t} $ is in $ k-WEAK-HAMPATH $, and if $ \braket{\tilde{G},s,t} $ is in $ k-WEAK-HAMPATH $ then $ G $ has a Hamiltonian path from $ s $ to $ t $, as no path can visit the extra vertices. The reduction is obviously polynomial time, since it only adds a finite number of vertices to the existing data. Thus we have shown that $ HAMPATH\leq_T k-WEAK-HAMPATH $. Since $ HAMPATH $ is NP-complete, we conclude that $ k-WEAK-HAMPATH $ is NP-hard, and it follows by Part 3.1 that $ k-WEAK-HAMPATH $ is NP-complete.
	\end{proof}
	\subsection*{Part 3.3}
	We now consider $ FULLPATH $, which can be formulated as laguage \begin{equation}
	\begin{aligned}
	FULLPATH=\{\braket{G,s,t}\pipe G\text{ is a directed graph and there exist a path, $ \gamma $,}\\\text{ from $ s $ to $ t $ such that }\abs{\gamma}=\abs{V(G)}\}.
	\end{aligned}
	\end{equation}
	\subsection*{Part 3.4}
	We show that $ FULLPATH $ is in P.
	\begin{proof}
		We use the hint in the problem at let $ M $ be the TM that in polynomial time, on input $ H $, where $ H $ is a directed graph, computes a partition $ (V_1,...,V_k) $ of the vertices of $ H $ such that $ H(V_i) $ is strongly connected for each $ i $, and in $ H $ there are edges from $ V_i $ to $ V_j $ if and only if $ i\leq j $. A graph $ H $ has a path from vertices $ s $ to $ t $ if and only if $ s\in V_1 $, $ t\in V_k $, and there are edges from $ V_i $ to $ V_{i+1} $ for each $ i<k $. On one hand if these conditions are satisfied we can construct the path from $ s $ to $ t $ by going $ s $ to all vertices in $ V_1 $ and then to $ V_2 $ and around all vertices in $ V_2 $ and so on, untill we end at $ V_k $ in which case we visit all vertices in $ V_k $ ending up at $ t $. On the contrary if there is no edge from $ V_i $ to $ V_{i+1} $ for some $ i $, then we either not get past $ V_i $, since we cannot go back to $ V_l $ for $ l<i $, or we cannot visit $ V_i $ at all, if we have already skipped it, again since we can not go back from $ V_j $ with $ j>i $. Thus we construct the following TM, $ M_1= $"On input $ \braket{G,s,t} $
		\begin{enumerate}
			\item Run $ M $ on $ G $.
			\item Check that $ s\in V_1 $ and $ t\in V_k $ (where $ V_k $ is the last element in the partition produced by $ M $) if not, \emph{reject}.
			\item For $ i=1 $ to $ n $ choose a vertex $ v_i\in V_i $ and run the TM that desides $ PATH $ in polynomial time, on $ \braket{G,v_i,v_{i+1}} $, if yes for all $ i $, \emph{accept}, else, \emph{reject.}  
		\end{enumerate}
		Since $ V_i $ is strongly connected for all $ i $, any vertex $v_i\in V_i  $ suffices in step $ 3 $.
	\end{proof} 
	
	\section*{Question 4}
	Consider the laguage\begin{equation*}
		\begin{aligned}
		\text{NO-FIERY-DEATH}=\{\braket{G,M,D}\pipe &G=(V,E)\text{ is a directed graph,}\\
		&M\subseteq V,\\
		&D\subseteq V,\\
		&\text{there is no path from any vertex in $ M $ to any vertex in $ D $}\}
		\end{aligned}
	\end{equation*}
	We show that $ \text{NO-FIERY-DEATH} $ is NL-complete.
	\begin{proof}
		We show that $ \text{NO-FIERY-DEATH} $ is in NL by reducing to $ \overline{PATH} $ which is in NL by Theorem 8.27. Consider the following log-space transducer, $ T= $"On input $ \braket{G,M,D} $ Construct the graph $ \tilde{G} $ by\begin{enumerate}
			\item For each pair $ (u,v)\in M\times D $ add a copy of $ G $, call the copy $ G^{u,v} $. 
			\item Add vertex $ s $ and for each $ u\in M $ the line $ (s,u^{(u,v)}) $ where $ u^{(u,v)} $ denotes the vertex $ u $ in $ G^{(u,v)} $.
			\item Add vertex $ t $ and for each $ v\in D $ the line $ (v^{(u,v)},t) $.
		\end{enumerate}
		Clearly, we see that if there is a path from a vertex $ u_0\in M $ to $ v_0\in D $ then there is a path from $ s\to u_0^{(u_0,v_0)}\to...\to v_0^{(u_0,v_0)}\to t $ in $ \tilde{G} $, so $ \braket{\tilde{G},s,t}\in PATH $, on the other hand if there is no path from any $ u\in M $ to any $ v\in D $, then there is also no path from $ s $ to $ t $, as such a path would be bound to go from some vertex in $ M $ to some vertex $ D $ along the way. Therefore, $ \braket{G,M,D}\in \text{NO-FIERY-DEATH} $ if and only if $ \braket{\tilde{G},s,t}\in \overline{PATH} $. Thus $ T $ shows that $ \text{NO-FIERY-DEATH}\leq_L \overline{PATH} $, and it follows that $ \text{NO-FIERY-DEATH} $ is in NL.\\
		That $ \text{NO-FIERY-DEATH} $ is NL-hard follows by the completely trivial reduction $ \overline{PATH}\leq_L \text{NO-FIERY-DEATH} $. This follows by viewing $ \overline{PATH} $ as being a subset of $ \text{NO-FIERY-DEATH} $ by restricting to cases where $ M=\{s\} $ and $ D=\{t\} $ are singeltons. This restriction can clearly be computed by the log-space tranducer, $ T'= $On input $ \braket{G,s,t} $ output $ \braket{G,\{s\},\{t\}} $. Thus we need only notice that $ \overline{PATH} $ is NL-complete, which follows by Theorem 8.27, saying that NL$ = $coNL. Thus for any language $ A $ in NL, we have $ \overline{A}\leq_L PATH $, by PATH being NL-hard. But then since complementation, and noting that the definition of $ \leq_L $ is symmetric under complementation, we notice that $ A\leq_L \overline{PATH} $. Since $ A $ was any language in NL, we see that $ \overline{PATH} $ is NL-hard. The reduction $ \overline{PATH}\leq_L \text{NO-FIERY-DEATH} $ hence shows that $ \text{NO-FIERY-DEATH} $ is NL-hard, and by the above it follows that it is NL-complete.
	\end{proof}
	\subsection*{Part 4.2}
	We consider now the language \begin{equation*}
		\begin{aligned}
		\text{NO-FIERY-DEATH-WITHOUT-ROD}=\{\braket{G,M,D,R}\pipe &G=(V,E)\text{ is a directed graph,}\\
		&M\subseteq V,\ D\subseteq V,\ R\subseteq V,\\
		&\text{If there is a path from a vertex in $ M $}\\
		&\text{to a vertex in $ D $, then it contains}\\
		&\text{a vertex in $ R $}\}
		\end{aligned}
	\end{equation*}
	We show that $ \text{NO-FIERY-DEATH-WITHOUT-ROD} $ is NL-complete.
	\begin{proof}
		We notice that if $ \braket{G,D,M,R}\in \text{NO-FIERY-DEATH-WITHOUT-ROD} $ if and only if either $ \braket{G,D,M}\in\text{NO-FIERY-DEATH} $, or $ \braket{G,D,R}\in\text{NO-FIERY-DEATH} $, or\\ $ \braket{G,R,M}\in\text{NO-FIERY-DEATH} $. In the following we let $ N $ be the log-space NTM that desides $ \text{NO-FIERY-DEATH} $, whose existence was shown in Part 4.1. Thus we may construct the log-space NTM $ N_1= $"On input $ \braket{G,M,D,R} $\begin{enumerate}
			\item Non-deterministically run $ N $ on $ \braket{G,M,D} $, $ \braket{G,M,R} $, and $ \braket{G,R,D} $. If one of them accepts, \emph{accept}.
			\item \emph{Reject}.
		\end{enumerate}
		Clearly, by the above observation, this NTM accept, $\text{NO-FIERY-DEATH-WITHOUT-ROD} $ and runs in logarithmic space. Hence $ \text{NO-FIERY-DEATH-WITHOUT-ROD} $ is in NL. On the other hand we may easily log-space reduce, $ \text{NO-FIERY-DEATH} $ to $ \text{NO-FIERY-DEATH-WITHOUT-ROD} $ by the following tranducer, $ T_2= $"On input $ \braket{G,M,D} $\begin{enumerate}
			\item Add vertex $ \dot{v} $ to $ G $ (call the new graph $ \dot{G} $).
			\item Write to output tape $ \braket{\dot{G},M,D,\{\dot{v}\}} $.
		\end{enumerate}
		Clearly if $ \braket{G,M,D} $ is in $ \text{NO-FIERY-DEATH}  $ then $ \braket{G,M,D,\{\dot{v}\}} $ is in\\ $ \text{NO-FIERY-DEATH-WITHOUT-ROD} $, on the other hand, since $ \dot{v} $ is disconnected from all other vertices in $ \dot{G} $ we see that no path can pass through $ \dot{v} $. Hence if $ \braket{\dot{G},M,D,\{\dot{v}\}} $ is in $ \text{NO-FIERY-DEATH-WITHOUT-ROD} $ we clearly must have $ \braket{G,M,D}\in \text{NO-FIERY-DEATH} $. Therefore, it follows that $ \text{NO-FIERY-DEATH}\leq_L \text{NO-FIERY-DEATH-WITHOUT-ROD} $, which show that $ \text{NO-FIERY-DEATH-WITHOUT-ROD} $ is NL-hard, and it follows that its is NL-complete.
	\end{proof}
\end{document}