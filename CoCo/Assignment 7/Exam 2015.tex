\documentclass[a4paper,11pt]{article}
\usepackage[utf8]{inputenc}
\usepackage[margin=1in]{geometry}
\usepackage{pdfpages}
\usepackage{mathrsfs}
\usepackage{amsfonts}
\usepackage{amsmath}
\DeclareMathOperator\arctanh{arctanh}
\usepackage{amssymb}
\usepackage{bbm}
\usepackage{amsthm}
\usepackage{graphicx}
\usepackage{centernot}
\usepackage{caption}
\usepackage{subcaption}
\usepackage{braket}
\usepackage{pgfplots}
\usepackage{lastpage}
\usepackage{enumitem}
\usepackage{setspace}
\usepackage{xcolor}
\usepackage{cancel}
\usepackage{scrextend}
\usepackage[english]{babel} 

\usepackage[square,sort,comma,numbers]{natbib}
\usepackage[colorlinks=true,linkcolor=blue]{hyperref}

\usepackage{fancyhdr}
\newcommand{\euler}[1]{\text{e}^{#1}}
\newcommand{\Real}{\text{Re}}
\newcommand{\Imag}{\text{Im}}
\newcommand{\supp}{\text{supp}}
\newcommand{\pare}[1]{\left( #1 \right)}
\newcommand{\norm}[1]{\left\lVert #1 \right\rVert}
\newcommand{\abs}[1]{\left\lvert #1 \right\rvert}
\newcommand{\floor}[1]{\left\lfloor #1 \right\rfloor}
\newcommand{\Span}[1]{\text{span}\left(#1\right)}
\newcommand{\dom}[1]{\mathscr D\left(#1\right)}
\newcommand{\Ran}[1]{\text{Ran}\left(#1\right)}
\newcommand{\conv}[1]{\text{co}\left\{#1\right\}}
\newcommand{\Ext}[1]{\text{Ext}\left\{#1\right\}}
\newcommand{\vin}{\rotatebox[origin=c]{-90}{$\in$}}
\newcommand{\vnotin}{\rotatebox[origin=c]{-90}{$\notin$}}
\newcommand{\interior}[1]{%
	{\kern0pt#1}^{\mathrm{o}}%
}
\newcommand*\diff{\mathop{}\!\mathrm{d}}
\newcommand{\ie}{\emph{i.e.} }
\newcommand{\eg}{\emph{e.g.} }
\newcommand{\dd}{\partial }
\newcommand{\N}{\mathbb{N}}
\newcommand{\Z}{\mathbb{Z}}
\newcommand{\R}{\mathbb{R}}
\newcommand{\C}{\mathbb{C}}
\newcommand{\w}{\mathsf{w}}

\newcommand{\Gliminf}{\Gamma\text{-}\liminf}
\newcommand{\Glimsup}{\Gamma\text{-}\limsup}
\newcommand{\Glim}{\Gamma\text{-}\lim}
\newcommand{\pipe}{\ \vert \ }
\newcommand{\tm}{\text{-}}

\newtheorem{theorem}{Theorem}
\newtheorem{definition}{Definition}
\newtheorem{proposition}{Proposition}
\newtheorem{lemma}{Lemma}
\newtheorem{corollary}{Corollary}

\numberwithin{equation}{section}
\linespread{1.3}

\pagestyle{fancy}
\fancyhf{}
\rhead{CoCo - Exam 2015}
\lhead{}
\rfoot{\thepage}
\lfoot{Dated: \today}
\author{Johannes Agerskov}
\date{Dated: \today}
\title{CoCo - Exam 2015}
\begin{document}
	
	\maketitle
	\section*{Question 1}

	Let \begin{equation}
		\begin{aligned}
		L_0&=\{a^kb^lc^md^n\pipe k,l,m,n\geq0\}\\
		L_1&=\{a^kb^lc^md^n\pipe k,l,m,n\geq0\text{ and at least two of $ k,l,m,n $ are equal}\},\\
		L_3&=\{a^kb^lc^md^n\pipe k,l,m,n\geq0\text{ and at least three of $ k,l,m,n $ are equal}\}
		\end{aligned}
	\end{equation}
		\subsection*{Part 1.1}
		We show that $ L_0 $ is regular. \begin{proof}
			This follows since $ L_0=L(a^*b^*c^*d^*) $, \ie $ L_0 $ is the language of a regular exression.
		\end{proof}
		\subsection*{Part 1.2}
		We show that $ L_1 $ is context-free, but not regular.
		\begin{proof}
			That $ L_1 $ is context free follows from the fact, that it is generated by the CFG
			\begin{equation}
			\begin{aligned}
			S&\to E_{ab}A_{cd}\pipe	A_{ab}E_{cd}\pipe E_{ac}A_d\pipe E_{ad}\pipe A_aE_{bc}A_d\pipe A_aE_{bd},\\
			E_{ab}&\to aE_{ab}b\pipe \epsilon,\\	
			A_{cd}&\to cA_{cd}\pipe A_{cd}d\pipe\epsilon,\\
			E_{ac}&\to aE_{ac}c\pipe A_b,\\
			A_b&\to bA_b\pipe \epsilon,\\
			E_{cd}&\to cE_{cd}d\pipe\epsilon,\\
			A_{ab}&\to aA_{ab}\pipe A_{ab}b\pipe\epsilon,\\
			E_{ad}&\to aE_{ad}d\pipe A_{bc},\\
			A_{bc}&\to bA_{bc}\pipe A_{bc}c\pipe\epsilon,\\
			A_a&\to aA_a\pipe\epsilon,\\
			A_d&\to A_dd\pipe\epsilon,\\
			E_{bc}&\to bE_{bc}c\pipe\epsilon,\\
			E_{bd}&\to bE_{bd}d\pipe A_c,\\
			A_c&\to cA_c\pipe\epsilon.	
			\end{aligned}
			\end{equation}
			Evidently, any derivation of this CFG will first choose a branch, each of which correspond to choosing which two letters will appear in the same multiplicity (this is what $ E_ij $ does) in the final expression, and the rest of such a derivation then simply produce the remaining letters at arbitrary multiplicity ($ A_{ij},\ A_i $).
			That $ L_1 $ is not regular follows from the pumping lemma, by the following argument: Assume that $ L_1 $ is regular and let $ p $ denote the pumping length as given by the pumping lemma. Then $ a^pb^pc $ has length greater than $ p $, so by the pumping lemma it may be pumped. Thus $ a^pb^pc=xyz $ with $ \abs{y}>0 $ and $ \abs{xy}\leq p $. Hence we see that $ xy=a^k $ and hence $ y=a^m $ for some $ 0<k,m\leq p $. But then $ xy^iz=a^{p+m(i-1)}b^pc $, which for $ i=1 $ has no two letters which are of equal multiplicity, and therefore $ xy^iz\notin L_1 $ and we have a contradiction. We thus conclude that $ L_1 $ is not regular.
		\end{proof}
		\subsection*{Part 1.3}
		We show that $ L_2 $ is in L.
		\begin{proof}
			This follows by constructing a log-space TM that desides $ L_2 $. Consider the machine $ M= $"On input $ w $\begin{enumerate}
				\item Check that $ w=a^kb^lc^md^n $ for some $ k,l,m,n\geq0 $. If not \emph{reject}.
				\item Scan the worktape from left to right, and store counters for the numbers of $ a $s, $ b $s, $ c $s, and $ d $s.
				\item Check if any three counters are equal, if yes, \emph{accept}, else, \emph{reject}. 
			\end{enumerate}
			Clearly this TM desides $ L_2 $, and it stores only counters which are log-space. Notice that step one can be done in log-space since it essentially just checks that $ w\in L_1 $, and $ L_1 $ is regular from which it follows that $ L_1\in L $. Thus it shows that $ L_2\in \text{L} $
		\end{proof}
		\section*{Question 2}
	In the following $ M $ and $ N $ are NFAs. Let $ E_{NFA}=\{\braket{M}\pipe L(M)=\emptyset\} $ and\\ $ EQ_{NFA}=\{\braket{M,N}\pipe L(M)=L(N)\} $.
	\subsection*{Part 2.1}
	We show that $ E_{NFA} $ is in NL. \begin{proof}
		By theorem 8.27 it is sufficient to show that $ E_{NFA} $ is in coNL. Consider therefore\\ $ \overline{E_{NFA}}=\{\braket{M}\pipe L(M)\neq \emptyset\} $. We claim that the following non-deterministic log-space TM desides $ \overline{E_{NFA}} $, $ M= $"On input $ \braket{M} $\begin{enumerate}
			\item Store the start state of $ M $ on the worktape.
			\item Set $ i:=0 $ on the worktape.
			\item While $ M $ is not in the accept state:
			\item \begin{addmargin}[2em]{0em}
				Non-deterministically feed $ M $ a letter from the input alphabet of $ M $, and non-deterministically update the state according to $ M $s transition function on the current state and the fed letter.
			\end{addmargin} 
			\item \qquad Update $ i:=i+1 $
			\item \qquad If $ i>n $, where $ n $ is the number of states in $ M $, \emph{reject}
			\item \emph{accept}.
		\end{enumerate}
		Clearly, this machine stores only a counter and the current state of $ M $ at all times in the calculation and therefore, it uses at most log-space. On the other hand it desides $ \overline{E_{NFA}} $ since it clearly accepts if it find any sequence of letters that consitute an accept string of $ M $, on the other hand, if such a string exists, we know by the pigeon hole principle, that there is an accepting string of length at most the number of states in $ M $. Thus  $ \overline{E_{NFA}} $  is in NL, and therefore $ E_{NFA} $ is in coNL$ = $NL.
	\end{proof}
	\subsection*{Part 2.2}
	We show that $ EQ_{NFA} $ is in PSPACE.\begin{proof}
		Notice first that PSPACE, since PSPACE is based in deterministic TMs, is closed under complementation. If $ A $ is in PSPACE there is a deterministic polynomial space TM, $ M $, that desides $ A $. Define $ \overline{M} $ by flipping accept and reject states, then we see that $ \overline{M} $ is a polynomial space determinitstic TM that desides $ \overline{A} $.\\
		Now we show that $ \overline{EQ_{NFA}}=\{\braket{M,N}\pipe L(M) \neq L(N)\} $ is in PSPACE. Eqiuvalently we may show that $ \overline{EQ_{NFA}} $ is in $ NPSPACE=PSPACE $ by Savitch's theorem.
		Consider the following machine, $ M= $"On input $ \braket{M,N} $\begin{enumerate}
			\item Non-deterministically store a string, $ w $, on the worktape of length less that $ nm $ where $ n $ is number of states in $ N $ and $ m $ is number of states in $ M $.
			\item Run $ M $ and $ N $ on $ w $, if one accepts and the other rejects, \emph{accept}
			\item \emph{reject.}
		\end{enumerate}
		Notice that if any string is longer than $ nm $, then there must be some substring on which $ N $ and $ M $ start and end in the same state, since $ N $ and $ M $ has $ nm $ possible different combined configurations. Thus if there is a string that is longer than $ nm $, then we may cut away a substring, making it shorter than $ nm $, without changing the outcome of $ N $ and $ M $. Therefore, $ M $ desides $ \overline{EQ_{NFA}} $, furthermore, it clearly runs in polynomial space. Thus we conclude that $ EQ_{NFA} $ is in PSPACE.
	\end{proof}
	
	\section*{Question 3}
	We define $ ONE\tm HALT=\{\braket{A,B}\pipe A,B\text{ are TMs and on every input exactly one of them halts}\} $.
	\subsection*{Part 3.1}
	We show that $ ONE\tm HALT $ is not Turing-recognizable\begin{proof}
		We use that $ \overline{HALT_{TM}} $ is not Turing-recognizable and show that $ \overline{HALT_{TM}}\leq_m ONE\tm HALT $.
		That $ \overline{HALT_{TM}} $ is not Turing recognizable follows by the simply observation that $ HALT_{TM} $ is trivially Turing recognizable but by Theorem 5.1 it is not decidable, thus the claim follows by theorem 4.22\\
		Consider now the following mapping reduction: Given $ \braket{M,w} $ construct the TM $ M_w $ which ignores it input and run $ M $ on $ w $, and output $ \braket{M_w,N} $ where $ N $ is the TM that immeadiately accepts (halts) on all input. Clearly if $ M $ does not halt on $ w $ we have $ \braket{M_w,N}\in ONE\tm HALT $, and if $ M $ does halt on $ w $ we have $ \braket{M_w,N}\notin ONE\tm HALT $. Clearly this mapping is computable, and we conclude that $ \overline{HALT_{TM}}\leq_m ONE\tm HALT $. By Corollary 5.29 it follows that $ ONE\tm HALT $ is not Turing-recognizable. 
	\end{proof}
	\subsection*{Part 3.2}
	We show that every finite language is decidable. \begin{proof}
		This follows from the fact that every finite language is regular, and every regular language is context-free and by theorem 4.9 every context-free language is decidable.
	\end{proof}
	\subsection*{Part 3.3}
	We show that a language $ L $ is decidable if and only if it is enumerated in lexicographical order by some enumerater, that never halts. \begin{proof}
		Clearly if $ L $ is decidable there is a decider, $ D $ for $ L $, and we can construct an enumerator that for all string, in lexicographical order runs $ D $ on the string, and prints it (with suitable seperator symbols) if and only if $ D $ accept. This enumerator will enumerate $ L $ in lexicographical order, but it will never halt. On the contrary if there is such an enumerator that enumerates $ L $ in lexicographical order then either $ L $ is finite in which case it is decidable or it is Turing recognizable by theorem 3.21. On the other side if there is a lexicographical order enumerator for $ L $, then we can construct a lexicographical order enumerator for $ \overline{L} $ that simply checks for each string, if the enumerator for $ L $ prints (can be done because of ordering) that string, if not print it. But then also $ \overline{L} $ is Turing recognizable and by theorem 4.22 $ L $ is decidable.
	\end{proof}
	\section*{Question 4}
	Define \begin{equation*}
		SUBGRAPH\tm 100\tm ISOMORPHIMSM=\{\braket{G_1,G_2}\pipe G_1\text{ is a subgraph of $ G_2 $ and }\abs{V(G_1)}\leq100\} 
	\end{equation*}
	\subsection*{Part 4.1}
	We show that $ SUBGRAPH\tm 100\tm ISOMORPHISM $ belongs to P.
	\begin{proof}
		Consider the following polynomial time TM, $ M= $"On input $ \braket{G_1,G_2} $\begin{enumerate}
			\item Check that $ G_1 $ has less than 100 vertices, and that $ G_2 $ has more vertices than $ G_1 $, if not \emph{reject}.
			\item In all possible ways, one at a time: For each vertex, $ v $, of $ G_1 $, associate a vertex, $ \tilde{v} $, in $ G_2 $. 
			\item \qquad  Check for each edge $ (u,v)\in E(G_1) $ that $ (\tilde{u},\tilde{v})\in E(G_2) $.
			\item If step 4 is affirmative for some association of vertices, \emph{accept}, if step 4 fails for all associations of vertices \emph{reject.}
		\end{enumerate}
		Clearly this TM tries to map the vertices of $ G_1 $ one-to-one to the vertices of $ G_2 $ in all possible ways. If for some one-to-one map, the image of $ G_1 $ is a subgraph of $ G_2 $, $ M $ accept and if this is not the case for any map, $ M $ rejects. Thus $ L(M)=SUBGRAPH\tm 100\tm ISOMORPHISM $. Furhtermore, Each of the steps are polynomial in time: Step 1 is clearly polynomial time, Step 2 is polynomial in time, since there are at most $ \abs{V(G_2)}^{100} $ ways of associating less than 100 vertices to $ \abs{V(G_2)} $ vertices, and Step 3 is for each of these associations trivially polynomial in time ($ O(n^2) $). Therefore, we conclude that $ SUBGRAPH\tm 100\tm ISOMORPHISM $ belongs to $ P $.
	\end{proof}
	\end{document}