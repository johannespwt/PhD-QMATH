\documentclass[a4paper,11pt]{article}
\usepackage[utf8]{inputenc}
\usepackage[margin=1in]{geometry}
\usepackage{pdfpages}
\usepackage{mathrsfs}
\usepackage{amsfonts}
\usepackage{amsmath}
\DeclareMathOperator\arctanh{arctanh}
\usepackage{amssymb}
\usepackage{bbm}
\usepackage{amsthm}
\usepackage{graphicx}
\usepackage{centernot}
\usepackage{caption}
\usepackage{subcaption}
\usepackage{braket}
\usepackage{pgfplots}
\usepackage{lastpage}
\usepackage{enumitem}
\usepackage{setspace}
\usepackage{xcolor}
\usepackage{cancel}
\usepackage{scrextend}
\usepackage[english]{babel} 

\usepackage[square,sort,comma,numbers]{natbib}
\usepackage[colorlinks=true,linkcolor=blue]{hyperref}

\usepackage{fancyhdr}
\newcommand{\euler}[1]{\text{e}^{#1}}
\newcommand{\Real}{\text{Re}}
\newcommand{\Imag}{\text{Im}}
\newcommand{\supp}{\text{supp}}
\newcommand{\pare}[1]{\left( #1 \right)}
\newcommand{\norm}[1]{\left\lVert #1 \right\rVert}
\newcommand{\abs}[1]{\left\lvert #1 \right\rvert}
\newcommand{\floor}[1]{\left\lfloor #1 \right\rfloor}
\newcommand{\Span}[1]{\text{span}\left(#1\right)}
\newcommand{\dom}[1]{\mathscr D\left(#1\right)}
\newcommand{\Ran}[1]{\text{Ran}\left(#1\right)}
\newcommand{\conv}[1]{\text{co}\left\{#1\right\}}
\newcommand{\Ext}[1]{\text{Ext}\left\{#1\right\}}
\newcommand{\vin}{\rotatebox[origin=c]{-90}{$\in$}}
\newcommand{\interior}[1]{%
	{\kern0pt#1}^{\mathrm{o}}%
}
\newcommand*\diff{\mathop{}\!\mathrm{d}}
\newcommand{\ie}{\emph{i.e.} }
\newcommand{\eg}{\emph{e.g.} }
\newcommand{\dd}{\partial }
\newcommand{\N}{\mathbb{N}}
\newcommand{\Z}{\mathbb{Z}}
\newcommand{\R}{\mathbb{R}}
\newcommand{\C}{\mathbb{C}}
\newcommand{\w}{\mathsf{w}}

\newcommand{\Gliminf}{\Gamma\text{-}\liminf}
\newcommand{\Glimsup}{\Gamma\text{-}\limsup}
\newcommand{\Glim}{\Gamma\text{-}\lim}
\newcommand{\pipe}{\ \vert \ }

\newtheorem{theorem}{Theorem}
\newtheorem{definition}{Definition}
\newtheorem{proposition}{Proposition}
\newtheorem{lemma}{Lemma}
\newtheorem{corollary}{Corollary}

\numberwithin{equation}{section}
\linespread{1.3}

\pagestyle{fancy}
\fancyhf{}
\rhead{CoCo - Assignment 5}
\lhead{}
\rfoot{\thepage}
\lfoot{Dated: \today}
\author{Johannes Agerskov, Mads Friis Frand-Madsen, and Sifan Huang}
\date{Dated: \today}
\title{CoCo - Assignment 5}
\begin{document}

\maketitle
\section*{7.7}
We show that NP is closed under union and concatination.\begin{proof}
	We start by considering union. Let $ A,B\in \text{NP} $ and consider $ L=A\cup B $. We construct a polynomial time verifier for $ L $. We know that there exist polynomial time verifiers $ V_A $ and $ V_B $ for $ A $ and $ B $. Now Consider the verifier $ V $ that on input $ \braket{w,c} $, run $ V_A $ and $ V_B $ on $ \braket{w,c} $ and if one of them accepts, $ V $ accepts, otherwise it rejects. This is clearly polynomial in time, since it is just two polynomial time verifiers in series. Also $ V $ accepts $ w $ for some $ c $ if and only if $ w\in A $ or $ w\in B $ or equivalently if and only if $ w\in L=A\cup B $.\\
	 For the concatenation, let $ L'=AB $. We construct polynomial time verifier $ V' $ such that $ V' $ verifies $ L' $. A certificate, $ c $, in this case, is an encoding of three things: Where to split the input $ w $, a certificate for $ V_A $ on the left part, and a certificate for $ V_B $ on the right part. Given such a $ c=(\braket{k},c_A,c_B) $, where $ \braket{k} $ denotes a suitable encoding of the number $ k $, and input $ w=w_1...w_n $, $ V' $ acts as following on input $ \braket{w,c} $: $ V' $ splits the input, $ w $, at the $ k $th position, according to the certificate, $ c $. It then runs $ V_A $ on $ w_1...w_k $ with certificate $ c_A $ and $ V_B $ on $ w_{k+1}...w_n $ with $ c_B $. If both accept $ V' $ accepts. Clearly, $ V' $ accepts $ w $ for some $ c $ if $ w\in L'=AB $. On the other hand, if $ V' $ accepts $ w $ for some $ c $ there exist a splitting of $ w $ such that the left part is in $ A $ and the right part is in $ B $ or equivalently, $ w\in L'=AB $
\end{proof}

\section*{7.9}
We show that $ TRAINGLE=\{\braket{G}\ \vert\ G\text{ is an undirected graph that contains a }3\text{-clique}\} $ is in P.
\begin{proof}
	We consider the following algorithm. $ D= $"On input $ \braket{G} $\begin{enumerate}
		\item check that $ \braket{G} $ encodes an udirected graph. (say with vertices $ v_1,...,v_n $)
		\item For $ i=1 $ to $ n-2 $ select $ v_i $ and:
		\item \qquad For $ j=i+1 $ to $ n-1 $ select $ v_j $ and:
		\item \qquad \qquad For $ k=j+1 $ to $ n $ select $ v_k $ and:
		\item \qquad\qquad\qquad Check that $ \{v_i,v_j,v_k\} $ forms a $ 3 $-clique. If true accepts. If false and $ i=n-2 $, $ j=n-1 $ and $ k=n $, reject."
	\end{enumerate}
	Since there are $ \frac{n(n-1)(n-3)}{6}=O(n^3) $ ways to choose $ 3 $ vertices out of $ n $ and each selection process if polynomial in time, this algorithm is clearly polynomial in time. It is also clear that the algorithm accepts if and only if $ G $ contains a $ 3 $-clique (Triangle).\\
	Notice that we use that for any reasonable encoding of a $ \braket{G} $ the input length is polynomial in the number of vertices.
\end{proof}

\section*{7.10}
We show that $ ALL_{DFA}=\{\braket{A}\ \vert\ A\text{ is a DFA and }L(A)=\Sigma^* \} $ is in P.
\begin{proof}
	We show this, by noting that $ ALL_{DFA} $ can be solved with the TM, $ M $ from the proof that $ PATH $ is in P. Simply construct the TM, $ M'= $"On input $ \braket{A} $\begin{enumerate}
		\item Construct the state diagram, $ G $ of $ A $, vieved as a directed graph.
		\item For all states $ q\in Q\setminus F $ ($ F $ is the set of accepts states of $ A $), run $ M $ on $ \braket{G,q_{\text{start}},q} $, if it accepts for some $ q $, \emph{reject} if it rejects for all $ q $, \emph{accept}.
	\end{enumerate}
	Evidently, $ M' $ accepts exactly those DFAs that can never reach a non-accepts state, \ie those that accept the language $ \Sigma^* $. Since $ PATH $ is in P, $ M $ runs in polynomial time, $ \braket{G,q_{\text{start}},q} $ is polynomial in the input length, $ \abs{\braket{A}} $, and since the set $ Q $ is polynomial in the input length, we conclude that $ M' $ is a polynomial time TM that accepts $ ALL_{DFA} $.
\end{proof}

\section*{7.42}
We show that P is closed under the star operation.
\begin{proof}
	Let $ A $ be any language in P. We follow the hint, and construct a polynomial time TM that accepts $ A^* $. Let $ D $ be the polynomial time TM that accepts $ A $, and let $ M $ be the polynomial TM that accepts $ PATH $. Consider then, $ M'= $"On input $ y=y_1...y_n $
	\begin{enumerate}
		\item Build $ n\times n $-table, with entries $ T_{i,j} $, for $ i,j=1,...,n $, by the following procedure:\\
		For $ i=1 $ to $ n $
		\item\qquad For $ j=i $ to $ n $
			\item\begin{addmargin}[4em]{0em} Run $ D $ on $ y_i...y_j $, if it accepts, set $ T_{ij}=1 $, if it rejects set $ T_{ij}=0 $. For $ i>j $ set $ T_{ij}=0 $.
				\end{addmargin}
		\item Let $ G $ be the directed graph with adjacency matrix $ T $, \ie view $ \braket{T_{ij}}_{i,j=1}^{n} $ as an encoding of the graph $ G $ (we label the vertices of $ G $ by $ 1,...,n $).
		\item Run $ M $ on $ \braket{\braket{T_{i,j}}_{i,j=1}^{n},1,n} $, if it accepts, \emph{accept}, if it rejects, \emph{reject}. 
	\end{enumerate}
	By design, we that $ M' $ accepts $ y $ if and only if there is a path $ 1\to i_1\to i_2\to...\to i_l\to n $ such that $ T_{1,i_1}=T_{i_j,i_{j+1}}=T_{i_l,n}=1 $ for all $ j=1,...,l $, which is equivalent to $ y_1...y_{i_1}\in A $, $ y_{i_j}...y_{i_j+1}\in A $, and $ y_{i_l}...y_{n}\in A $ for all $ j=1,...,l $, or equivalently that $ y\in A^* $. Furthermore, we see that all steps are polynomial time the input length, and the number of steps is also polynomial in the input length from which we conclude that $ M' $ is a polynomial time TM that accepts $ A^* $. 
\end{proof}

\section*{7.44}
Let $ UNARY\text{-}SSUM $ be the subset sum problem where all numbers are represented in unary numbers.
\begin{proof}
	We describe an algorithm that accepts $ UNARY\text{-}SUM $ in polynomial time. Let $ M $ be an algorithm that solves the SUBSET-SUM problem in exponential time, which is known to exist, since $ SUBSET-SUM $ is in NP  by Theorem 7.25, and by Theorem 7.20 there then exist a non-deterministic TM that decides it in polynomial time $ O(n^m) $ for some $ m\geq 1 $, and then by Theorem 7.11 there exist a deterministic single tape TM that desides it in exponential time ($ 2^{O(n^m)} $). Consider $ D= $" on input $ \braket{\{i_1,...,i_n\},t} $ where $ i_1,...,i_n,t $ are unary numbers\begin{enumerate}
		\item For all $ j=1 $ to $ n $: Encode $ i_j $ as a binary number, $ b_j $, by straightforwardly starting with $ b_j=0 $ and then updating $ b_j $ while reading across $ i_j $. \eg $ i_j=1111 $ we would have $(b_j=0,\ 1111)\to (b_j=1,\ \dot{1}111)\to (b_j=10,\ \dot{1}11)\to (b_j=11,\ 11\dot{1}1)\to (b_j=100,\ 111\dot{1}) $, where the dot, symbolises the current position of, say the tape head in a TM. 
		\item Encode $ t $ as a binary $ \tilde{t} $ by same proceedure as in step 1. 
		\item For each subset of $ \{b_1,...b_n\} $, $ I $, calculate the sum of $ I $, and check if it equals $ \tilde{t} $.
	\end{enumerate}
	Notice that the input length of $ M $ in the above algorithm $ \abs{\braket{\{b_1,...,b_n\},\tilde{t}}}=O(\log(\abs{\braket{\{i_1,...,i_n\},t}})) $. Thus $ M $ will run in time $ 2^{O(\log(n)^m)}\leq2^{c\log(n)^m}\leq=n^c $ for some $ c $, thus it runs in polynomial time. Clearly converting unary to binary is also polynomial in time. Thus $ D $ runs in polynomial time, and accepts exactly $ UNARY\text{-}SSUM $. 
\end{proof}
\begin{proof}
	We describe an algorithm that accepts $ UNARY\text{-}SUM $ in polynomial time. Let $ A $ be any language in P, and let $ M_A^* $ be the algorithm from 7.42, that desides $  A^*  $ in polynomial time. Now given an input $ \braket{\{i_1,...,i_n\},t} $ where $ i_1,...,i_n,t $ are unary, we notice that $ \{i_1,...,i_n\} $ is trivially a language in P, \eg because membership can be checked by trial an error or because it is context free. Now consider the algorithm $ D= $" on input $ \braket{A,t} $ where $A=\{i_1,...,i_n\}$, and $ i_1,...,i_n,t $ are unary\begin{enumerate}
		\item Run $ M_{A^*} $ on $ t $, if it accepts, \emph{accept}, if it rejects, \emph{reject.}
	\end{enumerate}
	Clearly, $ D $ accepts $ \braket{\{i_1,...,i_n\},t} $ if and only if $ t=i $
\end{proof}

Let $ UNARY\text{-}SSUM $ be the subset sum problem where all numbers are represented in unary numbers. We show that $ UNARY\text{-}SSUM $ is in P.
\begin{proof}
	Consider the CFG, $ G $:\begin{equation*}
	\begin{aligned}\\
	S&\to A\vert B\\
	A&\to 1A1\vert \# B\vert \$\\
	B&\to1B\vert \#B\vert \#A\vert \$
	\end{aligned}
	\end{equation*}
	By inspection, it is clear that this CFG can produce strings of the form $ u_1\#...\#u_m\$u $ where for some $ 1\leq i_1<...<i_k\leq m $, we have $ u_{i_1}...u_{i_k}=u $, \eg $ 111\#11\#111111\#1\$1111 $. This is easily seen from the fact that we can produce a $ 1 $ to the right of $ \$ $ only by producing one to the left of $ \$ $ as well, furthermore, one can only stop producing $ 1 $s to the right of $ \$ $ by putting down a $ \# $ on the left. However, this language is clearly equivalent to $ UNARY\text{-}SSUM $, by the bijective the map $ \braket{\{u_1,...,u_m\},u}\mapsto \braket{u_1\#...\#u_m\$u} $ which is clearly both polynomial time comptable and its inverse is also polynomial time computable. Thus we conclude that $ UNARY\text{-}SSUM\leq_P L(G) $, but by theorem 7.16 $ L(G) $ is in P, and thus by Theorem 7.31 $ UNARY\text{-}SSUM $ is in P.
\end{proof}


\section*{Exam 2019, Question 3}
\subsection*{3.1}
\begin{equation*}
	\begin{aligned}
	i\text{-}RSP=\left\{\braket{G}\ \vert\ \text{G}\text{ is a graph and there exist tree subgraph $ T\subset G $ such that $ V(T)=V(G) $,}\right.\\\left.\text{ and for any vertex in $ T $ the degree is 0, 1, or $ i $} \right\} 
	\end{aligned}
\end{equation*}
\subsection*{3.2}
We show that $ 3\text{-}RSP $ is NP-complete.\begin{proof}
	We do this by reducing the NP-complete problem $ UHAMPATH $ to $ 3\text{-}RSP $. Consider the algorithm, $ D= $"On input $ \braket{G,s,t} $ where $ G $ is a graph and $ s,t $ are vertices in $ G $\begin{enumerate}
		\item Construct the graph $ G' $ such that $ G\subset G' $ by adding one vertex $ b_i $ to $ G $ for every vertex $ v_i\in G\setminus{s,t} $ where we identify $ V(G)=\{v_1,...,v_m,s,t\} $, and add one line between each $ v_i $ and $ b_i $.
	\end{enumerate}
	this is clearly polynomial in time. Furthermore, notice that if $ G $ has a Hamiltonian path from $ s $ to $ t $, then $ G' $ has a 3-regular spanning tree, since the path from $ s $ to $ t $ go through all $ v_i $ exactly once, and by adding the lines from $ v_i $ to $ b_i $ we see that the vertices $ \{v_1,...,v_m,b_1,...,b_m,s,t\} $ with the lines given by the Hamiltonian path from $ s $ to $ t $ and the lines from $ v_i $ to $ b_i $ forms a 3-regular spanning tree, such that $ v_i $ has degree $ 3 $ for all $ i=1,...,m $ and $ \{s,t,b_1,...,b_m\} $ are the leaves of degree $ 1 $. On the contrary, if $ G' $ has a 3-regular spanning tree, we see that $ \{b_1,...,b_m\} $ must be leaves since they have degree $ 1 $. Hence $ \{v_1,...,v_m\} $ has degree 3, but since $ s,t $ only connects to $ \{v_1,...,v_n\} $ they must themselves be leaves. Notice then that $ s $ and $ t $ actually must be at the bottom of the tree, since any $ v_i $ exept the bottom one can have at most $ 1 $ leaf, since they have degree $ 3 $. Thus a Hamiltonian path from $ s $ to $ t $ exist by going from $ s $ up the tree all the way to the root, and down the other branch all the way down to $ t $ at the bottom. This path goes through every vertex $ v_1,..,v_m $ exactly once. Therefore, we see that $ D $ maps $ UHAMPATH $ to $ 3\text{RSP} $ and $ \overline{UHAMPATH} $ to $ \overline{3\text{-}RSP} $. Thus the map $ G\mapsto G' $ is a polynomial time computable function, and we conclude that $ UHAMPATH\leq_P 3\text{-}RSP $ by Theorem 7.36 that $ 3\text{-}RSP $ is NP-complete.
\end{proof}
\end{document}