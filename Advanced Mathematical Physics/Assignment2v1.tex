\documentclass[a4paper,11pt]{article}
\usepackage[utf8]{inputenc}
\usepackage[margin=1in]{geometry}
\usepackage{pdfpages}
\usepackage{mathrsfs}
\usepackage{amsfonts}
\usepackage{amsmath}
\usepackage{amssymb}
\usepackage{bbm}
\usepackage{amsthm}
\usepackage{graphicx}
\usepackage{centernot}
\usepackage{caption}
\usepackage{subcaption}
\usepackage{braket}
\usepackage{pgfplots}
\usepackage{lastpage}
\usepackage{enumitem}
\usepackage{setspace}
\usepackage[english]{babel} 

\usepackage[square,sort,comma,numbers]{natbib}
\usepackage[colorlinks=true,linkcolor=blue]{hyperref}

\usepackage{fancyhdr}
\newcommand{\euler}[1]{\text{e}^{#1}}
\newcommand{\Real}{\text{Re}}
\newcommand{\Imag}{\text{Im}}
\newcommand{\floor}[1]{\left\lfloor #1 \right\rfloor}
\newcommand{\norm}[1]{\left\lVert #1 \right\rVert}
\newcommand{\abs}[1]{\left\lvert #1 \right\rvert}
\newcommand{\Span}[1]{\text{span}\left(#1\right)}
\newcommand{\dom}[1]{\mathscr D\left(#1\right)}
\newcommand{\Ran}[1]{\text{Ran}\left(#1\right)}
\newcommand{\conv}[1]{\text{co}\left\{#1\right\}}
\newcommand{\Ext}[1]{\text{Ext}\left\{#1\right\}}
\newcommand{\vin}{\rotatebox[origin=c]{-90}{$\in$}}
\newcommand{\interior}[1]{%
	{\kern0pt#1}^{\mathrm{o}}%
}
\newcommand*\diff{\mathop{}\!\mathrm{d}}
\newcommand{\ie}{\emph{i.e.} }
\newcommand{\eg}{\emph{e.g.} }
\newcommand{\dd}{\partial }
\newcommand{\R}{\mathbb{R}}
\newcommand{\C}{\mathbb{C}}
\newcommand{\w}{\mathsf{w}}

\newcommand{\Gliminf}{\Gamma\text{-}\liminf}
\newcommand{\Glimsup}{\Gamma\text{-}\limsup}
\newcommand{\Glim}{\Gamma\text{-}\lim}

\newtheorem{theorem}{Theorem}
\newtheorem{definition}{Definition}
\newtheorem{proposition}{Proposition}
\newtheorem{lemma}{Lemma}
\newtheorem{corollary}{Corollary}

\numberwithin{equation}{section}
\linespread{1.3}

\pagestyle{fancy}
\fancyhf{}
\rhead{Advanced Mathematical Physics, Assignment 1}
\lhead{Johannes Agerskov}
\rfoot{\thepage}
\lfoot{Dated: \today}
\author{Johannes Agerskov}
\date{Dated: \today}
\title{Advanced Mathematical Physics, Assignment 1}
\begin{document}

	\maketitle

\section{Stability through Lieb-Oxford inequality}
We are given the Lieb-Oxford inequality: For any bosonic or fermionic wave function $ \psi\in L^2(\R^{3N}) $ with $ \norm{\psi}_2=1 $ we have \begin{equation}
\sum_{1\leq i< j\leq N}\int_{\R^{3N}}\frac{\abs{\psi(x_1,...,x_N)}^2}{\abs{x_i-x_j}}\diff x_1...\diff x_N-D(\rho_\psi,\rho_\psi)\geq-C_{LO}\int_{\R^3}\rho_\psi(x)^{4/3}\diff x,
\end{equation}  
with constant $ 0\leq C_{LO}\leq1.636 $ independent of $ \psi $ and $ N $. We now proceed to prove stability of the second kind through this inequality.
\subsection*{(a)}
Let $ \delta>0 $ then \begin{equation}
\int_{\R^3}\rho_\psi(x)^{4/3}\diff x\leq\frac{\delta}{2}\int_{\R^3}\rho_{\psi}(x)^{5/3}\diff x+\frac{N}{2\delta}.
\end{equation}
\begin{proof}
	Notice first first that $ \rho_{\psi}(x)^{4/3}=\rho_{\psi}(x)^{5/6}\rho_{\psi}(x)^{1/2} $. Thus by Cauchy-Schwartz inequality, we have \begin{equation}
	\int_{\R^3}\rho_\psi(x)^{4/3}\diff x\leq\left(\int_{\R^3}\rho_\psi(x)^{5/3}\diff x\right)^{\frac{1}{2}}\left(\int_{\R^3}\rho_\psi(x)\diff x\right)^{\frac{1}{2}}=\left(\int_{\R^3}\rho_\psi(x)^{5/3}\diff x\right)^{\frac{1}{2}}\sqrt{N},
	\end{equation}
	where we used that $ \int_{\R^3}\rho_\psi(x)\diff x=N $. Now using that for $ \delta>0 $ and $ a,b\in \R $ is holds that $ \frac{\delta}{2}a^2+\frac{1}{2\delta}b^2\geq ab $ (this is simply $ (\sqrt{\delta} a-\frac{1}{\sqrt{\delta}}b)^2\geq0 $) we find that \begin{equation}
	\int_{\R^3}\rho_\psi(x)^{4/3}\diff x\leq\frac{\delta}{2}\int_{\R^3}\rho_\psi(x)^{5/3}\diff x+\frac{N}{2\delta}
	\end{equation}
\end{proof}

\subsection*{(b)}
Let $ V_\text{C} $ be defined as in the lecture notes with fixed $ R_1,...,R_M\in\R^3 $ and $ Z_1=....=Z_N=Z $. We prove that if $ \psi\in H^1(\R^{3N}) $ is fermionic, then\begin{equation*}
\begin{aligned}
\mathcal{E}(\psi)&=T_{\psi}+(V_\text{C})_{\psi}\\
&\geq C_1\int_{\R^3}\rho_\psi(x)^{5/3}\diff x +D(\rho_\psi,\rho_\psi)-\sum_{j=1}^{M}\int_{\R^3}\frac{Z\rho_{\psi}}{\abs{x-R_j}}\diff x+\sum_{1\leq j< k\leq M}\frac{Z^2}{\abs{R_j-R_k}}-C_2N,
\end{aligned}
\end{equation*} 
with some constants $ C_1,C_2>0 $ independent of $ \psi $ and $ N $.
\begin{proof}
	By definition we have \begin{equation}
	(V_\text{C})_{\psi}=\int_{\R^{3N}}\sum_{1\leq i<j\leq N}\frac{\abs{\psi(x_1,...,x_N)}^2}{\abs{x_i-x_j}}-\sum_{i=1}^{N}\sum_{j=1}^{M}\frac{Z\abs{\psi(x_1,...,x_N)}^2}{\abs{x_i-R_j}}\diff x_1 ...\diff x_N+\sum_{1\leq j< k\leq M}\frac{Z^2}{\abs{R_j-R_k}}.
	\end{equation}
	Using that $ \psi $ is fermionic we find that \begin{equation}
	\int_{\R^{3N}}\sum_{i=1}^{N}\sum_{j=1}^{M}\frac{Z\abs{\psi(x_1,...,x_N)}^2}{\abs{x_i-R_j}}\diff x_1 ...\diff x_N=\sum_{j=1}^{M}\frac{1}{N}\sum_{i=1}^{N}\int_{\R^3}\frac{Z\rho_\psi(x_i)}{\abs{x_i-R_j}}\diff x_i=\sum_{j=1}^{M}\int_{\R^3}\frac{Z\rho_\psi(x)}{\abs{x-R_j}}\diff x.
	\end{equation}
	Furthermore, using the Lieb-Oxford inequality we find that \begin{equation}
	(V_{\text{C}})_{\psi}\geq -C_{LO}\int_{\R^3}\rho_{\psi}(x)^{4/3} \diff x+D(\rho_\psi,\rho_\psi)-\sum_{j=1}^{M}\int_{\R^3}\frac{Z\rho_\psi(x)}{\abs{x-R_j}}\diff x+\sum_{1\leq j< k\leq M}\frac{Z^2}{\abs{R_j-R_k}}.
	\end{equation}
	Therefore, by (a) we have \begin{equation}\label{CoulombEst1}
	(V_{\text{C}})_{\psi}\geq -C_{LO}\left(\frac{\delta}{2}\int_{\R^3}\rho_{\psi}(x)^{5/3}\diff x+\frac{N}{2\delta}\right) \diff x+D(\rho_\psi,\rho_\psi)-\sum_{j=1}^{M}\int_{\R^3}\frac{Z\rho_\psi(x)}{\abs{x-R_j}}\diff x+\sum_{1\leq j< k\leq M}\frac{Z^2}{\abs{R_j-R_k}}
	\end{equation}
	Now we use the fact that there exist a constant $ C>0 $ such that $ T_\psi\geq C\int_{\R^3}\rho_\psi(x)^{5/3}\diff x $. This can be seen by considering the Lieb-Thirring inequality with potential $ V=-\alpha\rho_\psi^{2/3} $ with some $ \alpha>0 $. Notice that then $ V\in L^{5/2}(\R^3) $ by Sobolev's inequality and the fact that $ \rho_\psi\in L^{3/2}(\R^3) $. Thus we may apply the Lieb-Thirring inequality \begin{equation}
	\sum_{i}\abs{E_i}\leq L_{1,3}\int_{\R^3}V_{-}(x)^{5/2}\diff x=\alpha^{5/2}L_{1,3}\int_{\R^3} \rho_\psi(x)^{5/3}\diff x.
	\end{equation}
	Notice however, that from the very definition of the eigenvalues we have $ T_\psi\geq -V_\psi+E_0 $. Thus we may conclude that \begin{equation}
	T_\psi\geq\alpha\int_{\R^3}\rho_\psi(x)^{5/3}\diff x-\alpha^{5/2}L_{1,3}\int_{\R^3}\rho_\psi(x)^{5/3}\diff x.
	\end{equation}
	Thereby we see that if we choose $ \alpha<1 $ and $ \alpha^{3/2}<L_{1,3}^{-1} $ we see that there exist some constant $C=\alpha(1-\alpha^{3/2}L_{1,3})>0  $ such that \begin{equation}
	T_\psi\geq C\int_{\R^3}\rho_\psi(x)^{5/3}\diff x.
	\end{equation}
	Combining this with \eqref{CoulombEst1} we find that \begin{equation}
	\begin{aligned}
	\mathcal{E}(\psi)\geq \left(C-C_{LO}\frac{\delta}{2}\right)\int_{\R^3}\rho_\psi(x)^{5/3}\diff x+D(\rho_\psi,\rho_\psi)-\sum_{j=1}^{M}\int_{\R^3}\frac{Z\rho_\psi(x)}{\abs{x-R_j}}\diff x\\+\sum_{1\leq j< k\leq M}\frac{Z^2}{\abs{R_j-R_k}}-C_{LO}\frac{N}{2\delta}
	\end{aligned}.
	\end{equation}
	Now choosing $ 0<\delta<\frac{2C}{C_{LO}} $, we find that $ C_1=\left(C-C_{LO}\frac{\delta}{2}\right)>0 $ and $ C_2=\frac{C_{LO}}{2\delta}>0 $ and 
	\begin{equation}
	\begin{aligned}
	\mathcal{E}(\psi)\geq C_1\int_{\R^3}\rho_\psi(x)^{5/3}\diff x+D(\rho_\psi,\rho_\psi)-\sum_{j=1}^{M}\int_{\R^3}\frac{Z\rho_\psi(x)}{\abs{x-R_j}}\diff x+\sum_{1\leq j< k\leq M}\frac{Z^2}{\abs{R_j-R_k}}-C_2N
	\end{aligned}.
	\end{equation}
	as desired.
\end{proof}
 \subsection*{(c)}
 We now prove that for any $ \psi\in H_1(\R^{3N}) $ that is fermionic it hold for any $ b>0$ that \begin{equation}
 \mathcal{E}(\psi)\geq C_1\int_{\R^3}\rho_\psi(x)^{5/3}\diff x-Z\int_{\R^3}\rho_\psi(x)\left(\frac{1}{\mathfrak{D}(x)}-b\right)\diff x-ZbN-C_2N.
 \end{equation} 
 with some constants $ C_1,C_2>0 $ independent of $ \psi $ and $ N $.
 \begin{proof}
 	First notice that by the basic electrostatic inequality with measure $ \mu(\diff x)=\rho_\psi(x)\diff x $ (which indeed defines a measure since $\rho_\psi\in L^1(\R^3)$ and $ \rho_\psi\geq0 $) and the result of (b) it follows that\begin{equation}
 	\begin{aligned}
 	\mathcal{E}(\psi)\geq C_1\int_{\R^3}\rho_\psi(x)^{5/3}\diff x-Z\int_{\R^3}\rho_\psi(x)\frac{1}{\mathfrak{D}(x)}\diff x-C_2N
 	\end{aligned}.
 	\end{equation}
 	Now using that $ \int_{\R^3}\rho_\psi(x) \diff x=N $ we see that \begin{equation}
 	-Z\int_{\R^3}\rho_\psi(x)\frac{1}{\mathfrak{D}(x)}\diff x=-Z\int_{\R^3}\rho_\psi(x)\left(\frac{1}{\mathfrak{D}(x)}-b\right)\diff x-ZbN,
 	\end{equation}
 	from which the claim follows:\begin{equation}
 	\mathcal{E}(\psi)\geq C_1\int_{\R^3}\rho_\psi(x)^{5/3}\diff x-Z\int_{\R^3}\rho_\psi(x)\left(\frac{1}{\mathfrak{D}(x)}-b\right)\diff x-ZbN-C_2N.\label{EnergyEst1}
 	\end{equation}
 \end{proof}
 \subsection*{(d)}
 From calculus of variations it can be shown that the functional obtained in (c) is minimized by some $ \rho_\psi $ of the form \begin{equation}
 \rho_{\psi}(x)=d\left(\frac{1}{\mathfrak{D}(x)}-b\right)^{3/2}\chi_{\{\frac{1}{\mathfrak{D}(x)}-b\geq c\}}(x)
 \end{equation}
 for some $ d>0$ and $c\geq0 $ independent of $ \psi $ and $ N $. Thereby, we may conclude that $ \mathcal{E}(\psi)\geq C(Z)(N+M) $. To see this notice that by inserting the minimizer on the left-hand side of \eqref{EnergyEst1} we obtain \begin{equation}\begin{aligned}
 \mathcal{E}(\psi)&\geq (C_1d^{5/3}-Zd)\int_{\{\frac{1}{\mathfrak{D}(x)}-b\geq c\}}\left(\frac{1}{\mathfrak{D}(x)}-b\right)^{5/2}\diff x-ZbN-C_2N\\&\geq\text{min}\left\{0,(C_1d^{5/3}-Zd)\right\}\int_{\{\frac{1}{\mathfrak{D}(x)}\geq c+b\}}\left(\frac{1}{\mathfrak{D}(x)}\right)^{5/2}\diff x-(Zb+C_2)N
 \end{aligned}
 \end{equation}
 Now defining $ \alpha:=(c+b)^{-1} $ we have \begin{equation}
 \int_{\{\frac{1}{\mathfrak{D}(x)}\geq c+b\}}\left(\frac{1}{\mathfrak{D}(x)}\right)^{5/2}\diff x\leq\sum_{j=1}^{M}\int_{\{\abs{x-R_j}\leq \alpha\}}\left(\frac{1}{\abs{x-R_j}}\right)^{5/2}\diff x=8\pi\sqrt{\alpha} M,
 \end{equation}
 where we used that $\left(\frac{1}{\mathfrak{D}(x)}\right)^{5/2} \chi_{\{\frac{1}{\mathfrak{D}(x)}\geq \frac{1}{\alpha}\}}\leq\sum_{j=1}^{M}\left(\frac{1}{\abs{x-R_j}}\right)^{5/2}\chi_{\{\abs{x-R_j}\leq \alpha\}} $, which is obvious from the fact that, for any $ x\in \R^3 $ the left-hand side will equal at least one of the terms on the right-hand side, and since all on the terms on the right-hand side are non-negative the inequality follows.
 From this it follows that \begin{equation}
 \mathcal{E}(\psi)\geq -K_1(Z)M-K_2(Z)N\geq-C(Z)(N+M)
 \end{equation}
 with $ K_1(Z)=\text{max}\left\{0,-(C_1d^{5/3}-Zd)\right\}8\pi\sqrt{\alpha} $, $ K_2(Z)=(Zb+C_2) $, and\\ $C(Z)=\max\{K_1(Z),K_2(Z)\}$. Many of these estimates were quite rough and can be optimized. For example one can optimize w.r.t $ b $. Notice to find the exact $ d $ and $ c $ we would have to minimize w.r.t to $ d $ and $ c $. Thus we find $ d=\left(\frac{3Z}{5C_1}\right)^{3/2} $ and $ c=0 $.
 
 \section{The volume occupied by matter}
 Let $ \psi\in L^2(\R^{3N}) $ ($ \psi\in H^1(\R^{3N}) $) be a fermionic wave function  with $ \norm{\psi}_2=1 $.
 \subsection*{(a)}
 It holds that $ \mathcal{E}(\psi)=T_\psi+(V_\text{C})_\psi\geq-CN $ where $ C>0 $ depends on $ Z $ and the ratio $ M/N $. This is a direct consequence of the result from problem 1. Since we have $ \mathcal{E}(\psi)\geq -C(Z)(M+N)=-C(Z)(M/N+1)N=-CN $ where $ C=C(Z)(M/N+1) $.
 \subsection*{(b)}
 Using a scaling argument, it is possible to conclude from (a) that \begin{equation}
 (1-\lambda)T_\psi+(V_\text{C})_\psi\geq-\frac{CN}{1-\lambda},\label{scaling1}
 \end{equation}
 for any $ 0<\lambda<1 $. From this it follows that \begin{equation}
 T_\psi\leq\frac{\mathcal{E}(\psi)+CN}{\lambda}+\frac{CN}{1-\lambda} \label{ineq1}
 \end{equation}
 \begin{proof}
 	To see this, notice that from \eqref{scaling1} we have \begin{equation}
 	-\lambda T_\psi\geq-\frac{CN}{1-\lambda}-\mathcal{E}(\psi),
 	\end{equation}
 	from which it follows that\begin{equation}
 	T_\psi\leq \frac{CN}{\lambda(1-\lambda)}+\frac{\mathcal{E}(\psi)}{\lambda}=\frac{\mathcal{E}(\psi)+CN}{\lambda}+\frac{CN}{1-\lambda},
 	\end{equation}
 	where we in the last equality used the partial fraction decomposition $ \frac{CN}{\lambda(1-\lambda)}=\frac{CN}{\lambda}+\frac{CN}{1-\lambda} $.
 \end{proof}
 	From this we may conclude that \begin{equation}
 	T_\psi\leq(\sqrt{\mathcal{E}(\psi)+CN}+\sqrt{CN})^2.
 	\end{equation}
 	\begin{proof}
 		For $ \mathcal{E}(\psi)=0 $ it follows by choosing $ \lambda=1/2 $ in \eqref{ineq1}. Now assume $ \mathcal{E}(\psi)\neq0 $,
 		we then optimize \eqref{ineq1} w.r.t $ \lambda $:\begin{equation}
 		\frac{\diff}{\diff\lambda}\left(\frac{\mathcal{E}(\psi)+CN}{\lambda}+\frac{CN}{1-\lambda}\right)=-\frac{\mathcal{E}(\psi)+CN}{\lambda^2}+\frac{CN}{(1-\lambda)^2}=0
 		\end{equation}
 		using that $ 0<\lambda<1 $, this is equivalent\begin{equation}
 		-(1-\lambda)^2(\mathcal{E}(\psi)+CN)-\lambda^2CN=0,
 		\end{equation}
 		which has the solutions $ \lambda_\pm=\frac{\mathcal{E}(\psi)+CN\pm\sqrt{\mathcal{E}(\psi)CN+C^2N^2}}{\mathcal{E}(\psi)} $, where we see that only the $ \lambda_- $ solution is consistent with $ 0<\lambda<1 $ (it is consistent since $ \mathcal{E}(\psi)\geq-CN $). Inserting this $ \lambda_- $ back into \eqref{ineq1} we find that \begin{equation}
 		T_\psi\leq(\sqrt{\mathcal{E}(\psi)+CN}+\sqrt{CN})^2,
 		\end{equation}
 		as desired.
 	\end{proof}
\end{document}