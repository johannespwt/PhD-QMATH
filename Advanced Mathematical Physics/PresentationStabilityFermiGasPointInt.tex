\documentclass{beamer}[10]
\usepackage{pgf}
\usepackage[danish]{babel}
\usepackage[utf8]{inputenc}
%\usepackage{beamerthemesplit}
\usepackage{graphics,epsfig, subfigure}
\usepackage{url}
\usepackage{srcltx}

\usepackage{cancel}

\usepackage{pdfpages}

\usepackage{amsmath}

\usepackage{amssymb}

\usepackage{bbm}

\usepackage{amsthm}

\usepackage{hyperref}

\usepackage{mathrsfs}

\usepackage{amsfonts}

\usepackage{bbm}

\usepackage{amsmath}

\usepackage{amssymb}

\usepackage{amsthm}

\usepackage{blkarray}

\usepackage{enumerate}

\usepackage{graphicx}

\usepackage{centernot}

\usepackage{caption}

\usepackage{braket}

\usepackage{slashed}

\usepackage{pgfplots}

\usepackage{feynmp-auto}

\usepackage{lastpage}

\usepackage{fancyhdr}
\usepackage[square,sort,comma,numbers]{natbib}

\usepackage{fancyhdr}
\newcommand{\euler}[1]{\text{e}^{#1}}
\newcommand{\Real}{\text{Re}}
\newcommand{\Imag}{\text{Im}}
\newcommand{\norm}[1]{\left\lVert #1 \right\rVert}
\newcommand{\abs}[1]{\left\lvert #1 \right\rvert}
\newcommand{\floor}[1]{\left\lfloor #1 \right\rfloor}
\newcommand{\Span}[1]{\text{span}\left(#1\right)}
\newcommand{\dom}[1]{\mathscr D\left(#1\right)}
\newcommand{\Ran}[1]{\text{Ran}\left(#1\right)}
\newcommand{\conv}[1]{\text{co}\left\{#1\right\}}
\newcommand{\Ext}[1]{\text{Ext}\left\{#1\right\}}
\newcommand{\vin}{\rotatebox[origin=c]{-90}{$\in$}}
\newcommand{\interior}[1]{%
	{\kern0pt#1}^{\mathrm{o}}%
}
\newcommand*\diff{\mathop{}\!\mathrm{d}}
\newcommand{\ie}{\emph{i.e.} }
\newcommand{\eg}{\emph{e.g.} }
\newcommand{\dd}{\partial }
\newcommand{\R}{\mathbb{R}}
\newcommand{\C}{\mathbb{C}}
\newcommand{\w}{\mathsf{w}}

\newcommand{\Gliminf}{\Gamma\text{-}\liminf}
\newcommand{\Glimsup}{\Gamma\text{-}\limsup}
\newcommand{\Glim}{\Gamma\text{-}\lim}

% TikZ til at lave figurer - for de avancerede
\usepackage{tikz}
% Div. pakker (muligt at ikke alle er i brug)
\usetikzlibrary{decorations.pathmorphing}
\usetikzlibrary{arrows.meta}
\usetikzlibrary{arrows}
\usetikzlibrary{decorations.pathreplacing,decorations.markings}
\usetikzlibrary{patterns}
\usetikzlibrary{fadings}
\usetikzlibrary{calc}
\usetikzlibrary{tikzmark,fit,shapes.geometric}

\definecolor{kugreen}{RGB}{50,93,61}
\definecolor{kugreenlys}{RGB}{132,158,139}
\definecolor{kugreenlyslys}{RGB}{173,190,177}
\definecolor{kugreenlyslyslys}{RGB}{214,223,216}
\setbeamercovered{transparent}
\mode<presentation>
%\usetheme[numbers,totalnumber,compress,sidebarshades]{PaloAlto}
\setbeamertemplate{footline}[frame number]
\usecolortheme[named=kugreen]{structure}
\useinnertheme{circles}
\usefonttheme[onlymath]{serif}
\setbeamercovered{transparent}
\setbeamertemplate{blocks}[rounded][shadow=true]
\setbeamercolor{block title}{bg=kugreen!50,fg=black}
\setbeamercolor{block body}{bg=kugreen!10,fg=black}
\logo{\includegraphics[width=0.8cm]{kuscience-logo}}
%\useoutertheme{infolines} 

\title{Stability of the $ N+1 $ Fermi gas with point interactions}
\subtitle{Advanced Mathematical Physics}
\author{Johannes Agerskov}
\institute{Institute for Mathematical Sciences \\ University of Copenhagen}
\date{May 15, 2020.}

\setbeamercovered{invisible}

\begin{document}
\frame{\titlepage \vspace{-0.5cm}
}

\frame
{
\frametitle{Overview}
\linespread{1.5}
\tableofcontents%[pausesection]
}
\section{Motivation}
\begin{frame}
	\frametitle{Motivation}
	Model of fermions interacting via point interactions are of great interest as they appear as
	\begin{itemize}
		\item Models of cold atomic gases.\\
		\item Models of nuclear interaction.\\
		\item Approximations of models with short-range interactions.
	\end{itemize}
	However, they are mathematically not very well understood.
\end{frame}

\begin{frame}
	\frametitle{Thomas collapse}
	\textbf{Thomas collaps:}
	It it known that a bosonic system of three or more bosons with attractive zero-range interactions is unstable (of the first kind) i.e. there is no ground state energy. This can be seen from the variational principle.\\\vspace*{0.5cm}
	\textbf{No Thomas collapse for (spin-1/2) fermions:} The Thomas collapse is a collective phenomenon where three (or more) bosons interacts in a single point. This can never happen for spin-1/2 fermions, due to the Pauli principle.\\\vspace*{0.5cm}
	\textbf{Stability of the first kind:} Is still an unsolved problem for general $ N+M $ systems ($ N $ spin up and $ M $ spin down).
\end{frame}
\section{Moser and Seiringer: Stability of $ N+1 $ system}
\begin{frame}
	\frametitle{Results of Moser and Seiringer}
	\begin{block}{Results}
		\small Moser, T., Seiringer, R. Stability of a Fermionic N + 1 Particle System with Point Interactions. Commun. Math. Phys. 356, 329–355 (2017).\normalsize
		\begin{itemize}
			\item Prove stability of the $ N+1 $ system, within a certain mass ratio interval.
			\item Prove existence of self-adjoint bounded from below Hamiltonian.
			\item Prove Tan relations.
		\end{itemize}
	\end{block}
	We focus on the first two.
\end{frame}
\section{Model}
\begin{frame}
	\frametitle{Formal Hamiltonian}
	\begin{block}{Formal Hamiltonian}
	The Hamiltonian of a system of $ N $ fermions of one species of mass $ 1 $ interacting with $ 1 $ fermion of another species of mass $ m $ can be described by the formal Hamiltonian\begin{equation}
	H=-\frac{1}{2m}\Delta_{x_0}-\frac{1}{2}\sum_{i=1}^{N}\Delta_{x_i}+\gamma\sum_{i=1}^{N}\delta(x_i-x_0)
	\end{equation}
	\end{block}
\end{frame}
\begin{frame} 
	\frametitle{Formal Hamiltonian}
	\begin{block}{Centre of mass separation}
		We can split the Hamiltonian in two \begin{equation}
		H=H_{\text{CM}}+\frac{m+1}{2m}H_{\text{rel}},
		\end{equation}
		with $ x_{\text{cm}}=(mx_0+\sum_{i=1}^{N}x_i)/(m+N) $, $ y_i=x_i-x_0 $, and \begin{equation}\begin{aligned}
		H_\text{CM}&=\frac{1}{2(N+m)}\Delta_{x_{\text{cm}}},\\ H_{\text{rel}}&=-\sum_{i=1}^{N}\Delta_{y_i}-\frac{2}{m+1}\sum_{1\leq i<j\leq N}\nabla_{y_i}\cdot\nabla_{y_j}+\tilde{\gamma}\sum_{i=1}^{N}\delta(y_i)
		\end{aligned}
		\end{equation}
	\end{block}
\end{frame}
\begin{frame}
	\frametitle{Quadratic form}
	The formal Hamiltonian can be given precise meaning through a quadratic form, which can be obtained by considering more regularized models such as rank-one perturbations of a free Hamiltonian. One obtains\begin{equation}
	\begin{aligned}
	F_\alpha(u)=&\int_{\R^{3N}}\diff k \hat{G}(k)^{-1}|\hat{w}|^2-\mu\norm{u}_{L^2(\R^{3N})}\\&+N\left(T_{\text{diag}}(\xi)+T_{\text{off}}(\xi)+\alpha\norm{\xi}_{L^2(\R^{3(N-1)})}^2\right)
	\end{aligned}
	\end{equation} 
	with $ \hat{u}(k)=\hat{w}(k)+\sum_{i=1}^{N}(-1)^{i-1}\hat{G}(k)\xi(\bar{k}^i) $, $ \mu>0 $, $ \hat{G}(k)=\left(\sum_{i=1}^{N}k_i^2+\frac{2}{m+1}\sum_{1\leq i<j\leq N}k_i\cdot k_j+\mu\right)^{-1} $,
\end{frame}
\begin{frame}
	\begin{equation}
	\begin{aligned}
	T_{\text{diag}}(\xi)&=\int_{\R^{3(N-1)}}\diff \bar{k}^N L(\bar{k}^N) \abs{\xi(\bar{k}^N)}^2,\\
	T_{\text{off}}(\xi)&=(N-1)\int_{\R^{3(N-2)}}\diff \bar{q}\int_{\R^3}\diff s\int_{\R^3}\diff t \overline{\xi(s,\bar{q})}\hat{G}(s,t,\bar{q})\xi(t,\bar{q}).
	\end{aligned}
	\end{equation} 
	with \small\begin{equation}
	L(\bar{k}^N)=2\pi^2\left(\frac{m(m+2)}{(m+1)^2}\sum_{i=1}^{N-1}k_i^2+\frac{2m}{(m+1)^2}\sum_{1\leq i<j\leq N-1}k_i\cdot k_j+\mu\right)^{1/2}.
	\end{equation}
	The domain is \begin{equation}
	\begin{aligned}
	\mathscr{D}(F_\alpha)&=\left\{u\in L_{\text{as}}^2(\R^{3N})\ \Big\vert\ \right.\\ &\qquad \left.\hat{u}=\hat{w}+\widehat{\rho G},\ w\in H_{\text{as}}^1(\R^{3N}),\ \xi\in H_{\text{as}}^{1/2}(\R^{3(N-1)}) \right\}.
	\end{aligned}
	\end{equation}
\end{frame}
\begin{frame}
	\frametitle{Theorem 1}
	Introduce the function for $ m>0 $ \tiny\begin{equation}
	\begin{aligned}
	\Lambda(m)=&\sup_{\substack{s,K\in\R^3,\\Q\geq0}}\frac{s^2+Q^2}{\pi^2(1+m)}\ell_m(s,K,Q)^{-1/2}\int_{\R^3}\diff t \frac{1}{t^2}\ell_m(t,K,Q)^{-1/2}\\
	&\times\frac{\abs{(s+AK)\cdot(t+AK)}}{\left[(s+AK)^2+(t+AK)^2+\frac{m}{1+m}(Q^2+AK^2)\right]^2-\left[\frac{2}{(1+m)}(s+AK)\cdot(t+AK)\right]^2}
	\end{aligned}
	\end{equation}
	\normalsize
	where $ A=(2+m)^{-1} $ and \begin{equation}
	\ell_m(s,K,Q)=\left(\frac{m}{(1+m)^2}(s+K)^2+\frac{m}{1+m}(s^2+Q^2)\right).
	\end{equation}
	It is then showed that \begin{equation}
	\Lambda(m)\leq\frac{4(1+m)^2(2+4m+m^2)^{3/2}}{\sqrt{2}\pi\left[m(m+2)\right]^3}.
	\end{equation}
\end{frame}
\section{Theorem 1}
\begin{frame}
	\frametitle{Theorem 1}
	\begin{block}{Theorem 1, Moser, T., Seiringer, R., 2017.}
		\emph{For any $ \xi\in H^{1/2}_{\textnormal{as}}(\R^{3(N-1)}) $, $ \mu>0 $ and $ N\geq2 $,\begin{equation}
			T_{\textnormal{off}}(\xi)\geq-\Lambda(m)T_{\textnormal{diag}}(\xi).
			\end{equation}
		In particular, if $ \Lambda(m)<1 $, then $ F_\alpha $ is closed and bounded from below by\begin{equation}
		F_\alpha(u)\geq\begin{cases}
		0&\text{for }\alpha\geq0,\\
		-\left(\frac{\alpha}{2\pi^2(1-\Lambda(m))}\right)^2\norm{u}^2_{L^2( \R^{3N})}& \text{for }\alpha<0.
		\end{cases}
		\end{equation}}
	\end{block}
\end{frame}
\begin{frame}
	\frametitle{Proof idea}
	\begin{itemize}
		\item Define $ \phi=L^{1/2}\xi $ such that $ T_{\text{diag}}=\norm{\phi}^2_{L^2(\R^{3(N-1)})} $.
		\item Notice that \small$ T_{\text{off}}(\xi)=\int_{\R^{3(N-2)}}\diff q\int_{\R^3} \diff s\int_{\R^3}\diff t\overline{\phi(s,q)}\phi(t,q)L(s,q)^{-1/2}L(t,q)^{-1/2}\hat{G}(s,t,q) $.
		\item Throw away positive part.
		\item Use Schur test $ \norm{\sigma}\leq\sup_{t}\left(h(t)\int \diff s \sigma(s,t)\frac{1}{h(s)}\right) $ on $ (L^{-1/2}GL^{-1/2})_- $.
		\item Choose $ h $ carefully and use (anti-)symmetry of $ \phi $ to arrive at $ \norm{(L^{-1/2}GL^{-1/2})_-}\leq\Lambda(m) $.\\
		\item Closedness is now standard argument since all terms are indvidually bounded from below.
	\end{itemize}
\end{frame}
\begin{frame}
	\frametitle{Physical consequences}
	$ \Lambda(m)<1 $ for $ m\geq0.36 $ so the critical mass ratio is less that $ 0.36 $
	\begin{itemize}
		\item Stability of first and second kind.
		\item Existence of self-adjoint bounded from below Hamiltonian.
		\end{itemize}
		
		 For $ m\to\infty $ we have $ \Lambda(m)\to 0 $. Thus, the system with $ \alpha<0 $ has energy bounded from below by $ -\left(\alpha/(2\pi^2)\right)^2 $.
		 \begin{itemize} 
		\item For $ m\to\infty $ the heavy fermion can bind at most one light fermion. This is the Pauli principle.
	\end{itemize}
\end{frame}
\section{Theorem 2}
\begin{frame}
	\frametitle{Theorem 2}
	Since $ T_{\textnormal{diag}}+T_{\textnormal{off}} $ is symmetric, closed, and bounded from below, we may define the unique self-adjoint operator $ \Gamma $ by\begin{equation}
	T_{\textnormal{diag}}(\xi)+T_{\textnormal{off}}(\xi)=\braket{\xi|\Gamma\xi}.
	\end{equation}
	It can be shown that $ H^1_{\text{as}}(\R^{3(N-1)})\subset\dom{\Gamma} $
\end{frame}
\begin{frame}
	\frametitle{Theorem 2}
	\begin{block}{Theorem 2, Moser, T., Seiringer, R., 2017}
	\emph{For any $ \xi\in H^1_{\textnormal{as}}(\R^{3(N-1)}),\mu>0$, and $ N\geq2 $\begin{equation}
		\norm{\Gamma \xi}^2_{L^2(\R^{3(N-1)})}\geq(1-\Lambda_1(m))\norm{L \xi}^2_{L^2(\R^{3(N-1)})}.
		\end{equation}
		\xcancel{In particular}, if $ \Lambda_1(m)<1 $, Then $ \dom{\Gamma}=\dom{L}=H^1_{\textnormal{as}}(\R^{3(N-1)}) $. More generally for $ 0\leq\beta\leq2 $,\small{\begin{equation}
		\norm{L^{(\beta-1)/2}\Gamma \xi}^2_{L^2(\R^{3(N-1)})}\geq(1-\Lambda_\beta(m))\norm{L^{(\beta+1)/2} \xi}^2_{L^2(\R^{3(N-1)})},
		\end{equation}}
		\normalsize for all $ \xi\in H^{(\beta+1)/2}_{\textnormal{as}}(\R^{3(N-1)}) $.
		}
	\end{block}
\end{frame}
\begin{frame}
	\frametitle{Proof idea}
	\begin{itemize}
		\item Write $ \Gamma=L+J $, i.e. $ \braket{\xi|J\xi}=T_{\text{off}}(\xi) $.\\
		\item Notice that define  $ \phi=L^{(\beta+1)/2}\xi$ and notice that \small$ \norm{L^{(\beta-1)/2}\Gamma\xi}^2=\braket{\phi|L^{-(\beta+1)/2} (L+J)L^{\beta-1}(L+J)L^{-(\beta+1)/2}|\phi}$\\
		\item Throw away positive term $ \braket{\phi|L^{-(\beta+1)/2} JL^{\beta-1}JL^{-(\beta+1)/2}|\phi} $.
		\item Claim is now equivalent to $ \braket{\phi|L^{(\beta-1)/2}JL^{-(\beta+1)/2}+L^{-(\beta+1)/2}JL^{(\beta-1)/2}|\phi}\geq-\Lambda_\beta(m)\norm{\phi}^2_2 $.
		\item Use Cauchy-Schwartz, and similar proof to that of theorem 1 to obtain the desired result.
	\end{itemize}
\end{frame}
\begin{frame}
	\frametitle{Consequences}
	Knowing that the quadratic form $ F_\alpha $ is symmetric, closed, and bounded from below, it is straightforward to obtain the Hamiltonian: 
	\begin{equation}
	\mathscr{D}(H_\alpha)=\left\{u\in\mathscr{D}(F_\alpha)\ \vert\ F_\alpha(\cdot,u)\text{ is $ L^2 $ bounded on }\mathscr{D}(F_\alpha)\right\}.
	\end{equation}
	Notice that for $ u\in\dom{H_\alpha} $, we have $ F_\alpha(\cdot,u)=\braket{\cdot,x} $ and we set $ H_\alpha u=x $. Moser and Seiringer obtains\begin{equation}
	\begin{aligned}
	\dom{H_\alpha}=\left\{u\in L^2_{\textnormal{as}}(\R^{3N})\ \vert\ u=w+G\xi,w\in H^2_{\textnormal{as}}(\R^{3(N-1)}),\right.\\
	\left.\xi\in\dom{\Gamma}, w\vert_{y_N=0}=(2\pi)^{-3/2}(-1)^{N+1}(\alpha+\Gamma)\xi  \right\},
	\end{aligned}
	\end{equation}
	$ (H_\alpha+\mu)(w+G\xi)=(H_{\textnormal{free}}+\mu)w $, with $ (G\xi)(x):=\left(\sum_{i=1}^{N}(-1)^{i-1}\hat{G}(k)\xi(\bar{k}^i)\right)^\vee(x) $
\end{frame}
\section{Conclusion}
\begin{frame}
	\frametitle{Conclusion}
	Thus stability of the fermionic $ N+1 $ system is established for $ m\geq0.36 $, and a rigorous version of the formal Hamiltonian is found.
\end{frame}
\begin{frame}
	\centering
	\Large Thank you for your attention.\\
\end{frame}	
\begin{frame}
	\tiny\begin{equation}
	\begin{aligned}
	\hspace{-0.3cm}\Lambda_\beta(m)=&\sup_{\substack{s,K\in\R^3,\\Q\geq0}}\frac{s^2+Q^2}{\pi^2(1+m)}\int_{\R^3}\diff t \frac{1}{t^2}\left(\frac{\ell_m(s,K,Q)^{(\beta-1)/2}}{\ell_m(t,K,Q)^{(\beta+1)/2}}+\frac{\ell_m(t,K,Q)^{(\beta-1)/2}}{\ell_m(s,K,Q)^{(\beta+1)/2}}\right)\\
	&\times\frac{\abs{(s+AK)\cdot(t+AK)}}{\left[(s+AK)^2+(t+AK)^2+\frac{m}{1+m}(Q^2+AK^2)\right]^2-\left[\frac{2}{(1+m)}(s+AK)\cdot(t+AK)\right]^2}
	\end{aligned}
	\end{equation}
\end{frame}	
\end{document}
