\documentclass[a4paper,11pt]{article}
\usepackage[utf8]{inputenc}
\usepackage[margin=1in]{geometry}
\usepackage{pdfpages}
\usepackage{mathrsfs}
\usepackage{amsfonts}
\usepackage{amsmath}
\usepackage{amssymb}
\usepackage{bbm}
\usepackage{amsthm}
\usepackage{graphicx}
\usepackage{centernot}
\usepackage{caption}
\usepackage{subcaption}
\usepackage{braket}
\usepackage{pgfplots}
\usepackage{lastpage}
\usepackage{enumitem}
\usepackage{setspace}
\usepackage[english]{babel} 

\usepackage[square,sort,comma,numbers]{natbib}
\usepackage[colorlinks=true,linkcolor=blue]{hyperref}

\usepackage{fancyhdr}
\newcommand{\euler}[1]{\text{e}^{#1}}
\newcommand{\Real}{\text{Re}}
\newcommand{\Imag}{\text{Im}}
\newcommand{\floor}[1]{\left\lfloor #1 \right\rfloor}
\newcommand{\norm}[1]{\left\lVert #1 \right\rVert}
\newcommand{\abs}[1]{\left\lvert #1 \right\rvert}
\newcommand{\Span}[1]{\text{span}\left(#1\right)}
\newcommand{\dom}[1]{\mathscr D\left(#1\right)}
\newcommand{\Ran}[1]{\text{Ran}\left(#1\right)}
\newcommand{\conv}[1]{\text{co}\left\{#1\right\}}
\newcommand{\Ext}[1]{\text{Ext}\left\{#1\right\}}
\newcommand{\vin}{\rotatebox[origin=c]{-90}{$\in$}}
\newcommand{\interior}[1]{%
	{\kern0pt#1}^{\mathrm{o}}%
}
\newcommand*\diff{\mathop{}\!\mathrm{d}}
\newcommand{\ie}{\emph{i.e.} }
\newcommand{\eg}{\emph{e.g.} }
\newcommand{\dd}{\partial }
\newcommand{\R}{\mathbb{R}}
\newcommand{\C}{\mathbb{C}}
\newcommand{\w}{\mathsf{w}}

\newcommand{\Gliminf}{\Gamma\text{-}\liminf}
\newcommand{\Glimsup}{\Gamma\text{-}\limsup}
\newcommand{\Glim}{\Gamma\text{-}\lim}

\newtheorem{theorem}{Theorem}
\newtheorem{definition}{Definition}
\newtheorem{proposition}{Proposition}
\newtheorem{lemma}{Lemma}
\newtheorem{corollary}{Corollary}

\numberwithin{equation}{section}
\linespread{1.3}

\pagestyle{fancy}
\fancyhf{}
\rhead{Advanced Mathematical Physics, Assignment 1}
\lhead{Johannes Agerskov}
\rfoot{\thepage}
\lfoot{Dated: \today}
\author{Johannes Agerskov}
\date{Dated: \today}
\title{Advanced Mathematical Physics, Assignment 1}
\begin{document}

	\maketitle

\section{Stability in two dimensions}
We define the energy functional for a particle in $ \R^2 $ as $ \mathcal{E}(\psi)=T_\psi+V_\psi $, with \begin{equation}
T_\psi=\int_{\R^2}\lvert\nabla\psi(x)\rvert^2 \diff x,\quad\text{and}\quad V_\psi=\int V(x)\lvert\psi(x)\rvert^2 \diff x.
\end{equation}
The ground state energy is defined by\begin{equation}
E_0=\inf\{\mathcal{E}(\psi),\ \psi\in H^1(\R^2),\ \lVert\psi\rVert_2=1,\ V_\psi \text{ well defined.}\}.
\end{equation} 
Now assuming that $ V\in L^{1+\epsilon}(\R^2)+L^\infty(\R^2) $ for some fixed $ \epsilon>0 $, we prove that $ E_0>-\infty $.
\begin{proof}
	Let $ V=v+w $ with $ v\in L^{1+\epsilon}(\R^2)  $ and $ w\in L^\infty(\R^2) $. Notice first, that by Sobolev's inequality we have \begin{equation}
	\lVert\nabla\psi\rVert_2^2\geq S_{2,p}\lVert\psi\rVert_2^{\frac{-4}{p-2}}\lVert\psi\rVert_p^{\frac{2p}{p-2}},\qquad 2<p<\infty.
	\end{equation}
	It follows that $ \psi\in L^p(\R^2) $ for $ 2\leq p<\infty $, whenever $ \psi\in H^1(\R^2) $.
	Assuming that $ V_\psi $ is well defined we know from H\"older's inequality that\begin{equation}
	\begin{aligned}
	V_\psi=\int V(x)\lvert\psi(x)\rvert^2 \diff x&\geq\int v(x)\lvert\psi(x)\rvert^2 \diff x-\lVert w\rVert_\infty \lVert \psi\rVert_2^2\\&\geq-\lVert v\rVert_q\lVert|\psi|^2\rVert_{\frac{q}{q-1}}-\lVert w\rVert_\infty \lVert \psi\rVert_2^2\\&=-\lVert v\rVert_q\lVert\psi\rVert_{\frac{2q}{q-1}}^{2}-\lVert w\rVert_\infty \lVert \psi\rVert_2^2.
	\end{aligned}
	\end{equation} Thus setting $ p=\frac{2q}{q-1}=2+\frac{2}{\epsilon} $,
	 with $ \epsilon>0 $, we find that \begin{equation}
	V_\psi\geq-\lVert v\rVert_{1+\epsilon}\lVert \psi\rVert_p^2-\lVert w\rVert_\infty \lVert \psi\rVert_2^2.
	\end{equation}
	Now using Sobolev's inequality we find that\begin{equation}
	T_\psi\geq S_{2,p}\lVert \psi\rVert_2^{-2\epsilon}\lVert \psi \rVert_p^{\frac{2p}{p-2}}=S_{2,p}\lVert \psi\rVert_2^{-2\epsilon}\lVert \psi \rVert_p^{2(1+\epsilon)}.
	\end{equation}
	Thus we conclude that $ \mathcal{E}(\psi)\geq S_{2,p}\lVert \psi\rVert_2^{-2\epsilon}\lVert \psi \rVert_p^{2(1+\epsilon)}-\lVert v\rVert_{1+\epsilon}\lVert \psi\rVert_p^2-\lVert w\rVert_\infty \lVert \psi\rVert_2^2 $.
	Consider now the case in which $ \psi\in H^1(\R^2) $, $ \lVert \psi \rVert_2=1 $ and $ V_\psi $ is well defined. It then follows that \begin{equation}
	\mathcal{E}(\psi)\geq S_{2,p}\lVert \psi \rVert_p^{2(1+\epsilon)}-\lVert v\rVert_{1+\epsilon}\lVert \psi\rVert_p^2-\lVert w\rVert_\infty.
	\end{equation}
	Therefore, we may conclude that \begin{equation}
	\begin{aligned}
	E_0&=\inf\{\mathcal{E}(\psi):\psi\in H^1(\R^2),\ \lVert\psi \rVert_2=1,\ V_\psi\text{ well defined} \}\\&\geq \inf\{S_{2,p}\lVert \psi \rVert_p^{2(1+\epsilon)}-\lVert v\rVert_{1+\epsilon}\lVert \psi\rVert_p^2-\lVert w\rVert_\infty:\psi\in H^1(\R^2),\ \lVert\psi \rVert_2=1,\ V_\psi\text{ well defined} \}\\
	&\geq\inf\{S_{2,p} x^{(1+\epsilon)}-\lVert v\rVert_{1+\epsilon}x-\lVert w\rVert_\infty : x\in\R,\  x\geq0\}>-\infty,
	\end{aligned}
	\end{equation}
	where we have used that fact that\\ $ \{\lVert \psi \rVert_p^2 : \psi\in H^1(\R^2),\ \lVert\psi \rVert_2=1,\ V_\psi\text{ well defined}\}\subseteq \{x\in\R :  x\geq0\} $.
\end{proof}
\section{Stability of hydrogen through ground state positivity}
\subsection*{(a)}
Let $ \Omega\in\R^3 $ be an open set and $ V\in\mathcal{C}(\Omega) $. Assume that $ \psi\in \mathcal{C}^2(\Omega) $ satisfies $ (-\Delta+V)\psi=E\psi $ for some $ E\in\R $ and furthermore $ \psi>0 $. Then it holds that \begin{equation}
\int_{\Omega}\lvert(\nabla\varphi)(x) \rvert^2\diff x+\int_{\Omega}V(x)\lvert\varphi(x)\rvert^2 \diff x\geq E\int_{\Omega}\lvert\varphi(x)\rvert^2 \diff x,
\end{equation}
for all $ \varphi\in\mathcal{C}_0^1(\Omega) $.
\begin{proof}
	Let $ \varphi\in\mathcal{C}_0^1(\Omega) $, and write $ \varphi=g\psi $. Since $ \psi>0 $ we clearly have $ g=\varphi/\psi\in\mathcal{C}_0^1(\Omega) $. Notice that $ \nabla\varphi=(\nabla g)\psi+g(\nabla\psi) $ and therefore \begin{equation}
	\lvert\nabla\varphi\rvert^2=\lvert\psi\rvert^2\lvert\nabla g\rvert^2+\lvert g\rvert^2\lvert\nabla\psi\rvert^2+(\nabla g)\cdot(\nabla \psi)\bar{g}\psi+(\nabla \psi)\cdot(\nabla \bar{g})\psi g.
	\end{equation}
	Using that $ (\nabla g)\cdot(\nabla \psi)\bar{g}\psi=\nabla\cdot(g(\nabla\psi)\bar{g}\psi)-\lvert g \rvert^2(\Delta\psi)\psi-g(\nabla\psi)\cdot(\nabla\bar{g})\psi-\lvert g \rvert^2\lvert\nabla\psi\rvert^2 $, we find\begin{equation}
		\lvert\nabla\varphi\rvert^2=\lvert\psi\rvert^2\lvert\nabla g\rvert^2+\nabla\cdot(g(\nabla\psi)\bar{g}\psi)-\lvert g\rvert^2(\Delta\psi)\psi.
	\end{equation}
	Applying Stokes' (or Gauss') theorem, as well as using the fact that $ g $ has compact support\footnote{Notice that since $ g $ is continuous, the support of $ g $, $ \text{supp}(g)=\{x\in \R^3 : f(x)\neq0\} $, is necessarily open. However, $ S=\overline{\text{supp}(g)} $ is compact by assumption. Furthermore, by continuity of $ g $, we must have $ g\rvert_{\dd S}=0 $. Thus we may split the integral $$ \int_{\Omega}\nabla\cdot(g(\nabla\psi)\bar{g}\psi) \diff x=\int_{S}\nabla\cdot(g(\nabla\psi)\bar{g}\psi) \diff x+\int_{\Omega\setminus S}\nabla\cdot(g(\nabla\psi)\bar{g}\psi) \diff x=\int_{\dd S}(g(\nabla\psi)\bar{g}\psi)\cdot \hat{n} \diff a=0. $$ } we conclude\begin{equation}
	\int_{\Omega}\lvert(\nabla\varphi)(x) \rvert^2\diff x=\int_{\Omega}\lvert\psi(x)\rvert^2\lvert\nabla g(x)\rvert^2-\lvert g(x)\rvert^2(\Delta\psi(x))\psi(x) \diff x\geq\int_{\Omega}\lvert g(x) \rvert^2\psi(x)(-\Delta\psi(x)).
	\end{equation} 
	Therefore we conclude\begin{equation}
	\begin{aligned}
	\int_{\Omega}\lvert(\nabla\varphi)(x) \rvert^2\diff x+\int_{\Omega}V(x)\lvert\varphi(x)\rvert^2 \diff x&\geq\int_{\Omega}\lvert g(x) \rvert^2\psi(x)(-\Delta\psi(x))+\lvert g(x) \rvert^2\psi(x)(V(x)\psi(x))\diff x\\
	&=\int_{\Omega}\lvert g(x) \rvert^2\psi(x)\left[(-\Delta+V(x))\psi(x)\right]\diff x\\
	&=E\int_{\Omega}\lvert g(x) \rvert^2\lvert\psi(x)\rvert^2\diff x\\
	&=E\int_{\Omega}\lvert \varphi(x)\rvert^2 \diff x
	\end{aligned}
	\end{equation}
	this concludes the proof.
\end{proof}
\subsection*{(b)} Consider now the function $ \psi(x)=\exp(-\alpha\abs{x}) $. We show that this function indeed satisfies $ \psi\in \mathcal{C}^2(\R^3\setminus\{0\}) $ and that there exist an $ \alpha $ such that $ (-\Delta-Z/\abs{x})\psi=E_0\psi $ for some $ E_0 $. First we notice that $ \psi $ is a composition of $ \mathcal{C}^\infty(\R^3\setminus\{0\}) $, thus $ \psi\in\mathcal{C}^2(\R^3\setminus\{0\})\subset\mathcal{C}^\infty(\R^3\setminus\{0\}) $. Furthermore, by going to spherical coordinates $ (r,\theta,\varphi) $, with $ \theta $ the azimuthal angle and $ \varphi $ the polar angle, we can express $ \tilde{\psi}(r,\theta,\phi):=\psi(x(r,\theta,\varphi))=\exp(-\alpha r) $. It is well known that the Laplacian on $ \mathcal{C}^2(\R\setminus\{0\}) $, $ \Delta $, can be exressed in polar coordinates as\footnote{Notice that we use $ \Delta $ to denote the Laplacian in both spherical and Cartesian coordinates.} \begin{equation}
\Delta\phi=\frac{1}{r}\frac{\partial^2}{\partial r^2}(r\phi)+\frac{1}{r^2\sin\varphi}\frac{\partial}{\partial\varphi}(\sin\varphi\frac{\partial\phi}{\partial\varphi})+\frac{1}{r^2\sin^2\varphi}\frac{\partial^2\phi}{\partial^2\theta}, \quad r>0,\ 0\leq\theta<2\pi,\ 0\leq\varphi\leq\pi.
\end{equation}
Thereby we see that \begin{equation*}
\begin{aligned}
(-\Delta-Z/\abs{x})\psi(x)\rvert_{x=x(r,\theta,\varphi)}=(-\Delta-Z/r)\tilde{\psi}(r,\theta,\varphi)&=-\frac{1}{r}\frac{\partial^2}{\partial r^2}(r \exp(-\alpha r))-Z/r\exp(-\alpha r)\\
&=(-\alpha^2+2\alpha/r-Z/r)\exp(-\alpha r).
\end{aligned}
\end{equation*} 
Thus choosing $ \alpha=Z/2 $ we find that $ (-\Delta-Z/r)\psi=E_0\psi $, with  $ E_0=-Z^2/4 $. From problem 2.(a) with $ \Omega=\R^3\setminus\{0\} $, which is clearly open, we then conclude that for all $ \varphi\in\mathcal{C}_0^1(\R^3\setminus\{0\}) $ we have 
\begin{equation}
\int_{\R^3\setminus\{0\}}\lvert(\nabla\varphi)(x) \rvert^2\diff x-\int_{\R^3\setminus\{0\}}\frac{Z}{\abs{x}}\lvert\varphi(x)\rvert^2 \diff x\geq E_0\int_{\R^3\setminus\{0\}}\lvert\varphi(x)\rvert^2 \diff x.
\end{equation}
\section{Lieb-Thirring inequalities in one dimension}
We show that in one dimension a Lieb-Thirring inequality of the form \begin{equation}
\sum_{j\geq0}\abs{E_j}^\gamma\leq L_\gamma\int_{\R}V_{-}(x)^{\gamma+1/2}\diff x, \label{LTineq1}
\end{equation}
cannot hold for $ 0\leq\gamma < 1/2 $. We show this by contradiction. Consider the Hamiltonian $ H=-\frac{\diff^2}{\diff x^2}+\alpha(\alpha+1)(\tanh(x)^2-1) $ with eigenfunction $ \psi(x)=\frac{1}{\cosh(x)^\alpha}$, $ \alpha>0 $,
\begin{equation}
-\frac{\diff^2}{\diff x^2}\psi(x)+\alpha(\alpha+1)(\tanh(x)^2-1)\psi(x)=-\alpha^2\psi(x).
\end{equation}
This can be seen by the following calculations:
\begin{equation}
\frac{\diff}{\diff x}\left(\frac{1}{\cosh(x)^\alpha}\right)=-\alpha\frac{\sinh(x)}{\cosh(x)^{\alpha+1}},
\end{equation}
\begin{equation}
\frac{\diff^2}{\diff x^2}\left(\frac{1}{\cosh(x)^\alpha}\right)=-\alpha\frac{1}{\cosh(x)^\alpha}+\alpha(\alpha+1)\frac{1}{\cosh(x)^\alpha}\tanh(x)^2.
\end{equation}
From which it clearly follows that\begin{equation}
-\frac{\diff^2}{\diff x^2}\psi(x)+\alpha(\alpha+1)(\tanh(x)^2-1)\psi(x)=-\alpha^2\psi(x).
\end{equation}
The potential of $ H $, is clearly given by $ V(x)=\alpha(\alpha+1)(\tanh(x)^2-1) $. Since $ \tanh(x)<1 $ for $ x\in\R $. we have that $ V_-(x)=\alpha(\alpha+1)(1-\tanh(x)^2) $. Assume now that a Lieb-Thirring inequality of the form \eqref{LTineq1} with $ 0\leq\gamma<1/2 $ holds. Let us then compute the right-hand side of the inequality with the potential $ V_-(x)=\alpha(\alpha+1)(1-\tanh(x)^2) $\begin{equation}
\begin{aligned}
L_\gamma\alpha(\alpha+1)\int_{\R}(1-\tanh(x)^2)^{\gamma+1/2} \diff x&=L_\gamma\alpha(\alpha+1)\int_{(-1,1)}(1-u^2)^{\gamma-1/2} \diff u\\
&=2L_\gamma\alpha(\alpha+1)\int_{(0,1)}(1+u)^{\gamma-1/2}(1-u)^{\gamma-1/2} \diff u,\label{int1}
\end{aligned}
\end{equation}
where we have made the change of variables $ u=\tanh(x) $ in the first line, Notice that then $ \frac{\diff u}{\diff x}=1-\tanh(x)^2 $. In the second line we simply exploited the fact that the integrand is even in $ u $ and factorized the integrand. Since the integrand is positive we can by monotone convergence theorem express it as a limit of integrals over the intervals $ (1/n,1) $ with $ n\to \infty $. Then we can rewrite all these integrals to Riemann integrals. By a simple comparison to integrals of the type $ \int_{0}^{1}\frac{1}{x^p}dx $, it is clear that the integral of \eqref{int1} is convergent if and only if $ \gamma<1/2 $. In this case we simply define $ C_\gamma=2L_\gamma\int_{(-1,1)}(1-u^2)^{\gamma-1/2} \diff u $, and we see that the Lieb-Thirring inequality is of the form\begin{equation}
\sum_j\abs{E_j}^\gamma\leq\alpha(\alpha+1)C_\gamma.
\end{equation}
On the other hand we know that $ \alpha^{2\gamma}\leq\sum_j\abs{E_j}^\gamma $, since we have shown $ -\alpha^2 $ to be one of the energies. Thus we conclude that \begin{equation}
\alpha^{2\gamma}\leq\alpha(\alpha+1)C_\gamma,\quad 0\leq \gamma<1/2
\end{equation}
However, this is clearly a contradiction since $ \alpha>0 $ was chosen arbitrarily. To see this, simply choose $ 0<\alpha<1 $ such that  $ \alpha^{2\gamma-1}> 2C_\gamma $. This conlcudes that in one dimension, there can be no Lieb-Thirring inequality of the form \eqref{LTineq1} with $ 0\leq\gamma<1/2 $.
\section{Thomas-Fermi theory}
\textbf{Notation:} We say in the following that an integral $ \int f(x)\diff x $ act as a bounded linear functional on some $ L^p $-space if the linear functional $ F:L^p\ni g\mapsto\int f(x)g(x) \diff x\in\C$ is bounded on $ L^p $\vspace{0.5cm}\\
Let $ \rho\in L^1(\R^3)\cap L^{5/3}(\R^3) $, $ \rho>0 $. The direct Couloumb energy is defined as \begin{equation}
D(\rho)=\int_{\R^3}\int_{\R^3}\frac{\rho(x)\rho(y)}{\abs{x-y}}\diff x\diff y. 
\end{equation}
Consider then the \emph{Thomas-Fermi} energy function \begin{equation}
\mathcal{E}^{TF}(\rho)=\int_{\R^3}\rho(x)^{5/3} \diff x -\int_{\R^3}\frac{\rho(x)}{\abs{x}}+D(\rho),
\end{equation} 
and for a fixed $ N>0 $ the minimization problem \begin{equation}
\begin{aligned}
E_0&:=\inf\left\{\mathcal{E}^{TF}(\rho) : \rho\in\mathcal{D}_N \right\}\\
\mathcal{D}_N&:=\left\{\rho\in L^1(\R^3)\cap L^{5/3}(\R^3) : \rho\geq0,\ \norm{\rho}_1\leq N \right\}.
\end{aligned}
\end{equation}
\subsection*{(a)}
We prove that for $ \rho\in \mathcal{D}_N $, we have\begin{equation}
\int_{\R^3}\frac{1}{\abs{x}}\rho(x)\diff x\leq cN+d\norm{\rho}_{5/3},
\end{equation}
with some constants $ c,d>0 $ independent of $ \rho $. To see this, let $a>0$, we then split the integral \begin{equation}
\int_{\R^3}\frac{1}{\abs{x}}\rho(x)\diff x=\int_{\abs{x}\leq a}\frac{1}{\abs{x}}\rho(x)\diff x+\int_{\abs{x}>a}\frac{1}{\abs{x}}\rho(x)\diff x\leq\int_{\abs{x}\leq a}\frac{1}{\abs{x}}\rho(x)\diff x+a^{-1}\norm{\rho}_1.
\end{equation}
For the remaining integral we use H\"older's inequality with $ q=5/2 $  and $ p=5/3 $. Then we get \begin{equation}
\int_{\abs{x}\leq a}\frac{1}{\abs{x}}\rho(x)\diff x\leq\abs{\int_{\abs{x}\leq a}\frac{1}{\abs{x}^{5/2}}\diff x}^{\frac{2}{5}}\norm{\rho}_{5/3}=\abs{4\pi \int_{(0,a)}\frac{1}{r^{1/2}}\diff r}^{\frac{2}{5}}\norm{\rho}_{5/3}=(8\pi\sqrt{a})^{2/5}\norm{\rho}_{5/3}
\end{equation}
where we in the first equality changed to spherical coordinates with Jacobian $ r^2\sin(\varphi) $ and computed the angular integrals directly. Thus we have for $ \rho\in \mathcal{D}_N $ \begin{equation}
\int_{\R^3}\frac{1}{\abs{x}}\rho(x)\diff x\leq a^{-1}N+(8\pi\sqrt{a})^{2/5}\norm{\rho}_{5/3},
\end{equation}
where we used that $ \norm{\rho}_1\leq N $.
Knowing that $ 0\leq D(\rho)<\infty $ and choosing $ 0<a\leq 1/(8\pi)^2 $ we may conclude that \begin{equation}
E_0\geq\inf\left\{(1-(8\pi\sqrt{a})^{2/5})\norm{\rho}_{5/3}-a^{-1}N+D(\rho) : \rho\in\mathcal{D}_N\right\}\geq-a^{-1}N>-\infty.
\end{equation}
Choosing $ a=1/(8\pi)^2 $ we may of course conclude that \begin{equation}
E_0\geq-(8\pi)^2N,
\end{equation}
However, choosing $ a<(8\pi)^2 $ we may conclude that any minimizing sequence, $ \rho^j $, is norm bounded in $ L^{5/3}(\R^3) $. 
\subsection*{(b)}
Let $ (\rho^j)_{j\geq1}\subset \mathcal{D}_N $ be a sequence such that $ \mathcal{E}^{TF}(\rho^j)\to E_0 $. Then $ \norm{\rho^j}_{5/3} $ is bounded. From Banach-Alaoglu's theorem and the fact that the predual of $ L^{5/3}(\R^3) $, namely $ L^{5/2}(\R^3) $, is reflexive\footnote{Clearly $ L^p(\R^3) $ is reflexive for all $ 1<p<\infty $ as $ (L^p(\R^3))^\ast=L^q(\R^3) $, with $ 1/p+1/q=1 $.}, we may conclude after restricting to a subsequence that we have $ \rho^j\rightharpoonup\rho_0 $ for some $ \rho_0\in L^{5/3}(\R^3) $. We prove now that \begin{equation}
\norm{\rho_0}_{5/3}\leq\liminf_{j\geq1}\norm{\rho^j}_{5/3}.
\end{equation}
In fact we can prove the more general statement: Let $ X $ be a Banach space and $ (x^j)_{j\geq1}\subset X $ be a sequence converging weakly to $ x\in X $, then $ \norm{x}\leq\liminf_{j\geq1}\norm{x^j} $.
\begin{proof}
	By the Hahn-Banach (extension) theorem, there exist a linear functional $ f:X\to \C $ such that $ f(x)=\norm{x} $ and such that $ \norm{f}=1 $. By weak convergence of $ x^j $ we then have \begin{equation}
	\norm{x}=\abs{f(x)}=\liminf_{j\geq 1}\abs{f(x^j)}\leq\liminf_{j\geq1}\norm{x^j},
	\end{equation}
	which proves the claim.
\end{proof}
Since $ L^{5/3}(\R^3) $ is a Banach space by the Riez-Fischer theorem, we have desired result.
\subsection*{(c)}
We prove now that $ \rho_0>0 $ almost everywhere. First we notice that $ \rho_0 $ is measurable. Consider therefore the set $ M_R=\{x\in\R^3 : \rho_0(x)<0\}\cap B_R(0) $, where $ B_R(0) $ denotes the ball of radius $ R $ centred at $ 0 $. Assume for contradiction that this set has measure greater than zero, $ \lambda(M_R)>0 $ for some $ R>0 $. Then $ \int_{M_R}\rho_0(x) \diff x <0$.\footnote{Consider the sets $ A^R_n=\{\rho_0<-1/n\}\cap B_R(0) $. Since $ M_R=\cup_{n=1}^{\infty}A_n^R $, we must have $ \lambda(A_m^R)>0 $ for some $ m\geq1 $. Thus $ \int_{M_R}\rho_0(x)\diff x\leq-\frac{1}{m}\lambda(A_m^R)<0 $.} However, this is a contradiction, since $ \int_{M_R}\rho^j(x) \diff x\geq0 $, and $ \int_{M_R} \diff x $ acts as a bounded linear functional on $ L^{5/3}(\R^3) $ by H\"older's inequality. As we have already established weak convergence of $ \rho^j $ in $ L^{5/3}(\R^3) $, we may conclude that $ \int_{M_R}\rho_0(x) \diff x=\lim\limits_{j\to\infty}\int_{M_R}\rho^j(x) \diff x\geq0 $. Thus we have established that $ \rho_0\geq0 $ on $ B_R(0) $ a.e. for all $ R>0 $, from which it follows that $ \rho_0\geq0 $ a.e.\\
\\
Now that we have established that $ \rho_0>0 $ a.e., we show that $ \int_{\R^3}\rho_0 \diff x \leq N $. To see this consider the sequence $( \chi_{B_n(0)}\rho_0)_{n\geq1} $. By the monotone convergence theorem (MCT) we know that \begin{equation}
\int_{\R^3}\rho_0(x) \diff x=\lim\limits_{n\to\infty}\int_{\R^3}\chi_{B_n(0)}(x)\rho_0(x) \diff x=\lim\limits_{n\to\infty}\int_{B_n(0)}\rho_0(x)\diff x.
\end{equation}
Since $ \int_{B_n(0)}\diff x $ acts as a bounded linear functional on $ L^{5/3}(\R^3) $ we may conclude from weak convergence of $ \rho^j $ that \begin{equation}
\int_{B_n(0)}\rho_0(x)\diff x=\lim\limits_{j\to\infty}\int_{B_n(0)}\rho^j(x)\diff x.
\end{equation}
From the fact that $ \int_{B_n(0)}\rho^j(x)\diff x\leq\int_{\R^3}\rho^j(x)\diff x\leq N $ we may conclude that \begin{equation}
\int_{B_n(0)}\rho_0(x)\diff x\leq N,\quad \text{for all }n\geq1.
\end{equation}
Thus it follows from the MCT that $ \int_{\R^3}\rho_0(x)\diff x\leq N $, so $ \rho_0\in\mathcal{D}_N $.
\subsection*{(d)}
It can be shown that $ \rho^j\rightharpoonup\rho_0 $ in $ L^q(\R^3) $ for some $ 1<q<3/2 $. Using this we can show that \begin{equation}
\int_{\R^3}\frac{1}{\abs{x}}\rho^j(x)\diff x\to\int_{\R^3}\frac{1}{\abs{x}}\rho_0(x)\diff x.
\end{equation}
To see this, we split the integral in two\begin{equation}
\int_{\R^3}\frac{1}{\abs{x}}\rho^j(x)\diff x=\int_{\abs{x}\leq1}\frac{1}{\abs{x}}\rho^j(x)\diff x+\int_{\abs{x}>1}\frac{1}{\abs{x}}\rho^j(x)\diff x.
\end{equation}
We then notice that the integral $ \int_{\abs{x}\leq1}\frac{1}{\abs{x}}\diff x $ acts as a bounded linear function on $ L^{5/3}(\R^3) $. This can be seen by the fact that $ \chi_{\abs{x}\leq1}\frac{1}{\abs{x}}\in L^{5/2}(\R^3) $.\footnote{This can be seen analogously to the computation we did in 4.(a) by changing to spherical coordinates.} Thus by H\"older's inequality we have for $ f\in L^{5/3}(\R^3) $ that\begin{equation}
\abs{\int_{\abs{x}\leq 1}\frac{1}{\abs{x}}f(x)\diff x}\leq\norm{\chi_{\abs{x}\leq1}\frac{1}{\abs{x}}}_{5/2}\norm{f}_{5/3}.
\end{equation}
Therefore, we may conclude the convergence of the first integral by weak convergence of $ \rho^j $ in $ L^{5/3}(\R^3) $. For the second integral, we instead use that $ \chi_{\abs{x}>1}\frac{1}{\abs{x}}\in L_p(\R^3) $ for all $ p>3 $.\footnote{Can also bee seen by changing to spherical coordinates.} Again by H\"older's inequality $ \int_{\abs{x}>1}\frac{1}{\abs{x}}\diff x $ acts as a bounded linear functional on $ L^q(\R^3) $ for all $ 1<q<3/2 $. Thus, by the fact that $ \rho^j $ converges weakly in $ L_q(\R^3) $ for some $ 1<q<3/2 $ we may conclude the convergence of the second integral. Thereby we have \begin{equation}
\underbrace{\int_{\abs{x}\leq1}\frac{1}{\abs{x}}\rho^j(x)\to\int_{\abs{x}\leq1}\frac{1}{\abs{x}}\rho_0(x)}_{\text{weak convergence in }L^{5/3}(\R^3)},\qquad \underbrace{\int_{\abs{x}>1}\frac{1}{\abs{x}}\rho^j(x)\to\int_{\abs{x}>1}\frac{1}{\abs{x}}\rho_0(x)}_{\text{weak convergence in }L^{q}(\R^3)\text{ for some }1<q<3/2}.
\end{equation}
From which we obtain the desired result \begin{equation}
\int_{\R^3}\frac{1}{\abs{x}}\rho^j(x)\to\int_{\R^3}\frac{1}{\abs{x}}\rho_0(x).
\end{equation}
\subsection*{(e)}
Collecting all the result from problem $ 4.(a) $ to $ 4.(d) $ and assuming that $ D(\rho_0)\leq\liminf_{j\to\infty}D(\rho^j) $, we can now show that \begin{equation}
\mathcal{E}^{TF}(\rho_0)=E_0,
\end{equation}
i.e. $ \rho_0 $ is a minimizer of $ \mathcal{E}^{TF} $. This follows directly from the inequalities\begin{equation}
E_0\leq\mathcal{E}^{TF}(\rho_0)\leq\liminf_{j\geq1}\mathcal{E}^{TF}(\rho^j)=E_0,
\end{equation}
where the first inquality follows from $ E_0 $ being the minimum energy. The second inequality follows from \begin{equation}
\mathcal{E}^{TF}(\rho_0)\leq\liminf_{j\geq1}\int_{\R^3}\rho^j(x)^{5/3}\diff x+\lim\limits_{j\to\infty}\int_{\R^3}\frac{1}{\abs{x}}\rho^j(x) \diff x+\liminf_{j\geq1}D(\rho^j)\leq\liminf_{j\geq1}\mathcal{E}^{TF}(\rho^j)
\end{equation}
where we for the first inequality have used the result of 5.(b) with the (lower semi-)continuity and non-decreasing monotonicity of $ x\mapsto x^{5/3} $ in the first term (see below), we have used the result of 5.(d) in the second term, and the assumption stated above in the third term. Furthermore we used, for the second inequality, the superadditivity of $ \liminf $, \ie $ \liminf_{j\geq1}(a^j)+\liminf_{j\geq1}b^j\leq\liminf_{j\geq1}(a^j+b^j) $.
Thus we may conclude that \begin{equation}
\mathcal{E}^{TF}(\rho_0)=E_0,
\end{equation}
hence $ \rho_0 $ is a minimizer.\vspace{0.5cm}\\
\textbf{Composition of a monotone non-decreasing lower semicontinuous function with a lower semicontinuous function}
Above, we used the following lemma: Let $ f:X\to\R $ be a (sequentially) lower semicontinuous functional and $ g:\R\to\R $ be a monotone non-decreasing (sequentially) lower semicontinuous function. Then $ g\circ f $ is (sequentially) lower semi continuous.
\begin{proof} We prove this for the case of lower semicontinuity by using nets, however the proof works also for sequential lower semicontinuity, by restricting to sequences.
	Let $ (a_i)_{i\in I}\subset X $ be a net converging to $ a\in X $. Then $ f(a)\leq\liminf_{i\in I}f(a_i) $ by lower semicontinuity of $f $. Assume that $ \liminf_{i\in I}f(a_i)<\infty $, then $$ (g\circ f)(a)=g(f(a))\leq g(\liminf_{i\in I}f(a_i))\leq\liminf_{i\in I} g(\inf_{j\geq i}f(a_j))\leq\liminf_{i\in I}g(f(a_i))=\liminf_{i\in I}(g\circ f)(a_i) $$
	where the first inequality follows from the monotonicity of $ g $, the second from lower semicontinuity of $ g $, and the third again from  the monotonicity of $ g $. On the other hand, if $ \liminf_{i\in I}f(a_i)=\infty $, we know there exist $ i_0\in I $ such that $ f(a_i)>f(a) $ for $ i\geq i_0 $, \ie the net $ (f(a_i))_{i\in I} $ is eventually in the set $ \{x\in\R : x>f(a)\} $. Therefore, by the monotonicity of $ g $ we may conclude that the net $ (g(f(a_i)))_{i\in I} $ is eventually in the set $ \{x\in \R : x\geq g(f(a))\} $. From this it follows that $ \liminf_{i\in I}\{(g\circ f)(a_i)\}\geq (g\circ f)(a) $. Hence $ g\circ f $ is lower semicontinuous. 
\end{proof}
\end{document}