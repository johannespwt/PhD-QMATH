\documentclass{beamer}[10]
\usepackage{pgf}
\usepackage[danish]{babel}
\usepackage[utf8]{inputenc}
%\usepackage{beamerthemesplit}
\usepackage{graphics,epsfig, subfigure}
\usepackage{url}
\usepackage{srcltx}

\usepackage{pdfpages}

\usepackage{amsmath}

\usepackage{amssymb}

\usepackage{bbm}

\usepackage{amsthm}

\usepackage{hyperref}

\usepackage{mathrsfs}

\usepackage{amsfonts}

\usepackage{bbm}

\usepackage{amsmath}

\usepackage{amssymb}

\usepackage{amsthm}

\usepackage{blkarray}

\usepackage{enumerate}

\usepackage{graphicx}

\usepackage{centernot}

\usepackage{caption}

\usepackage{braket}

\usepackage{slashed}

\usepackage{pgfplots}

\usepackage{feynmp-auto}

\usepackage{lastpage}

\usepackage{fancyhdr}
\usepackage[square,sort,comma,numbers]{natbib}

% TikZ til at lave figurer - for de avancerede
\usepackage{tikz}
% Div. pakker (muligt at ikke alle er i brug)
\usetikzlibrary{decorations.pathmorphing}
\usetikzlibrary{arrows.meta}
\usetikzlibrary{arrows}
\usetikzlibrary{decorations.pathreplacing,decorations.markings}
\usetikzlibrary{patterns}
\usetikzlibrary{fadings}
\usetikzlibrary{calc}
\usetikzlibrary{tikzmark,fit,shapes.geometric}

\definecolor{kugreen}{RGB}{50,93,61}
\definecolor{kugreenlys}{RGB}{132,158,139}
\definecolor{kugreenlyslys}{RGB}{173,190,177}
\definecolor{kugreenlyslyslys}{RGB}{214,223,216}
\setbeamercovered{transparent}
\mode<presentation>
%\usetheme[numbers,totalnumber,compress,sidebarshades]{PaloAlto}
\setbeamertemplate{footline}[frame number]
\usecolortheme[named=kugreen]{structure}
\useinnertheme{circles}
\usefonttheme[onlymath]{serif}
\setbeamercovered{transparent}
\setbeamertemplate{blocks}[rounded][shadow=true]
\setbeamercolor{block title}{bg=kugreen!50,fg=black}
\setbeamercolor{block body}{bg=kugreen!10,fg=black}
\logo{\includegraphics[width=0.8cm]{kuscience-logo}}
%\useoutertheme{infolines} 

\title{From integers to real numbers}
\subtitle{eIUP lecture}
\author{Johannes Agerskov\\
	\small{PhD student}}
\institute{Institute for Mathematical Sciences \\ University of Copenhagen}
\date{May 15, 2020.}

\setbeamercovered{invisible}

\begin{document}
\frame{\titlepage \vspace{-0.5cm}
}

\frame
{
\frametitle{Overview}
\linespread{1.5}
\tableofcontents%[pausesection]
}
\section{ILOs}
\begin{frame}
	\frametitle{ILOs}
	After this lecture the student will be able to
	\begin{itemize}
		\item Define and construct the natural numbers, the integers and the rational numbers.
		\item Understand the concept of a \emph{proof by contradiction}.
		\item Show that $ \sqrt{2} $ is not a rational number.
		\item If time permits; be familiar with the construction of real numbers as the completion of the rational numbers.
	\end{itemize}
\end{frame}

\section{Naturals and integers}
\begin{frame}
	\frametitle{Naturals and integers}
	\begin{block}{Natural numbers}
		The \emph{natural numbers} are defined as the infinite \emph{totally ordered set} \begin{equation}
		\mathbb{N}=\{0,1,2,3,...\}.
		\end{equation}
		These come defined with two compatible binary operations: addition and multiplication denoted by $ +:\mathbb{N}\times\mathbb{N}\to\mathbb{N} $ and $ \cdot:\mathbb{N}\times\mathbb{N}\to\mathbb{N} $ respectively.
		\pause\\
		We also define the set \begin{equation}
		\mathbb{N}_\times=\mathbb{N}\setminus\{0\}=\{1,2,3,...\}.
		\end{equation}
	\end{block}
\end{frame}

\begin{frame}
	\frametitle{Naturals and integers}
	\begin{block}{Integers}
		Requiring all naturals to have additive inverses we obtain integers\begin{equation}
		\mathbb{Z}=\{...,-2,-1,0,1,2,...\}.
		\end{equation}
		With addition and (extended to the negatives) multiplication.\\
		\pause
	We define the even and odd integers by \begin{equation}
	\mathbb{Z}^{\text{even}}=\{...,-4,-2,0,2,4,...\},\quad\mathbb{Z}^{\text{odd}}=\{...,-3,-1,1,3,...\},
				\end{equation}
				such that $ \mathbb{Z}=\mathbb{Z}^{\text{even}}\cup \mathbb{Z}^{\text{odd}} $.
	\end{block}
\end{frame}
\begin{frame}
	\frametitle{The rational numbers}
	\begin{block}{Construction of rational numbers}
		By requiring all integers to have multiplicative inverses and extending addition to these inverses we obtain the rational numbers \begin{equation}
		\mathbb{Q}=\left\{\frac{n}{m}\ \big\vert\ n\in\mathbb{Z},\ m\in\mathbb{N}_{\times}\right\}.
		\end{equation}
	\end{block}
	\begin{block}{Reduced fractions}
		We notice that all rational numbers can be written in a reduced form $ \frac{n}{m} $ where $ n\in\mathbb{Z} $ and $ m\in\mathbb{N}_{\times} $ have no common divisors. 
	\end{block}
\end{frame}
\section{Square-roots and rational numbers}
\begin{frame}
	\frametitle{Square-roots and rational numbers}
	\begin{block}{The square-root}
		We say that $ a=\sqrt{b} $ if and only if $ a>0 $ and $ a^2=b $, \emph{i.e.} \begin{equation}
		 \sqrt{b}^2=\sqrt{b}\cdot\sqrt{b}=b.
		\end{equation}
	\end{block}
	\begin{block}{Square-roots and rational numbers}
		\textbf{Question} (we will answer shortly together): \emph{If $ a $ is a rational number, is $ \sqrt{a} $ then also always a rational numbers?}\\
		\emph{e.g.} $ \sqrt{\frac{9}{16}}=\frac{3}{4},\ \sqrt{\frac{1}{144}}=\frac{1}{12} $, \emph{etc.} What about $ \sqrt{2}? $
	\end{block}
\end{frame}
\section{Proof by contradiction}
\begin{frame}
	\frametitle{Proof by contradiction}
	\begin{block}{Reductio ad absurdum}
		A proof by contradiction is a mathematical proof technique, that takes the following form: Let A be a logic statement,\begin{itemize}
			\item Assume A is true.\\
			\item Prove that A implies B.\\
			\item Prove that A implies -B (not B \emph{or} opposite of B).\\
			\item Conclude that A is false (\emph{i.e.} -A is true).
		\end{itemize}
		\emph{e.g.} We want to show that there is no highest natural number:
		Thus A="There is a largest natural number (say $ N $)". B="All natural numbers are smaller than $ N $". However, if $ N $ is a natural number then $ N+1 $ is a natural number and $ N+1>N $ which proves -B . Thus we may conclude that -A is true.
	\end{block}
\end{frame}
\begin{frame}
	\frametitle{Is $ \sqrt{2} $ a rational number?}
	\begin{theorem}
		$ \sqrt{2} $ is not a rational number.
	\end{theorem}
	\begin{block}{Proof}
		\begin{itemize}
			\item Assume that (A) $ \sqrt{2} $ is a rational number.\\
			\item Then (B) $ \sqrt{2}=\frac{a}{b} $ with $ a\in\mathbb{Z} $ and $ b\in\mathbb{N}_{\times} $ (can assume reduced).\\
			\item Notice that then $ \frac{a^2}{b^2}=\left(\frac{a}{b}\right)^2=2 $.\\
			 \item Continued on next slide...
		\end{itemize}
	\end{block}
\end{frame}
\begin{frame}
	\frametitle{Is $ \sqrt{2} $ a rational number?}
	\begin{block}{Proof continued}
		\begin{itemize}
			\item Knowing that $ \frac{a^2}{b^2}=2 $, is $ a^2 $ and even or odd number?\pause\pause \\(\textbf{Even} since $ a^2=2b^2 $.)\pause\\
			\item Knowing that $ a^2 $ is even, is $ a $ even or odd?\pause\pause \\
			(\textbf{Even} since $ a^2 $ is odd if $ a $ is odd.)\pause
			\item Knowing that $ a $ is even, is $ \frac{a^2}{2} $ even or odd?\pause\pause \\
			(\textbf{Even} since $ \frac{a^2}{2}=a\cdot\frac{a}{2} $ is even if $ a $ is even.)\pause
			\item Knowing that $ \frac{a^2}{2} $ is even, is $ b^2 $ even or odd?\pause\pause \\
			(\textbf{Even} since $ b^2=\frac{a^2}{2} $. Notice then $ b $ is also \textbf{Even}.)\pause\\
			\item We go to breakout rooms and finish the proof!
		\end{itemize}
	\end{block}
\end{frame}
\begin{frame}
	\frametitle{Is $ \sqrt{2} $ a rational number?}
	Go to Padlet \url{https://padlet.com/johannesas/wrgamlv82s844qfs}
	\begin{block}{Proof continued}
		Questions for Padlet:
		\begin{enumerate}
		\item Knowing that $ a $ and $ b $ are even, is $\frac{a}{b}$ a reduced fraction?
		\item What may we then conclude about our initial assumption that $ \sqrt{2} $ is a rational numbers?
	\end{enumerate}\pause\pause\pause
	My answers:
	\begin{enumerate}
		\item No, $ \frac{a}{b} $ is not reduced since $ 2 $ is a common divisor of $ a $ and $ b $.
		\item By reductio ad absurdum, we may conclude that $ \sqrt{2} $ is \textbf{not} a rational number.
	\end{enumerate}
	\end{block}
\end{frame}
\begin{frame}
	\frametitle{The real numbers}
	\begin{block}{What number is $ \sqrt{2} $?}
		\begin{itemize}
			\item Representing $ \sqrt{2} $ by decimal number $ \sqrt{2}=1.414... $ we find that $ \sqrt{2} $ can always be approximated by a rational number by simply cutting off the decimals at some point, \emph{e.g.} $ 1.41=\frac{141}{100} $ and $ \sqrt{2}-1.41<0.01 $.
			\item We call all numbers that can be approximated by rational numbers in such a way for real numbers, and we say that they are obtained by completing the rational numbers.
		\end{itemize}	
	\end{block}
	\begin{block}{Fun fact!}
		There are equally many natural numbers, integers, and rational numbers but the real numbers form a substantially larger set. If you were to say how big a percentage of the real numbers the rational numbers make up, it is $ 0\% $
	\end{block}
\end{frame}
\begin{frame}
	\centering
	\Large Thank you for your attention.\\
\end{frame}		
\end{document}
