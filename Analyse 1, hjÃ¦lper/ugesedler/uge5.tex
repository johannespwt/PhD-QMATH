\begin{itemize}
    \item Hvis $f\colon U \to \mR$ ($\mC$) er en uendeligt ofte differentiabel funktion (her er definitionsmængden $U \subseteq \mR$ åben), kan man knytte en særlig potensrække til $f$ kaldet dens Taylor-række. Tirsdag afslutter vi emnet potensrækker med at undersøge sådanne Taylorrækker (MC 4.3).
    
    Vi påbegynder derefter et nyt kapitel. Man kan spørge sig selv, hvad der sker på randen af det åbne konvergensområde for en potensrække ($\abs{x}=r$), og i undersøgelsen af dette spørgsmål kan man uden tab af generalitet antage, at $r=1$. Vi kan da skrive $x$ på polarformen $x=e^{i \theta}$ og betragte rækken som en række i $\theta\in\mR$, nemlig $\sum_{n=0}^\infty a_n e^{i n \theta}$. Det viser sig naturligt også at betragte rækken $\sum_{n=0}^\infty a_{-n} e^{-i n \theta}$, hvor $a_{0}, a_{-1}, a_{-2}, \ldots$ er en følge der udvider følgen $\{a_n\}_{n \in \mN}$ til en dobbelsidet uendelig følge $\{a_n\}_{n \in \mZ}$, og endda begge rækker samtidigt som $\sum_{n=-\infty}^\infty a_n e^{i n \theta}$. Hvis denne række er konvergent (defineret på passende vis), er dens sumfunktion $f$ en $2\pi$-periodisk funktion i $\theta$. Rækker af denne type kaldes trigonometriske rækker (husk Eulers identitet $e^{i x}= \cos(x)+ i \sin(x)$, der knytter de trigonometriske funktioner sammen med den komplekse eksponentialfunktion). Hovedformålet er nu at undersøge konvergensegenskaberne for sådanne rækker, og at finde ud af hvornår og hvordan $2\pi$-periodiske funktioner kan skrives som sumfunktioner af trigonometriske rækker.	Dette emne hedder Fourieranalyse og viser sig at være yderst vigtigt	i både abstrakt matematik (harmonisk analyse) og anvendt matematik (signalanalyse). I denne uge starter ved med definitioner og eksempler (MC 5.1-5.2).
    
    I øvelsestimerne behandler vi Opgave 1-3, eventuelt Opgave 4.
\end{itemize}

\begin{opg}[Potensrækker]
I denne opgaver benytter vi konventionen, at $x$ er reel og $z$ er kompleks.
\begin{enumerate}
	\item
	Find konvergensområdet for følgende rækker, 
	\begin{tasks}{3}
		\item $\displaystyle \sum_{n=1}^\infty \sqrt{n} \left(\frac x2\right)^n, \quad x\in\mR$.
		\item $\displaystyle \sum_{n=1}^\infty \frac{z^n}{n^2}, \quad z\in\mC$
		\item $\displaystyle \sum_{n=1}^\infty \sqrt{\frac{x}{n}}^n, \quad x\in\mR_+$
	\end{tasks}

    \item Lad $p\in\mN$. Vis, at hvis 
    $$ \sum_{n=0}^\infty a_n x^n, \QUAD x\in\mR $$
    har konvergensradius $r$, så har 
    $$ \sum_{n=0}^\infty a_n x^{pn}, \QUAD x\in\mR $$
    konvergensradius $\sqrt[p]{r}$. 
    
    \item Find konvergensområdet og kontinuitetsområdet for følgende rækker og deres sumfunktioner:
    \begin{tasks}{2}
	    \item $\displaystyle \sum_{n=0}^\infty (z-2)^n$ %TL 12.6.1 a
	    \item $\displaystyle \sum_{n=0}^\infty \max\{0,(-2)^n\}{(x-1)^n}$
    \end{tasks}
\end{enumerate}
\end{opg}

\begin{opg}[Potensrækker, integration og differentiation] \hfill
\begin{enumerate}
	\item Vis, at rækken ${\displaystyle \sum_{n=0}^\infty \pare{-1}^nx^n}$ konvergerer uniformt mod $\displaystyle \frac1{1+x}$ på intervallet $[-a,a]$, for ethvert $a\in[0,1)$. Brug dette til at vise uniform konvergens af følgende funktionsrækker	mod de angivne funktioner på samme interval:
	\begin{tasks}{2}
		\item $\displaystyle \sum_{n=1}^\infty \pare{-1}^nnx^{n-1}$ \; mod \; $\displaystyle -\frac{1}{(1+x)^2}$
		\item $\displaystyle \sum_{n=0}^\infty \frac{\pare{-1}^n}{n+1}x^{n+1}$ \; mod \; $\displaystyle \log(1+x)$
	\end{tasks}
	
	\item Betragt rækken fra ii) i forrige delopgave. Kan du finde en talrække med sum $\log(2)$?
\end{enumerate}
\end{opg}

\begin{opg}[Bessels funktion, af Henrik Schlichtkrull] \hfill \\
Lad $\alpha\in\mN$ og definér \textit{Bessels funktion} $J_\alpha\colon \mR\to\mR$ ved
$$ J_\alpha\pare{x}\coloneqq \sum_{n=0}^\infty
    \frac{\pare{-1}^n}{2^{2n+\alpha}n!\pare{n+\alpha}!}x^{2n+\alpha} $$
\begin{enumerate}
    \item Bestem konvergensradius $r$ for Bessels funktion.
    
    \item Vis, at $J_\alpha$ er en løsning til differentialligningen
    $$ x^2\frac{\partial^2 y}{\partial x^2} + x\frac{\partial y}{\partial x}
        + \pare{x^2-\alpha^2}y = 0 $$
\end{enumerate}
\end{opg}

\begin{opg}[Den svære, af Erik Lange]
Lad $f_1\colon\mR\to\mR$ være givet ved $f_1\pare{x} = 0$, og definér følgen $\seq f$ af funktioner $f_n\colon\mR\to\mR$ givet ved
$$ f_n\pare{x} = \int_0^x f_{n-1}\pare{t} + \sin\pare{t} \,dt $$
Bemærk at følgen er veldefineret, eftersom $f_n$ er kontinuert for alle $n\in\mN$.

\begin{enumerate}
    \item Vis ved induktion, at
    $$ \abs{f_{n+1}\pare{x} - f_n\pare{x}} \leq \frac{\abs{x}^{n+1}}{\pare{n+1}!} $$
    for alle $n\in\mN$ og $x\in\mR$. Brug dette til at vise, at
    $$ \abs{f_m - f_n} \leq \frac{\abs{x}^{n+1}}{\pare{n+1}!} + \ldots + \frac{\abs{x}^m}{m!} $$
    for alle $m\geq n$. 
    
    \item Vis, at der findes en funktion $f\colon\mR\to\mR$, så $\seq f$ konvergerer punktvist mod $f$.
    
    \item Vis, at konvergens er uniform på intervaller af typen $[-a,a]$.
    
    \item Vis, at $f$ løser differentialligningen
    $$ \frac{\partial y}{\partial x} = y + \sin $$
    på hele $\mR$ med begyndelsesbetingelsen $y\pare{0} = 0$.
\end{enumerate}
Idéerne fra denne opgave kan generaliseres og bruges til et bevis for Picard-Lindlöfs sætning om eksistens af løsninger til visse typer differentialligninger.
\end{opg}