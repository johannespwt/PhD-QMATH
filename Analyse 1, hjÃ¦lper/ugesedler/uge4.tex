\begin{itemize}
    \item Tirsdag i denne uge taler vi om funktionsrækker (MC 3.2-3.3). Vi ser på eksempler og ser på en test, der sikrer uniform konvergens af funktionsrækker. Til øvelserne snakker vi om Opgave 1-3.

	\item Torsdag starter vi med at se på en vigtig gruppe af eksempler af funktionsrækker, potensrækkerne $\sum_{n=0}^\infty a_n x^n$, som generaliserer den geometriske række $\sum_{n=0}^\infty x^n$ (MC 4.1 og 4.2). Ligesom den geometriske række konvergerer mod $\frac{1}{1-x}$ på ''enhedsskiven'' ($|x|<1$), så finder vi, at alle potensrækker har en konvergensradius $r \in [0,\infty]$ udenfor hvilken ($|x|>r$) rækken divergerer, og indenfor hvilken ($|x|<r$) rækken konvergerer. Til øvelserne snakker vi om Opgave 4-6.
\end{itemize}

\begin{opg}[Funktionsfølger]
Lad, for $n \in \mN$, funktionen $h_n : \mR \to \mR$ være givet ved $h_n(x) = x^n$.
\begin{enumerate}
	\item Skitsér på intervallet $[-2,2]$ grafen for funktion $h_n$ for mindst fire forskellige værdier af $n$.
	
	Bestem talfølgerne $\cbrac{h_n(x)}_{n \in \mN}$ for $x=-1, x= 0, x= \frac{1}{2}, x= 1$ og $x=2$. For hvilke af disse $x$ er talfølgen konvergent, og hvad er grænseværdien i så fald?
	

	\item Bestem \textit{konvergensområdet} for funktionsfølgen, altså mængden af værdier $x \in \mR$ for hvilke $\{ h_n(x)\}_{n\in\mN}$ konvergerer. 
	
    Angiv den funktion $h$, som følgen $\seq h$ konvergerer punktvist imod i sit konvergensområde.  
    
    \item Betragt nu $\seq h$ som en følge af funktioner defineret på mængden $A= (-1,1]$. Indikér grafisk $\varepsilon$-båndet omkring den punktvise grænse $h$, og beregn afstanden
	$$ d_{(-1,1]}(h_n,  h) $$
    for $n \in \mN$, hvor $d_A(f,g) = \sup_{x \in A} \cbrac{|f(x)-g(x)|}$ er den uniforme afstand.

    Er funktionsfølgen $\seq h$ uniformt konvergent på intervallet $(-1, 1]$?
\end{enumerate}
\end{opg}

\begin{opg}[Konvergens af funktionsfølger]
For $n \in \mN$ lad funktionerne $f_n, g_n\colon \mR \to \mR$ og $h_n\colon [0, \infty) \to [0,\infty)$ være givet ved 
\begin{align*}
    f_n(x) &= \frac{1}{(x-n)^2+1}, \qquad 
    g_{n}(x)  =  e^{-\left(x-\frac{1}{n}\right)^2} \quad \text{og} \\
    h_1(x) &= x, \quad h_{n+1}(x) = \sqrt{h_n(x)} 
\end{align*}
\begin{enumerate}
    \item Skitsér i samme koordinatsystem grafen for $f_n$ for flere forskellige værdier af $n\in\mN$. Gør det samme for $g_n$ og $h_n$.
    
    Vis, at alle tre funktionsfølger $\seq f$, $\seq g$ og $\seq h$ konvergerer punktvist på deres  definitionsområder, og angiv grænsefunktionerne $f$, $g$ og $h$.
    
    
    \item Vis, at funktionsfølgerne $\seq f$ og $\seq h$ ikke konvergerer uniformt på deres respektive definitionsområder. Afgør om følgen $\seq g$ konvergerer uniformt på $\mR$.

    \item Lad $a\in\mR$. Vis, at $\seq f$ konvergerer uniformt på $(-\infty, a]$. 
    
    Lad $b,c\in\mR$ og antag $0 < b < c$. Vis, at $\seq h$ konvergerer uniformt på $[b,c]$. 
\end{enumerate} 
\end{opg}

\begin{opg}[Forberedelse til funktionsrækker]
Lad $g_0: [0,1) \to \mR$ være givet ved $g_0(x)=1$, og lad $g_n\colon [0,1) \to \mR$ være givet ved $g_n(x)=g_{n-1}(x)+x^{n}$ for alle $n \in \mN$. 
	
\begin{enumerate}
	\item Opskriv de første fem led i funktionsfølgen $\{g_n\}_{ \in \mN_0}$ og tegn deres grafer. 
    
    Vis, at $\{g_n\}_{ \in \mN_0}$ konvergerer punktvist på $[0,1)$, og find grænsefunktionen $g$. 

    Afgør om konvergensen er uniform.
	
	\item Lad $a \in [0,1)$. Vis, at funktionsfølgen $\{g_n\}_{ \in \mN_0}$ konvergerer uniformt på $[0,a]$. 
	
	Diskutér i gruppen eller med jeres instruktor, om dette viser, at der er uniform konvergens på $[0,1)$, når $a$ kan vælges vilkårligt tæt på $1$.
\end{enumerate}
\end{opg}

\begin{opg}[Funktionsfølger og integration]
Lad $f_1 : \mR \to [0, \infty)$ være kontinuert med $ \int_{- \infty}^{\infty} f_1(x) \mathrm{d} x = 1$. Vi kan tænke på $f_1$ som tæthedsfunktionen for en sandsynlighedsfordeling på $\mR$. Definér, for ethvert $n \in \mN$, funktionen $f_n : \mR \to [0, \infty)$ ved 
\begin{equation} \label{fn}
    f_n(x) = n f_1(nx).
\end{equation}

\begin{enumerate}
	\item Tegn graferne for $f_1, f_2, f_3$ og $f_4$ i det tilfælde, hvor $f_1$ er tæthedsfunktionen for standardnormalfordelingen, d.v.s. $f_1(x) = \frac{1}{\sqrt{2 \pi}} e^{-x^2/2}$. 
	
	I det følgende er $f_1$ igen en vilkårlig kontinuert tæthedsfunktion, og $f_n$ defineret som i \eqref{fn}.
	
	\item Vis ved brug af substitution, at 
	
	\begin{equation} \label{intid}
	\int_{-a}^a f_n(x) \,dx =  \int_{-na}^{na} f_1(x) \,dx 
	\end{equation}
	for ethvert $a \geq 0$.	
	
	Begrund, at funktionen $f_n$ er en kontinuert tæthedsfunktion for en sandsynlighedsfordeling på $\mR$ (d.v.s. er kontinuert og ikke-negativ med $\int_{-\infty}^{\infty} f_n(x) \mathrm{d}x=1$).
	
	\item Lad $a> 0$. Begrund, at 
	$$ \lim_{n \to \infty} \int_{-a}^a f_n(x) \,dx = 1. $$
    Vis med udgangspunkt i MC 3.16 at funktionsfølgen $\seq f$ ikke kan konvergere uniformt mod nulfunktionen på intervallet $[-a,a]$. 
	
	Afgør om følgen kan konvergere uniformt mod en anden funktion på dette interval.
	
	Afgør om følgen kan konvergere punktvist mod nulfunktionen.
\end{enumerate}
\end{opg}

\begin{opg}
I denne opgave, bruger vi den konvention at $x$ er reel og $z$ er kompleks.
\begin{enumerate}
	\item Betragt rækken $\displaystyle \sum_{n=0}^\infty (2x)^n$. Opskriv de første fire afsnitssummer og tegn deres grafer sammen med grafen for funktionen $x \mapsto \displaystyle\frac{1}{1 - 2x}$ på intervallet $(-1,1)$.\footnote{Den sidste funktion er ikke defineret i $x=1/2$ og er ubegrænset i enhver omegn af $x=1/2$, men for at tegne grafen kan du lave en passende afskæring af $y$-aksen.}
	
	\item Find området hvor følgende rækker konvergerer punktvist, det vil sige betragt dem som talrækker for hvert $x$ eller $z$, og afgør for hvilke værdier de konvergerer.
	\begin{tasks}{2}
		\item $\displaystyle \sum_{n=0}^\infty (2z)^n$.
		\item $\displaystyle \sum_{n=1}^\infty  \frac{e^{-nx}}{ (-1)^n n}$.
	\end{tasks}

	Er konvergensen absolut i alle punkter i konvergensområdet?
	
	\item Vis uniform konvergens af rækkerne på de angivne mængder.
	\begin{tasks}{2}
		\item $\displaystyle \sum_{n=1}^\infty \frac{\cos(nx)}{n^2}$ på $\mR$
		\item $\displaystyle \sum_{n=1}^\infty \frac{x^n}{n^3}$ på $[-1,1]$;
		\item $\displaystyle \sum_{n=1}^\infty \frac{1}{n^z}$ på $\cbrac{z\in\mC:\textup{Re}(z)\geq p }$, hvor $p > 1$.
	\end{tasks}

    For $t>0$ definerer vi $t^z= t^{a+bi} := t^a t^{bi}$, $t^{bi} := e^{\ln(t)b i}$ for $z=a+bi$ med $a,b \in \mR$. 
    
    Afgør, om rækken fra iii) er uniformt konvergent på mængden $\{z \in \mC : \textup{Re}(z)>1\}$?
\end{enumerate} 
\end{opg}

\begin{opg}[Den svære, af Mathias Schack Rabing] \hfill \\
Betragt følgen $\seq a$ givet ved $a_n=e^{2\pi i \theta n}$.

\begin{enumerate}
    \item Vis, at $\seq a$ er konvergent for alle $\theta\in\mZ$.
    \item Vis, at $\seq a$ har endeligt mange fortætningspunkter for alle $\theta\in\mQ$.
    \item Vis, at ethvert punkt på enhedscirklen er et fortætningspunkt for $\seq a$ for alle $\theta\in\mR\setminus\mQ$.
\end{enumerate}
\end{opg}