\setcounter{section}{1}

\begin{center}
\begin{tabular}{|l||c|c|c|c|c|c|c|}\hline
Forel\ae{}sning tirsdag&1a1&1a2&1a3&1a4&1a5&1a6&\\\hline
Opgaver tirsdag&1.1&1.2&1.3&1.4&&&\\\hline
Forel\ae{}sning torsdag&1b1&1b2&1b3&1b4&1b5&1b6&1b7\\\hline
Opgaver torsdag&1.5&1.6&1.7&1.8&1.9&&\\\hline
Forankringssp\o{}rgsm\aa{}l&(1.x)&(1.y)&(1.z)&&&&\\\hline
\end{tabular}
\end{center}

\section*{Forel\ae{}sningsprogram}
I l\o{}bet af kursets f\o{}rste uge d\ae{}kker vi MC Afsnit 1.1--1.4. Afsnit 1.1 er en generel diskussion af analyse i de komplekse tal $\complex$, navnlig for funktioner fra (en delm\ae{}ngde af) $\reals$ til $\complex$, og Afsnit 1.2 og 1.3 d\ae{}kker to fundamentale begreber om talf\o{}lger: Fort\ae{}tningspunkter og konvergens. Alt stof der gennemg\aa{}s i denne uge er fuldst\ae{}ndig fundamentalt for alt hvad der f\o{}lger i kursets l\o{}b, m\aa{}ske med undtagelse af Lemma 1.8 og Proposition 1.30 hvis beviser vi henholdsvis overspringer og gennemg\aa{}r lidt overfladisk i forel\ae{}sningsvideoerne.

Deltagerne bedes bide s\ae{}rligt m\ae{}rke i hvordan vi bygger videre p\aa{} Analyse 0 dels i forhold til at give mening til differentiabilitet og integration for funktioner med v\ae{}rdier i $\complex$ ved at sammenligne med teorien for differentiabilitet og integration for funktioner med v\ae{}rdier i $\reals^2$, og ved at inddrage viden om gr\ae{}nseovergang i Analyse 0-forstand til at konkludere om 
gr\ae{}nseovergang i Analyse 1-forstand.

I forel\ae{}sningsvideo 1b3 opstiller og beviser vi ``Klemmelemmaet'' der fortjener at blive eksplicit n\ae{}vnt her til brug ved opgaveregningen:

\noindent{\bf Klemmelemmaet} Lad tre reelle f\o{}lger $\{a_n\}_{n\in \naturals}$, $\{b_n\}_{n\in \naturals}$, $\{c_n\}_{n\in \naturals}$ v\ae{}re givet med
\[
a_n\leq b_n\leq c_n
\]
for alle $n\in\naturals$. Hvis $\{a_n\}_{n\in \naturals}$ og $\{c_n\}_{n\in \naturals}$ er konvergente med
\[
\lim_{n\to\infty}a_n=\lim_{n\to\infty}c_n=d
\]
s\aa{} er ogs\aa{} $\{b_n\}_{n\in \naturals}$ konvergent med 
\[
\lim_{n\to\infty}b_n=d.
\]


\section*{Forankringssp\o{}rgsm\aa{}l}

\begin{itemize}
\item[(1.x)] G\ae{}lder det at $\frac d{dx}(e^{zx})=ze^{zx}$ for alle fastholdte $z\in \complex$?
\item[(1.y)] Hvad er forskellen p\aa{} at vide at $a_n$ konvergerer og p\aa{} at kunne bestemme gr\ae{}nsev\ae{}rdien? Hvad skal der til ud over konvergens til at for at fx bestemme de f\o{}rste 5 betydende cifre af gr\ae{}nsev\ae{}rdien? Hvordan relaterer dette sig til ideen om approksimation?
\item[(1.z)] Vi har bem\ae{}rket at An0-udsagnet $\lim_{x\to \infty}f(x)=a$ er logisk st\ae{}rkere end An1-udsagnet $\lim_{n\to \infty}f(n)=a$. Hvordan kan det s\aa{} v\ae{}re at det ofte er lettere for os at etablere at An0-udsagnet er sandt?
\end{itemize}

%Ugens forl\o{}b:
%
%\begin{center}
%\begin{tabular}{|l||c|c|c|c|c|c|c|}\hline
%Forel\ae{}sning tirsdag&1a1&1a2&1a3&1a4&1a5&1a6&\\\hline
%Opgaver tirsdag&1.1&1.2&1.3&1.4&&&\\\hline
%Forel\ae{}sning torsdag&1b1&1b2&1b3&1b4&1b5&1b6&1b7\\\hline
%Opgaver torsdag&1.5&1.6&1.7&1.8&1.9&&\\\hline
%Forankringssp\o{}rgsm\aa{}l&1.x&1.y&1.z&&&&\\\hline
%\end{tabular}
%\end{center}
%\begin{itemize}
%\item Inden \o{}velserne tirsdag 26/4: Se alle videoerne 1a1--1a6 og arbejd p\aa{} at l\o{}se opgaverne 1.1--1.4.
%\item Tirsdag 26/4 10:15--12:00: \O{}velser hvor opgave 1.1--1.4 diskuteres. Alle hold m\o{}der fysisk frem og deltagere der kun kan deltage online er velkomne hos hold X i An1-zoomrummet. 
%\item Inden sp\o{}rgetimen torsdag 28/4: Se alle videoerne 1b1--1b7 og fors\o{}g at svare p\aa{} fundamentalsp\o{}rgsm\aa{}lene. Klarg\o{}r sp\o{}rgsm\aa{}l til S\o{}ren om ugens gennemg\aa{}ede stof.
%\item Inden \o{}velserne torsdag 26/4: Arbejd p\aa{} at l\o{}se opgaverne 1.5--1.9.
%\item Torsdag 28/4 11:15--12:00: Sp\o{}rgetime med S\o{}ren  i An1-zoomrummet. 
%\item Torsdag 28/4 13:15--16:00: \O{}velser hvor opgave 1.5--1.9 (og eventuelt tidligere opgaver) diskuteres. Hold Y afholdes ikke fysisk og deltagerne der henvises til An1-zoomrummet
%\end{itemize}

%\begin{opg}
%Reelle tal, supremum og infumum
%\begin{enumerate}
%    \item  Vælg et $a_0>0$ og definer følgen $\seq{a}$ ved
%    $$ a_n = \frac{a_{n-1}}{2} + \frac{3}{2 a_{n-1}}\quad \text{for } n \in\mN $$
%    Beregn de første ti elementer af følgen, $a_1, a_2, \ldots, a_{10}$, for et konkret valg af $a_0$ (f.eks. $a_0=1$). Hvordan ser følgen ud til at opføre sig?
%
%    Begrund, at alle følgens elementer er rationale tal, såfremt $a_0$ er et rationalt tal.   
%%    \item Lad 
%%    $$ A = \cbrac{ 1/n : n \in \mN}, \QUAD
%%        B = \cbrac{x\in\mQ: x^3 >2}. $$
%%    Begrund kort, hvorfor Dedekind-fuldstændigheden (= supremumsegenskaben) af $\mR$ medfører, at $A$ har et supremum og et infimum, og at $B$ har et infimum. 
%%    
%%    Find $\sup A$, $\inf A$ og $\inf B$. Begrund dine svar så omhyggeligt som muligt.
%%    
%%    Gælder det, at $ \sup A \in A$? Hvad med $\inf A \in A$ og $\inf B \in B$?
%%    
%%    \item Lad $A$ være en delmængde af $\mR$, og lad $m \in \mR$. Vi siger, at \emph{$m$ er et maksimum for $A$}, hvis $m$ er en øvre grænse for $A$, og $m$ er element i $A$ (altså $m \in A$). Hvis $A$ har et maksimum, er dette entydigt (overvej), og vi betegner det $\max A$. 
%%
%%\begin{enumerate}[label=\roman*)]
%%    \item Giv et eksempel på en mængde $A$, der har et supremum, men ikke et maksimum. 
%%    
%%    \item Vis, at det modsatte ikke kan lade sig gøre: Hvis $A$ har et maksimum, så har $A$ også et supremum, nemlig $\sup A = \max A$. Forsøg som tidligere at argumentere så omhyggeligt som muligt. 
%%    
%%    \item Afgør, om følgende påstand er sand: \emph{Mængden $A$ har et maksimum, hvis og kun hvis $A$ har et supremum, og $\sup A \in A$.}
%%    
%%\end{enumerate}
%\end{enumerate}
%\end{opg} 

%TIL TIRSDAG:

\section*{Opgaver}

\begin{opg} \emph{Algebra og komplekse tal}
\begin{enumerate}
    \item Skriv følgende komplekse tal på formen $a+ bi$ med $a, b \in \mR$.
    \begin{multicols}{3}
    \begin{enumerate}[label=\roman*)]
        \item $(3+2i)-(6- \sqrt{13} i)$
	    \item $(3+2i) \cdot (6- \sqrt{13} i)$ 
		\item  $\dfrac{3+2i}{6- \sqrt{13} i}$ 
	    \item $\sqrt{2}\, e^{i \pi /4}$
		\item  $i^{2021}$
	    \item $(1+i)^{2021}$
    \end{enumerate}
    \end{multicols}
    
%    \item Find alle komplekse løsninger $z$ til følgende ligninger:
%    \begin{multicols}{4}
%    \begin{enumerate}[label=\roman*)]
%        \item $z^2=1$
%        \item $z^3=1$
%        \item $z^4=1$
%        \item $z^n=1$, $n \in \mN$
%    \end{enumerate}
%    \end{multicols}
%    Skriv løsningerne både på formen med real- og imaginærdel og på polarform. 
%
%    Tegn dem i den komplekse talplan. (I iv) behøver du kun tegne dem for endeligt mange af dine yndlings-$n$.)
%    
    \item Lad $x=\frac 1{\sqrt 2}(\sqrt{3}+ i)$. Skitsér i hånden følgende punkter i planen, for $n=1,\dots,8$:
    \begin{multicols}{3}
    \begin{enumerate}[label=\roman*)]
        \item $(\textup{Re}(x^n),\text{Im}(x^n))$
        \item $(n,\textup{Re}(x^n))$
        \item $(n,\textup{Im}(x^n))$
    \end{enumerate}
    \end{multicols}
  %  Her betegner $\text{Re}(z)$ og $\text{Im}(z)$ hhv. real- og imaginærdelen af et komplekst tal $z$, d.v.s. hvis $z=a+bi$ med $a,b \in \mR$, så er $\text{Re}(z)=a$ og $\text{Im}(z)=b$.
\end{enumerate}
\end{opg}
\begin{opg} \emph{Differentiation og integration af komplekse funktioner}
	\begin{enumerate}
		\item Lad $ I\subset \R $ være et åbent interval. Vis formlen $ (fg)'=f'g+fg' $ for differentiable funktioner $ f,g:I\to \C $ på to måder:
		\begin{enumerate}[label=\roman*]
			\item Opskriv $ \Re(fg) $ og $ \Im(fg) $ ud fra $ \Re(f) $, $ \Re(g) $, $ \Im(f) $ og $ \Im(g) $, og benyt almindelige regneregler for differentiation af produkt.
			\item Inspicer beviset for den almindelige regneregel for differentiation af produkt (Sætning 4.9 i [EHM]) og argumenter for, at beviset også gælder for funktioner $ f: I\to\C $ ``mutatis mutandis"\footnote{Slå ``mutatis mutandis" op hvis du ikke ved, hvad det betyder.}.
		\end{enumerate}
		\item Lad $ n\in \mZ $. Betragt funktionen $ f:\R \to \C $ givet ved $ f(x)=x\e^{inx} $ for alle $ x\in \R $. Find $ f'(x) $ for ethvert $ x\in \R $.
		\item Find en stamfunktion til $ f $.
		%Hint: Brug resultatet fra a).
		\item Lad $ g:[0,2\pi]\to\C $ og $ h:[0,2\pi]\to\C $ være givet ved $ g(x)=\Re f(x) $ og $ h(x)=\Im f(x) $ for alle $ x\in[0,2\pi] $. Argumenter for at $ g $ og $ h $ er Riemann integrable og udregn $ \int_{0}^{2\pi}x\e^{inx} \diff x $
	\end{enumerate}
\end{opg}

\begin{opg} \emph{Rekursiv versus direkte definition af talf\o{}lger}
\begin{enumerate}
\item F\o{}lgen $\{a_n\}_{n\in\N}$ er givet direkte ved
\[
a_n=(-1)^n.
\]
 Find en rekursiv definition af  $\{a_n\}_{n\in\N}$.
\item \item F\o{}lgen $\{b_n\}_{n\in\N}$ er givet rekursivt ved
\[
b_1=1\qquad b_n=b_{n-1}+n, n>1
\]
 Find en direkte definition af  $\{b_n\}_{n\in\N}$.
\end{enumerate}

\end{opg}

%TIRSDAG ELLER TORSDAG:

\begin{opg} \emph{Newtons metode}
	\begin{enumerate}
		\item Beregn afstanden fra $ \sqrt{2} $ til den $ n $'te approksimation af $ \sqrt{2} $ givet i Eksempel 1.15, for $ n $ mellem $ 1 $ og $ 15 $. Lav et plot der illustrerer, hvor hurtigt denne metode konvergerer.\\
		Forslag: Lav et semilogaritmisk plot. 
		\item Betragt nu istedet følgen $ x_0=2 $, $ x_n=\dfrac{1}{3}\left(\dfrac{2}{x_{n-1}^2}+2x_{n-1}\right) $, for $ n=1,2,3,... $. Beregn $ x_1 $, $ x_2 $,..., $ x_{10} $. Hvilken rod af $ 2 $ ser denne f\o{}lge ud til at approksimere?
	\end{enumerate}
\end{opg}


%TIL TORSDAG:

\begin{opg} \emph{Konvergens}
\begin{enumerate}
    \item Find grænseværdierne for nedenstående talfølger ($n = 1,2,3, \ldots$):
	\begin{multicols}{3}
	\begin{enumerate}[label=\roman*)]
	    \item $\dfrac{n^3 }{2^{-n}+3n^3}$
		\item  $ \dfrac{\cos( n)}{n}$
		\item $3 \sin(1/n) + \dfrac{\pi}{2+ 1/n^4}$
		\item $ \sin(1/n)n $
		\item $ \dfrac{\left((\log(n)\right)^{2021}}{n} $
		\item $ \dfrac{(n!)^2}{(2n)!} $
		\item $\dfrac{n+in^2}{2n-n^2+1}$
		\item $ \e^{i2\pi(n+1/n)} $
	\end{enumerate}
	\end{multicols}
%    \item Lad $ \{a_n\}_{n\in\N} $ være en talfølge og lad $ a\in \C $. Vis at $ a $ er et fortætningspunkt for $ \{a_n\}_{n\in\N} $ hvis og kun hvis der gælder følgende:
 %   $$
  %  \forall\epsilon>0\forall N\in\N \exists n>N : \abs{a_n-a}<\epsilon
   % $$
\end{enumerate}
\end{opg}

\begin{opg} \emph{Trekantsuligheden}
\begin{enumerate}
%    \item Vis, ved brug af trekantsuligheden, at en talfølge $\seq{a}$ højst kan konvergere mod ét tal $a$ (d.v.s. vis at hvis $\seq{a}$ konvergerer mod $a$ og mod $a'$, da er $a=a'$).
%
%    \item Vis, at der for vilkårlige komplekse tal $z_1, \dots, z_n$ gælder, at
%	$\abs{\sum_{i=1}^n z_i } \leq  \sum_{i=1}^n \abs{z_i}$.
%	
	\item Vis, at der for alle komplekse tal $z$ og $w$ gælder
	$\abs{z-w}\geq\abs{|z|-|w|}$. Begrund herudfra, at hvis $\seq{z}$ er en kompleks følge, der konvergerer mod $z$, så vil følgen $\{\abs{z_n}\}_{n \in \mN}$ konvergere mod $\abs{z}$.
\end{enumerate}
\end{opg}

\begin{opg} \emph{Forståelse af konvergens, divergens og fortætningspunkter}
\begin{enumerate}
    \item Lad $a_n= \frac {1}{n^2}$. Følgen $\seq{a}$ konvergerer. Angiv dens grænseværdi $a$, og find for hvert $\varepsilon>0$, et $N\in\mN$ sådan at der for alle $n \geq N$
	gælder $\abs{a_n-a} < \varepsilon$ (med andre ord: giv et eksplicit bevis for konvergensen mod $a$).
	
	\item Lad $b_n = n^2$. Find for hvert $K>0$ et $N\in\mN$, sådan at der
	for alle $n>N$ gælder $ b_n >K$ (med andre ord: giv et eksplicit bevis for, at $\seq{b}$ divergerer mod $+\infty$).
	
	\item Lad $c_n = (-1)^n + \frac{1}{2^n}$. Angiv alle fortætningspunkterne for følgen $\seq{c}$.  Vis, for hvert fortætningspunkt $\tilde{c}$, at der for hvert $\varepsilon>0$ findes uendeligt mange $n\in\mN$ med $\abs{c_n-\tilde{c}} <\varepsilon$ (med andre ord: giv et eksplicit bevis for, at $\tilde{c}$ er fortætningspunkt). Vis tilsvarende eksplicit, at ingen andre punkter er fortætningspunkter for følgen.
\end{enumerate}
\end{opg}



\begin{opg} \emph{Mere om divergens og fortætningspunkter}
\begin{enumerate}
    \item Lad $\seq{a}$ og $\seq{b}$ være komplekse talfølger givet ved
    $$ a_n = \begin{cases} 0 & n\text{ lige} \\
            1 & n\text{ ulige} \end{cases}, \QUAD
        b_n=e^{\frac{2\pi n i}{3}}. $$
    Bestem alle fortætningspunkter for følgerne. Hvorfor kan du med det samme konkludere, at begge følger er divergente?
    
    
    \item Lad $\seq{a}$ være en følge som divergerer mod $\infty$, og lad $\seq{b}$ være en følge som opfylder $b_n\geq a_n$ for alle $n\in\mN$. 

    \begin{enumerate}[label=\roman*)]
	\item Vis, at $\seq{b}$ divergerer mod $\infty$.
	\item Vis, at ingen af følgerne har nogen fortætningspunkter.
	\end{enumerate}
    Er det sandt, at enhver reel talfølge, der ikke har nogen fortætningspunkter, enten divergerer mod $\infty$ eller mod $-\infty$?
    
     \item Find en talfølge, for hvilken m\ae{}ngden af  fortætningspunkter består af alle de naturlige tal, $ \N $.
   
\end{enumerate}
\end{opg}

\begin{opg} \emph{Bernoullis ulighed}\\
Betragt nedenstående følger
$$ \seq{a}, \quad\text{hvor } a_n = \pare{1+\frac{1}{n}}^n, \QUAD
    \seq{b}, \quad\text{hvor } b_n = \pare{1+\frac{1}{n}}^{n+1} $$
\begin{enumerate}
    \item  Lad $x\in\mR$ og $n\in\mN$. Vis at for $x\geq -1$ haves
    $$ \pare{1+x}^n \geq 1+nx $$
    \item Vis, at $\seq{a}$ og $\seq{b}$ begge er begrænsede og monotone. Har de to følger samme grænse?\vspace{0.1cm}\\
    \emph{\textbf{Vink}: Lad $ n\in\N $ være vilkårlig. Betragt så $ (a_{n+1}-a_n) $ og $ (b_n-b_{n+1}) $. Benyt i begge tilfælde Bernoullis ulighed fra 1.9.a) til at konkludere, at følgerne hver især er enten voksende eller aftagende.}
    \item Bestem Eulers tal, $ \e $, ned til fjerde decimal.
\end{enumerate}
\end{opg}