\begin{itemize}
    \item Tirsdag vil vi vise Bessels ulighed, som sammen med Pythagoras' og Parsevals Lemma kredser om spørgsmålet
    $$ \norm{f}_2^2 \overset{?}{\sim} \sum_{k=-\infty}^\infty \inProd{f,e_k}. $$
    Dette spørgsmål udspringer af det velkendte resultat $\norm{v}_2^2 = \sum_{n=1}^d \inProd{v,a_n}$, hvor $\cbrac{e_n}_{n=1}^d$ udgør en ortonormal basis for et $d$-dimensionalt vektorrum. Med henblik på at styrke vores intuition om trigonometriske rækker skal vi se en række eksempler samt undersøge trigonometriske rækker i konteksten af det fysiske fænomen kendt som usikkerhedsprincippet. (MC 5.5)
    
    Slutteligt skal vi forbinde vores studie af Fourierrækker til vores kendte begreber, og tirsdag starter vi med at undersøge punktvis konvergens. (MC 5.6)
    
    Til øvelsesmøderne taler vi om Opgave 1-3.
    
    \item Torsdag færdiggør vi studiet af Fourierrækker ved at se nærmere på uniform konvergens. (MC 5.7)
    
    Vi påbegynder desuden introduktionen af metriske rum, som kan forstås som en generaliseret ramme for mange af de iagttagelser, vi allerede har gjort os. Vi skal således (gen)se mange af de tidligere resultater i en mere generel, abstrakt opsætning. (MC 6.1)
    
    Til øvelsesmøderne taler vi om Opgave 4-6.
    
    \item Ugens svære Opgave 7 diskuteres mest naturligt torsdag.
\end{itemize}

