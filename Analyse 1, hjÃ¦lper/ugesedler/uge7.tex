\begin{itemize}
    \item Tirsdag vil vi vise Bessels ulighed, som sammen med Pythagoras' og Parsevals Lemma kredser om spørgsmålet
    $$ \norm{f}_2^2 \overset{?}{\sim} \sum_{k=-\infty}^\infty \inProd{f,e_k}. $$
    Dette spørgsmål udspringer af det velkendte resultat $\norm{v}_2^2 = \sum_{n=1}^d \inProd{v,a_n}$, hvor $\cbrac{e_n}_{n=1}^d$ udgør en ortonormal basis for et $d$-dimensionalt vektorrum. Med henblik på at styrke vores intuition om trigonometriske rækker skal vi se en række eksempler samt undersøge trigonometriske rækker i konteksten af det fysiske fænomen kendt som usikkerhedsprincippet. (MC 5.5)
    
    Slutteligt skal vi forbinde vores studie af Fourierrækker til vores kendte begreber, og tirsdag starter vi med at undersøge punktvis konvergens. (MC 5.6)
    
    Til øvelsesmøderne taler vi om Opgave 1-3.
    
    \item Torsdag færdiggør vi studiet af Fourierrækker ved at se nærmere på uniform konvergens. (MC 5.7)
    
    Vi påbegynder desuden introduktionen af metriske rum, som kan forstås som en generaliseret ramme for mange af de iagttagelser, vi allerede har gjort os. Vi skal således (gen)se mange af de tidligere resultater i en mere generel, abstrakt opsætning. (MC 6.1)
    
    Til øvelsesmøderne taler vi om Opgave 4-6.
    
    \item Ugens svære Opgave 7 diskuteres mest naturligt torsdag.
    
    \begin{opg}[Fourierrækker I]\hfill \\
\begin{enumerate}
 \item Lad $g: \mathbb{R} \rightarrow \mathbb{R}$ være funktionen givet ved
 $$
 g(x)=e^{-i x}-3 e^{-3 i x}+\frac{1}{2} \sin (x)
 $$
 Bestem Fourierkoefficienterne og Fourierrækken for $g$, både på formen med den komplekse eksponentialfunktion og på formen med sinus og cosinus. Kan du trække nogle analogier til Opgave $6.1 \mathrm{c}) \mathrm{i})$ ?
 
 \item Vi ved fra opgave $6.5 \mathrm{~b}$ ), at den trigonometriske række $\sum_{n=-\infty}^{\infty} 2^{-|n|} e^{i n x}$ er uniformt konvergent. Bestem Fourierkoefficienterne og Fourierrækken for dens sumfunktion $f$\footnote{Opgaven kan faktisk loses uden at kende $f$ eksplicit - du kan evt. finde inspiration på side 125 i MC.}
 \item Betragt den $2 \pi$ -periodiske funktion $f: \mathbb{R} \rightarrow \mathbb{R}$ givet ved
 $$
 f(x)=\left\{\begin{array}{ll}
 0 & x \in(-\pi, 0) \\
 x & x \in[0, \pi) \\
 \pi / 2 & x=\pi
 \end{array}\right.
 $$
 Beregn Fourierkoefficienterne, og bestem Fourierrækken for $f$, både på formen med den komplekse eksponentialfunktion og på formen med sinus og cosinus.
 Kan du sige noget om, hvorvidt Fourierrækken konvergerer uniformt?
 \item Lad $f: \mathbb{R} \rightarrow \mathbb{R}$ være givet ved $f(x)=\sin (x)^{3}$. Vis, at $f$ er $2 \pi$ -periodisk. Bestem Fourierkoefficienterne $c_{k}(f)$ hørende til den komplekse skrivemåde, og angiv den tilsvarende Fourierrække for $f$\footnote{Hint: Det er fordelagtigt først at omskrive $f$ ved hjælp af eksponentialfunktioner.}
\end{enumerate}
    \end{opg}
    
\begin{opg}[Vektorrumsstruktur]\hfill \\
	$\operatorname{Lad} N \in \mathbb{N}_{0}$ og lad $V_{N}$ være mængden af trigonometriske polynomier af $\operatorname{grad} N$, det vil sige mængden af funktioner $f: \mathbb{R} \rightarrow \mathbb{R}$ på formen
	$$
	f(x)=\sum_{k=-N}^{N} c_{k} e^{i k x}
	$$
	$\operatorname{med} c_{-N}, \ldots, c_{N} \in \mathbb{C}$
	\begin{enumerate}
	 \item Overvej følgende punkter uden at forsøge at lave et formelt bevis.
	 \begin{enumerate}[label=\roman*)]
	\item Overvej kort, hvorfor $V_{N}$ er et vektorrum over $\mathbb{C}$ (sammenlign med MC 5.18) udstyret med den naturlige skalarmultiplikation og addition af funktioner. Overvej også, hvorfor
	$$
	V_{N}=\operatorname{span}_{\mathrm{C}}\left\{e_{-N}, \ldots, e_{N}\right\}
	$$
	hvor $e_{k}: \mathbb{R} \rightarrow \mathbb{R}$ som sædvanlig betegner funktionen givet ved $e_{k}(x)=e^{i k x}$ for $k \in \mathbb{Z} .$ Hvad siger dette om dimensionen $\operatorname{dim}_{\mathbb{C}}\left(V_{N}\right)$ af $V_{N} ?$
	\item Argumentér for, at $V_{0}=\operatorname{span}_{C}\{1\}$ og at $V_{1}=\operatorname{span}_{C}\{1, \cos , \sin \}$, hvor 1 betegner den konstante funktion $x \mapsto 1$.
	\item Begrund, at $V_{M}$ er et underrum af $V_{N}$ hvis $M \leq N$, og giv et eksempel på en kontinuert, $2 \pi$ -periodisk funktion $f: \mathbb{R} \rightarrow \mathbb{R}$, som ikke tilhører vektorrummet $V_{N}$ for noget $N \in \mathbb{N}$
	\end{enumerate}
	\item Vis, at afbildningen $\langle\cdot, \cdot\rangle_{V_{N}}: V_{N} \times V_{N} \rightarrow V_{N}$ givet ved
	$$
	\langle f, g\rangle_{V_{N}}=\frac{1}{2 \pi} \int_{-\pi}^{\pi} f(x) \overline{g(x)} \mathrm{d} x
	$$
	definerer et indre produkt (MC side 121 ) på vektorrummet $V_{N}$, det vil sige vis, at ...
	\begin{enumerate}[label=\roman*)]
	\item ... afbildningen $f \mapsto\left\langle f, g_{0}\right\rangle_{V_{N}}$ er lineær for ethvert fastholdt $g_{0} \in V_{N} ;$
	\item ... $\langle g, f\rangle_{V_{N}}=\overline{\langle f, g\rangle}_{V_{N}}$ for alle $f, g \in V_{N} ;$ og
	\item ... $\langle f, f\rangle_{V_{N}} \geq 0$ for alle $f \in V_{N}$ med lighed kun hvis $f=0$.
	\end{enumerate}
	\item Vis, at funktionerne $\left\{e_{k}\right\}_{k=-N, \ldots, N}$ udgor et ortonormalt system i $V_{N}$ med hensyn til det indre produkt $\langle\cdot, \cdot\rangle_{V_{N}} .$
	Konkludér, at systemet er lineært uafhængigt over $\mathbb{C}, \operatorname{og}$ at $\operatorname{dim}_{\mathbb{C}} V_{N}=2 N+1$.
	\item Vis, at $V_{N}$ er isomorft med $\mathbb{C}^{2 N+1}$ som vektorrum med indre produkt, det vil sige find en lineær bijektiv afbildning $\Pi_{N}: V_{N} \rightarrow \mathbb{C}^{2 N+1}$, sådan at
	$$
	\langle f, g\rangle_{N}=\left\langle\Pi_{N}(f), \Pi_{N}(g)\right\rangle_{\mathbb{C}^{2 N+1}} \quad \text { for alle } f, g \in V_{N}
	$$
	hvor $\langle x, y\rangle_{\mathbb{C}^{2 N+1}}=\sum_{i=0}^{2 N} x_{i} \overline{y_{i}}$ for $x, y \in \mathbb{C}^{2 N+1}$.
\end{enumerate}
\end{opg}

\begin{opg}[Fourierrækker II]\hfill \\
	\begin{enumerate}
		\item Betragt funktionerne $F_{\text {fløjte }}: \mathbb{R} \rightarrow \mathbb{R} \operatorname{og} \mathcal{F}_{N}: \mathbb{R} \rightarrow \mathbb{R}, N \in \mathbb{N}$, givet ved
		$$
		\begin{aligned}
		F_{\text {fløjte }}(x) &=\sin (x)+9 \sin (2 x)+\frac{15}{4} \sin (3 x)+\frac{9}{5} \sin (4 x), \\
		\mathcal{F}_{N}(x) &=\frac{1}{N} \sum_{n=0}^{N-1} D_{n}(x)
		\end{aligned}
		$$
		hvor $D_{n}$ som tidligere betegner Dirichlet-kernen. Bestem Fourierkoefficienterne og Fourierrækken for $F_{\text {fløjte }}$ og $ \mathcal{F}_{N}$
		Fun fact: $F_{\text {fløjte }}$ beskriver kammertonen (A440) for en fløjte, og $\mathcal{F}_{N}$ kaldes Fejér-kernen.
		
		\item Beregn Fourierkoefficienterne for de $2 \pi$ -periodiske funktioner $f$ og $F$ defineret ved
		$$
		f(x)=\begin{cases}
		-\frac{\pi}{2}-\frac{x}{2}, & x \in[-\pi, 0) \\
		0, & x=0 \\
		\frac{\pi}{2}-\frac{x}{2} & x \in(0, \pi)
		\end{cases}, \quad F(x)=\begin{cases}
		\frac{\pi^{2}}{12}-\frac{1}{4}(x+\pi)^{2}, & x \in[-\pi, 0) \\
		\frac{\pi^{2}}{12}-\frac{1}{4}(x-\pi)^{2}, & x \in[0, \pi)
		\end{cases}
		$$
		Vis, at Fourierrækkerne for $f$ og $F$ er lig de trigonometriske rækker fra Opgave $6.4$. Diskutér med din underviser, hvad det kan bruges til.
	\end{enumerate}
\end{opg}

\begin{opg}[Pythagoras, Bessel og Parseval]
	
\end{opg}
\end{itemize}

