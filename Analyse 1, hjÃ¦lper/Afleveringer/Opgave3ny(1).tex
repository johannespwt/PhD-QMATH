\setcounter{section}{3}

\begin{opg}\hfill \\
	\begin{enumerate}
		\item Find konvergensradius, konvergensomr\aa{}de  og sumfunktion for potensr\ae{}kken $\displaystyle \sum_{n=0}^{\infty}(n+1/2)z^n ,  z\in\C $. %Find konvergensradius og sumfunktionen til denne række.
			\ifanswers
			\begin{proof}[Løsning]
%				Konvergensradius findes ved Sætning 4.8 til at være $ r=\left(\lim\limits_{n\to\infty}\left(\frac{n+3/2}{n+1/2}\right)\right)^{-1}=1 $. Vi noterer, at rækken åbenlyst er divergent hvis $ \abs{z}=1 $ per divergenstesten, hvorfor vi finder sumfunktionen defineret på $ B_1 $. Bemærk at $ \sum_{n=0}^{\infty}(-1/2)z^n $ har konvergensradius $ 1 $, da det er den sædvanlige (generaliserede) geometriske række, og $ \sum_{n=0}^{\infty}(n+1)z^n $ har konvergensradius $ 1 $ per Sætning 4.22 (7). Dermed gælder ved Sætning 4.18 (5), at $ \sum_{n=0}^{\infty}(n+1/2)z^n=\sum_{n=0}^{\infty}(n+1-1/2)z^n= $ har sumfunktion $ \sum_{n=0}^{\infty}(n+1)z^n+\sum_{n=0}^{\infty}(-1/2)z^n  $. Der gælder yderligere per eksempel 4.17 og Sætning 4.15(1), at $ \sum_{n=0}^{\infty}(-1/2)z^n=-\frac{1}{2}\frac{1}{1-z} $ for $z\in B_1 $, og per Eksempel 4.19 gælder det, at $ \sum_{n=0}^{\infty}(n+1)z^{n}=\frac{1}{(1-z)^2} $ for $ z\in B_1 $. Dermed findes sumfunktionen $ \sum_{n=0}^{\infty}(n+1/2)z^n =\frac{1}{(1-z)^2}-\frac{1}{2}\frac{1}{1-z}=\frac{1}{2}\frac{1+z}{(1-z)^2} $ for $ z\in B_1 $.
			Ved eksempel 4.19 ved vi, at $ \sum_{n=0}^{\infty}(n+1)z^n=\frac{1}{(1-z)^2} $ for $ z\in B_1 $ og ved Eksempel 3.23 og Sætning 4.15(2) gælder det, at $ \sum_{n=0}^{\infty}z^n=\frac{1}{1-z} $ for $ z\in B_1 $. Ved 4.15(1) fåes $ \sum_{n=0}^{\infty}(-\frac{1}{2})z^n=-\frac{1}{2(1-z)} $, og ved 4.18(5) fåes $ \sum_{n=0}^{\infty}(n+1/2)z^n=\sum_{n=0}^{\infty}(n+1)z^n+\sum_{n=0}^{\infty}(-\frac{1}{2})z^n=\frac{1}{(1-z)^2}-\frac{1}{2}\frac{1}{1-z}=\frac{1}{2}\frac{1+z}{(1-z)^2}  $ for $ z\in B_1 $. Det ses let ved divergenstesten at rækken er divergent for $ \abs{z}=1 $, hvorfor det konkluderes at konvergensradius er $ 1 $ og konvergensområdet er $ B_1 $. 
				\end{proof}
			\fi

		\item Find konvergensradius,  konvergensomr\aa{}de  og sumfunktion for potensrækken $\displaystyle \sum_{n=0}^{\infty}\frac{(-1)^nx^{2n}}{(2n+2)!} ,  x\in\R $. 
		\ifanswers
		\begin{proof}[Løsning]
		Det vides fra Eksempel 4.32, at $ \cos(x)=\sum_{n=0}^{\infty}\frac{(-1)^nx^{2n}}{(2n)!} $ for alle $ x\in \R $. Ækvivalent gælder det, at $ 1+\sum_{n=1}^{\infty}\frac{(-1)^nx^{2n}}{(2n)!}=\cos(x)  $ eller at $ \sum_{n=1}^{\infty}\frac{(-1)^nx^{2n}}{(2n)!}=\cos(x)-1 $. Ved brug af 4.15(1) kan vi skrive dette som $ -x^{2}\sum_{n=1}^{\infty}\frac{(-1)^{n-1}x^{2n-2}}{(2n)!}=\cos(x)-1 $ eller med $ m=n-1 $ har vi $ -x^{2}\sum_{m=0}^{\infty}\frac{(-1)^{m}x^{2m}}{(2m+2)!}=\cos(x)-1 $. 
		Alle manipulationer bevarer konvergensradius $ r=\infty $, og vi konkluderer for alle $ x\neq 0 $, at $ s(x)=\sum_{n=0}^{\infty}\frac{(-1)^{n}x^{2n}}{(2n+2)!}=\frac{1-\cos(x)}{x^2} $. For $ x=0 $ findes $ s(0)=a_0=\frac{1}{2!}=\frac{1}{2} $. Dermed er konvergensradius $ r=\infty $ og konvergensområdet hele $ \R $.
		\end{proof}
		\fi
		
		\item Find konvergensradius for potensrækken $\displaystyle \sum_{n=1}^{\infty}\frac{n^n}{n!}x^n, x\in\R $.
		\ifanswers
		\begin{proof}[Løsning]
			Bemærk at $ \frac{(n+1)^{n+1}}{(n+1)!}\Big/ \frac{n^n}{n!}=\frac{(n+1)^n}{n^n}=\left(1+\frac{1}{n}\right)^n $. Det fra Eksempel 1.45 at $ \lim\limits_{n\to\infty}\left(\frac{(n+1)^{n+1}}{(n+1)!}\Big/ \frac{n^n}{n!}\right)=e $.
			Det ses nu let, at alle antagelser for sætning 4.8 er opfyldt, og det følger derfor, at konvergensradius er $ r=\left(\lim\limits_{n\to\infty}\left(\frac{(n+1)^{n+1}}{(n+1)!}\Big/ \frac{n^n}{n!}\right)\right)^{-1}=e^{-1}  $.
		\end{proof}
		\fi
	\end{enumerate}
\end{opg}


\begin{opg}\hfill \\
		\begin{enumerate}
			\item Find Taylorrækken for funktionen $$ f(x)=2^x,\qquad   x\in\R $$ 
			og vis, at den konvergerer mod $ f(x) $ for alle $ x\in \R $.
			\ifanswers
			\begin{proof}[Løsning]
				Det ses, at $ f(x)=e^{\log(2)x} $, og det følger, at $ f^{(n)}(x)=\left(\log(2)\right)^nf(x) $. Dermed ses at Taylorrækken for $ f $ er $ \sum_{n=0}^{\infty}\frac{\left(\log(2)\right)^nx^n}{n!} $. Ved brug af Sætning 4.15 (3) med $ c=\log(2) $ samt Eksempel 4.10, konkluderes at konvergensradius er $ r=\infty $, samt at $ \sum_{n=0}^{\infty}\frac{\left(\log(2)\right)^nx^n}{n!}=\exp(\log(2)x)=2^x $ som er det ønskede resultat.
				
				
%				Det observeres at der gælder $ \norm{f^{(n)}}_{(-x,x)}=\left(\log(2)\right)^n2^x $, og dermed gælder at $ \left\{\frac{\norm{f^{(n)}}_{(-x,x)}}{n!}\abs{x}^n\right\}_{n\in\N}=\left\{\frac{\left(\log(2)\right)^n}{n!}2^x\abs{x}^n\right\}_{n\in\N} $ konvergerer mod nul for alle $ x\in\R $. Derfor gælder det per Lemma 4.31 at $ \sum_{n=0}^{\infty}\frac{\left(\log(2)\right)^nx^n}{n!}=f(x) $ for alle $ x\in\R $.
			\end{proof}
			\fi
				\item Find Taylorrækken for funktionen $$ g(x)=\int_{0}^{x}e^{t^3}\diff t,\qquad   x\in\R $$ 
				og vis, at den konvergerer mod $ g(x) $ for alle $ x\in \R $.
				\ifanswers
				\begin{proof}[Løsning]
%				Det gælder at $ f'(x)=e^{x^3} $. Det vides derfor, at $ f'(x)=\sum_{n=0}^{\infty}\frac{1}{n!}x^{3n} $, for alle $ x\in\R $. Det følger af Sætning 4.22 (8) at $ \int_{0}^{x}e^{t^3}\diff t=\sum_{n=0}^{\infty}\frac{1}{(3n+1)n!}x^{3n+1} $ for alle $ x\in \R $, hvilket dermed per Sætning 4.28 udgør Taylorrækken.
				Det vides at $ e^x=\sum_{n=0}^{\infty}\frac{x^n}{n!} $ for alle $ x\in\R $. Så ved 4.15(4) ses det at $ e^{x^3}=\sum_{n=0}^{\infty}\frac{x^{3n}}{n!} $ for alle $ x\in\R $ og ved 4.22(8) ses da, at $ \int_{0}^{x}e^{t^3}\diff t=\sum_{n=0}^{\infty}\frac{x^{3n+1}}{(3n+1)n!} $ for alle $ x\in\R $. Per Sætning 4.28 konkluderes det at $ \sum_{n=0}^{\infty}\frac{x^{3n+1}}{(3n+1)n!} $ udgør Taylorrækken for $ g $.
				\end{proof}
				\fi
		\end{enumerate}
	\end{opg}


\begin{opg}\hfill\\
	Vi ser på potensrækken $ \sum_{n=0}^{\infty}a_n x^n, x\in \R $, med $ a_0=0 $ og $$ a_n=\sum_{k=1}^{n}\frac1k \qquad\text{ for }\qquad n\geq1 $$. Lad $ r $ betegne rækkens konvergensradius og $ s $ sumfunktionen $ s: (-r,r)\to \C $
\begin{enumerate}
    \item Vis at rækken divergerer for $ x=1 $, men at den konvergerer for $ x=\frac{1}{2} $. Slut, at $ \frac{1}{2}\leq r\leq1 $.
    \ifanswers
    \begin{proof}[Løsning]
    	For $ x=1 $ har vi $ \sum_{n=0}^{\infty}a_n x^n=\sum_{n=1}^{\infty}\sum_{k=1}^{n}\frac{1}{k} $. Da $ \sum_{k=1}^{n}\frac{1}{k}\geq \frac1n $, og det vides at den harmoniske række er divergent, følger det af sammenligningstesten, at $ \sum_{n=0}^{\infty}a_n x^n $ er divergent for $ x=1 $. For $ x=1/2 $ benyttes at $ a_n=\sum_{k=1}^{n}\frac{1}{k}\leq \sum_{k=1}^{n}1=n $. Og da det vides fra Eksempel 4.19, at $ \sum_{n=1}^{\infty}\frac{n}{2^n} $ er konvergent, konkluderes ved sammenligningstesten (Korollar 2.17), at $ \sum_{n=0}^{\infty}a_n \left(\frac{1}{2}\right)^n $ konvergerer. Det følger af Sætning 4.3 (1) og (2), at $  1/2\leq r\leq 1 $.
    	
%    	For $ x=1/2 $ bemærkes at rækken $ \sum_{n=0}^{\infty}a_n x^n=\sum_{n=1}^{\infty}a_n\left(\frac{1}{2}\right)^n $ er positiv. Da det gælder at $ 1<a_{n+1}/a_n\leq1+\frac{1}{n+1} $ følger det af klemmelemmaet, at $ \lim\limits_{n\to\infty}(a_{n+1}/a_n)=1 $. Derfor gælder det, at\\ $ \lim\limits_{n\to\infty}\left((a_{n+1}(1/2)^{n+1})/(a_n(1/2)^n)\right)=1/2 $, og det følger af kvotienttesten, at $ \sum_{n=0}^{\infty}a_n x^n=\sum_{n=1}^{\infty}a_n\left(\frac{1}{2}\right)^n $ er konvergent. Det følger du direkte af Sætning 4.3(1)  $ r\leq 1 $ og af Sætning 4.3(2) at $ \frac12\leq r $.
    \end{proof}
    \fi
    \item Bestem Taylorr\ae{}kken for $ (1-x)s(x) $, og redegør for, at den også har konvergensradius $ r $.
    \ifanswers
    \begin{proof}[Løsning]
    	Betragt potensrækken $ \sum_{n=1}^{\infty}(a_n-a_{n-1})x^n $. Det er klart ved Sætning 4.15(3) og Sætning 4.18(5) at $ \sum_{n=1}^{\infty}(a_n-a_{n-1})x^n $ har konvergensradius mindst $ r $ og at der gælder $ \sum_{n=1}^{\infty}(a_n-a_{n-1})x^n=\sum_{n=1}^{\infty}a_nx^n-x\sum_{n=0}^{\infty}a_nx^n=(1-x)\sum_{n=0}^{\infty}a_nx^n=(1-x)s(x) $ for $ \abs{x}< r $. Det følger af Sætning 4.28, at dette er Taylorrækken for $ (1-x)s(x) $. På den anden side, gælder der, at $ a_n=\sum_{k=1}^{n}a_k-a_{k-1}=\sum_{k=1}^{n}(a_k-a_{k-1})1 $. Bemærk nu, at det vides at $ \sum_{n=1}^{\infty}x^n $ har konvergensradius $ 1 $. Lad da $ r' $ betegne konvergensradius af $ \sum_{n=1}^{\infty}(a_n-a_{n-1})x^n $, da gælder ved Sætning 4.18(6) (Cauchy-multiplikation), at $ \sum_{n=0}^{\infty}a_nx^n $ har konvergensradius mindst $ \min(1,r') $. Vi ser fra resultatet i c) nedenfor, at $ r'=1 $, hvoraf det følger at $ r\geq 1=r' $, således at $ r=r' $.
%    	Bemærk, at det er unaturligt at vise at konvergensradius er størst $ r $, da det ikke holder som et abstrakt resultat. Det følger dog i dette tilfælde af opgave c) nedenunder.
    \end{proof}
    \fi 
    \item Vis, at $ r=1 $ og bestem et lukket udtryk for $ s:(-1,1)\to \C $.
    \ifanswers
    \begin{proof}[Løsning]
     Rækken $ \sum_{n=1}^{\infty}\frac{x^n}{n} $ genkendes fra eksempel 4.24, og det vides, at $ r'=1 $, således at $ r=r'=1 $ samt at $ \sum_{n=1}^{\infty}\frac{x^n}{n}=-\log(1-x) $, hvoraf det ses at $ (1-x)s(x)=-\log(1-x) $. Altså har vi $ s(x)=\frac{\log(1-x)}{x-1} $, for $ x\in(-1,1) $.
    \end{proof}
    \fi
\end{enumerate}
\end{opg}

	
		\begin{opg}\hfill \\
		En funktion $ f:(-r,r)\to \R $ siges at være lige hvis $ f(x)=f(-x) $. Lad $ \sum_{n=0}^{\infty}a_nx^n $ være en potensrække med konvergensradius $ r>0 $ og sumfunktion $ s:(-r,r)\to \R $
		\begin{enumerate}
			\item Vis at hvis $ a_{2n+1}=0 $ for alle $ n\geq 0 $ så er $ s $ lige.
			\ifanswers
			\begin{proof}[Løsning]
				Der gælder åbenlyst at $$ s(-x)=\sum_{n=0}^{\infty}a_n(-x)^n=\sum_{n=0}^{\infty}a_{2n}(-x)^{2n}=\sum_{n=0}^{\infty}a_{2n}x^{2n}=\sum_{n=0}^{\infty}a_nx^n=s(x), $$
				hvoraf det følger, at $ s $ er lige.
			\end{proof}
			\fi
			\item Vis at potensrækken $$ \sum_{n=0}^{\infty}a_n(1-(-1)^n)x^n $$ har sum $ s(x)-s(-x) $. Brug entydighedssætningen for potensr\ae{}kker til at konkludere, at hvis $ s $ er lige, så er $ a_{2n+1}=0 $ for alle $ n\geq 0 $.
			\ifanswers
			\begin{proof}[Løsning]
				Det ses ved Sætning 4.18(5) at $  \sum_{n=0}^{\infty}a_n(1-(-1)^n)x^n= \sum_{n=0}^{\infty}(a_nx^n-a_n(-x)^n)=s(x)-s(-x) $ for $ x\in(-r,r) $. Hvis $ s $ er lige ses dermed, at $ \sum_{n=0}^{\infty}a_n(1-(-1)^n)x^n=2\sum_{n=0}^{\infty}a_{2n+1}x^{2n+1}=0 $ for $ x\in(-r,r) $. Det følger entydighedssætningen (Sætning 4.35) at $ a_{2n+1}=0 $ for alle $ n\geq 0 $
			\end{proof}
			\fi
			%			\item Vis at $ s $ er ulige hvis og kun hvis $ a_{2n}=0 $ for alle $ n\geq 0 $.
%			\ifanswers
%			\begin{proof}[Løsning]
%				Det er klart hvis $ a_{2n}=0 $, at $$ s(-x)=\sum_{n=0}^{\infty}a_n(-x)^n=\sum_{n=0}^{\infty}a_{2n+1}(-x)^{2n+1}=-\sum_{n=0}^{\infty}a_{2n+1}x^{2n+1}=-\sum_{n=0}^{\infty}a_nx^n=-s(x), $$
%				og at $ s $ dermed er ulige. Modsat gælder der, hvis $ s $ er ulige, at $ \sum_{n=0}^{\infty}a_n(1+(-1)^n)x^n=\sum_{n=0}^{\infty}(a_nx^n+a_n(-x)^n)=s(x)+s(-x)=0 $. Heraf konkluderes, at $ \sum_{n=0}^{\infty}a_n(1+(-1)^n)x^n=2\sum_{n=0}^{\infty}a_{2n}x^{2n}=0 $, og det følger af entydighedssætningen (Sætning...), at $ a_{2n}=0 $ for alle $ n\geq 0 $
%			\end{proof}
%			\fi
		\end{enumerate}
		
	\end{opg}