\documentclass{article}

\usepackage{ANpreamble}
%\usepackage{enumerate}

\newif\ifanswers

\answerstrue %udkommenter denne linje for at skjule løsninger


\title{Analyse 1 2020/2021 - Rettevejledning til Hjemmeopgave 2}
\author{Johannes Agerskov}
\date{\today}
\setcounter{section}{2}
\begin{document}
	
	\maketitle
Følgende rettevejledning er, som det fremgår af navnet, \emph{vejledende}. Det er derfor op til den enkelte instruktor at afgøre i hvilken grad, de angivne forslag i denne vejledning er passende til den enkelte besvarelse. Vejledningen er på ingen måde dækkende over alle typer af besvarelser, og ofte vil der i retteprocessen også indgå et helhedsintryk når der gives point. Det er derimod vejledningens formål, at give et overordnet billede af, hvor hårdt der slåes ned på nogle udvalgte (måske typiske) fejl eller mangler der kan optræde i besvarelserne. Det er forventeligt, at fejl og mangler, som er gennemgået i denne vejledning, vil optræde i forskellige grader eller variationer i besvarelserne, og i så fald, må det vurderes hvorvidt der skal trækkes færre eller flere point i disse tilfælde. Det er håbet, at fejl, der ikke er gennemgået i denne vejledning, kan relateres til eller sammenlignes med fejl gennemgået i vejledningen, og at pointgivningen for disse besvarelse dermed kan ekstra-/interpoleres fra vejledningen.
\vspace*{0.5cm}\\
Som beskrevet i "Regler og vejledning for aflevering a Hjemmeopgave 2" på sidste side i opgavesættet, lægges der vægt på klar og og præcis formulering. Det er dermed essentielt, at opgaverne er letlæselige, og at det er klart, hvad den studerende mener. For at der kan gives fuld point for en opgave, skal argumentet være fuldkomment, og der bør ikke være brug for at færdiggøre argumenterne i hovedet, for at forstå dem.
\vspace*{0.2cm}
\\
Hvis der skulle være brug for let tolkning af en forklaring, dvs. at det er tydeligt, hvad der er menes, men formuleringen er delvis utilstrækkelig, fratrækkes der som udgangs punkt få point, altså $ 1 $ eller $ 2 $ point afhængig af graden af utilstrækkelighed.
\vspace*{0.2cm}\\
Hvis der er brug for større tolkning af en forklaring, eller brug for færdiggørelse af argumentet for at verificere dets korrekthed, anses denne del af opgaven for værende ikke løst, og der kan trækkes point svarende til den del af opgaven.
\vspace*{0.2cm}\\
Hvis der er givet et argument et forkert sted i besvarelsen, f.eks. arguementet står i 2.a) men er først nødvendigt (og brugbart) i 2.b, da fratrækkes 1 point, med mindre selvfølglig, at der i opgave 2.b henvises til det tidligere argument.
\vspace*{0.2cm}\\
Vi retter efter et oppefra og ned princip, hvilket vil sige, at enhver besvarelse som udgangspunkt har 100/100 point. Der fratrækkes så point for fejl, mangler og andre utilstrækkeligheder i besvarelsen.
\vspace*{0.2cm}\\
I følgende tilfælde trækkes der \textbf{ikke} point:
\begin{itemize}
	\item Der trækkes ikke point for ligegyldig tekst.
	\item Der trækkes ikke point for et fejlagtigt argument, \textbf{hvis} der efterfølgende korrigeres ved at give et korrekt argument for samme resultat. 
	\item Hvis der gives to besvarelser for en delopgave tildeles gennemsnittet af pointene for de to besvarelser.
	\item Der fratrækkes ikke point for følgefejl, altså fejl som skyldes fejl i tidligere opgaver.
\end{itemize}

En forløbig rettevejledning for Hjemmeopgave 2 ser således ud:

\begin{opg}[30 \emph{point}]\hfill
	\begin{enumerate}
		\item (10 \emph{point})\begin{enumerate}[label=(\roman*)]
			\item Glemmer at kigge på absolutværdien når der udføres konvergenstest ($ -5 $ point.)
		\end{enumerate}
		\item (10 \emph{point}) \begin{enumerate}[label=(\roman*)]
			\item Bemærker ikke, at rækken kun er alternerende fra $ n=3 $. ($ -1 $ point)
			\item Argumenterer ikke for / nævner ikke, at rækken er alternerende fra $ n=3 $ ($ -3 $ point)
			\item Argumenterer ikke for / nævner ikke, at absolutværdien af ledne er monoton aftagende fra $ n=3 $ før brug af Leibniz' test ($ -4 $ point)
			\item Viser ikke at rækken \emph{ikke} er absolut konvergent, for at kunne konkludere betinget konvergens ($ -4 $ point)
		\end{enumerate}
		\item (10 \emph{point})\begin{enumerate}[label=(\roman*)]
			\item Bemærk, at nogle kunne falde i fælden, at man kunne tro, at Leibniz' test virker på denne række.
			\item Splitter rækken op i to rækker (med lighedstegn), hvor den ene er divergent ($ -5 $ point)
		\end{enumerate}
	\end{enumerate}	
\end{opg}
\begin{opg}[20 \emph{point}]\hfill
	\begin{enumerate}
		\item (10 $ \emph{point} $)
		\begin{enumerate}[label=(\roman*)]
			\item Indser ikke at $ b=-p $ i forhold til $ p $-rækken i eksempel 2.23 og konkluderer konvergens for $ b<1 $ ($ -5 $ point)
		\end{enumerate}
		\item (10 \emph{point}) 
		\begin{enumerate}[label=(\roman*)]
			\item Argumenterer ikke for divergens hvis $ a>0 $ ($ -4 $ point)
			\item Nævner ikke $ a=0 $ tilfælde ($ -1 $ point, da de har vist det i a) )
		\end{enumerate}

	\end{enumerate}
\end{opg}
\begin{opg}[10 \emph{point}]\hfill
	\begin{enumerate}
			\item (10 \emph{point}) Der kunne være mange kreative gode eller mindre gode løsninger til denne opgave, så denne vejledning tager udgangspunkt i at den studerende har fuldt samme strategi som den vejledende besvarelse. Desuden, er opgaven lidt mere regnetung, og så jeg synes ikke vi skal slå for hårdt ned på småting, som at glemme at nævne Riemann integrabilitet, før de regner et integrale osv.
			\begin{enumerate}[label=(\roman*)]
				\item Glemmer at argumentere for konvergens før der laves manipulationer der kun er lovlige for konvergente rækker. ($ -3 $ point).
				\item Glemmer at vurdere rækken større end $ 1 $. ($ -2 $ point)
				\item Halen af rækken skal vurderes oppefra af en positiv række med aftagende led, for at kunne vurdere denne med et integral. Hvis dette ikke er gjort korrekt ($ -3 $ point)
				\item Når halen af rækken vurderes oppefra med et integral har man glemt at lægge det første led af halen til ($ -2 $ point).
				\item Hvis besvarelsen har form: (1) Rækkens hale vurderes mod en anden række. (2) Den anden række er tastet ind i maple og giver...(uden teoretisk underbyggelse) (3) konklusion. ($ -5 $ point)
			\end{enumerate}
	\end{enumerate}
\end{opg}

\begin{opg}[40 \emph{point}]\hfill
	\begin{enumerate}
		\item (10 $ \emph{point} $)
		\begin{enumerate}[label=(\roman*)]
			\item Vurderer ikke rækken \emph{ledvist} ved Weierstrass' M-test ($ -3 $ point)
		\end{enumerate}
		\item (10 \emph{point}) Bemærk i denne opgave, at det relevante resultat i noterne kun er beskrevet for funktionsfølger, og ikke for rækker. Hvis der henvises til dette (Sætning ...) bør der uddybes, at en funktionsrække, blot er en funktionsfølge.
		\begin{enumerate}[label=(\roman*)] 
			\item Nævner ikke, at alle led er kontinuert differentiable på $ [-a,a] $. ($ -2 $ point).
			\item Viser ikke uniform konvergens af den ledvist afledte række ($ -3 $ point)
			\item Henviser ikke til opgave a) for at vise konvergens i mindst ét punkt. ($ -1 $ point svarende til at argumentet står i forkert delopgave)
			\item Glemmer at argumentere for $ x^2<1 $, når sumfunktionen af den ledvist afledte række findes. ($ -2 $ point) 
		\end{enumerate}
		\item (10 \emph{point})
		\begin{enumerate}[label=(\roman*)]
			\item Konkluderer direkte at de to funktioner er ens, fordi de har samme afledte ($ -4 $ point)
		\end{enumerate}
		\item (10 \emph{point})\hfill\\
		Maple siger... (giver ingen point). Derudover, hvis de ikke genkender rækken som værende $ f $ og henviser til tidligere delopgaver:
		\begin{enumerate}[label=(\roman*)]
			\item Bruger resultater for ledvis integration eller differentiation uden at tjekke antagelser for resultaterne ($ -3 $ point)
			\item Finder sum af geometrisk række uden at tjekke om betingelser for konvergens er opfyldt.
		\end{enumerate}
		
	\end{enumerate}
\end{opg}
\end{document}