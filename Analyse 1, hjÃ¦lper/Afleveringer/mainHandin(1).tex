\documentclass{article}

\usepackage{ANpreamble}
%\usepackage{enumerate}

\newif\ifanswers
\newif\ifrules

\answerstrue 	%udkommenter denne linje for at skjule løsninger
%\rulestrue 	%udkommenter denne linje for at skjule regler

\title{Analyse 1 2020/2021 - Hjemmeopgave 2}
\author{}
\date{\vspace{-1cm}Afleveres senest kl 13:00 på Absalon, 27. maj 2021}

\begin{document}
	
	\maketitle
	
	\noindent
	
	\setcounter{section}{2}
\begin{opg}\hfill
\begin{enumerate}
    \item Afgør om følgende række er absolut konvergent, betinget konvergent eller divergent.
    $$ \sum_{n=1}^\infty \frac{(-1)^n n! }{(2n)!}. $$
    \ifanswers
    \begin{proof}[Løsning]
    	Bemærk at $ \abs{\frac{(-1)^n n! }{(2n)!}}=\frac{ n! }{(2n)!} $ og der gælder \begin{equation*}
    	\lim\limits_{n\to\infty}\left(\frac{ (n+1)! }{(2(n+1))!}\Bigg/\frac{ n! }{(2n)!}\right)=\lim\limits_{n\to\infty}\left(\frac{n+1}{(2n+1)(2n+2)}\right)=0,
    	\end{equation*}
    	hvor det følger af sidste lighedstegn, at grænsen eksisterer. Det konkluderes dermed ved kvotienttesten (Sætning 2.24), at $ \sum_{n=1}^\infty \frac{ n! }{(2n)!}<\infty $ og rækken er absolut konvergent.
    \end{proof}
    \fi
    
    \item Afgør om følgende række er absolut konvergent, betinget konvergent eller divergent.
    $$ \sum_{n=2}^\infty
        \frac{\cos\pare{\pi n}}{\log\pare{\log\pare{n}}}.  $$
        \ifanswers
        \begin{proof}[Løsning]
        	Bemærk at rækken $ \sum_{n=3}^{\infty}\frac{\cos\pare{\pi n}}{\log\pare{\log\pare{n}}} $ er alternerende, da $ \cos(\pi n)=(-1)^n $.
        	Ydermere gælder at $ \left\{\abs{\frac{(-1)^n}{\log\pare{\log\pare{n}}}}\right\}_{n=3}^\infty=\left\{\frac{1}{\log\pare{\log\pare{n}}}\right\}_{n=3}^\infty $ er monotont aftagende med $\lim\limits_{n\to\infty}\left(\frac{1}{\log\pare{\log\pare{n}}}\right)= 0 $. Per Leibniz' test (Sætning 2.30) konkluderes at $ \sum_{n=3}^\infty
        	\frac{\cos\pare{\pi n}}{\log\pare{\log\pare{n}}} $ er konvergent, hvoraf det følger at den oprindelige række er konvergent per Proporsition 2.12. Det ses samtidig let at $ \sum_{n=3}^{\infty}\frac{1}{\log\pare{\log\pare{n}}} $ er divergent ved sammeligningstesten (Korollar 2.17) da $ 1/\log(\log(n))>1/n $ for $ n\geq 3 $, hvoraf det følger at den oprindelige række er divergent per Proporsition 2.12	. Altså er rækken betinget konvergent.
        \end{proof}
        \fi
    \item Afgør om følgende række er absolut konvergent, betinget konvergent eller divergent.
    $$
    \sum_{n=2}^{\infty}\left(\frac{1}{n}+\frac{(-1)^n}{\log(n)}\right)
    $$
    
    \ifanswers
    \begin{proof}[Løsning]
    	Betragt den alternerende række $ \sum_{n=2}^{\infty}(-1)^n/\log(n) $. Per Leibniz' test (Sætning 2.30) ses det let, at denne række er konvergent, da absolutværdien af ledne er monotont aftagende og gående mod $ 0 $. Derfor konkluderes, at $ \sum_{n=1}^{\infty}(-1)^n/\log(n)<\infty $. Det følger da umiddelbart af Korollar 2.11, samt at $ \sum_{n=2}^{\infty}\frac{1}{n} $ er divergent, at $ \sum_{n=2}^{\infty}\left(\frac{1}{n}+\frac{(-1)^n}{\log(n)}\right) $ er divergent.
    	
%    	Antag nu at $ \sum_{n=2}^{\infty}\left(\frac{1}{n}+\frac{(-1)^n}{\log(n)}\right) $ er konvergent. Da gælder per sætning 2.9 at $ \sum_{n=2}^{\infty}\left(\frac{1}{n}+\frac{(-1)^n}{\log(n)}-\frac{(-1)^n}{\log(n)}\right)=\sum_{n=2}^{\infty}\frac{1}{n} $ er konvergent, hvilket er en modstrid med Sætning 2.5. Det konkluderes derfor at $ \sum_{n=2}^{\infty}\left(\frac{1}{n}+\frac{(-1)^n}{\log(n)}\right) $ er divergent.
    \end{proof}
    \fi
\end{enumerate}
\end{opg}

%\begin{opg}
%Lad $a,b > 0$ og betragt rækken givet ved
%$$ \sum_{n=0}^\infty \pare{b^n - a^n}. $$
%\begin{enumerate}
%    \item Vis at rækken er konvergent for $a = \frac{1}{3}$, $b = \frac{1}{2}$, og bestem summen.
%    
%    
%    \item Bestem samtlige værdier $a,b>0$, hvor rækken er konvergent.
%    
%\end{enumerate}
%\end{opg}

\begin{opg}
	Lad $ a,b\in\R $ og betragt rækken $ \displaystyle\sum_{n=1}^{\infty}n^{an+b} $
	\begin{enumerate}
		\item Lad $ a=0 $. Bestem m\ae{}ngden af  $ b\in\R $ hvor rækken er konvergent.
		\ifanswers
		\begin{proof}[Løsning]
			For $ a=0 $ har vi $ \sum_{n=1}^{\infty}n^b $. Det genkendes som en $ p $-række med $ p=-b $. Det konkluderes fra Eksempel 2.23 at rækken er konvergent hvis og kun hvis $ b<-1 $.
		\end{proof}
		\fi
		\item Bestem m\ae{}ngden af  $ a,b\in\R $ for hvilke rækken er konvergent.
		\ifanswers
		\begin{proof}[Løsning]
			For $ a<0 $ bemærkes, at der eksistere $ N\in\N $ således at $ an+b<-2 $ for alle $ n>\N $. Men da gælder at $ n^{an+b}<n^{-2} $ for alle $ n>\N $, og ved sammeligningtesten (Korollar 2.17) ses at $ \sum_{n=N+1}^{\infty}n^{an+b} $ er konvergent, da det vides at $ \sum_{n=N+1}^{\infty}\frac{1}{n^2} $ er konvergent per Eksempel 2.23,\\
			For $ a>0 $ vides det, at der eksistere $ N'\in\N $ således at $ an+b>1 $ for $ n>N' $. Da gælder klart at $ n^{an+b}>n $ for alle $ n>N' $, hvoraf det konkluderes ved divergenstesten (kontraposition af Sætning 2.2), at $ \sum_{n=1}^{\infty}n^{an+b} $ er divergent, da $ n^{an+b} $ ikke konvergerer mod $ 0 $.
			
			
%			For $ a< 0 $ benytter vi rodtesten. Vi ser først at \begin{equation*}
%			\left(n^{an+b}\right)^{1/n}=n^{a+b/n}=n^an^{b/n}.
%			\end{equation*}
%			Bemærk at $n^{b/n}=\exp(\log(n^{b/n}))=\exp\left(\frac{b}{n}\log(n)\right) $. Det vides at $ x\mapsto\exp(x) $ er kontinuert på hele $ \R $. Det følger yderligere af L'H\^opital's regel og Observation 1.42 at $ \lim\limits_{n\to\infty}b\frac{\log(n)}{n}=\lim\limits_{x\to\infty}b\frac{\log(x)}{x}=\lim\limits_{x\to\infty}b\frac{1}{x}=0 $. Dermed gælder per Sætning 1.43, at $ \lim\limits_{n\to\infty}n^{b/n}=\exp(0)=1 $. For $ a<0 $ har vi nu ved regnereglerne for konvergente følger (Sætning 1.39) \begin{equation*}
%			\lim\limits_{n\to\infty}\left(n^{an+b}\right)^{1/n}=\lim\limits_{n\to\infty}n^an^{b/n}=0,
%			\end{equation*}
%			hvor vi har brugt den velkendte grænseværdi $ \lim\limits_{n\to\infty}n^a=0 $ for $ a<0 $. Det følger af rodtesten (Sætning 2.26), at rækken er konvergent.\\
%			Hvis $ a>0 $ har vi derimod $ \lim\limits_{n\to\infty}n^{an+b}=\infty $. Dette kan ses ved at bemærke, at $ n^{an+b}=\exp(\log(n^{an+b}))=\exp((an+b)\log n) $. Det er tydeligt at $ \lim\limits_{n\to\infty}(an+b)\log n=\infty $, og det gælder per observation 1.42 samt $ \lim\limits_{x\to\infty}e^x=\infty $, at $ \lim\limits_{n\to\infty}n^{an+b}=\infty $. Det følger da ved kontraposition af (Sætning 2.2) at rækken er divergent.\\
			Tilfældet $ a=0 $ er dækket i opgave a). Vi konkluderer alt i alt, at rækken konvergerer hvis og kun hvis $ a<0 $ eller $ a=0 $ \emph{og} $ b<-1 $.
		\end{proof}
		\fi
	\end{enumerate}
\end{opg}

\begin{opg}
		Betragt rækken $$ \sum_{n=1}^{\infty}\frac{\ceil{\frac{n}{\pi}}}{n^4}. $$
		\begin{enumerate}
			\item Vis at rækken er konvergent med en sum, der ligger i intervallet $\left[1,\frac{11}{10}\right] $	
		\end{enumerate}
		\textsl{[Estimater skal som altid begrundes, og i det omfang der er brugt computeralgebrasystemer skal output herfra underbygges teoretisk.]}

		\ifanswers
		\begin{proof}[Løsning]
			At rækken er konvergent konkluderes let ved sammeligningstesten, da $ \ceil{n/\pi}\leq n $ og dermed gælder $ \frac{\ceil{\frac{n}{\pi}}}{n^4}\leq \frac{1}{n^3} $. Det følger da af $ \sum_{n=1}^{\infty}\frac{1}{n^3}<\infty $ (Eksempel 2.23), at rækken konvergerer.\\
			For at estimere rækken, bemærkes først at $ \sum_{n=1}^{\infty}\frac{\ceil{\frac{n}{\pi}}}{n^4}=1+\sum_{n=2}^{\infty}\frac{\ceil{\frac{n}{\pi}}}{n^4}>1 $. På den anden side gælder det at $ \ceil{n/\pi}<n/\pi+1 $. Dermed har vi \begin{equation*}
			\sum_{n=1}^{\infty}\frac{\ceil{\frac{n}{\pi}}}{n^4}\leq \sum_{n=1}^{N}\frac{\ceil{\frac{n}{\pi}}}{n^4}+\sum_{n=N+1}^{\infty}\frac{\frac{n}{\pi}+1}{n^4}\leq \sum_{n=1}^{N}\frac{\ceil{\frac{n}{\pi}}}{n^4}+\frac{\frac{N+1}{\pi}+1}{(N+1)^4}+\int_{N+1}^{\infty}\frac{\frac{x}{\pi}+1}{x^4}\diff x,
			\end{equation*}
			hvor den anden ulighed følger af Observation 2.21.
			Vælger vi $ N=3 $ findes\begin{equation*}
			\sum_{n=1}^{\infty}\frac{\ceil{\frac{n}{\pi}}}{n^4}\leq 1+\frac{1}{2^4}+\frac{1}{3^4}+\frac{4/\pi+1}{4^4}+\int_{4}^{\infty}\frac{x/\pi+1}{x^4}\diff x=1+\frac{1}{2^4}+\frac{1}{3^4}+\frac{4/\pi+1}{4^4}+\frac{1}{2\pi}\frac{1}{4^2}+\frac{1}{3}\frac{1}{4^3}=1.0988...
			\end{equation*}
			Altså konkluderes at $\sum_{n=1}^{\infty}\frac{\ceil{\frac{n}{\pi}}}{n^4}\in(1,1.099)\subset\left[1,\frac{11}{10}\right]  $.
		\end{proof}
		\fi
\end{opg}

\begin{opg}
	Betragt rækken givet ved
	$$ \sum_{n=1}^\infty \frac{x^{2n-1}}{n-\frac12}, \QUAD\text{hvor } x\in\mR $$
	og lad $0\leq a < 1$. 
	\begin{enumerate}
		\item Vis, at rækken konvergerer uniformt på intervallet $[-a,a]$.
		\ifanswers
		\begin{proof}[Løsning]
			Bemærk at $ \abs{\frac{x^{2n-1}}{n-\frac{1}{2}}}\leq 2a^n $ for $ -1<-a\leq x\leq a<1 $ med $ 0\leq a $, og siden $ \sum_{n=1}^{\infty}2a^n<\infty $, for $ 0\leq a<1 $, følger det af Weierstrass' majorenttest (Sætning 3.24), at $ \sum_{n=1}^\infty \frac{x^{2n-1}}{n-\frac12} $ er uniformt konvergent på intervallet $ [-a,a] $.
		\end{proof}
		\fi
		
		\item Vis, at rækkens sumfunktion $f\colon [-a,a]\to\mR$ er differentiabel, og der gælder
		$$ f'(x)=\frac{2}{1-x^2}, \Quad x\in[-a,a]. $$
		\ifanswers
		\begin{proof}[Løsning]
			Den ledvist afledte række findes\begin{equation*}
			\sum_{n=1}^{\infty}2\left(n-\frac{1}{2}\right)\frac{x^{2n-2}}{n-\frac12}=2\sum_{n=0}^{\infty}(x^2)^n.
			\end{equation*}
			Det ses at den ledvist afledte række er uniform konvergent på intervallet $ [-a,a] $, da det blot er en (generaliseret) geometrisk række. Det ses yderligere at hvert led i den oprindelige række er kontinuert differentiable på intervallet $ [-a,a] $.
			Det følger af Korollar 3.20 fra forelæsningsslides, at rækkens sumfunktion er differentiable, og at den afledte sumfunktion er givet ved den ledvist afledte række. vi får dermed \begin{equation*}
			f'(x)=\sum_{n=1}^{\infty}2x^{2n-2}=2\sum_{n=0}^{\infty}(x^2)^n=\frac{2}{1-x^2},
			\end{equation*}
			hvor vi i sidste lighedstegn benyttede Sætning 2.4 samt $ \abs{x^2}\leq a^2<1 $.
		\end{proof}
		\fi
		
		
		\item Vis at $ g(x)=\log(1+x)-\log(1-x) $ også opfylder at $ g'(x)=\frac2{1-x^2} $ og vis at formlen
		$\sum_{n=1}^\infty \frac{x^{2n-1}}{n-\frac12}=\log(1+x)-\log(1-x) $ gælder for alle $ x\in [-a,a] $.
		\ifanswers
		\begin{proof}[Løsning]
			Det gælder åbenlyst at $ g'(x)=\left(\log(1+x)-\log(1-x)\right)'=\frac{1}{1+x}-\frac{1}{1-x}=\frac{1-x+1+x}{1-x^2}=\frac{2}{1-x^2} $. Derfor gælder $ (g(x)-f(x))'=g'(x)-f'(x)=0 $, og vi konkluderer at $ g(x)-f(x)=\text{konst.}=g(0)-f(0) $. Men der gælder klart at $ f(0)=0=g(0) $, hvorfor vi konkluderer $ f(x)=g(x) $.
		\end{proof}
		\fi
		\item
		Vis at $ \displaystyle \sum_{n=0}^{\infty}\frac{1}{2n+1}\frac{1}{4^n}=\log(3) $.
		\ifanswers
		\begin{proof}[Løsning]
			Vi bermærker først at rækken åbenlyst er konvergent per sammenlignstesten da $ \frac{1}{2n+1}\frac{1}{4^n}<\frac{1}{4^n} $. Ydermere gælder det at  $$ \sum_{n=0}^{\infty}\frac{1}{2n+1}\frac{1}{4^n}=\sum_{n=1}^{\infty}\frac{1}{2n-1}\frac{1}{4^{n-1}}=\sum_{n=1}^{\infty}\frac{1}{n-\frac12}\frac{1}{2^{2n-2+1}}=\sum_{n=1}^{\infty}\frac{1}{n-\frac12}\left(\frac12\right)^{2n-1}. $$ Ved at lade $ a=1/2 $ i c) ses det nu let fra resultatet i c) at $$ \sum_{n=1}^{\infty}\frac{1}{n-\frac12}\left(\frac{1}{2}\right)^{2n-1}=\log(1+1/2)-\log(1-1/2)=\log(3/2)-\log(1/2)=\log(3/2)+\log(2)=\log(3), $$ hvoraf det ønskede resulat følger. 
		\end{proof}
		\fi
		
	\end{enumerate}
\end{opg}

%\begin{opg}
%Lad $p,r\in\mR$ og betragt rækken
%$$ \sum_{n=2}^\infty \frac{\log\pare{\log\pare{n}}^r}{\log\pare{n}n^p}. $$
%\begin{enumerate}
%    \item Lad $p = 1$. Vis, at rækken er konvergent hvis og kun hvis $ r<-1 $.
%   \ifanswers
%    \begin{proof}[Løsning]
%    Definér $ f:[2,\infty)\to \R  $ ved $ f(x)=\frac{\log\pare{\log\pare{x}}^r}{\log\pare{x}x} $. Den afledte funktion af $ f $ findes ved $ f'(x)=\left(\frac{r\log\pare{\log\pare{x}}^{r-1}}{(\log(x)x)^2}-\frac{\log\pare{\log\pare{x}}^r(1+\log(x))}{(\log(x)x)^2}\right) $. Det er klart at der for alle $ r\in \R $ eksisterer et $ M_r $ således at $ f'(x)<0 $ for $ x>M_r $. Dermed gælder per integraltesten, at $ \sum_{n=2}^{\infty}\frac{\log\pare{\log\pare{n}}^r}{\log\pare{n}n^p} $ er konvergent hvis og kun hvis $ \int_{M_r}^{\infty}f(x)\diff x<\infty $. Men bemærk at \begin{equation}
%    \int_{M_r}^{\infty}f(x)\diff x=\lim\limits_{a\to\infty}\int_{M_r}^{a}f(x)\diff x=\lim\limits_{a\to\infty}\int_{M_r}^{a}\frac{\log\pare{\log\pare{x}}^r}{\log\pare{x}x}\diff x=\lim\limits_{a\to\infty}\int_{M_r}^{a}u(x)^r u'(x)\diff x
%    \end{equation}
%    hvor vi har defineret $ u(x)=\log\pare{\log\pare{x}} $. Integration ved subsitution giver derfor \begin{equation}
%    \int_{M_r}^{\infty}f(x)\diff x=\lim\limits_{a\to\infty}\int_{\log\pare{\log\pare{M_r}}}^{\log\pare{\log\pare{a}}}u^r\diff u=\int_{C_r}^{\infty}u^r\diff u
%    \end{equation}
%    hvor $ C_r=\log\pare{\log\pare{M_r}} $ og sidste lighedstegn holder grundet $ \lim\limits_{a\to\infty}\log\pare{\log\pare{a}}=\infty $. Det er nu velkendt at $ \int_{M_r}^{\infty}f(x)\diff x<\infty $ hvis og kun hvis $ r<-1 $, og det følger at $ \sum_{n=2}^{\infty}\frac{\log\pare{\log\pare{n}}^r}{\log\pare{n}n^p} $ er konvergent hvis og kun hvis $ r<-1 $.
%    \end{proof}
%	\fi
%    
%    \item Bestem samtlige $p,r\in\mR$, hvor rækken er konvergent.
%    \ifanswers
%    \begin{proof}
%    	Hvis $ p>1 $, da ses det, siden at $ \lim\limits_{n\to\infty}\left(\log\pare{\log(n)^r}/\log(n)\right)=0 $, som følger af L'H\^opital's regel brugt $ \max\pare{0,\lceil r\rceil} $ gange, at der eksisterer et $ N\in\N $ således at $ \frac{\log\pare{\log\pare{n}}^r}{\log\pare{n}n^p}<\frac{1}{n^p} $ for $ n>N $, og dermed ses at $ \sum_{N}^{\infty}\frac{\log\pare{\log\pare{n}}^r}{\log\pare{n}n^p}<\infty $, hvoraf det følger at $ \sum_{n=2}^{\infty}\frac{\log\pare{\log\pare{n}}^r}{\log\pare{n}n^p}<\infty $ for alle $ r\in \R $.\\
%    	Hvis på den anden side at $ p<1 $, da ses at $ \frac{\log\pare{\log\pare{n}}^r}{\log\pare{n}n^p}=\frac{\log\pare{\log\pare{n}}^rn^{1-p}}{\log\pare{n}n}\geq\frac{n^{1-p}}{\log\pare{\log\pare{n}}^{\abs{r}} \log(n)n}\geq\frac{n^{1-p}}{\log(n)^{\abs{r}+1}}\frac1n $, men siden at $ \lim\limits_{n\to\infty}\left(\frac{\log(n)^{\abs{r}+1}}{n^{1-p}}\right)=0 $, hvilket igen kan ses ved at bruge L'H\^opital's regel $ \abs{r}+1 $ gange, følger det at der eksisterer et $ N\in\N $ således at $ \frac{\log\pare{\log\pare{n}}^r}{\log\pare{n}n^p}>\frac{1}{n} $ for $ n>N $ og dermed per sammeligningstesten (Sætning ...) er $ \sum_{n=2}^{\infty}\frac{\log\pare{\log\pare{n}}^r}{\log\pare{n}n^p} $ divergent for alle $ r\in\R $
%    \end{proof}
%    \fi
%   
%    
%    \item Betragt funktionsrækken
%    $$ \sum_{n=2}^\infty \frac{\log\pare{\log\pare{n}}^x}{\log\pare{n}n}, \QUAD x\in\mR. $$
%    Lad $s<-1$ og vis at funktionsrækken er uniformt konvergent for alle $x \leq s$. 
%    \ifanswers
%    \begin{proof}[Løsning]
%    	Det ses let at $ \frac{\log\pare{\log\pare{n}}^x}{\log\pare{n}n}\leq \frac{\log\pare{\log\pare{n}}^s}{\log\pare{n}n} $ for $ n\geq 2 $ og $ x\leq s $. Men det vides fra 4.a), at $ \sum_{n=2}^\infty \frac{\log\pare{\log\pare{n}}^s}{\log\pare{n}n}<\infty $ for $ s<-1 $. Det følger derfor af Weierstrass' majorenttest (Sætning 3.24), at $ \sum_{n=2}^\infty \frac{\log\pare{\log\pare{n}}^x}{\log\pare{n}n} $ er uniformt konvergent for alle $ x\leq s $.
%    \end{proof}
%    \fi
%
%\end{enumerate}
%\end{opg}
	
	\ifrules
	\newpage
	\noindent
	{\LARGE Regler og vejledning for aflevering af Hjemmeopgave 2}
	
	\noindent\hrulefill \\
	
	\noindent
	Besvarelsen skal udarbejdes individuelt, og afskrift behandles efter universitetets regler om eksamenssnyd. Besvarelsen vil blive bedømt på en skala fra 0 til 100. Denne bedømmelse indgår med en vægt på omtrent en fjerdedel af den endelige karakter. På tværs af de fire Hjemmeopgaver skal man have mindst 50 point i gennemsnit for at bestå.
	
	Ved bedømmelsen lægges vægt på klar og præcis formulering og på argumentation på grundlag af og med henvisning til relevante resultater i pensum, herunder opgaver regnet ved øvelserne. I kan bruge følgende som rettesnor for henvisninger.
	\begin{itemize}
		\item Tænk på henvisninger som en hjælp til at forklare sig. Hvis det er klart af fra konteksten, hvilke resultater man bruger, så er det ikke nødvendigt at henvise.
		
		\item I må henvise til resultater fra noterne [MC], øvelsesopgaverne og forelæsningsslides, samt til alle l\ae{}reb\o{}ger brugt p\aa{} andre f\o{}rste\aa{}rskurser p\aa{} matematikstudierne. Hvis I citerer fra materiale n\ae{}vnt i litteraturlisten i [MC] kan I uden videre benytte forkortelserne brugt her (fx [EHM], [Li]). Det er tilladt at henvise til pensum, som endnu ikke er gennemgået. Det er ikke nødvendigt at angive sidetal på henvisninger. 
		
		\item Det er næsten aldrig relevant at henvise til definitioner.
	\end{itemize}
	%	I skal argumentere med en grundighed svarende til de vejledende besvarelser af de tidligere hjemmeopgaver.
	
%	\bigskip  
%	\noindent
%	Det er ikke n�dvendigt at bruge computeralgebrasystemer til l�sningen af opgaverne, Alle estimater skal som altid begrundes, og der gives ikke fuld point, s�fremt der er brugt CAS uden at dette er forklareti opgaven.".
	
	\bigskip  
	\noindent
	Besvarelsen må udfærdiges i hånden eller med \LaTeX\ eller lignende, men skal være ensartet og letlæselig. Billeder, plots og lignende må gerne udfærdiges i andre programmer. Det forventes, at håndskrevne besvarelser ikke fylder mere end 7 sider,
	og at besvarelser udarbejdet elektronisk ikke fylder mere end 5 sider. Billeder, plots og lignende tæller med i sideantallet. Håndskrevne besvarelser skal være tydeligt læsbare.
	
	\bigskip 
	\noindent
	P\aa{} hver side af den afleverede l\o{}sning skal I skrive jeres navn og KU-id (``svenske nummerplade''). Vi anbefaler at I ogs\aa{} skriver ``side $x$ ud af $y$'' p\aa{} hver side.
	
	\bigskip 
	\noindent
	Det er kun tilladt at aflevere gennem Absalon, og man skal uploade sin besvarelse som én .pdf. Vi opfordrer jer til hvis muligt at aflevere i god tid for at undgå at Absalon går ned, fordi alle afleverer samtidig.
	
	\bigskip
	\noindent
	Bedømmelsen og en .pdf med et udvalg af forklarende kommentarer vil blive uploadet til Absalon inden for en uge.
	\fi
\end{document}