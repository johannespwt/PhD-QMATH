\documentclass{article}

\usepackage{ANpreamble}
%\usepackage{enumerate}

\newif\ifanswers

\answerstrue %udkommenter denne linje for at skjule løsninger


\title{Analyse 1 2020/2021 - Rettevejledning til Hjemmeopgave 3}
\author{Johannes Agerskov}
\date{\today}
\setcounter{section}{3}
\begin{document}
	
	\maketitle
Følgende rettevejledning er, som det fremgår af navnet, \emph{vejledende}. Det er derfor op til den enkelte instruktor at afgøre i hvilken grad, de angivne forslag i denne vejledning er passende til den enkelte besvarelse. Vejledningen er på ingen måde dækkende over alle typer af besvarelser, og ofte vil der i retteprocessen også indgå et helhedsintryk når der gives point. Det er derimod vejledningens formål, at give et overordnet billede af, hvor hårdt der slåes ned på nogle udvalgte (måske typiske) fejl eller mangler der kan optræde i besvarelserne. Det er forventeligt, at fejl og mangler, som er gennemgået i denne vejledning, vil optræde i forskellige grader eller variationer i besvarelserne, og i så fald, må det vurderes hvorvidt der skal trækkes færre eller flere point i disse tilfælde. Det er håbet, at fejl, der ikke er gennemgået i denne vejledning, kan relateres til eller sammenlignes med fejl gennemgået i vejledningen, og at pointgivningen for disse besvarelse dermed kan ekstra-/interpoleres fra vejledningen.
\vspace*{0.5cm}\\
Som beskrevet i "Regler og vejledning for aflevering a Hjemmeopgave 3" på sidste side i opgavesættet, lægges der vægt på klar og og præcis formulering. Det er dermed essentielt, at opgaverne er letlæselige, og at det er klart, hvad den studerende mener. For at der kan gives fuld point for en opgave, skal argumentet være fuldkomment, og der bør ikke være brug for at færdiggøre argumenterne i hovedet, for at forstå dem.
\vspace*{0.2cm}
\\
Hvis der skulle være brug for let tolkning af en forklaring, dvs. at det er tydeligt, hvad der er menes, men formuleringen er delvis utilstrækkelig, fratrækkes der som udgangs punkt få point, altså $ 1 $ eller $ 2 $ point afhængig af graden af utilstrækkelighed.
\vspace*{0.2cm}\\
Hvis der er brug for større tolkning af en forklaring, eller brug for færdiggørelse af argumentet for at verificere dets korrekthed, anses denne del af opgaven for værende ikke løst, og der kan trækkes point svarende til den del af opgaven.
\vspace*{0.2cm}\\
Hvis der er givet et argument et forkert sted i besvarelsen, f.eks. arguementet står i 2.a) men er først nødvendigt (og brugbart) i 2.b, da fratrækkes 1 point, med mindre selvfølglig, at der i opgave 2.b henvises til det tidligere argument.
\vspace*{0.2cm}\\
Vi retter efter et oppefra og ned princip, hvilket vil sige, at enhver besvarelse som udgangspunkt har 100/100 point. Der fratrækkes så point for fejl, mangler og andre utilstrækkeligheder i besvarelsen.
\vspace*{0.2cm}\\
I følgende tilfælde trækkes der \textbf{ikke} point:
\begin{itemize}
	\item Der trækkes ikke point for ligegyldig tekst.
	\item Der trækkes ikke point for et fejlagtigt argument, \textbf{hvis} der efterfølgende korrigeres ved at give et korrekt argument for samme resultat. 
	\item Hvis der gives to besvarelser for en delopgave tildeles gennemsnittet af pointene for de to besvarelser.
	\item Der fratrækkes ikke point for følgefejl, altså fejl som skyldes fejl i tidligere opgaver.
\end{itemize}

En forløbig rettevejledning for Hjemmeopgave 3 ser således ud:

\begin{opg}[30 \emph{point}]\hfill
	\begin{enumerate}
		\item (10 \emph{point})\begin{enumerate}[label=(\roman*)]
			\item Glemmer at kigge på absolutværdien når der udføres konvergenstest ($ -5 $ point.)
			\item Viser bare konvergens ($ -7 $ point)
		\end{enumerate}
		\item (10 \emph{point}) \begin{enumerate}[label=(\roman*)]
			\item Forkert brug af Leibniz' test (op til $ -5 $ point) Følgende er vejledende:
			\item Argumenterer ikke for / nævner ikke, at rækken er alternerende fra $ n=3 $ ved brug af Leibniz' test ($ -1 $ point)
			\item Argumenterer ikke for / nævner ikke, at absolutværdien af ledne er monoton aftagende fra $ n=3 $ før brug af Leibniz' test ($ -4 $ point)
			\item Viser ikke at rækken \emph{ikke} er absolut konvergent, for at kunne konkludere betinget konvergens ($ -4 $ point)
		\end{enumerate}
		\item (10 \emph{point})\begin{enumerate}[label=(\roman*)]
			\item Bemærk, at nogle kunne falde i fælden, at man kunne tro, at Leibniz' test virker på denne række.
			\item Splitter rækken op i to rækker (med lighedstegn), hvor den ene er divergent ($ -1\text{ til }-2 $ point)
			\item Viser ikke kovergens af $ \sum_{n=2}^{\infty}\frac{(-1)^n}{\log(n)} $ ($ -5 $ point).
		\end{enumerate}
	\end{enumerate}	
\end{opg}

\end{document}