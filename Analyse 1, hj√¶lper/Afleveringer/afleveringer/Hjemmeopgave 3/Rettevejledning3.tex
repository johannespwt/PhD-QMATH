\documentclass{article}

\usepackage{ANpreamble}
%\usepackage{enumerate}

\newif\ifanswers

\answerstrue %udkommenter denne linje for at skjule løsninger


\title{Analyse 1 2020/2021 - Rettevejledning til Hjemmeopgave 3}
\author{Johannes Agerskov}
\date{\today}
\setcounter{section}{3}
\begin{document}
	
	\maketitle
Følgende rettevejledning er, som det fremgår af navnet, \emph{vejledende}. Det er derfor op til den enkelte instruktor at afgøre i hvilken grad, de angivne forslag i denne vejledning er passende til den enkelte besvarelse. Vejledningen er på ingen måde dækkende over alle typer af besvarelser, og ofte vil der i retteprocessen også indgå et helhedsintryk når der gives point. Det er derimod vejledningens formål, at give et overordnet billede af, hvor hårdt der slåes ned på nogle udvalgte (måske typiske) fejl eller mangler der kan optræde i besvarelserne. Det er forventeligt, at fejl og mangler, som er gennemgået i denne vejledning, vil optræde i forskellige grader eller variationer i besvarelserne, og i så fald, må det vurderes hvorvidt der skal trækkes færre eller flere point i disse tilfælde. Det er håbet, at fejl, der ikke er gennemgået i denne vejledning, kan relateres til eller sammenlignes med fejl gennemgået i vejledningen, og at pointgivningen for disse besvarelse dermed kan ekstra-/interpoleres fra vejledningen.
\vspace*{0.5cm}\\
Som beskrevet i "Regler og vejledning for aflevering a Hjemmeopgave 3" på sidste side i opgavesættet, lægges der vægt på klar og og præcis formulering. Det er dermed essentielt, at opgaverne er letlæselige, og at det er klart, hvad den studerende mener. For at der kan gives fuld point for en opgave, skal argumentet være fuldkomment, og der bør ikke være brug for at færdiggøre argumenterne i hovedet, for at forstå dem.
\vspace*{0.2cm}
\\
Hvis der skulle være brug for let tolkning af en forklaring, dvs. at det er tydeligt, hvad der er menes, men formuleringen er delvis utilstrækkelig, fratrækkes der som udgangs punkt få point, altså $ 1 $ eller $ 2 $ point afhængig af graden af utilstrækkelighed.
\vspace*{0.2cm}\\
Hvis der er brug for større tolkning af en forklaring, eller brug for færdiggørelse af argumentet for at verificere dets korrekthed, anses denne del af opgaven for værende ikke løst, og der kan trækkes point svarende til den del af opgaven.
\vspace*{0.2cm}\\
Hvis der er givet et argument et forkert sted i besvarelsen, f.eks. arguementet står i 2.a) men er først nødvendigt (og brugbart) i 2.b, da fratrækkes 1 point, med mindre selvfølglig, at der i opgave 2.b henvises til det tidligere argument.
\vspace*{0.2cm}\\
Vi retter efter et oppefra og ned princip, hvilket vil sige, at enhver besvarelse som udgangspunkt har 100/100 point. Der fratrækkes så point for fejl, mangler og andre utilstrækkeligheder i besvarelsen.
\vspace*{0.2cm}\\
I følgende tilfælde trækkes der \textbf{ikke} point:
\begin{itemize}
	\item Der trækkes ikke point for ligegyldig tekst.
	\item Der trækkes ikke point for et fejlagtigt argument, \textbf{hvis} der efterfølgende korrigeres ved at give et korrekt argument for samme resultat. 
	\item Hvis der gives to besvarelser for en delopgave tildeles gennemsnittet af pointene for de to besvarelser.
	\item Der fratrækkes ikke point for følgefejl, altså fejl som skyldes fejl i tidligere opgaver.
\end{itemize}

En forløbig rettevejledning for Hjemmeopgave 3 ser således ud:

\begin{opg}[30 \emph{point}]\hfill
	\begin{enumerate}
		\item (10 \emph{point})\begin{enumerate}[label=(\roman*)]
			\item Løser opgaven reelt istedet for komplekst (Dvs med sætning 4.22 eller Eksempel 4.24) ($ -3 $ point)
			\item Glemmer at tjekke randtildfælde for konvergensområde ($ -2 $ point)
			\item Henviser ikke korrekt til manipulationsresultater potensrække (1)-(8) men bruger dem korrekt ($ -1 $ point)
		\end{enumerate}
		\item (10 \emph{point}) \begin{enumerate}[label=(\roman*)]
			\item Glemmer $ x=0 $ tilfælde ($ -2 $ point)
			\item Glemmer at tjekke konvergensområde ($ -2 $ point)
		\end{enumerate}
		\item (10 \emph{point})\begin{enumerate}[label=(\roman*)]
			\item Glemmer at invertere for $ r $. ($ -3 $ point)
		\end{enumerate}
	\end{enumerate}	
\end{opg}

\begin{opg}[20 \emph{point}]\hfill
	\begin{enumerate}
		\item (10 \emph{point})\begin{enumerate}[label=(\roman*)]
			\item Glemmer at vise konvergens af Taylorrækken ($ -5 $ point)
			\item Hvis de ikke eksplicit differentierer og finder Taylorrækken: Glemmer at henvise til Sætning 4.28, for at vise at hvis en potensrække er lig funktionen, så er den Taylorrækken.  ($ -3 $ point)
		\end{enumerate}
		\item (10 \emph{point})\begin{enumerate}[label=(\roman*)]
			\item Glemmer at henvise til Sætning 4.28, for at vise at hvis en potensrække er lig $ g $ så er den Taylorrækken for $ g $.  ($ -3 $ point)
		\end{enumerate}
	\end{enumerate}
\end{opg}

\begin{opg}[30 \emph{point}]\hfill
	\begin{enumerate}
		\item (10 \emph{point})\begin{enumerate}[label=(\roman*)]
			\item Henviser ikke til Sætning 4.3 eller nævner ækvivalent resultat for at vise $ 1/2\leq r\leq1 $. ($ -2 $ point)
			\item Glemmer at vise divergens ($ -3 $ point)
			\item Glemmer at vise konvergens ($ -3 $ point)
		\end{enumerate}
		\item (10 \emph{point})\begin{enumerate}[label=(\roman*)]
			\item Glemmer at redegøre for konvergensradius mindst $ r $ ($ -2  $ point) 
			\item BEMÆRK: der var fejl i opgaven, så der trækkes ikke point, hvis der ikke vises konvergensradius størst $ r $.
			\item BONUS: Hvis der vises konvergensradius størst $ r $ ($ +3 $ point)
		\end{enumerate}
		\item (10 \emph{point})\begin{enumerate}[label=(\roman*)]
			\item Glemmer at vise $ r=1 $ ($ -4 $ point)
		\end{enumerate}
	\end{enumerate}
\end{opg}
\begin{opg}[20 \emph{point}]\hfill
	\begin{enumerate}
		\item (10 \emph{point})\begin{enumerate}[label=(\roman*)]
			\item ??
		\end{enumerate}
		\item (10 \emph{point})\begin{enumerate}[label=(\roman*)]
			\item Henviser ikke til 4.18(5), når potensrækken skal splittes ($ -3 $ point)
			\item Glemmer at konkluderer at $ a_{2n+1}=0 $ vha. entydighedssætningen. ($ -4 $ point)
			\item Referer ikke til a) for "hvis". ($ -1 $ point)
		\end{enumerate}
	\end{enumerate}
\end{opg}

\end{document}