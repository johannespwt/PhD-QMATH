\documentclass{beamer}[10]
\usepackage[english]{babel}
\usepackage[utf8]{inputenc}
%\usepackage{beamerthemesplit}
\usepackage{graphics,epsfig, subfigure}
\usepackage{url}
\usepackage{srcltx}
\usepackage{mathrsfs}
\usepackage{amsfonts}
\usepackage{amsmath}
\DeclareMathOperator\arctanh{arctanh}
\usepackage{amssymb}
\usepackage{bbm}
\usepackage{amsthm}
\usepackage{graphicx}
\usepackage{centernot}
\usepackage{caption}
\usepackage{braket}
\usepackage{lastpage}
\usepackage{setspace}
\usepackage{xcolor}
\usepackage{mathtools}
\usepackage{cancel}

\renewcommand{\thempfootnote}{\arabic{mpfootnote}}


\newcommand{\euler}[1]{\text{e}^{#1}}
\newcommand{\Real}{\text{Re}}
\newcommand{\Imag}{\text{Im}}
\newcommand{\supp}{\text{supp}}
\newcommand{\norm}[1]{\left\lVert #1 \right\rVert}
\newcommand{\abs}[1]{\left\lvert #1 \right\rvert}
\newcommand{\floor}[1]{\left\lfloor #1 \right\rfloor}
\newcommand{\Span}[1]{\text{span}\left(#1\right)}
\newcommand{\dom}[1]{\mathscr D\left(#1\right)}
\newcommand{\Ran}[1]{\text{Ran}\left(#1\right)}
\newcommand{\conv}[1]{\text{co}\left\{#1\right\}}
\newcommand{\Ext}[1]{\text{Ext}\left\{#1\right\}}
\newcommand{\vin}{\rotatebox[origin=c]{-90}{$\in$}}
\newcommand{\interior}[1]{%
	{\kern0pt#1}^{\mathrm{o}}%
}
\renewcommand{\braket}[1]{\left\langle#1\right\rangle}
\newcommand*\diff{\mathop{}\!\mathrm{d}}
\newcommand{\ie}{\emph{i.e.} }
\newcommand{\eg}{\emph{e.g.} }
\newcommand{\dd}{\partial }
\newcommand{\R}{\mathbb{R}}
\newcommand{\C}{\mathbb{C}}
\newcommand{\w}{\mathsf{w}}
\newcommand{\rr}{\mathcal{R}}


\newcommand{\Gliminf}{\Gamma\text{-}\liminf}
\newcommand{\Glimsup}{\Gamma\text{-}\limsup}
\newcommand{\Glim}{\Gamma\text{-}\lim}

\newtheorem{mtheorem}{Theorem}
\newtheorem{mdefinition}{Definition}
\newtheorem{mproposition}{Proposition}
\newtheorem{mlemma}{Lemma}
\newtheorem{mcorollary}{Corollary}
\newtheorem{mremark}{Remark}


\definecolor{qmathblue}{RGB}{61,131,131}
\definecolor{kugreen1}{RGB}{50,93,61}
\definecolor{kugreen}{RGB}{70,116,60}
\definecolor{qmathbluelys}{RGB}{119,168,168}
\definecolor{kugreenlyslys}{RGB}{173,190,177}
\definecolor{kugreenlyslyslys}{RGB}{239,249,240}
\definecolor{qmathbluelyslyslys}{RGB}{236,243,243}
\setbeamercovered{transparent}
\mode<presentation>
%\usetheme[numbers,totalnumber,compress,sidebarshades]{PaloAlto}
\setbeamertemplate{footline}[frame number]
\setbeamertemplate{theorems}[numbered]

\usecolortheme[named=qmathblue]{structure}
\useinnertheme{circles}
\usefonttheme[onlymath]{serif}
\setbeamercovered{transparent}
\setbeamertemplate{blocks}[rounded][shadow=false]
\setbeamercolor{block body}{bg=qmathbluelyslyslys,fg=black}
\setbeamercolor{block title}{bg=qmathbluelyslyslys,fg=black}


\logo{\includegraphics[width=1.2cm]{qmath.jpg}}
%\useoutertheme{infolines} 
\title{The ground state energy of dilute 1D many-body quantum systems}
\subtitle{}
\author{Johannes Agerskov\\\vspace{0.2 cm}
	\scriptsize{Based on joint work with Robin Reuvers and Jan Philip Solovej\\
		arXiv:2203.17183}}
\institute{QMATH \\ University of Copenhagen}
\date{September 14th 2022}

\setbeamercovered{invisible}

\begin{document}
\frame{\titlepage \vspace{-0.5cm}
}

\frame
{
\frametitle{Overview}
\tableofcontents%[pausesection]
}

\section{The scattering length}

\begin{frame}
\frametitle{Background}
\vspace*{-0.2cm}
\begin{block}{The scattering length}
	\vspace*{-0.4cm}
	\small\begin{theorem}[Lieb-Yngvason 2001]
		For $ B_R=\{0\leq\abs{x}<R\}\subset \R^d $ with $ R>R_0\coloneqq\text{range}(v) $, let $ \phi\in H^1(B_{R}) $ satisfy
	 \begin{equation}
		 -\Delta \phi +\frac12 v\phi=0,\qquad \text{on }B_R,
		 \end{equation}
		 with boundary condition $ \phi(x)=1 $ for $ \abs{x}=R$.
		 Then $ \phi(x)=f(\abs{x}) $ for some $ f:(0,R]\to [0,\infty) $, and for $ \text{range}(v)<r<R $, we have \begin{equation}
		 f(r)=\begin{cases}
		 (r-a)/(R-a) &\text{for }d=1\\
		 \ln(r/a)/\ln(R/a) &\text{for }d=2\\
		 (1-ar^{2-d})/(1-aR^{2-d})&\text{for }d\geq 3,
		 \end{cases}
		 \end{equation}
		 with some constant $ a $ called the (s-wave) \textbf{scattering length}.
	\end{theorem}
\end{block}	
\end{frame}
\section{The model}
\begin{frame}
	\frametitle{Model}
	We consider a many-body system of bosons that interacts via a repulsive pair potential $ v_{ij}=v(\abs{x_i-x_j}) $, with $ v=v_{\text{reg}}+v_{\text{h.c.}} $\begin{equation}
	\mathcal{E}(\psi)=\int_{\Lambda_L}\left(\sum_{i=1}^{N}\abs{\nabla_i\psi}^2+\sum_{i<j} v_{ij}\abs{\psi}^2\right)\quad \text{on } L^2(\Lambda_L)^{\otimes_{\text{sym}} N}.
	\end{equation}
	The ground state energy is defined by 
	$$
	E(N,L)\coloneqq\inf_{\psi\in\mathcal{D}(\mathcal{E}),\ \norm{\psi}^2=1}\mathcal{E}(\psi).
	$$
\end{frame}
\section{2D and 3D}
\begin{frame}
\frametitle{2D and 3D}
\small For $ \Lambda_L=[0,L]^d $, let $ e(\rho)\coloneqq\lim\limits_{\substack{L\to\infty\\ N/L^{d}\to\rho}}E(N,L)/L^{d} $.
\vspace*{-0.3cm}
\begin{block}{}
	\small
	\vspace*{-0.2cm}
	\begin{theorem}[\small $ d=3 $ result, Lee-Huang-Yang 1957\footnotemark]
		\begin{equation}
		e(\rho)=4\pi\rho^2 a\left(1+\frac{128}{15\sqrt{\pi}}\sqrt{\rho a^3}+o(\sqrt{\rho a^3})\right).
		\end{equation}
	\end{theorem}
\vspace*{-0.3cm}
	\begin{theorem}[\small $ d=2 $ result\footnotemark]
		\begin{equation}
		e(\rho)=4 \pi \rho^2 Y\left(1-Y|\log Y|+\left(2 \Gamma+\frac{1}{2}+\ln (\pi)\right) Y\right)+o\left(\rho^2 Y^2\right),
		\end{equation}
		$Y=\abs{\ln(\rho a^2)}^{-1}$.
	\end{theorem}
\footnotetext[1]{\tiny Upper bound: Yau-Yin 2009, Basti-Cenatiempo-Schlein 2021. Lower bound: Fournais-Solovej 2021}
\footnotetext[2]{\tiny Fournais-Girardot-Junge-Morin-Olivieri 2022}
\end{block}
\end{frame}

\section{Main result}

\begin{frame}
	\frametitle{Main result}
	\begin{block}{}
		\begin{theorem}[A.-Reuvers-Solovej, 2022]
			\label{TheoremMain}
			Consider a 1D Bose gas with repulsive interaction  $v=v_{\text{reg}}+v_{\text{h.c.}}$ as defined above. Define the denisty $\rho=N/L$. For $\rho|a|$ and $\rho R_0$ sufficiently small, the ground state energy can be expanded as 
			\begin{equation}
			\label{result}
			E(N,L)=N\frac{\pi^2}{3}\rho^2\left(1+2\rho a+
			\mathcal{O}
			\left((\rho|a|)^{6/5}+(\rho R_0)^{6/5}+N^{-2/3}\right)\right),
			\end{equation}
			where $a$ is the scattering length of $v$.
		\end{theorem}
	\end{block}	
\end{frame}


\section{Examples in 1D}
\begin{frame}
	\frametitle{Examples in 1D}
	\begin{block}{The hard core gas}
		Energy behaves like free Fermi energy in volume $ L-NR $, \ie \begin{equation}
		\begin{aligned}
		E_{\text{hard core}}(N,L)&=N\frac{\pi^2}{3}\rho^2 (1-NR/L)^{-2}\\&= E_0\left(1+2\rho R+\mathcal{O}\left((\rho R)^2\right)\right).
		\end{aligned}
		\end{equation}
		Scattering length is $ a=R $.
	\end{block}
	\begin{block}{Lieb-Liniger model ($v(\cdot)=2c\delta(\cdot)$)}
		Energy behaves asymptotically like
		\begin{equation}
		E_{\text{LL}}(N,L,c)=N\frac{\pi^2}{3}\rho^2\left(1-4\rho/c+\mathcal{O}\left((\rho/c)^2\right)\right),
		\end{equation}
		with scattering length $ a=-\frac{2}{c} $.
	\end{block}
\end{frame}



\section{Fermions}
\begin{frame}
	\frametitle{Spinless/spin-polarized fermions}
	Spinless Fermions are unitarily equivalent to Bosons with a zero b.c. at all planes of intersection, \ie with an infinite delta potential. As a consequence we have the following corollary. 
	\begin{theorem}[A.-Reuvers-Solovej 2022]\label{TheoremFermion}
		Consider a 1D Fermi gas with repulsive interaction  $v=v_{\text{reg}}+v_{\text{h.c.}}$ as defined before. Let $ E_F(N,L)$ be its associated ground state energy. Write $\rho=N/L$. For $\rho a_o$ and $\rho R_0$ sufficiently small, the ground state energy can be expanded as 
		\begin{equation}
		E_F(N,L)=N\frac{\pi^2}{3}\rho^2\left(1+2\rho a_o+\mathcal{O}\left(\left(\rho R_0\right)^{6/5}+N^{-2/3}\right)\right),
		\end{equation}
		where $ a_o\geq0 $ is the odd wave scattering length of $v$. 
	\end{theorem} 
	This is consistent with lower bound $ E_F(N,L)\geq E_0 $, since $ a_o\geq 0 $.
\end{frame}

\begin{frame}
	\frametitle{A conjecture for spin-1/2 fermions}
	\vspace*{-0.2cm}
	Two solvable model for spin-1/2 fermion:
	\vspace*{-0.1cm}
	\begin{block}{\normalsize The hard core gas}
		\small Ground state energy is independent of spin
		so \begin{equation}
		E_{\text{hard core}}(N,L)=N\frac{\pi^2}{3}\rho^2 (1-NR/L)^{-2}\approx E_0(1+2\rho R).
		\end{equation}
		Scattering length is $ a_e=a_o=R $.
	\end{block}
\vspace*{-0.2cm}
	\begin{block}{\normalsize Yang-Gaudin model}
		\small Is the spin-1/2 version of the LL model, \ie $ H_{\text{YG}}=H_{\text{LL}} $.
		 Behaves asymptotically like
		 \begin{equation}
		 E_{\text{YG}}(N,L,c)=N\frac{\pi^2}{3}\rho^2\left(1-4\rho\ln(2)/c+\mathcal{O}\left((\rho/c)^2\right)\right),
		 \end{equation}
		 with scattering length $ a_e=-\frac{2}{c} $, $ a_o=0 $.
		\end{block}
\end{frame}
\begin{frame}
	\frametitle{A conjecture for spin-1/2 fermions}
	Based on the two solvable cases: \begin{block}{Conjecture}
		\small\begin{equation}
		\begin{aligned}
		E(N,L)=N\frac{\pi^2}{3}\rho^2\Big(1+2\ln(2)\rho a_e+2(1-\ln(2))\rho a_o\\+\mathcal{O}\left((\rho\max(\abs{a_e},a_o))^2\right)\Big)
		\end{aligned}
		\end{equation}
	\end{block}
\end{frame}

\begin{frame}
	\centering{Thanks for your attention!}
\end{frame}


\end{document}
