\documentclass[a4paper, 11pt]{article} % A4 paper size, default 11pt font size and oneside for equal margins

\usepackage[utf8]{inputenc} % Required for inputting international characters
\usepackage[T1]{fontenc} % Output font encoding for international characters
\usepackage{url}
\usepackage{amsmath,amssymb,amsfonts}
\title{Essay for RCR course\\
	\large Transparency of mathematics}
\author{Johannes Agerskov}
\begin{document}
	\maketitle
\subsection*{Description of my PhD project}
I am a PhD student at the Institute for Mathematical Science, specifically I work at the Centre for the Mathematics of Quantum Theory (QMATH). My project concerns the stability of the fermionic gas with zero-range interactions. In particular, I seek to investigate the ground-state energy of such systems. It is well known, see for example \cite{PhysRev.47.903}, that the Bose-gas, \emph{i.e.} a gas of bosons, with zero-range interactions is unstable in the sense that the ground-state cannot exist. Thus, such a gas, if physically realised, would continue decaying. This phenomenon, known as the \emph{Thomas effect} or the \emph{Thomas collapse}, is specific to bosons, and it is likely that such a collapse would not occur for fermions. In fact these gases have already been experimentally constructed, \cite{PhysRevLett.105.070402}, and it is thus expected that the mathematical model describing them, would be stable. The goal of my project is thus to construct a mathematical proof of stability \emph{or} instability of the Fermi-gas with zero-range interactions. A proof of instability would of course raise suspicion as to whether the mathematical model accurately describes the physically realised gases or not.

\subsection*{Good vs questionable scientific practise}
In the RCR course we focused, for good reasons, mainly on scientific misconduct, authorship, data handling, and conflicts of interest. Of course we also learned about intellectual property rights, however as mathematical and theoretical discoveries cannot be patented, it seems unlikely that this will be an issue during my PhD project. Data handling constitutes likewise a minimal part of my project. Scientific misconduct, authorship and conflicts of interest thus appears to be the most relevant subject of the RCR curriculum for my project. However, I will in this essay discuss another part of the RCR curriculum, which to me seems like the most relevant during my project, namely questionable scientific practise. The reason for me to deem this the most relevant part of the curriculum in the context of my project, lies with the fact that violations of good scientific practise in this case \emph{questionable scientific practise} can occur involuntarily, and I shall argue below how these types of violations might be inevitable to some extend in mathematics and theoretical physics. Questionable scientific practise is defined in the RCR course curriculum as "research which undermines research integrity – breaching principles of honesty, transparency, and
accountability – without amounting to research misconduct", \cite{4d9ee36a3be642848d14f8517fe0f94c} page 19. Thus questionable scientific practise may not be plagiarism, fabrication, falsification, or it may not be done wilfully or with grossly negligence in order for it not to amount to research misconduct \cite{4d9ee36a3be642848d14f8517fe0f94c} page 28 BOX 2. However by the BOX 7 on page 22 of \cite{4d9ee36a3be642848d14f8517fe0f94c} breaches of research integrity may be committed in may other ways. For example by not being transparent in your reasoning. Transparency is indeed what I will discuss below in relation to my own project.
\subsection*{Transparency of mathematics and theoretical physics}
Mathematics and to some degree theoretical physics relies entirely on pure reasoning and arguments. Thus it is important that the arguments or proofs presented in mathematical papers will be of great transparency, in order for the peers to be able to follow every step. However, it seems to me that the common practice within mathematics and theoretical physics dictates that one leaves out certain steps of the argumentation. This practise is not without reason, as mathematical papers and proofs have had increasing complexity throughout the history of the field. Thus one may argue that the author should contemplate the (human) readability of his paper, as it is in the interest of the author and his peers that the paper will be enjoyable to read. However, due to the great complexity of modern mathematics this is a fine balance. One might imagine that giving all details of a proof in a paper would not only make the paper incredibly long, but also obscure which points in the proof that were non-trivial realisations. On the other hand leaving out to much information may obscure the proof itself and make it hard to verify. As a concrete example I have been reading a paper the last few weeks, and in the second theorem of the paper, they have a phrase: "\emph{In particular, if} $ \Lambda(m)< 1 $\emph{, then }$ D(\Gamma)=D(L)=H^1_{\text{as}}(\mathbb{R}^{3(N-1)}) $." \cite{Moser_2017}. However, this does not seem to follow from the rest of the theorem which is otherwise implied by "\emph{In particluar, [...]}". This has led to discussions with my supervisor as well as postdocs in my group for the past three weeks. All of which would have been unnecessary if the authors had been more transparent in their argumentation.\\
As a consequence I will during my PhD project reflect on the transparency of my work, and maybe even keep a completely transparent version of any work I do, which can then be provided upon request by any potential reader.


\bibliographystyle{vancouver}
\bibliography{bibtex}
\end{document}

