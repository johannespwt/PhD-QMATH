\documentclass[a4paper,11pt]{article}
\usepackage[utf8]{inputenc}
\usepackage[margin=1in]{geometry}
\usepackage{pdfpages}
\usepackage{mathrsfs}
\usepackage{amsfonts}
\usepackage{amsmath}
\DeclareMathOperator\arctanh{arctanh}
\usepackage{amssymb}
\usepackage{bbm}
\usepackage{amsthm}
\usepackage{graphicx}
\usepackage{centernot}
\usepackage{caption}
\usepackage{subcaption}
\usepackage{braket}
\usepackage{pgfplots}
\usepackage{lastpage}
\usepackage{enumitem}
\usepackage{setspace}
\usepackage{xcolor}
\usepackage[english]{babel} 

\usepackage[square,sort,comma,numbers]{natbib}
\usepackage[colorlinks=true,linkcolor=blue]{hyperref}

\usepackage{fancyhdr}
\newcommand{\euler}[1]{\text{e}^{#1}}
\newcommand{\Real}{\text{Re}}
\newcommand{\Imag}{\text{Im}}
\newcommand{\supp}{\text{supp}}
\newcommand{\norm}[1]{\left\lVert #1 \right\rVert}
\newcommand{\abs}[1]{\left\lvert #1 \right\rvert}
\newcommand{\floor}[1]{\left\lfloor #1 \right\rfloor}
\newcommand{\Span}[1]{\text{span}\left(#1\right)}
\newcommand{\dom}[1]{\mathscr D\left(#1\right)}
\newcommand{\Ran}[1]{\text{Ran}\left(#1\right)}
\newcommand{\conv}[1]{\text{co}\left\{#1\right\}}
\newcommand{\Ext}[1]{\text{Ext}\left\{#1\right\}}
\newcommand{\vin}{\rotatebox[origin=c]{-90}{$\in$}}
\newcommand{\interior}[1]{%
	{\kern0pt#1}^{\mathrm{o}}%
}
\newcommand*\diff{\mathop{}\!\mathrm{d}}
\newcommand{\ie}{\emph{i.e.} }
\newcommand{\eg}{\emph{e.g.} }
\newcommand{\dd}{\partial }
\newcommand{\R}{\mathbb{R}}
\newcommand{\N}{\mathbb{N}}
\newcommand{\C}{\mathbb{C}}
\newcommand{\w}{\mathsf{w}}
\newcommand{\loc}{\text{loc}}

\newcommand{\Gliminf}{\Gamma\text{-}\liminf}
\newcommand{\Glimsup}{\Gamma\text{-}\limsup}
\newcommand{\Glim}{\Gamma\text{-}\lim}

\newtheorem{theorem}{Theorem}
\newtheorem{definition}{Definition}
\newtheorem{proposition}{Proposition}
\newtheorem{lemma}{Lemma}
\newtheorem{corollary}{Corollary}

\numberwithin{equation}{section}
\linespread{1.3}

\pagestyle{fancy}
\fancyhf{}
\rhead{Mandatory Assigment 1 - Diffun}
\lhead{Johannes Agerskov}
\rfoot{\thepage}
\lfoot{Dated: \today}
\author{Johannes Agerskov}
\date{Dated: \today}
\title{Mandatory Assigment 1 - Diffun}
\begin{document}
\maketitle	

\section*{Ex 1}
Let $ S $ be a symmetric and densely defined operator on a Hilbert space $ H $, such that $ S $ has a bounded, everywhere defined, inverse. Then $ S $ is self-adjoint.
\begin{proof}
	Since $ S^{-1} $ is everywhere defined we have $ \Ran{S}=H $, which is easily seen by noticing that for any $ x\in H $ we have $ SS^{-1}x=x $ and therefore $ x\in \Ran{S} $. It follows that $ \ker{S^*}=\Ran{S}^\perp=\{0\} $ and $ S^* $ is therefore injective. Furthermore, since $ S $ is symmetric we have by definition that $ S\subset S^* $ from which is trivially follows that $ \Ran{S}=H\subset\Ran{S^*} $. Thus $ \Ran{S^*}=H $, and we conclude that $ S^* $ has a bounded everywhere defined inverse, $ (S^*)^{-1} $. However, we notice that $ (S^*)^{-1}=S^{-1} $, by the following observation: Clearly we have $ S^*(S^{-1}x)=S(S^{-1}x)=x $ for any $ x\in H $, since $ S^* $ is an extension of $ S $. On the contrary we also have $ S^*(S^{-1}(S^*y))=S^*y $ for any $ y\in\mathcal{D}(S^*) $ by the previous observation, but by injectivity of $ S^* $ it then follows that $ S^{-1}S^*y=y $. Thus $ S^{-1} $ is exactly the unique two-sided inverse of $ S $, and we conclude that $ (S^*)^{-1}=S^{-1} $. It then follows that $ \mathcal{D}(S)=S^{-1}H=\mathcal{D}(S^*) $, such that $ S^*=S $ and $ S $ is self-adjoint. Notice that nowhere did we use that $ S^{-1} $ is bounded, and this follows automatically, in this case, if it is everywhere defined.
\end{proof}

\section*{Ex 2, Exam 2016, Q1}
\subsection*{(a)}
We show that for all $ u\in\mathcal{D}'(\R) $ we have $ (\euler{x^3/3}u)'=e^{x^3/3}(u'+x^2u) $. We furhermore use this conclude that any solution of $ u'+x^2u $ in $ \mathcal{D}'(\R) $ has the form $ u(x)=C\euler{-x^3/3} $.
\begin{proof}
	By the usual Leibniz rule for distributions and $ C^\infty $ functions, lemma 3.7, we have \begin{equation}
	(\euler{x^3/3}u)'=(\euler{x^3/3})'u+\euler{x^3/3}u'=\euler{x^3/3}(x^2 u+u')
	\end{equation}
	which is the desired result. Now notice that $ u'+x^2u=0 $ if and only if $ (\euler{x^3/3}u)'=0 $, since $ \euler{x^3/3} $ is non-zero. However, $ \euler{x^3/3}u $ is a distribution, and by theorem 4.19, we have that $ (\euler{x^3/3}u )'=0 $ if and only if $ \euler{x^3/3}u =C $ for some constant $ C\in\C $. Multiplying both sides of this relation by $ \euler{-x^3/3} $ we get $ u=C\euler{-x^3/3} $. Here we of cause used that $ f(gu)=(fg)u $ for $ f,g\in C^\infty(\R) $ and $ u\in\mathcal{D}'(\R) $, which is obvious by definition of $ (fg)u $ and $ f(gu) $ as adjoints of multiplication with $ C^\infty $ functions on $ C^\infty_0(\R) $, and by comutativity of products of functions. This concludes the proof.
\end{proof}
\subsection*{(b)}
We show that $ u=H(x)\euler{-x^3/3} $ solves \begin{equation} \label{eq1}
u'+x^2u=\delta_0,
\end{equation}
and we find all solutions to \eqref{eq1} in $ \mathcal{D}(\R) $.
By the usual Leibniz rule for distributions and $ C^\infty $ functions, lemma 3.7, we have \begin{equation}
u'=(H(x)\euler{-x^3/3})'=\euler{-x^3/3}((H(x))'-x^2 H(x))=\euler{-x^3/3}(\delta_0-x^2H(x))=\delta_0-x^2 u
\end{equation}
or equivalently that $ u'+x^2u=\delta_0 $. Here we used that $ (H(x))'=\delta_0 $ as shown on page 33 in the book, eq. (3.23), and we used that $ f\delta_0=f(0)\delta_0 $ for any $ f\in C^{\infty}(\R) $, which is a trivial fact by $ \braket{f\delta_0,\varphi}=\braket{\delta_0,f\varphi}=f(0)\varphi(0)=f(0)\braket{\delta_0,\varphi}=\braket{f(0)\delta_0,\varphi} $ for any $ f\in C^\infty(\R) $ and any $ \varphi\in C^{\infty}_0(\R) $.
Now let $ u_1 $ and $ u_2 $ be solutions of \eqref{eq1}. Then clearly $ (u_1-u_2)'+x^2(u_1-u_2)=\delta_0-\delta_0=0 $. Therefore, by (a) we have that $ u_1-u_2=C\euler{-x^3/3} $ for some $ C\in \C $. Hence, it is seen that all solutions of $ \eqref{eq1} $ must be of the form $ u=H(x)\euler{-x^3/3}+C\euler{-x^3/3}=(H(x)+C)\euler{-x^3/3} $ for some $ C\in \C $.



\section*{Ex 3 Exam 2016, Q2}
Consider the unbounded operator $ A $ on $ L^2(\R) $ defined by $ A=-\frac{\diff^2}{\diff x^2}+1 $ on the domain $ D(A)=C^\infty_0(\R\setminus\{-1,1\}) $.
\subsection*{(a)} 
We show that $ A $ is densely defined symmetric operator.
That $ \overline{D(A)}=L^2(\R) $ follows from from Theorem 2.15 ($ 3^\circ $) with $ \Omega=\R\setminus\{-1,1\} $ which is open, and $ p=2 $. That $ A $ is symmetric can be seen by the calculation: Let $ u,v\in D(A) $\begin{equation}
(Av,u)=\int_{\R} \left(-\frac{\diff^2}{\diff x^2}+1\right)v(x)\overline{u(x)} \diff x,
\end{equation}
where the integral makes sence, since $ \left(-\frac{\diff^2}{\diff x^2}+1\right)v $ and $ u $ are a.e. equal to $ L^2(\R) $ functions. For the sake of simplicity we will, without loss of generality, define $ v=u=0 $ on $ \{-1,1\} $, such that they are continuous. Since $ v $ and $ u $ have compact support, there exist an $ N\in \N $ such that $ \supp(u)\cap \supp(v)\subset [-N+1,N-1] $. But then we have \begin{equation}
\begin{aligned}
\int_{\R} \left(-\frac{\diff^2}{\diff x^2}+1\right)v(x)\overline{u(x)} \diff x&=\int_{[-N,N]} \left(-\frac{\diff^2}{\diff x^2}+1\right)v(x)\overline{u(x)}\diff x\\
&=\int_{-N}^{N}\left(-\frac{\diff^2}{\diff x^2}+1\right)v(x)\overline{u(x)} \diff x = \int_{\R} \left(-\frac{\diff^2}{\diff x^2}+1\right)v(x)\overline{u(x)} \diff x\\
&=\int_{[-N,N]} \left(-\frac{\diff^2}{\diff x^2}+1\right)v(x)\overline{u(x)}\diff x=\int_{-N}^{N}v(x)\left(-\frac{\diff^2}{\diff x^2}+1\right)\overline{u(x)} \diff x\\
&=\int_{-N}^{N}v(x)\overline{\left(-\frac{\diff^2}{\diff x^2}+1\right)u(x)} \diff x=\int_{[-N,N]}v(x)\overline{\left(-\frac{\diff^2}{\diff x^2}+1\right)u(x)} \diff x\\
&=\int_{\R}v(x)\overline{\left(-\frac{\diff^2}{\diff x^2}+1\right)u(x)} \diff x=(v,Au)
\end{aligned}
\end{equation}
where we used that we can convert the Lebesgue integrals of continuous functions over closed bounded intervalt to Riemann integrals, and also back again, and we used partial integration twice in the third line, together with the fact, that no boundary terms will contribute as the integrand is identically zero in a neighborhood of the boundary. This shows that $ A $ is symmetric.

\subsection*{(b)}
We show that $ u\in\ker(A^*) $ if and only if $ u\in L^2(\R) $ and the distribution $ u''-u\in\mathcal{D}'(\R) $ has support in $ \{-1,1\} $.
\end{document}