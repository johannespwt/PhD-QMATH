\documentclass[a4paper,11pt]{article}
\usepackage[utf8]{inputenc}
\usepackage[margin=1in]{geometry}
\usepackage{pdfpages}
\usepackage{mathrsfs}
\usepackage{amsfonts}
\usepackage{amsmath}
\DeclareMathOperator\arctanh{arctanh}
\usepackage{amssymb}
\usepackage{bbm}
\usepackage{amsthm}
\usepackage{graphicx}
\usepackage{centernot}
\usepackage{caption}
\usepackage{subcaption}
\usepackage{braket}
\usepackage{pgfplots}
\usepackage{lastpage}
\usepackage{enumitem}
\usepackage{setspace}
\usepackage{xcolor}
\usepackage[english]{babel} 

\usepackage[square,sort,comma,numbers]{natbib}
\usepackage[colorlinks=true,linkcolor=blue]{hyperref}

\usepackage{fancyhdr}
\newcommand{\euler}[1]{\text{e}^{#1}}
\newcommand{\Real}{\text{Re}}
\newcommand{\Imag}{\text{Im}}
\newcommand{\supp}{\text{supp}}
\newcommand{\norm}[1]{\left\lVert #1 \right\rVert}
\newcommand{\abs}[1]{\left\lvert #1 \right\rvert}
\newcommand{\floor}[1]{\left\lfloor #1 \right\rfloor}
\newcommand{\Span}[1]{\text{span}\left(#1\right)}
\newcommand{\dom}[1]{\mathscr D\left(#1\right)}
\newcommand{\Ran}[1]{\text{Ran}\left(#1\right)}
\newcommand{\conv}[1]{\text{co}\left\{#1\right\}}
\newcommand{\Ext}[1]{\text{Ext}\left\{#1\right\}}
\newcommand{\vin}{\rotatebox[origin=c]{-90}{$\in$}}
\newcommand{\interior}[1]{%
	{\kern0pt#1}^{\mathrm{o}}%
}
\newcommand*\diff{\mathop{}\!\mathrm{d}}
\newcommand{\ie}{\emph{i.e.} }
\newcommand{\eg}{\emph{e.g.} }
\newcommand{\dd}{\partial }
\newcommand{\R}{\mathbb{R}}
\newcommand{\N}{\mathbb{N}}
\newcommand{\C}{\mathbb{C}}
\newcommand{\w}{\mathsf{w}}
\newcommand{\loc}{\text{loc}}

\newcommand{\Gliminf}{\Gamma\text{-}\liminf}
\newcommand{\Glimsup}{\Gamma\text{-}\limsup}
\newcommand{\Glim}{\Gamma\text{-}\lim}

\newtheorem{theorem}{Theorem}
\newtheorem{definition}{Definition}
\newtheorem{proposition}{Proposition}
\newtheorem{lemma}{Lemma}
\newtheorem{corollary}{Corollary}

\numberwithin{equation}{section}
\linespread{1.3}

\pagestyle{fancy}
\fancyhf{}
\rhead{Mandatory Assigment 1 - Diffun}
\lhead{Johannes Agerskov}
\rfoot{\thepage}
\lfoot{Dated: \today}
\author{Johannes Agerskov}
\date{Dated: \today}
\title{Mandatory Assigment 1 - Diffun}
\begin{document}
\maketitle	

\section*{Ex 1}
\setcounter{section}{1}
Let $ S $ be a symmetric and densely defined operator on a Hilbert space $ H $, such that $ S $ has a bounded, everywhere defined, inverse. Then $ S $ is self-adjoint.
\begin{proof}
	Since $ S^{-1} $ is everywhere defined we have $ \Ran{S}=H $, which is easily seen by noticing that for any $ x\in H $ we have $ S(S^{-1}x)=x $ and therefore $ x\in \Ran{S} $. It follows that $ \ker{S^*}=\Ran{S}^\perp=\{0\} $ and $ S^* $ is therefore injective. Furthermore, since $ S $ is symmetric we have by definition that $ S\subset S^* $ from which is trivially follows that $ \Ran{S}=H\subset\Ran{S^*} $. Thus $ \Ran{S^*}=H $, and we conclude that $ S^* $ has an everywhere defined inverse, $ (S^*)^{-1} $ (which is bounded by the closed graph theorem, since the inverse of a closed operator is closed). However, we notice that $ (S^*)^{-1}=S^{-1} $, by the following observation: Clearly we have $ S^*(S^{-1}x)=S(S^{-1}x)=x $ for any $ x\in H $, since $ S^* $ is an extension of $ S $. On the contrary we also have $ S^*(S^{-1}(S^*y))=S^*y $ for any $ y\in\mathcal{D}(S^*) $ by the previous observation, but by injectivity of $ S^* $ it then follows that $ S^{-1}S^*y=y $. Thus $ S^{-1} $ is exactly the unique two-sided inverse of $ S $, and we conclude that $ (S^*)^{-1}=S^{-1} $. It then follows that $ D(S)=S^{-1}H=D(S^*) $, so $ S^*=S $ and $ S $ is self-adjoint. Notice that nowhere did we use that $ S^{-1} $ is bounded, and this follows automatically, in this case, if it is everywhere defined, by the closed graph theorem.
\end{proof}

\section*{Ex 2, Exam 2016, Q1}
\setcounter{section}{2}
\subsection*{(a)}
We show that for all $ u\in\mathcal{D}'(\R) $ we have $ (\euler{x^3/3}u)'=e^{x^3/3}(u'+x^2u) $. We furhermore use this to conclude that any solution of $ u'+x^2u=0 $ in $ \mathcal{D}'(\R) $ has the form $ u(x)=C\euler{-x^3/3} $.
\begin{proof}
	By the usual Leibniz rule for distributions and $ C^\infty $ functions, lemma 3.7, we have \begin{equation}
	(\euler{x^3/3}u)'=(\euler{x^3/3})'u+\euler{x^3/3}u'=\euler{x^3/3}(x^2 u+u')
	\end{equation}
	which is the desired result. Now notice that $ u'+x^2u=0 $ if and only if $ (\euler{x^3/3}u)'=0 $, since $ \euler{x^3/3} $ is non-zero. Indeed we have that $ \braket{u'+x^2u,\varphi}=0 $ for all $ \varphi\in C_0^\infty(\R) $, if and only if $ \braket{\euler{x^3/3}(u'+x^2u),\euler{-x^3/3}\varphi}=\braket{u'+x^2u,\varphi}=0 $ for all $ \varphi\in C_0^\infty(\R) $ and since $ M_{\euler{-x^3/3}}\rvert_{C_0^\infty(\R)}: C_0^\infty(\R)\to C_0^\infty(\R) $ is surjective, we see that $ (u'+x^2u)= 0 $ if and only if $ \euler{x^3/3}(u'+x^2u)= 0 $ for all $ \varphi\in C_0^\infty(\R) $. Now, we notice that $ \euler{x^3/3}u $ is a distribution, and by theorem 4.19, we have that $ (\euler{x^3/3}u )'=0 $ if and only if $ \euler{x^3/3}u =C $ for some constant $ C\in\C $. Multiplying both sides of this relation by $ M_{\euler{-x^3/3}}:\mathcal{D}'(\R)\to\mathcal{D}'(\R) $ we get $ u=C\euler{-x^3/3} $. Here we of cause used that $ f(gu)=(fg)u $ for $ f,g\in C^\infty(\R) $ and $ u\in\mathcal{D}'(\R) $, which is obvious by definition of $ (fg)u $ and $ f(gu) $ as adjoints of multiplication with $ C^\infty $ functions on $ C^\infty_0(\R) $, and by comutativity of products of functions. This concludes the proof.
\end{proof}
\subsection*{(b)}
We show that $ u=H(x)\euler{-x^3/3} $ solves \begin{equation} \label{eq1}
u'+x^2u=\delta_0,
\end{equation}
and we find all solutions to \eqref{eq1} in $ \mathcal{D}'(\R) $.
By the usual Leibniz rule for distributions and $ C^\infty $ functions, lemma 3.7, we have \begin{equation}
u'=(H(x)\euler{-x^3/3})'=\euler{-x^3/3}((H(x))'-x^2 H(x))=\euler{-x^3/3}(\delta_0-x^2H(x))=\delta_0-x^2 u
\end{equation}
or equivalently that $ u'+x^2u=\delta_0 $. Here we used that $ (H(x))'=\delta_0 $ as shown on page 33 in the book, eq. (3.23), and we used that $ f\delta_0=f(0)\delta_0 $ for any $ f\in C^{\infty}(\R) $, which is a trivial fact by $ \braket{f\delta_0,\varphi}=\braket{\delta_0,f\varphi}=f(0)\varphi(0)=f(0)\braket{\delta_0,\varphi}=\braket{f(0)\delta_0,\varphi} $ for any $ f\in C^\infty(\R) $ and any $ \varphi\in C^{\infty}_0(\R) $.
Now let $ u_1 $ and $ u_2 $ be solutions of \eqref{eq1}. Then clearly $ (u_1-u_2)'+x^2(u_1-u_2)=\delta_0-\delta_0=0 $. Therefore, by (a) we have that $ u_1-u_2=C\euler{-x^3/3} $ for some $ C\in \C $. Hence, it is seen that all solutions of $ \eqref{eq1} $ must be of the form $ u=H(x)\euler{-x^3/3}+C\euler{-x^3/3}=(H(x)+C)\euler{-x^3/3} $ for some $ C\in \C $.



\section*{Ex 3 Exam 2016, Q2}
\setcounter{section}{3}
Consider the unbounded operator $ A $ on $ L^2(\R) $ defined by $ A=-\frac{\diff^2}{\diff x^2}+1 $ on the domain $ D(A)=C^\infty_0(\R\setminus\{-1,1\}) $.
\subsection*{(a)} 
We show that $ A $ is densely defined symmetric operator.
That $ \overline{D(A)}=L^2(\R) $ follows from from Theorem 2.15 ($ 3^\circ $) with $ \Omega=\R\setminus\{-1,1\} $ which is open, and $ p=2 $. That $ A $ is symmetric can be seen by the calculation: Let $ u,v\in D(A) $\begin{equation}
(Av,u)=\int_{\R} \left(-\frac{\diff^2}{\diff x^2}+1\right)v(x)\overline{u(x)} \diff x,
\end{equation}
where the integral makes sence, since $ \left(-\frac{\diff^2}{\diff x^2}+1\right)v $ and $ u $ are a.e. equal to $ L^2(\R) $ functions. For the sake of simplicity we will, without loss of generality, define $ v=u=0 $ on $ \{-1,1\} $, such that they are continuous. Since $ v $ and $ u $ have compact support, there exist an $ N\in \N $ such that $ \supp(u)\cap \supp(v)\subset [-N+1,N-1] $. But then we have \begin{equation}
\begin{aligned}
\int_{\R} \left(-\frac{\diff^2}{\diff x^2}+1\right)v(x)\overline{u(x)} \diff x&=\int_{[-N,N]} \left(-\frac{\diff^2}{\diff x^2}+1\right)v(x)\overline{u(x)}\diff x\\
&=\int_{-N}^{N}\left(-\frac{\diff^2}{\diff x^2}+1\right)v(x)\overline{u(x)} \diff x\\
&=\int_{-N}^{N}v(x)\left(-\frac{\diff^2}{\diff x^2}+1\right)\overline{u(x)} \diff x\\
&=\int_{-N}^{N}v(x)\overline{\left(-\frac{\diff^2}{\diff x^2}+1\right)u(x)} \diff x\\
&=\int_{[-N,N]}v(x)\overline{\left(-\frac{\diff^2}{\diff x^2}+1\right)u(x)} \diff x\\
&=\int_{\R}v(x)\overline{\left(-\frac{\diff^2}{\diff x^2}+1\right)u(x)} \diff x=(v,Au)
\end{aligned}
\end{equation}
where we used that we can convert the Lebesgue integrals of continuous functions over closed bounded intervalt to Riemann integrals, and also back again, and we used partial integration twice in the third line, together with the fact, that no boundary terms will contribute as the integrand is identically zero in a neighborhood of the boundary. This shows that $ A $ is symmetric.

\subsection*{(b)}
We show that $ u\in\ker(A^*) $ if and only if $ u\in L^2(\R) $ and the distribution $ u''-u\in\mathcal{D}'(\R) $ has support in $ \{-1,1\} $.
\begin{proof}
	We show first the left direction. Let $ u\in L^2(\R) $ such that $ \supp(u''-u)\subset\{-1,1\} $. Then for any $ v\in D(A) $ we have \begin{equation}
	\begin{aligned}
	(u,Av)=\int_{\R}u(x) \overline{\left(-\frac{\diff^2}{\diff x^2}+1\right)v(x)} \diff x&=\int_{\R}u(x) \left(-\frac{\diff^2}{\diff x^2}+1\right)\overline{v(x)} \diff x\\&=\braket{u,-\overline{v}''+\overline{v}}=\braket{ (-u''+u),\overline{v}},
	\end{aligned}
	\end{equation}
	where we identified $ v $ with a $ C^\infty_0(\R) $ function via extension by $ 0 $, and we used the definition of the distributional derivative.
	Now by assumption $ \supp(u''(x)-u)\subset\{-1,1\} $ which, by definition 3.10, is equivalent to $$ \{-1,1\}^\complement\subset\supp(u''-u)^\complement=\bigcup\{w\ \vert\ w\text{ is open and }u''-u=0\text{ on }w \}. $$
	By lemma 3.11 we have that $ u''-u=0 $ on $ \supp(u''-u)^\complement $.
	Thus it clear that \begin{equation}
	(u,Av)=\braket{ (-u''+u),\overline{v}}=0=\braket{0,\bar{v}}, \qquad\text{for any } v\in D(A)
	\end{equation} 
	since we, by extension by $ 0 $, have that $ v\in C^{\infty}_0\left(\supp(u''-u)^\complement\right) $ for any $ v\in D(A) $. This concludes that $ u\in D(A^*) $ and that $ A^*u=0 $, so $ u\in\ker(A^*) $.\\
	For the right direction, notice that if $ u\in \ker(A^*) $ then trivially $ u\in D(A^*)\subset L^2(\R) $. Furthermore, we have \begin{equation}
	(u,Av)=\braket{ (-u''+u),\overline{v}}=0,\quad \text{for any }v\in D(A)=C_0^\infty(\R\setminus\{-1,1\}),
	\end{equation}
	which, by definition 3.10, implies that $ u''-u=0 $ on $ \R\setminus\{-1,1\}=\{-1,1\}^\complement $. Thus \begin{equation}
	\begin{aligned}
	\supp(u''-u)&=\left(\bigcup\{w\ \vert\ w\text{ is open and }u''-u=0\text{ on }w \}\right)^\complement\\&
	=\bigcap\left\{w^\complement \ \vert\ w\text{ is open and }u''-u=0\text{ on }w \right\}\\&
	\subset\left(\{-1,1\}^\complement\right)^\complement=\{-1,1\}.
	\end{aligned}
	\end{equation}
	This concludes the proof.
\end{proof}
\subsection*{(c)}
We find four linearly independt solutions in the subspace $ \ker(A^*)\subset L^2(\R) $, \ie four element $ v_i $, $ i\in\{1,2,3,4\} $ such that $ v_i\in D(A^*) $ and $ A^*v_i=0 $ for all $ i\in\{1,2,3,4\} $.\\
Consider the functions \begin{equation}
\begin{aligned}
f_1(x)&=\exp(x)\mathbbm{1}_{(-\infty,-1)}(x),\\
f_2(x)&=\exp(-x)\mathbbm{1}_{(1,\infty)}(x),\\
f_3(x)&=\exp(x)\mathbbm{1}_{(-1,1)}(x),\\
f_4(x)&=\exp(-x)\mathbbm{1}_{(-1,1)}(x).
\end{aligned}
\end{equation}
Evidently, these are linearly independent since they either have disjoint support except for $ f_3 $ and $ f_4 $, which are independent of each other since they are not a multiple of one another. They are also all in $ L^2(\R) $ since they are exponentially decaying and it hold that $ \norm{f_i}_{L^2(\R)}\leq\norm{\euler{2}\ \exp(-\abs{x})}_{L^2(\R)}<\infty $ for all $ i\in\{1,2,3,4\} $. Furhtermore, we may compute the distributional derivatives, by noticing that \eg that $ f_1+\frac{1}{\euler{}}\mathbbm{1}_{(-1,\infty)} $ satisfies all assumptions in lemma 3.6. Therefore we may conclude that \begin{equation}
f_1'=\left(f_1+\frac{1}{\euler{}}\mathbbm{1}_{(-1,\infty)}\right)'-\frac{1}{\euler{}}\mathbbm{1}_{(-1,\infty)}'
\end{equation}
where $ (f_1+\frac{1}{\euler{}}\mathbbm{1}_{(0,\infty)})'=f_1 $ by lemma 3.6 and the fact that $ (\euler{x})'=\euler{x} $ and $ 1'=0 $. Furthermore, $ \mathbbm{1}_{(-1,\infty)}'=\delta_{-1} $ by a translation of eq (3.23) in the book. Thus we have\begin{equation}
f_1'=f_1-\frac{1}{\euler{}}\delta_{-1}
\end{equation}
Similarly we may compute \begin{equation}
\begin{aligned}
f_2'=\left((f_2+\frac{1}{\euler{}}\mathbbm{1}_{(-\infty,1)}\right)'-\frac{1}{\euler{}}\mathbbm{1}_{(-\infty,1)}'=\left(f_2+\frac{1}{\euler{}}\mathbbm{1}_{(-\infty,1)}\right)'-\frac{1}{\euler{}}(\mathbbm{1}_{(-\infty,1)}-1)'\\=\left(f_2+\frac{1}{\euler{}}\mathbbm{1}_{(-\infty,1)}\right)'+\frac{1}{\euler{}}\mathbbm{1}_{(1,\infty)}'
\end{aligned}
\end{equation}
where we used that $ 1'=0 $, so $ \mathbbm{1}_{(-\infty,1)}'=(\mathbbm{1}_{(-\infty,1)}-1)'=-\mathbbm{1}_{(1,\infty)}' $. Thus we see that \begin{equation}
f_2'=-f_2+\frac{1}{\euler{}}\delta_1
\end{equation}
where we used lemma 3.6 and a translation of eq (3.23) again. For $ f_3 $ we have \begin{equation}
\begin{aligned}
f_3'=\left(f_3+\euler{}\mathbbm{1}_{(1,\infty)}+\frac{1}{\euler{}}\mathbbm{1}_{(-\infty,-1)}\right)'-\left(\euler{}\mathbbm{1}_{(1,\infty)}+\frac{1}{\euler{}}\mathbbm{1}_{(-\infty,-1)}\right)'=f_3-\euler{}\delta_1+\frac{1}{\euler{}}\delta_{-1},
\end{aligned}
\end{equation}
and similarly for $ f_4 $
\begin{equation}
\begin{aligned}
f_4'=\left(f_4+\frac{1}{\euler{}}\mathbbm{1}_{(1,\infty)}+\euler{}\mathbbm{1}_{(-\infty,-1)}\right)'-\left(\frac{1}{\euler{}}\mathbbm{1}_{(1,\infty)}+\euler{}\mathbbm{1}_{(-\infty,-1)}\right)'=-f_4-\frac{1}{\euler{}}\delta_1+\euler{}\delta_{-1},
\end{aligned}
\end{equation}
where we again used lemma 3.6 and translated versions of eq (3.23) of the book.
Thus we can now calculate
\begin{equation}\label{LinEqSys}
	\begin{aligned}
	f_1''-f_1&=\left(f_1-\frac{1}{\euler{}}\delta_{-1}\right)'-f_1=-\frac{1}{\euler{}}\delta_{-1}-\frac{1}{\euler{}}\delta_{-1}',\\
	f_2''-f_2&=\left(-f_2+\frac{1}{\euler{}}\delta_{1}\right)'-f_2=-\frac{1}{\euler{}}\delta_1+\frac{1}{\euler{}}\delta_1',\\
	f_3''-f_3&=\left(f_3-\euler{}\delta_1+\frac{1}{\euler{}}\delta_{-1}\right)'-f_3=-\euler{}\delta_1'+\frac{1}{\euler{}}\delta_{-1}'-\euler{}\delta_1+\frac{1}{\euler{}}\delta_{-1},\\
	f_4''-f_4&=\left(-f_4-\frac{1}{\euler{}}\delta_1+\euler{}\delta_{-1}\right)'-f_4=-\frac{1}{\euler{}}\delta_1'+\euler{}\delta_{-1}'+\frac{1}{\euler{}}\delta_1-\euler{}\delta_{-1}.
	\end{aligned}
\end{equation}
%
% \begin{equation}
%f_1''-f_1=\left(f_1-\frac{1}{\euler{}}\delta_{-1}\right)'-f_1=-\frac{1}{\euler{}}\delta_{-1}-\frac{1}{\euler{}}\delta_{-1}',
%\end{equation}
%\begin{equation}
%f_2''-f_2=\left(-f_2+\frac{1}{\euler{}}\delta_{1}\right)'-f_2=-\frac{1}{\euler{}}\delta_1+\frac{1}{\euler{}}\delta_1',
%\end{equation}
%\begin{equation}
%f_3''-f_3=\left(f_3-\euler{}\delta_1+\frac{1}{\euler{}}\delta_{-1}\right)'-f_3=-\euler{}\delta_1'+\frac{1}{\euler{}}\delta_{-1}'-\euler{}\delta_1+\frac{1}{\euler{}}\delta_{-1},
%\end{equation}
%\begin{equation}
%f_4''-f_4=\left(-f_4-\frac{1}{\euler{}}\delta_1+\euler{}\delta_{-1}\right)'-f_4=-\frac{1}{\euler{}}\delta_1'+\euler{}\delta_{-1}'+\frac{1}{\euler{}}\delta_1-\euler{}\delta_{-1}.
%\end{equation}
Since we have that $ \supp(\delta_x)=\supp(\delta_x')=\{x\} $, which is clear from the facts $ \delta_x\neq0,\ \delta_x'\neq 0 $, and $ \delta_x|_{\R\setminus\{x\}}=\delta_x'|_{\R\setminus\{x\}}=0 $, we conclude that all $ f_i $, $ i\in\{1,2,3,4\} $ satisfies that $ f_i\in L^2(\R) $ and $ \supp(f_i''-f_i)\subset \{-1,1\} $. Therefore, by (b), we find that $ f_i $ with $ i\in\{1,2,3,4\} $ are four linearly independent solutions in $ \ker A^* $.
\subsection*{(d)}
We find an element  $u\in\ker(A^*)$ that satisfies $ -u''+u=\delta_1+\delta_{-1} $. We find this solution in $ \Span{\{f_1,f_2,f_3,f_4\}} $. Let $ u=a_1f_1+a_2f_2+a_3f_3+a_4f_4 $, we may use the system of equations \eqref{LinEqSys}, to conclude that \begin{equation}
-u''+u=\begin{pmatrix}
\delta_{-1}&\delta_{1}&\delta_{-1}'&\delta_1'
\end{pmatrix}
\begin{pmatrix}
\frac{1}{\euler{}}&0&-\frac{1}{\euler{}}&\euler{}\\
0&\frac{1}{\euler{}}&\euler{}&-\frac{1}{\euler{}}\\
\frac{1}{\euler{}}&0&-\frac{1}{\euler{}}&-\euler{}\\
0&-\frac{1}{\euler{}}&\euler{}&\frac{1}{\euler{}}
\end{pmatrix}
\begin{pmatrix}
a_1\\a_2\\a_3\\a_4
\end{pmatrix}.
\end{equation}
Seeking solutions of the form $ -u''+u=\delta_{-1}+\delta_{1} $ is thus equivalent to solving the equation \begin{equation}
\begin{pmatrix}
\delta_{-1}&\delta_{1}&\delta_{-1}'&\delta_1'
\end{pmatrix}
\begin{pmatrix}
\frac{1}{\euler{}}&0&-\frac{1}{\euler{}}&\euler{}\\
0&\frac{1}{\euler{}}&\euler{}&-\frac{1}{\euler{}}\\
\frac{1}{\euler{}}&0&-\frac{1}{\euler{}}&-\euler{}\\
0&-\frac{1}{\euler{}}&\euler{}&\frac{1}{\euler{}}
\end{pmatrix}
\begin{pmatrix}
a_1\\a_2\\a_3\\a_4
\end{pmatrix}=\begin{pmatrix}
\delta_{-1}&\delta_{1}&\delta_{-1}'&\delta_1'
\end{pmatrix}\begin{pmatrix}
1\\1\\0\\0
\end{pmatrix}
\end{equation}
which by linear independent of $ \delta_{-1} $, $ \delta_{1} $, $ \delta_{-1}' $, and $ \delta_1' $ is again equivalent to \begin{equation}
\begin{pmatrix}
\frac{1}{\euler{}}&0&-\frac{1}{\euler{}}&\euler{}\\
0&\frac{1}{\euler{}}&\euler{}&-\frac{1}{\euler{}}\\
\frac{1}{\euler{}}&0&-\frac{1}{\euler{}}&-\euler{}\\
0&-\frac{1}{\euler{}}&\euler{}&\frac{1}{\euler{}}
\end{pmatrix}
\begin{pmatrix}
a_1\\a_2\\a_3\\a_4
\end{pmatrix}=\begin{pmatrix}
1\\1\\0\\0
\end{pmatrix}
\end{equation}
Subtracting row $ 3 $ from row $ 1 $ we get \begin{equation}\label{1-3}
2\euler{}a_4=1.
\end{equation}
Adding row $ 2 $ and $ 4 $ we get \begin{equation}\label{2+4}
2\euler{}a_3=1.
\end{equation}
Adding row $ 1 $ and $ 3 $ we get the equation \begin{equation}
\frac{2}{\euler{}}(a_1-a_3)=\frac{2}{\euler{}}a_1-\frac{1}{\euler{2}}=1,
\end{equation}
where we used \eqref{2+4} in the first equality. Subtracting row $ 4 $ from row $ 2 $ we get\begin{equation}
\frac{2}{\euler{}}(a_2-a_4)=\frac{2}{\euler{}}a_2-\frac{1}{\euler{2}}=1,
\end{equation}
where we used \eqref{1-3} in the first equality. Thus we find \begin{equation}
a_1=a_2=\frac{1}{2}\left(\euler{}+\frac{1}{\euler{}}\right)\text{  and  } a_3=a_4=\frac{1}{2\euler{}},
\end{equation}
which corresponds to \begin{equation}
u(x)=\frac{1}{2}\left(\euler{}+\frac{1}{\euler{}}\right)\euler{-\abs{x}}\ \mathbbm{1}_{\{\abs{x}>1\}}+\frac{1}{2\euler{}}(\euler{x}+\euler{-x})\mathbbm{1}_{\{\abs{x}<1\}}.
\end{equation}
This $ u $ constitutes a solution of the desired form.
\end{document}