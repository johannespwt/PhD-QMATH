\documentclass[a4paper,11pt]{article}
\usepackage[utf8]{inputenc}
\usepackage[margin=1in]{geometry}
\usepackage{pdfpages}
\usepackage{mathrsfs}
\usepackage{amsfonts}
\usepackage{amsmath}
\DeclareMathOperator\arctanh{arctanh}
\usepackage{amssymb}
\usepackage{bbm}
\usepackage{amsthm}
\usepackage{graphicx}
\usepackage{centernot}
\usepackage{caption}
\usepackage{subcaption}
\usepackage{braket}
\usepackage{pgfplots}
\usepackage{lastpage}
\usepackage{enumitem}
\usepackage{setspace}
\usepackage{xcolor}
\usepackage[english]{babel} 

\usepackage[square,sort,comma,numbers]{natbib}
\usepackage[colorlinks=true,linkcolor=blue]{hyperref}

\usepackage{fancyhdr}
\newcommand{\euler}[1]{\text{e}^{#1}}
\newcommand{\Real}{\text{Re}}
\newcommand{\Imag}{\text{Im}}
\newcommand{\supp}{\text{supp}}
\newcommand{\norm}[1]{\left\lVert #1 \right\rVert}
\newcommand{\abs}[1]{\left\lvert #1 \right\rvert}
\newcommand{\floor}[1]{\left\lfloor #1 \right\rfloor}
\newcommand{\Span}[1]{\text{span}\left(#1\right)}
\newcommand{\dom}[1]{\mathscr D\left(#1\right)}
\newcommand{\Ran}[1]{\text{Ran}\left(#1\right)}
\newcommand{\conv}[1]{\text{co}\left\{#1\right\}}
\newcommand{\Ext}[1]{\text{Ext}\left\{#1\right\}}
\newcommand{\vin}{\rotatebox[origin=c]{-90}{$\in$}}
\newcommand{\interior}[1]{%
	{\kern0pt#1}^{\mathrm{o}}%
}
\newcommand*\diff{\mathop{}\!\mathrm{d}}
\newcommand{\ie}{\emph{i.e.} }
\newcommand{\eg}{\emph{e.g.} }
\newcommand{\dd}{\partial }
\newcommand{\R}{\mathbb{R}}
\newcommand{\N}{\mathbb{N}}
\newcommand{\C}{\mathbb{C}}
\newcommand{\w}{\mathsf{w}}
\newcommand{\loc}{\text{loc}}

\newcommand{\Gliminf}{\Gamma\text{-}\liminf}
\newcommand{\Glimsup}{\Gamma\text{-}\limsup}
\newcommand{\Glim}{\Gamma\text{-}\lim}

\newtheorem{theorem}{Theorem}
\newtheorem{definition}{Definition}
\newtheorem{proposition}{Proposition}
\newtheorem{lemma}{Lemma}
\newtheorem{corollary}{Corollary}

\numberwithin{equation}{section}
\linespread{1.3}

\pagestyle{fancy}
\fancyhf{}
\rhead{Assigment 1 - Diffun}
\lhead{Johannes Agerskov}
\rfoot{\thepage}
\lfoot{Dated: \today}
\author{Johannes Agerskov}
\date{Dated: \today}
\title{Assigment 1 -Diffun}
\begin{document}
\maketitle	
\section*{Ex 1}
Let $ \varphi\in C^\infty_0(\R^2) $ and $ u\in\mathcal{D}'(\R) $. We show that $ f(x)=\braket{u,\varphi(x,\cdot)} $ defines a function in $ C^\infty_0(\R) $ with $ f'(x)=\braket{u,\partial_x\varphi(x,\cdot)} $.
\begin{proof}
	We first of all notice that in the product topology the projection maps $ \pi_1: \R^2\ni(x,y)\mapsto x\in \R $ and $ \pi_2: \R^2\ni(x,y)\mapsto y\in \R $ are continuous. Therefore, $ \pi_1(\supp(\varphi))\subset \R $ and $ \pi_2(\supp(\varphi))\subset \R $ are compact sets, as they are images of compact sets under continuous maps.	
	 Now clearly, we have $ \supp(\varphi(x,\cdot))\subset \pi_2(\supp(\varphi)) $ for all $ x\in \R $, so $ \supp(\varphi(x,\cdot)) $ is closed subset of a compact set, and therefore $ \varphi(x,\cdot) $ has compact support for all $ x\in \R $. Furthermore, $ \varphi(x,\cdot) $ is a $ C^\infty $ function, since all derivatives are equal to some partial derivative of $ \varphi $ which is continuous by assumtion. Hence $ \varphi(x,\cdot)\in C^{\infty}_0(\R) $ for all $ x\in \R $ and, $ f $ is well-defined. Now by a similar argument we have that $ \supp(\varphi(\cdot,y))\subset\pi_1\supp(\varphi) $ for all $ y\in R $ and therefore $ \varphi(x,\cdot)\neq0 $ only if $ x\in\pi_1\supp(\varphi) $. Therefore, we may conclude that $ \supp(f(x))\subset \pi_1(\supp(\varphi)) $. Thus $ \supp(f(x)) $ is a closed subset of a compact set, hence it is compact.\\
	 Thus we know that $ f(x) $ is well-defined and have compact support. to show that $ f $ is a $ C^\infty $ function. We compute the difference coefficient for $ f $\begin{equation}
	 \frac{f(x+h)-f(x)}{h}=\left\langle u, \frac{\varphi(x+h,\cdot)-\varphi(x,\cdot)}{h}\right\rangle,
	 \end{equation}
	 where we used linearity of $ \braket{u,\cdot} $.
	 Now we show that for any sequence $ h_n $, such that $ h_n\to 0 $, we have $ \frac{\varphi(x+h_n,\cdot)-\varphi(x,\cdot)}{h_n}\to\partial_x\varphi(x,\cdot) $ in $ C^\infty_0(\R) $. This is seen by the mean value theorem, since $ \frac{\varphi(x+h_n,\cdot)-\varphi(x,\cdot)}{h_n}=\partial_x\varphi(x+\xi_n(x,h_n,\cdot),\cdot) $ for some $ 0\leq\xi(x,h_n,\cdot)\leq h_n $. However, since we by the above argument have that $ \supp(\varphi(x,\cdot))\in\pi_2\supp(\varphi)\subset\subset\R $ for all $ x\in\R $, we see that $ \supp\left(\frac{\varphi(x+h_n,\cdot)-\varphi(x,\cdot)}{h_n}\right)\subset\pi_2\supp(\varphi) $ for all $ n\geq1 $. Thus there exist a $ j\geq1 $ such that $\supp\left(\frac{\varphi(x+h_n,\cdot)-\varphi(x,\cdot)}{h_n}\right)\in K_j  $ for all $ n\geq1 $, where $ K_j $ is the increasing sequence of compact sets defined in lemma 2.2 in the book. Furthermore, since $ \partial_x\varphi(\cdot,\cdot) $ is continuous with compact support, it a well known result that it is uniformly continuous. But then it is clear that $ \frac{\varphi(x+h_n,\cdot)-\varphi(x,\cdot)}{h_n}=\partial_x\varphi(x+\xi_n(x,h_n,\cdot),\cdot)\to\partial_x\varphi(x,\cdot) $ uniformly for all $ x\in \R $. The same holds for all the derivatives, $ \partial_y^m\varphi(x,\cdot) $, by the same argument applied to $ \partial_y^m\varphi(x,\cdot) $.
	 Thus we have shown that there exist a $ j\geq1 $ such that $ \frac{\varphi(x+h_n,\cdot)-\varphi(x,\cdot)}{h_n}\in C^\infty_{K_j}(\R) $ for all $ n\geq1 $ and \begin{equation}
	 \sup\left\{\abs{\partial^m_y\left(\frac{\varphi(x+h_n,y)-\varphi(x,y)}{h_n}-\partial_x\varphi(x,y)\right)}\ :\ y\in K_j,\ m\leq\alpha \right\}\to0 \text{ as }n\to\infty
	 \end{equation}
	 for all $ \alpha\geq0 $. Hence by theorem 2.5(a) we have that $ \frac{\varphi(x+h_n,\cdot)-\varphi(x,\cdot)}{h_n}\to \partial_x\varphi(x,\cdot) $ in $ C_0^\infty(\R) $.
	 It then follows from continuity of $ u: C^\infty_0(\R)\to\C $ that $ \frac{f(x+h_n)-f(x)}{h_n}\to\braket{u,\partial_x\varphi(x,\cdot)} $ as $ n\to\infty $, but since this was for any sequence, $ h_n $, converging to $ 0 $, we then may conclude that $ \frac{f(x+h)-f(x)}{h}\to \braket{u,\partial_x\varphi(x,\cdot)}$ as $ h\to 0 $, such that $ f'(x)=\braket{u,\partial_x\varphi(x,\cdot)} $. It then follows that $ f $ is continuous, since it is differentiable. Now to see that $ f $ is $ C^\infty $, we simply proceed by induction. Iterating the argument with $ \varphi $ replaced by $ \partial^m_x\varphi $, shows that $ f $ is $ m+1 $ times differentiable with $ f^{(m+1)}(x)=\braket{u,\partial^{m+1}_x\varphi(x,\cdot)} $, thus $ f\in C_0^m(\R) $. Therefore, by induction, $ f\in C^k_0(\R) $ for all $ k\geq0 $ such that $ f\in C^{\infty}_0(\R) $, which completes the proof.
\end{proof} 



\section*{Ex 2}
Consider the function $ u:\R\to\C $ given by $ u(x)=\exp(-\abs{x}) $, $ x\in\R $.\\
\textbf{1)} We show that $ u\in L^1(\R) $ and that in the sense of distributions we have \begin{equation}\label{dist.eq}
\left(1-\frac{\diff^2}{\diff x^2}\right)u=2\delta_0.
\end{equation}
where $ \delta_0 $ is the $ \delta $-distribution at $ 0 $.
That $ u\in L^1(\R) $ is easily verified: $ u $ is measurable, since it is continuous. Futhermore, \begin{equation}
\begin{aligned}
\int_\R \abs{u(x)}\diff x=2\int_{[0,\infty)}\exp(-x)\diff x=2\lim\limits_{N\to\infty}\int_{[0,N]}\exp(-x)\diff x\\
=2\lim\int_{0}^{N}\exp(-x)\diff x=2\lim\limits_{N\to\infty}\left[-\exp(-x)\right]_0^N=2<\infty.
\end{aligned}
\end{equation}
So $ u\in L^1(\R) $. To verify $ \eqref{dist.eq} $, notice that $ u $ is $ C^\infty $ on $ \R_+ $ and $ \R_- $, and that $ u $ is continuous on $ \R $. Therefore, by lemma 3.6 in G. Grubb, we have that \begin{equation}
 \frac{\diff}{\diff x}u=\begin{cases}
 -\exp(-x),&x>0\\
 \exp(x),&x<0.
 \end{cases}
\end{equation}
which is again an $ L^1(\R) $ function. Now notice that $ \frac{\diff}{\diff x}u+2H $ is extendible to a continuous function on $ \R $, where $ H $ is the Heaviside step function $ H=\mathbbm{1}_{(0,\infty)} $. Thus lemma 3.6 again applies to this function, giving us\begin{equation}
\frac{\diff}{\diff x}\left(\frac{\diff}{\diff x}u+2H\right)=\frac{\diff^2}{\diff x^2}u+2\delta_0=\begin{cases}
\exp(-x),&x>0,\\
\exp(x),&x<0.
\end{cases}=u
\end{equation}
which is equivalent to \begin{equation}
\left(1-\frac{\diff^2}{\diff x^2}\right)u=2\delta_0.
\end{equation}
as desired.\\
\textbf{2)} We show that if $ \phi\in C^\infty_0(\R) $ then $ u\ast\phi\in C^\infty(\R) $ and \begin{equation}\label{phi.eq}
\left(1-\frac{\diff^2}{\diff x^2}\right)u\ast\phi=2\phi.
\end{equation}
That $ u\ast \phi\in C^\infty(\R) $ follows from theorem 3.16 and by noticing that $ u\ast\phi=\phi\ast u $ (by definition as the adjoint operation). Now \eqref{phi.eq} follows from Eq (3.42) in G. Grubb. Using this relation and linearity of the convolution we may calculate\begin{equation}
\left(1-\frac{\diff^2}{\diff x^2}\right)u\ast\phi=\left(\left(1-\frac{\diff^2}{\diff x^2}\right)u\right)\ast\phi=2\delta_0\ast\phi=2\braket{\delta_0,\phi(x-\cdot)}=2\phi(x-0)=2\phi(x).
\end{equation}
where we also used theorem 3.16 in the third equality again.

\section*{Ex 3}
\textbf{a)}\\ Let $ f(x)=x^{-3/2}H(x) $, where $ H $ is the Heaviside step function. We show that $ f\lvert_{\R_+}\in L^1_{\loc}(\R_+) $, but $ f $ is not in $ L^1_\loc(\R) $.
\begin{proof}
	Notice first that $ f_n=\mathbbm{1}_{(1/n,\infty)}f $ is a non-negative increasing sequence of functions such that $ f_n\uparrow f $ pointwise. $ f_n $ are measurable, since \begin{equation}
	\{f_n>a\}=\begin{cases}
	(1/n,a^{-2/3})& 0< a<n^{3/2},\\
	(1/n,\infty)& a=0,\\
	\emptyset& a\geq n^{3/2},\\
	\R&a< 0
	\end{cases}
	\end{equation} which are all open sets, \ie $ \{f_n>a\}\in\mathcal{B}(\R) $ for all $ a\in\R $. Thus by the monotone convergence theorem we have  $ f $ is measurable. Now for any compact set $ K\in\R_+ $ we have that there exist $ a,b>0 $ such that $ K\in[a,b] $. Thus we estimate \begin{equation}
	\int_K \abs{f(x)} \diff x=\int_K f(x) \diff x\leq\int_{[a,b]}f(x) \diff x
	\end{equation}
	since $ f $ is continuous on the interval $ (a,b) $ we may rewrite this integral as a Riemann integral\begin{equation}
	\int_K \abs{f(x)} \diff x\leq \int_{a}^{b}f(x)\diff x=\int_{a}^{b}x^{-3/2}\diff x=\left[-2x^{-1/2}\right]_{a}^{b}=2(a^{-1/2}-b^{-1/2})<\infty.
	\end{equation}
	Thus $ f\in L^1_\loc(\R_+) $. On the other hand, $ [0,1] $ is clearly a compact set in $ \R $, and by the monotone convergence theorem we have \begin{equation}
	\int_{[0,1]}\abs{f(x)}\diff x=\int_{[0,1]}f(x)\diff x=\lim\limits_{n\to\infty}\int_{[0,1]}f_n(x)\diff x=\lim\limits_{n\to\infty}\int_{(1/n,1]}x^{-3/2}\diff x
	\end{equation}
	again since $ x^{-3/2} $ is continuous on $ (1/n,1) $ we may rewrite in terms of Riemann integrals \begin{equation}
	\int_{[0,1]}\abs{f(x)}\diff x=\lim\limits_{n\to\infty}\int_{1/n}^{1}x^{-3/2}\diff x=\lim\limits_{n\to\infty}\left[-2x^{-1/2}\right]_{1/n}^{1}=2\lim\limits_{n\to\infty}\left(n^{1/2}-1\right)=\infty,
	\end{equation}
	from which it follows that $ f\notin L^1_\loc(\R) $.
\end{proof}
We now show that $ \braket{\Lambda,\varphi}=\int_{(0,\infty)}x^{-3/2}\left(\varphi(x)-\varphi(0)\right) $ defines a distribution in $ \mathcal{D}'(\R) $, which is equal to $ f $ on $ \R_+ $ and on $ \R_- $. \begin{proof}
	We have already shown that $ x^{-3/2}\mathbbm{1}_{[0,\infty)} $ is measurable so $ \braket{\Lambda,\varphi} $ is well defined.  That $ \braket{\Lambda,\cdot} $ is a linear functional is obvious from linearity of the integral. $ \braket{\Lambda,\varphi}\neq\infty $ will follows from the proof of continuity below. We thus need to show, that $ \braket{\Lambda,\cdot} $ is also continuous on $ C^\infty_0(\R) $. To see this, let $ a>0 $ and let $ \varphi\in C_{K_j}^\infty(\R) $ and notice that \begin{equation}
\begin{aligned}
	\abs{\braket{\Lambda,\varphi}}&=\abs{\int_{(0,\infty)}x^{-3/2}\left(\varphi(x)-\varphi(0)\right)\diff x}\\&\leq\abs{\int_{(0,a]}x^{-3/2}\left(\varphi(x)-\varphi(0)\right)\diff x}+\abs{\int_{(a,\infty)}x^{-3/2}\left(\varphi(x)-\varphi(0)\right)\diff x}.
\end{aligned}
	\end{equation}
	Now by the mean value theorem $ \abs{\braket{\Lambda,\varphi}}\left(\varphi(x)-\varphi(0)\right)=\varphi'(\xi(x))x $ where $ 0\leq \braket{\Lambda,\varphi}=\xi(x)\leq x $. Thus we have \begin{equation}
	\begin{aligned}
	\abs{\braket{\Lambda,\varphi}}\leq\abs{\int_{(0,a]}x^{-1/2}\varphi'(\xi(x))\diff x}+\abs{\int_{(a,\infty)}x^{-3/2}\left(\varphi(x)-\varphi(0)\right)\diff x}\\
	\leq \max_{x\in \R }(\abs{\varphi'(x)})\int_{(0,a]} x^{-1/2}\diff x+2\max_{x\in \R }(\abs{\varphi(x)})\int_{(a,\infty)} x^{-3/2}\diff x.
	\end{aligned}
	\end{equation}
	where the maxima $ \max_{x\in \R}(\abs{\varphi'(x)})=\max_{x\in K_j }(\abs{\varphi'(x)}) $ and $ \max_{x\in \R }(\abs{\varphi(x)})=\max_{x\in K_j }(\abs{\varphi(x)}) $ exist since, $ \varphi \in C^{\infty}_{K_j}(\R) $. By the usual conversion of Lebsgue integrals to Riemann integrals, via \eg monotone convergence theorem, we get \begin{equation}
	\begin{aligned}
	\max_{x\in K_j }(\abs{\varphi'(x)})\int_{(0,a]} x^{-1/2}\diff x+2\max_{x\in K_j}(\abs{\varphi(x)})\int_{(a,\infty)} x^{-3/2}\diff x\qquad\qquad\qquad\qquad\qquad\qquad\\
	=2\max_{x\in K_j }(\abs{\varphi'(x)}) a^{1/2}+4\max_{x\in K_j}(\abs{\varphi(x)})a^{-1/2}\leq C \sup\left\{\abs{\varphi^{(m)}(x)} : x\in K_j,\ m\leq 1\right\}
	\end{aligned}
	\end{equation}
	where $ C $ might be choosen to be \eg $ C=6 $, which is easily seen by setting $ a=1 $. 
	Thereby we have shown for any $ j\in \N $ that \begin{equation}
	\braket{\Lambda,\varphi}\leq C \sup\left\{\abs{\varphi^{(m)}(x)} : x\in K_j,\ m\leq 1\right\},
	\end{equation} for all $ \varphi\in C^\infty_{K_j}(\R) $. Thus by theorem 2.5(d) we see that $ \braket{\Lambda,\cdot} $ defines a distribution in $ \mathcal{D}'(\R) $.\\
	That $ \Lambda=\Lambda_f $ on $ \R_+ $ is easily seen: Let $ \varphi\in C^\infty_0(\R_+) $, then \begin{equation}
	(\Lambda-\Lambda_f)(\varphi)=\int_{(0,\infty)} x^{-3/2} \left(\varphi(x)-\underbrace{\varphi(0)}_{=0}\right)\diff x-\int_{(0,\infty)} x^{-3/2}\varphi(x)\diff x=0.
	\end{equation}
	where we used that $ f\in L^1_\loc(\R) $ in the first equality and that $ \supp(\varphi)\subset(0,\infty) $ implies that $ \varphi(0)=0 $ and linearity of the integral in the second equality. Thus we have shown that $ \Lambda\lvert_{\R_+}-\Lambda_f\lvert_{\R_+}=0 $ which by definition means that $ \Lambda=\Lambda_f\ (=f) $ on $ \R_+ $. On $ \R_- $ both distributions are trivially zero, so $ \Lambda=\Lambda_f\ (=f) $ on $ \R_- $ as well. 
\end{proof}
\vspace{0.2cm}
\noindent\textbf{b)}\\
Let $ g(x)=-2x^{-1/2}H(x) $. We show that $ g\in L^1_\loc(\R) $ and that $ g'=\Lambda $.
\begin{proof}
Define $ g_n=\mathbbm{1}_{(1/n,\infty)}g $, then $ -g_n $ is an increasing sequence of non-negative functions such that $ -g_n\uparrow-g $. $ g_n $ are measurable, by a similar argument to one made in (a), or by noticing that $ g_n $ may be written as a product of a continuous function $ \tilde{g}(x)=\begin{cases}
g(x)&x>1/n\\
-2xn^{3/2}&x\leq1/n
\end{cases} $, and the measurable function $ \mathbbm{1}_{(1/n,\infty)} $. Thus, $ -g_n $ are measurable and by the monotone convergence theorem $ -g $ is measurable, from which it follows that $ g $ is measurable. Now let $ K $ be a compact subset of $ \R $, then there exist $ a>0 $ such that $ K\in(-a,a) $ therefore, we estimate \begin{equation}
\int_K\abs{g(x)}\diff x\leq\int_{[-a,a]}\abs{g(x)}\diff x=\int_{[0,a]}2x^{-1/2}=\lim\limits_{n\to\infty}\int_{(1/n,a]} 2x^{-1/2}\diff x,
\end{equation}
where we used the monotonce convergence theorem in the last equality. The last integrals may be rewritten as Riemann integrals and thus we have \begin{equation}
\int_K\abs{g(x)}\diff x\leq\lim\limits_{n\to\infty}\int_{1/n}^{a}2x^{-1/2}\diff x=2\lim\limits_{n\to\infty}\left[2x^{1/2}\right]_{1/n}^{a}=4a^{1/2}<\infty.
\end{equation}
Thus it follows that $ g\in L^1_\loc(\R) $. It therefore makes sense to compute the distributional derivative, $ g' $. This can be computed directly from definition, let $ \varphi\in C^\infty_0(\R) $ \begin{equation}
\braket{g',\varphi}=-\braket{g,\varphi'}=\int_{(0,\infty)} 2x^{-1/2}\varphi'(x)\diff x=\int_{(0,\infty)} 2x^{-1/2}\left(\varphi(x)-\varphi(0)\right)'\diff x,
\end{equation}
where we used that $ \left(\varphi(x)-\varphi(0)\right)'=\varphi'(x) $ in the last equality. Noticing that $ \abs{-2x^{1/2}\varphi'(x)}\in L^1(\R_+) $, since $ \varphi'\in C^\infty_0(\R) $, it follows from the dominated convergence theorem that \begin{equation}
\begin{aligned}
\braket{g',\varphi}=\lim\limits_{n\to\infty}\int_{(1/n,n)} 2x^{-1/2}\left(\varphi(x)-\varphi(0)\right)'\diff x
\end{aligned}
\end{equation}
By rewriting in terms of Riemann integrals we have
\begin{equation}
\begin{aligned}
\braket{g',\varphi}&=\lim\limits_{n\to\infty}\int_{1/n}^{n} 2x^{-1/2}\left(\varphi(x)-\varphi(0)\right)'\diff x\\&=\lim\limits_{n\to\infty}\left(\left[2x^{-1/2}\left(\varphi(x)-\varphi(0)\right)\right]_{1/n}^{n}+\int_{(1/n,n)} x^{-3/2}\left(\varphi(x)-\varphi(0)\right)\diff x\right)
\end{aligned}
\end{equation}
Now we use that \begin{equation}
\lim\limits_{n\to\infty}\left(\left[2x^{-1/2}\left(\varphi(x)-\varphi(0\right)\right]_{1/n}^{n}\right)=2\lim\limits_{n\to\infty}\left[n^{-1/2}(\varphi(n)-\varphi(0))-n^{1/2}(\varphi(1/n)-\varphi(0))\right]=0,
\end{equation}
which can be seen from the fact that $ \varphi(n) $ is bounded, and $ \varphi(1/n)-\varphi(0)=\varphi'(\xi_n)/n\leq C/n $ for some $ C>0 $ by the mean value theorem. Now notice also that $  \abs{x^{-3/2}\left(\varphi(x)-\varphi(0)\right)}\in L^1(\R_+) $ since, as was also used in part a), we have \begin{equation}
\abs{x^{-3/2}\left(\varphi(x)-\varphi(0)\right)}\leq\begin{cases}
\max(\abs{\varphi'})x^{-1/2}&0<x<1\\
2\max(\abs{\varphi})x^{-3/2} & x\geq 1
\end{cases}
\end{equation} where as usual the top estimate follows from the mean value theorem and the bottom one is straightforward. Clearly, as seen by above in part a), this shows that $ \abs{x^{-3/2}\left(\varphi(x)-\varphi(0)\right)}\in L^1(\R_+) $. But then notice that by the dominated convergence theorem it follows that \begin{equation}
\int_{(1/n,n)} x^{-3/2}\left(\varphi(x)-\varphi(0)\right)\diff x\to\int_{(0,\infty)} x^{-3/2}\left(\varphi(x)-\varphi(0)\right)\diff x,\text{ as }n\to\infty.
\end{equation}
We have thereby shown that $ \braket{g',\phi}=\int_{(0,\infty)} x^{-3/2}\left(\varphi(x)-\varphi(0)\right)\diff x=\braket{\Lambda,\varphi} $, for all $ \varphi\in C^\infty_0(\R) $, such that $ g'=\Lambda $.
\end{proof}
\end{document}