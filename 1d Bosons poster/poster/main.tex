\documentclass[final]{beamer}
%% Possible paper sizes: a0, a0b, a1, a2, a3, a4.
%% Possible orientations: portrait, landscape
%% Font sizes can be changed using the scale option.
\usepackage[size=a0,orientation=portrait,scale=1.1]{beamerposter}

\usetheme{gemini}
\usecolortheme{default}
\useinnertheme{rectangles}

\usepackage[english]{babel}
\usepackage[utf8]{inputenc}
%\usepackage{beamerthemesplit}
\usepackage{graphics,epsfig, subfigure}
\usepackage{url}
\usepackage{srcltx}
\usepackage{mathrsfs}
\usepackage{amsfonts}
\usepackage{amsmath}
\DeclareMathOperator\arctanh{arctanh}
\usepackage{amssymb}
\usepackage{bbm}
\usepackage{amsthm}
\usepackage{graphicx}
\usepackage{centernot}
\usepackage{caption}
\usepackage{braket}
\usepackage{lastpage}
\usepackage{setspace}
\usepackage{xcolor}
\usepackage{mathtools}
\usepackage{cancel}
\usepackage{booktabs}
\usepackage{tcolorbox}
\usepackage[normalem]{ulem}
\usepackage{anyfontsize}
\usepackage{t1enc}


\usepackage[square,sort,comma,numbers]{natbib}

\renewcommand{\ULdepth}{1pt}

\newcommand{\euler}[1]{\text{e}^{#1}}
\newcommand{\Real}{\text{Re}}
\newcommand{\Imag}{\text{Im}}
\newcommand{\supp}{\text{supp}}
\newcommand{\norm}[1]{\left\lVert #1 \right\rVert}
\newcommand{\abs}[1]{\left\lvert #1 \right\rvert}
\newcommand{\floor}[1]{\left\lfloor #1 \right\rfloor}
\newcommand{\Span}[1]{\text{span}\left(#1\right)}
\newcommand{\dom}[1]{\mathscr D\left(#1\right)}
\newcommand{\Ran}[1]{\text{Ran}\left(#1\right)}
\newcommand{\conv}[1]{\text{co}\left\{#1\right\}}
\newcommand{\Ext}[1]{\text{Ext}\left\{#1\right\}}
\newcommand{\vin}{\rotatebox[origin=c]{-90}{$\in$}}
\newcommand{\interior}[1]{%
	{\kern0pt#1}^{\mathrm{o}}%
}
\renewcommand{\braket}[1]{\left\langle#1\right\rangle}
\newcommand*\diff{\mathop{}\!\mathrm{d}}
\newcommand{\ie}{\emph{i.e.} }
\newcommand{\eg}{\emph{e.g.} }
\newcommand{\dd}{\partial }
\newcommand{\R}{\mathbb{R}}
\newcommand{\C}{\mathbb{C}}
\newcommand{\w}{\mathsf{w}}
\newcommand{\rr}{\mathcal{R}}


\newcommand{\Gliminf}{\Gamma\text{-}\liminf}
\newcommand{\Glimsup}{\Gamma\text{-}\limsup}
\newcommand{\Glim}{\Gamma\text{-}\lim}

\newtheorem{mtheorem}{Theorem}
\newtheorem{mdefinition}{Definition}
\newtheorem{mproposition}{Proposition}
\newtheorem{mlemma}{Lemma}
\newtheorem{mcorollary}{Corollary}
\newtheorem{mremark}{Remark}
\newtheorem{mconjecture}{Conjecture}


\definecolor{qmathblue}{RGB}{61,131,131}
\definecolor{kugreen1}{RGB}{50,93,61}
\definecolor{kugreen}{RGB}{70,116,60}
\definecolor{kugreenlys}{RGB}{132,158,139}
\definecolor{kugreenlyslys}{RGB}{173,190,177}
\definecolor{kugreenlyslyslys}{RGB}{239,249,240}
\definecolor{qmathbluelyslyslys}{RGB}{236,243,243}
\setbeamercovered{transparent}
\mode<presentation>
%\usetheme[numbers,totalnumber,compress,sidebarshades]{PaloAlto}
%\setbeamertemplate{footline}[frame number]
\setbeamertemplate{theorems}[numbered]

\usecolortheme[named=qmathblue]{structure}
\useinnertheme{circles}
\usefonttheme[onlymath]{serif}
\setbeamercovered{transparent}
\setbeamertemplate{blocks}[rounded][shadow=false]
\setbeamercolor{block body}{bg=qmathbluelyslyslys,fg=black}
\setbeamercolor{block title}{bg=qmathbluelyslyslys,fg=black}



\newlength{\sepwidth}
\newlength{\colwidth}
\setlength{\sepwidth}{0.03\paperwidth}
\setlength{\colwidth}{0.4625\paperwidth}
\newcommand{\separatorcolumn}{\begin{column}{\sepwidth}\end{column}}

\newlength{\sepwidthh}
\newlength{\colwidthh}
\setlength{\sepwidthh}{0.015\paperwidth}
\newcommand{\separatorcolumnn}{\begin{column}{\sepwidthh}\end{column}}
%\useoutertheme{infolines} 
%\input{custom-defs.tex}

\logoleft{\hspace*{2cm}\includegraphics[width=10cm]{qmath.jpg}}
\logoright{\hspace*{-5cm}\includegraphics[width=5cm]{kuscience-logo}}

\title{\Huge The ground state energy of 1D dilute many-body quantum systems}

\author{\underline{Johannes Agerskov}\textsuperscript{1}, Robin Reuvers\textsuperscript{2}, and Jan Philip Solovej\textsuperscript{1}}
\institute{1. Department of Mathematics, University of Copenhagen, Universitetsparken 5, DK-2100 Copenhagen \O, Denmark
	\\
	2. Universit\`{a} degli Studi Roma Tre, Dipartimento di Matematica e Fisica, L.go S. L. Murialdo 1, 00146 Roma, Italy}
\date{June 15th 2022}

\setbeamercovered{invisible}

\begin{document}
	\begin{frame}[t]
		\begin{columns}[t]
			\begin{column}{2\colwidth+\sepwidthh}
				\begin{tcolorbox}[colframe=qmathblue,colback=qmathbluelyslyslys,title=\centering Motivation]\nocite{agerskov2022ground}
					We study the ground state energy of a gas of 1D bosons with density $\rho$, interacting through a general, repulsive 2-body potential with scattering length $a$, in the dilute limit $\rho |a|\ll1$. The first terms in the expansion of the thermodynamic energy density are $\pi^2\rho^3/3(1+2\rho a)$, where the leading order is the 1D free Fermi gas. This result covers the Tonks--Girardeau limit of the Lieb--Liniger model as a special case, but given the possibility that $a>0$, it also applies to potentials that differ significantly from a delta function.
				\end{tcolorbox}\vspace{0.75cm}
			\end{column}
			
		\end{columns}
		
	\begin{columns}[t]
		\separatorcolumn
		
		\begin{column}{\colwidth}
		\begin{tcolorbox}[colframe=qmathblue,colback=qmathbluelyslyslys,title=Set up and previous results]
		We consider the model \begin{equation}
			H=-\sum^N_{i=1}\Delta_{x_i}+\sum_{1\leq i<j\leq N}v(x_i-x_j),
		\end{equation}	
		where $ \Delta_{x_i} $ is the $ d $-dimension Laplacian w.r.t $ x_i\in\R^d $, and $ v $ is a repulsive, spherically symmetry potential.\\
		In the hypercube $\Lambda_L\coloneqq[0,L]^d$, the corresponding bosonic energy functional is 
		\begin{equation}
		\mathcal{E}(\psi)=\int_{\Lambda_L}\left(\sum_{i=1}^{N}\abs{\nabla_i\psi}^2+\sum_{i<j} v_{ij}\abs{\psi}^2\right)\quad \text{on } L^2(\Lambda_L)^{\otimes_{\text{sym}} N},
		\end{equation}
		with $ v_{ij}=v\left(\abs{x_i-x_j}\right) $, and the ground state energy is defined by 
		\begin{equation}
			E(N,L)\coloneqq\inf_{\psi\in\mathcal{D}(\mathcal{E}),\ \norm{\psi}^2=1}\mathcal{E}(\psi).
		\end{equation}
		Let $ e(\rho)\coloneqq\lim\limits_{\substack{L\to\infty\\ N/L^{d}\to\rho}}E(N,L)/L^{d} $, then previously known result in $ d=2,3 $ are 
		\begin{theorem}[$ d=3 $ result, Lee-Huang-Yang \cite{lee1957eigenvalues}, \cite{yau2009second,fournais2020energy,basti2021new,fournais2021energy}]
			\begin{equation}
			e(\rho)=4\pi\rho^2 a\left(1+\frac{128}{15\sqrt{\pi}}\sqrt{\rho a^3}+o(\sqrt{\rho a^3})\right).
			\end{equation}
		\end{theorem}
		
		\begin{theorem}[$ d=2 $ result, \cite{lieb2001ground}]
			\begin{equation}
			e(\rho)=4\pi \rho^2\left(\abs{\ln(\rho a^2)}^{-1}+o(\abs{\ln(\rho a^2)}^{-1})\right) .
			\end{equation}
		\end{theorem}
		
		
		\end{tcolorbox}\vspace{0.75cm}
		\begin{tcolorbox}[colframe=qmathblue,colback=qmathbluelyslyslys,title=The scattering length]
			\begin{theorem}[\cite{lieb2001ground}]
				For $ B_R\coloneqq\{0<\abs{x}<R\}\subset \R^d $ with $ R>R_0\coloneqq\text{range}(v) $, let $ \phi\in H^1(B_{R}) $ satisfy
				\begin{equation}
				-\Delta \phi +\frac12 v\phi=0,\qquad \text{on }B_R,
				\end{equation}
				with boundary condition $ \phi(x)=1 $ for $ \abs{x}=R$.
				Then $ \phi(x)=f(\abs{x}) $ for some $ f:(0,R]\to [0,\infty) $, and for $ \text{range}(v)<r<R $, we have \begin{equation}
				f(r)=\begin{cases}
				(r-a)/(R-a) &\text{for }d=1\\
				\ln(r/a)/\ln(R/a) &\text{for }d=2\\
				(1-ar^{2-d})/(1-aR^{2-d})&\text{for }d\geq 3,
				\end{cases}
				\end{equation}
				with some constant $ a $ called the (s-wave, or even-wave in $ d=1 $) \textbf{scattering length}.
			\end{theorem}
			Similarly, one can define the p-wave, or odd-wave in $ d=1 $, scattering length by having a p-wave boundary condition at $ \abs{x}=1 $.
		\end{tcolorbox}\vspace{0.75cm}
			
		\begin{tcolorbox}[colframe=qmathblue,colback=qmathbluelyslyslys,title=Main result]
			Let $ v=v_{\text{reg}}+v_{\text{h.c.}} $, with $ v_{\text{reg}} $ a finite measure, and $ v_{\text{h.c.}} $ a positive linear combination of hard core potentials of the form 
			\begin{equation*}
			v_{[x_1,x_2]}(x):=\begin{cases*}
			\ \infty& \phantom{when} $|x|\in [x_1,x_2]$\\
			\ 0 & \phantom{when} \text{otherwise}
			\end{cases*}.
			\end{equation*}
			Then we have the result
			\begin{theorem}[Bosons {[A., R. Reuvers, J. P. Solovej, 2022]}]
				\label{TheoremMain}
				Consider a one dimensional ($ d=1 $) Bose gas with repulsive interaction  $v=v_{\text{reg}}+v_{\text{h.c.}}$ as defined above. Write $\rho=N/L$. For $\rho|a|$ and $\rho R_0$ sufficiently small, the ground state energy can be expanded as 
				\begin{equation}
				\label{result}
				E(N,L)=N\frac{\pi^2}{3}\rho^2\left(1+2\rho a+
				\mathcal{O}
				\left((\rho|a|)^{6/5}+(\rho R_0)^{6/5}+N^{-2/3}\right)\right),
				\end{equation}
				where $a$ is the scattering length of $v$.
			\end{theorem}
			Since spinless (or spin-aligned) fermions are unitarily equivalent to impenetrable bosons, we have the following theorem as a corolarry 
			\begin{theorem}[Spinless fermions {[A., R. Reuvers, J. P. Solovej, 2022]}]\label{TheoremFermion}
				Consider a spinless Fermi gas with repulsive interaction  $v=v_{\text{reg}}+v_{\text{h.c.}}$ as defined before. Let $ E_F(N,L)$ be its associated ground state energy. Write $\rho=N/L$. For $\rho a_o$ and $\rho R_0$ sufficiently small, the ground state energy can be expanded as 
				\begin{equation}
				E_F(N,L)=N\frac{\pi^2}{3}\rho^2\left(1+2\rho a_o+\mathcal{O}\left(\left(\rho R_0\right)^{6/5}+N^{-2/3}\right)\right),
				\end{equation}
				where $ a_o\geq0 $ is the odd wave scattering length of $v$. 
			\end{theorem} 
		\end{tcolorbox}\vspace{0.75cm}	
		
		\begin{tcolorbox}[colframe=qmathblue,colback=qmathbluelyslyslys,title=Upper bound proof sketch]
			Use Variational principle
			$$
			E(N,L)\leq \frac{\mathcal{E}(\Psi)}{\norm{\Psi}^2},\quad \text{for any }  \Psi\in \mathcal{D}(\mathcal{E}) .
			$$
			Let $ M=N(\rho b)^{3/2} $ and choose the trial state $ \Psi=\prod_{i=1}^{M}\Psi_i $, where \begin{equation}
				\Psi_i(x)=\begin{cases}
				\omega(\rr(x))\frac{\abs{\Psi^i_F(x)}}{\rr(x)}& \text{if }\rr(x)<b\\
				\abs{\Psi^i_F(x)}&\text{if }\rr(x)\geq b,
				\end{cases}
			\end{equation}
			and $ \omega $ is the (suitably normalized) solution to the two-body scattering equation,  $\Psi^i_F$ is the free Fermi ground state with $ N/M $ particles in the box $ [(i-1)L/M,iL/M] $, and $ \rr(x)\coloneqq \min_{i<j}(\abs{x_i-x_j}) $ is uniquely defined a.e.\\ The trial state energy computation amounts, after some manipulations and integration by parts, to computing integrals involving reduced densities and reduced density matrices of the free Fermi gas.
		\end{tcolorbox}\vspace{0.75cm}
		\end{column}
	\separatorcolumnn

	\begin{column}{\colwidth}
		\begin{tcolorbox}[colframe=qmathblue,colback=qmathbluelyslyslys,title=Upper bound proof sketch]
			For the free Fermi gas, then one-particle reduced density matrix is known
			\begin{equation}
			\begin{aligned}
			\gamma^{(1)}(x,y)&=\frac{2}{l}\sum_{j=1}^{N}\sin\left(\frac{\pi}{l}jx\right)\sin\left(\frac{\pi}{l} jy\right)\\
			&=\frac{\pi}{l}\left(D_{N}\left(\pi\frac{x-y}{l}\right)+D_{N}\left(\pi\frac{x+y}{l}\right)\right),
			\end{aligned}
			\end{equation}
			where $ l=L/M $ and $ D_N(x)=\frac{1}{2\pi}\sum_{k=-N}^{N}\euler{ikx}=\frac{\sin((N+1/2)x)}{2\pi\sin(x/2)} $ is the Dirichlet kernel.
			Bounds on $ k $-particle reduced density matrices are provided by Wick's theorem, \eg:
			\begin{mlemma}\label{Lemma rho2 bound}
				$$ \rho^{(2)}(x_1,x_2)\leq\left(\frac{\pi^2}{3}\rho^4+f(x_2)\right)(x_1-x_2)^2+\mathcal{O}(\rho^6(x_1-x_2)^4), $$ 
				with $ \int f(x_2)\diff x_2\leq \textnormal{ const. }\rho^3\log(N). $
			\end{mlemma}
			Collecting everything and choosing $ b=\max(\rho^{-1/5}\abs{a}^{4/5}, R_0) $ we find
			
			\begin{mproposition}[Upper bound Theorem \ref{TheoremMain}]
				\label{PropositionUpperBound}
				There exists a constant $C_\text{U}>0$ such that for $\rho|a|$, $\rho R_0\leq C_U^{-1}$, the ground state energy $E^D(N,L)$ satisfies
				\small{\begin{equation}
					\label{equpper}
					E^D(N,L)\leq N\frac{\pi^2}{3}\rho^2\left(1+2\rho a + C_\text{U}\left((\rho\abs{a})^{6/5}+(\rho R_0)^{3/2}+N^{-1}\right)\right).
					\end{equation}}
			\end{mproposition}
		\end{tcolorbox}\vspace{0.75cm}
		
	\begin{tcolorbox}[colframe=qmathblue,colback=qmathbluelyslyslys,title=Lower bound proof sketch]
		Proof of lower bound consists of the following steps:
		\begin{itemize}
			\item Use Dyson's lemma to reduce to a nearest neighbor double delta-barrier potential.
			\item Reduce to the Lieb Liniger model by discarding \textbf{a small part} of the wave function.
			\item Use a known lower bound for the Lieb Liniger model.
		\end{itemize}	
		\begin{mlemma}[Dyson]\label{LemmaDyson} Let $ R>R_0=\textnormal{range}(v) $ and $ \varphi\in H^1(\R) $, then for any interval $ \mathcal{I}\ni 0 $ 
			\begin{equation}
			\int_{\mathcal{I}} \abs{\partial \varphi}^2+\frac12 v\abs{\varphi}^2\geq \int_{\mathcal{I}}\frac{1}{R-a}\left(\delta_R+\delta_{-R}\right)\abs{\varphi}^2,
			\end{equation}
			where $ a $ is the s-wave scattering length.
		\end{mlemma}
		Define $ \psi\in L^2([0,\ell-(n-1)R]^n) $ by 
				$$ \psi(x_1,x_2,...,x_n)=\Psi(x_1,R+x_2,...,(n-1)R+x_n), $$
				for $ x_1\leq x_2\leq...\leq x_n $ and symmetrically extended.\\\vspace{0.2cm}
				Then by Lemma \ref{LemmaDyson} \begin{equation}
				\begin{aligned}
				\mathcal{E}(\Psi)&\geq E^N_{LL}(n,\ell-(n-1)R,2/(R-a))\braket{\psi|\psi}\\
				&\geq n\frac{\pi^2}{3}\rho^2\left(1+2\rho(a-\cancel{R})+\cancel{2\rho R}-\text{const. }\frac{1}{N^{2/3}}\right)\braket{\psi|\psi}.
				\end{aligned}
				\end{equation}
		The mass of $ \psi $ can be bounded by the following lemma.
		\begin{mlemma}\label{LemmaImprovedMassBound}
			For some constant $C>0$ and for $ n(\rho R)^2\leq  \frac{3}{16\pi^2}C $, $ \rho R\ll 1 $ and $ R>2\abs{a} $ we have
			\begin{equation}\label{EqImprovedMassBound}
			\begin{aligned}
			\braket{\psi|\psi} \geq 1-\textnormal{const. }\left(n(\rho R)^3+n^{1/3}(\rho R)^2\right).
			\end{aligned}
			\end{equation}
		\end{mlemma}
		and we find the following proposition. \begin{mproposition}\label{PropositionLowerBoundSpecN}
			For assumptions as in lemma \ref{LemmaImprovedMassBound} we have \begin{equation}
			E^N(n,\ell)\geq n\frac{\pi^2}{3}\rho^2\left(1+2\rho a+\textnormal{const. }\left(\frac{1}{n^{2/3}}+n(\rho R)^3+n^{1/3}(\rho R)^2\right)\right).
			\end{equation}
		\end{mproposition}
		Error grows with $n$, so localization is needed. Localization is justified by going to the grand canonical ensemble. We localize to boxes with $n=(\rho R)^{-9/5}$ giving
		\begin{mproposition}[Lower bound Theorem \ref{TheoremMain}]
			\label{PropositionLowerBound}
			There exists a constant $C_\text{L}>0$ such that the ground state energy $E^N(N,L)$ satisfies
			\begin{equation}
			\label{eqlower}
			E^N(N,L)\geq N\frac{\pi^2}{3}\rho^2\left(1+2\rho a-C_\text{L}\left((\rho\abs{a})^{6/5}+(\rho R_0)^{6/5}+N^{-2/3}\right)\right).
			\end{equation}
		\end{mproposition}
	\end{tcolorbox}\vspace{0.75cm}
	\begin{tcolorbox}[colframe=qmathblue,colback=qmathbluelyslyslys,title=Spin-1/2 conjecture]
		For the spin-1/2 case, there are two solvable models. The Yang-Gaudin model, and the hard core model. Based on the ground state energies of these, one might conjecture
		\begin{mconjecture}
		For assumptions as in Theorem \ref{TheoremMain}, the spin-1/2 Fermi ground state energy can be expanded as
			\begin{equation}
			\begin{aligned}
			E(N,L)=N\frac{\pi^2}{3}\rho^2\Big(1+2\ln(2)\rho a_e+2(1-\ln(2))\rho a_o\\+\mathcal{O}\left((\rho\max(\abs{a_e},a_o))^2\right)\Big).
			\end{aligned}
			\end{equation}
			where $ a_e $ is the even-wave scattering length, and $a_o$ is the odd-wave scattering length.
		\end{mconjecture}
	\end{tcolorbox}\vspace{0.75cm}
	\begin{tcolorbox}[colframe=qmathblue,colback=qmathbluelyslyslys,title=References]
		\vspace*{0 cm}
		\bibliographystyle{amsplain}
		{\fontsize{15}{18}\selectfont\bibliography{bibliography}}
		\vspace*{0 cm}
	\end{tcolorbox}\vspace{0.75cm}
	\end{column}	
	\separatorcolumn
	\end{columns}
	\end{frame}
\end{document}