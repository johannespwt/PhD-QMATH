\documentclass[final]{beamer}
%% Possible paper sizes: a0, a0b, a1, a2, a3, a4.
%% Possible orientations: portrait, landscape
%% Font sizes can be changed using the scale option.
\usepackage[size=a0,orientation=portrait,scale=1.1]{beamerposter}

\usetheme{gemini}
\usecolortheme{default}
\useinnertheme{rectangles}

\usepackage[english]{babel}
\usepackage[utf8]{inputenc}
%\usepackage{beamerthemesplit}
\usepackage{graphics,epsfig, subfigure}
\usepackage{url}
\usepackage{srcltx}
\usepackage{mathrsfs}
\usepackage{amsfonts}
\usepackage{amsmath}
\DeclareMathOperator\arctanh{arctanh}
\usepackage{amssymb}
\usepackage{bbm}
\usepackage{amsthm}
\usepackage{graphicx}
\usepackage{centernot}
\usepackage{caption}
\usepackage{braket}
\usepackage{lastpage}
\usepackage{setspace}
\usepackage{xcolor}
\usepackage{mathtools}
\usepackage{cancel}
\usepackage{booktabs}
\usepackage{tcolorbox}


\newcommand{\euler}[1]{\text{e}^{#1}}
\newcommand{\Real}{\text{Re}}
\newcommand{\Imag}{\text{Im}}
\newcommand{\supp}{\text{supp}}
\newcommand{\norm}[1]{\left\lVert #1 \right\rVert}
\newcommand{\abs}[1]{\left\lvert #1 \right\rvert}
\newcommand{\floor}[1]{\left\lfloor #1 \right\rfloor}
\newcommand{\Span}[1]{\text{span}\left(#1\right)}
\newcommand{\dom}[1]{\mathscr D\left(#1\right)}
\newcommand{\Ran}[1]{\text{Ran}\left(#1\right)}
\newcommand{\conv}[1]{\text{co}\left\{#1\right\}}
\newcommand{\Ext}[1]{\text{Ext}\left\{#1\right\}}
\newcommand{\vin}{\rotatebox[origin=c]{-90}{$\in$}}
\newcommand{\interior}[1]{%
	{\kern0pt#1}^{\mathrm{o}}%
}
\renewcommand{\braket}[1]{\left\langle#1\right\rangle}
\newcommand*\diff{\mathop{}\!\mathrm{d}}
\newcommand{\ie}{\emph{i.e.} }
\newcommand{\eg}{\emph{e.g.} }
\newcommand{\dd}{\partial }
\newcommand{\R}{\mathbb{R}}
\newcommand{\C}{\mathbb{C}}
\newcommand{\w}{\mathsf{w}}
\newcommand{\rr}{\mathcal{R}}


\newcommand{\Gliminf}{\Gamma\text{-}\liminf}
\newcommand{\Glimsup}{\Gamma\text{-}\limsup}
\newcommand{\Glim}{\Gamma\text{-}\lim}

\newtheorem{mtheorem}{Theorem}
\newtheorem{mdefinition}{Definition}
\newtheorem{mproposition}{Proposition}
\newtheorem{mlemma}{Lemma}
\newtheorem{mcorollary}{Corollary}
\newtheorem{mremark}{Remark}


\definecolor{kugreen}{RGB}{50,93,61}
\definecolor{kugreenlys}{RGB}{132,158,139}
\definecolor{kugreenlyslys}{RGB}{173,190,177}
\definecolor{kugreenlyslyslys}{RGB}{239,249,240}
\setbeamercovered{transparent}
\mode<presentation>
%\usetheme[numbers,totalnumber,compress,sidebarshades]{PaloAlto}
%\setbeamertemplate{footline}[frame number]
\setbeamertemplate{theorems}[numbered]

\usecolortheme[named=kugreen]{structure}
\useinnertheme{circles}
\usefonttheme[onlymath]{serif}
\setbeamercovered{transparent}
\setbeamertemplate{blocks}[rounded][shadow=false]
\setbeamercolor{block body}{bg=kugreenlyslyslys,fg=black}
\setbeamercolor{block title}{bg=kugreenlyslyslys,fg=black}



\newlength{\sepwidth}
\newlength{\colwidth}
\setlength{\sepwidth}{0.03\paperwidth}
\setlength{\colwidth}{0.45\paperwidth}
\newcommand{\separatorcolumn}{\begin{column}{\sepwidth}\end{column}}
%\useoutertheme{infolines} 
%\input{custom-defs.tex}

\logoleft{\hspace*{2cm}\includegraphics[width=10cm]{qmath.jpg}}
\logoright{\hspace*{-5cm}\includegraphics[width=5cm]{kuscience-logo}}

\title{The ground state energy of 1D dilute many-body quantum systems}

\author{Johannes Agerskov}
\institute{QMATH \\ University of Copenhagen}
\date{June 15th 2022}

\setbeamercovered{invisible}

\begin{document}
	\begin{frame}[t]
		\begin{columns}[t]
			
			\begin{column}{2\colwidth+\sepwidth}
				\begin{tcolorbox}[colframe=kugreen,colback=kugreenlyslyslys,title=\centering Abstract]
					We study the ground state energy of a gas of 1D bosons with density $\rho$, interacting through a general, repulsive 2-body potential with scattering length $a$, in the dilute limit $\rho |a|\ll1$. The first terms in the expansion of the thermodynamic energy density are $\pi^2\rho^3/3(1+2\rho a)$, where the leading order is the 1D free Fermi gas. This result covers the Tonks--Girardeau limit of the Lieb--Liniger model as a special case, but given the possibility that $a>0$, it also applies to potentials that differ significantly from a delta function.
				\end{tcolorbox}
			\end{column}
			
		\end{columns}
		
	\begin{columns}[t]
		\separatorcolumn
		
		\begin{column}{\colwidth}
		\begin{tcolorbox}[colframe=kugreen,colback=kugreenlyslyslys,title=Set up and previous results]
			
		\end{tcolorbox}
		
		\begin{tcolorbox}[colframe=kugreen,colback=kugreenlyslyslys,title=The scattering length]
			\begin{theorem}[ref needed]
				For $ B_R\coloneqq\{0<\abs{x}<R\}\subset \R^d $ with $ R>R_0\coloneqq\text{range}(v) $, let $ \phi\in H^1(B_{R}) $ satisfy
				\begin{equation}
				-\Delta \phi +\frac12 v\phi=0,\qquad \text{on }B_R,
				\end{equation}
				with boundary condition $ \phi(x)=1 $ for $ \abs{x}=R$.
				Then $ \phi(x)=f(\abs{x}) $ for some $ f:(0,R]\to [0,\infty) $, and for $ \text{range}(v)<r<R $, we have \begin{equation}
				f(r)=\begin{cases}
				(r-a)/(R-a) &\text{for }d=1\\
				\ln(r/a)/\ln(R/a) &\text{for }d=2\\
				(1-ar^{2-d})/(1-aR^{2-d})&\text{for }d\geq 3,
				\end{cases}
				\end{equation}
				with some constant $ a $ called the (s-wave) \textbf{scattering length}.
			\end{theorem}
			
		\end{tcolorbox}
			
		\begin{tcolorbox}[colframe=kugreen,colback=kugreenlyslyslys,title=Main result]
			\begin{theorem}[A., R. Reuvers, J. P. Solovej, 2022]
				\label{TheoremMain}
				Consider a Bose gas with repulsive interaction  $v=v_{\text{reg}}+v_{\text{h.c.}}$ as defined above. Write $\rho=N/L$. For $\rho|a|$ and $\rho R_0$ sufficiently small, the ground state energy can be expanded as 
				\begin{equation}
				\label{result}
				E(N,L)=N\frac{\pi^2}{3}\rho^2\left(1+2\rho a+
				\mathcal{O}
				\left((\rho|a|)^{6/5}+(\rho R_0)^{6/5}+N^{-2/3}\right)\right),
				\end{equation}
				where $a$ is the scattering length of $v$.
			\end{theorem}
		\end{tcolorbox}	
			
		\end{column}
		
		\separatorcolumn
		
		
		\separatorcolumn
	\end{columns}
	\end{frame}
\end{document}