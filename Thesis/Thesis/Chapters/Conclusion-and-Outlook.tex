\chapter{Conclusion and Outlook}
\label{ChapterConclusionAndOutlook}
In this chapter, we summarize both the findings of this thesis and questions that are left open.
\section{Conclusion}
In Chapter \ref{ChapterMany-BodyQuantumMechanics}, we started by reviewing basic many-body quantum mechanics. Here we defined relevant systems and quantities. We also proved results on the generality of the potentials that would be \emph{allowed} for the energy quadratic form to be associated with a unique self-adjoint Hamiltonian. In particular, we showed that in one dimension potentials of the form $ v=v_{\sigma\text{--finite}}+v_{\text{meas.}} +c\delta_0 $ with $ c\in[0,\infty] $, were allowed. We then proceeded by reviewing the \emph{scattering length} and known results on \emph{dilute} quantum systems in dimensions two and three. We also revisited the bosonic and fermionic one dimensional point interacting models, \ie the Lieb-Liniger model and the Yang-Gaudin model.
For the Yang-Gaudin model, we proved a lower bound on the thermodynamic ground state (within the Yang-Bethe ansatz states).

Chapter \ref{ChapterTheGroundStateEnergyOfTheOneDimensionalDiluteBoseGas} consisted of a paper written in collaboration with Robin Reuvers and Jan Philip Solovej. Here we proved matching upper and lower bounds on the ground state energy of the one dimensional dilute Bose gas, resulting in a next-to-leading order ground state energy expansion. As a corollary, we found a ground state energy expansion for spinless or equivalently spin-polarized (spin-aligned) fermions as well. Finally, as another corollary, we found the ground state energy expansion for a one dimensional dilute gas of anyons. Interestingly, the expansions we found in one dimension exhibit universality, as is the case in dimensions two and three. However, in one dimension, the error must depend on the range of the potential, as the scattering length may vanish. The one dimensional expansion also suggests that no Bose-Einstein condensate is formed. This is evident in the proof, where the formation of a Bose-Einstein condensate is absent both in the low-energy trial state and in the lower-bounding Lieb-Liniger model. The ground state energy expansion also appears to resemble a perturbed Fermi sea, rather than a perturbed condensate, as the leading order term is equal to the free Fermi energy.

In Chapter \ref{ChapterTheGroundStateEnergyOfTheOneDimensionalDiluteSpin1/2FermiGas}, we generalized some of the results obtained in Chapter \ref{ChapterTheGroundStateEnergyOfTheOneDimensionalDiluteBoseGas} to the spin--$ 1/2 $ Fermi gas, and to other symmetries or spin-dependent potentials. Most noteworthy, we proved an upper bound, which we conjectured to be tight based on the solvable models at hand. This upper bound exhibits the same universality as was found for the Bose or spin-polarized Fermi gases. Perhaps even more interestingly, the upper bound seemed connected with magnetism in terms of the Heisenberg chain. For spin--$ 1/2 $ fermions, with $ a_o\geq a_e $, we found that the energy of the variational trial state was related to the antiferromagnetic (Heisenberg chain) energy of the spin part on an ordered sector. Thus the optimal spin part of the trial state on a sector was shown to be the antiferromagnetic Heisenberg ground state. Phrased differently, the energy of the variational trial state was shown to be determined by an effective Heisenberg chain model. For other symmetries or spin-dependent potentials, both the ferro- and antiferromagnetic Heisenberg chains were shown to be effective models for the energy of the variational trial state.\\
We then proceeded in Chapter \ref{ChapterTheGroundStateEnergyOfTheOneDimensionalDiluteSpin1/2FermiGas} by motivating a matching lower bound for the ground state energy of the spin--$ 1/2 $ fermions. We generalized in this case certain results from Chapter \ref{ChapterTheGroundStateEnergyOfTheOneDimensionalDiluteBoseGas}. However, the lower bounding model in these generalizations was shown to be the Lieb-Liniger-Heisenberg model. We stated a conjecture about the ground state energy of this model that, if proven true, implies a rigorous lower bound on the one dimensional dilute spin--$ 1/2 $ Fermi gas. This conjecture was further motivated by heuristic arguments, but never proven. Finally, we noted that for certain different symmetries of the domain and properties of the interaction potential or certain spin-dependent interaction potentials, the lower bounding Lieb-Liniger-Heisenberg model admits a tight lower bound by a Lieb-Liniger model. This reduced the completion of the lower bound proof to the case of Chapter \ref{ChapterTheGroundStateEnergyOfTheOneDimensionalDiluteBoseGas}. Hence for these symmetries and potentials, and spin-dependent potentials, we obtained a lower bound, matching the upper bound already found previously.

\section{Open Problems and Outlook}
 In the making of this thesis, we have encountered some problems which are, at the time of writing, still left open. We give an overview of these problems here:
 \begin{itemize}
 	\item The first problem, which, to the best of our knowledge, seems to have been left open in the literature, is proving that the ground state of the Yang-Gaudin model is among the Yang-Bethe ansatz states. Furthermore, a proof of the existence of solutions to the equations \eqref{EqYG1}--\eqref{EqYG4} seems also to be absent. Solving this problem appears to be a key ingredient in giving a rigorous proof of Conjecture \ref{ConjectureLLHGroundStateEnergy}. Thus this may be a step in the direction of proving Conjecture \ref{ConjectureEqCSpin1/2FermiGroundStateEnergy}.
 	\item The proof of Conjecture \ref{ConjectureEqCSpin1/2FermiGroundStateEnergy} is of course left open. We have already identified a possible strategy by proving Conjecture \ref{ConjectureLLHGroundStateEnergy}. However, one may also consider following entirely different strategies.
 	\item Conjecture \ref{ConjectureLLHGroundStateEnergy}, was heuristically motivated but left open. Given a solution to the first open problem above, the conjecture can be proved by adding space and reducing the model to a Yang-Gaudin model in the case of two particles, as was seen in Chapter \ref{ChapterTheGroundStateEnergyOfTheOneDimensionalDiluteSpin1/2FermiGas}. This may be a strategy for more particles as well if one can suitably generalize the methods. One may also prove the conjecture only in certain regimes of the couplings $ c',c $ by making degenerate perturbation theory rigorous.
 \end{itemize}
Some of the results of this thesis of course also generate new problems that one can pursue in the future. We list some of these in the following:
\begin{itemize}
	\item Finding the next order term in the ground state energy expansion. Although we obtain ground state energy expansions to next-to-leading order (first order in the diluteness parameter) in the bosonic case, the solvable models seem to suggest that the expansion may be valid even to second order in the diluteness parameter.
	\item Giving an approximate momentum distribution of the ground states of dilute quantum gases in one dimension. In Chapter \ref{ChapterTheGroundStateEnergyOfTheOneDimensionalDiluteBoseGas} it was briefly touched upon that the momentum distributions of these ground states are predicted to exhibit some universality as well.
	\item Making rigorous the effective Heisenberg chain model for spin-$ 1/2$ fermions (and the other symmetries and spin-dependent potential analyzed in Chapter \ref{ChapterTheGroundStateEnergyOfTheOneDimensionalDiluteSpin1/2FermiGas}). In the upper bound proof, this effective model was evident. Similarly, such effective models are predicted for point interacting models in one dimension in recent physics literature \cite{yang2016effective,volosniev2015engineering}.
	\item Generalizing the results of Chapter \ref{ChapterTheGroundStateEnergyOfTheOneDimensionalDiluteSpin1/2FermiGas} to higher spin. For higher spin, the upper bound from Chapter \ref{ChapterTheGroundStateEnergyOfTheOneDimensionalDiluteSpin1/2FermiGas} seems to have a straightforward generalization. However, we have no intuition for whether this bound is still tight. To answer this, one may start by lower bounding the Yang-Gaudin ground state energy for higher spin models.
\end{itemize}



