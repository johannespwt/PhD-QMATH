\chapter{Conclusion and Outlook}
\label{ChapterConclusionAndOutlook}
We summarize in this chapter both the findings of this thesis and the questions that are left open.

In Chapter \ref{ChapterMany-BodyQuantumMechanics}, we started by reviewing basic many-body quantum mechanics. Here we defined relevant systems and quantities. We also proved results on generality of the potentials that would be \emph{allowed} in order for the energy quadratic form to be associated to a unique self-adjoint Hamiltonian. In particular, we showed that in one dimension potentials of the form $ v=v_{\sigma\text{--finite}}+v_{\text{meas.}} +c\delta_0 $ with $ c\in[0,\infty] $, were allowed. We then proceeded by reviewing the \emph{scattering length} and known results on \emph{dilute} quantum systems in dimensions two and three. We also revisited the bosonic and fermionic one dimensional point interacting models, \ie the Lieb-Liniger model and the Yang-Gaudin model.
For the Yang-Gaudin model we proved a lower bound on the thermodynamic ground state (within the Yang-Bethe ansatz states).

Chapter \ref{ChapterTheGroundStateEnergyOfTheOneDimensionalDiluteBoseGas} consisted of a paper written in collaboration with Robin Reuvers and Jan Philip Solovej. Here we prove an upper and a matching lower bound on the ground state energy of the one dimensional dilute Bose gas resulting in a next-to-leading order ground state energy expansion. As a corollary we find the also a ground state energy expansion for spinless or equivalently spin-polarized (spin-aligned) fermions. Finally, as  another corollary we find the ground state energy expansion for a one dimensional dilute gas of anyons. Interestingly, the expansions we find in one dimension exhibit universality, as is the case in dimensions two and three. However, the expansions resemble perturbations of 


In Chapter \ref{ChapterTheGroundStateEnergyOfTheOneDimensionalDiluteSpin1/2FermiGas}



