 \chapter{Introduction}
Since the seminal work of Lee, Huang, and Yang in 1957 \cite{lee1957eigenvalues}, there has been a tremendous interest in dilute quantum gases and their ground state energy expansions. Finding good approximations for the ground state energy, at least in two and three dimensions, is intimately related to understanding the formation of \emph{Bose-Einstein condensates}. Furthermore, such ground state energy expansions often exhibit universality. More specifically, the ground state energy of dilute systems tend to depend on the interaction potential only through the so called \emph{scattering length}.

This interest has in the mathematical physics literature grown in during the last decades culminating in the recent completion of a rigorous proof of the Lee-Huang-Yang formula in 2019 \cite{yau2009second,fournais2020energy}\footnote{While the lower bound was made fully general in terms of assumptions on the interaction potential in 2021 \cite{fournais2021energy}, weakening the assumptions under which the upper bound can be proven is still an active field of research. }. The ground state energy expansion in two dimensional

One dimension: No previous results (even sparse in the physics literature) Hohenberg-Mermin-Wagner. absence of condensation. Universality. Magnetism....
