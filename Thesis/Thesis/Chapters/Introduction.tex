 \chapter{Introduction}
Since the seminal work of Lee, Huang, and Yang in 1957 \cite{lee1957eigenvalues, lee1957many, huang1957quantum}, there has been a tremendous interest in dilute quantum gases and their ground state energy expansions. Finding good approximations for the \emph{bosonic} ground state energy, at least in two and three dimensions, is intimately related to understanding the formation of \emph{Bose-Einstein condensates}. Furthermore, such ground state energy expansions often exhibit universality. More specifically, the ground state energy of dilute systems tend to depend on the interaction potential only through the so called \emph{scattering length}. This interest has in the mathematical physics literature grown in during the last decades culminating in the recent completion of a rigorous proof of the Lee-Huang-Yang formula in 2019 \cite{yau2009second,fournais2020energy}\footnote{While the lower bound was made fully general in terms of assumptions on the interaction potential in 2021 \cite{fournais2021energy}, weakening the assumptions under which the upper bound can be proven is still an active field of research. }. With the problem essentially solved for the three dimensional Bose gas, it is natural to seek similar ground state energy expansions in other dimensions or with different particle statistics. Recently, the two dimensional bosonic ground state energy expansion was proven to analogous precision in \cite{fournais2022ground}, and previously the fermionic ground state energy expansions has been studied in both two- and three dimensions \cite{lieb2005ground}

The general one dimensional dilute Bose gas, or quantum gas in general, has been surprisingly little studied both in the physics and mathematics literature. This may be partly due to presence of solvable models in one dimension. In 1963 Lieb and Liniger showed that the one dimensional Bose gas with point (delta-function) interactions is solvable by Bethe ansatz \cite{lieb1963exact}. In practice, this means that one may obtained algebraic equations for the ground state and excited energies, by postulating eigenstates to be a superposition of plane waves with suitable scattering boundary conditions. Similarly in 1967, the one dimensional spin--$ 1/2 $ Fermi gas with point interactions was shown, in the physics literature, to be solvable by means on a generalized Bethe ansatz \cite{yang1967some}. This argument was one year later further generalized to accommodate any symmetry of the domain and hence any spin \cite{sutherland1968further}. Some effort has since then gone into arguing that various confined three dimensional systems may be well approximated by such point interacting systems in one dimension, leaving the analysis of the spectrum already complete \cite{olshanii1998atomic,petrov2000regimes,dunjko2001bosons,lieb2003one,lieb2004one,seiringer2008lieb}. In \cite{lieb2003one,lieb2004one,seiringer2008lieb} it was shown that such an approximation indeed is valid in certain confinement regimes. We call this regime for the \emph{weak confinement regime}, and it is described by having the trapping length scale, in the transverse direction much longer than the three dimensional \emph{scattering length} scale. This means that transverse excitations cannot be neglected. However, in the \emph{strong confinement regime}, described by having the transverse trapping length scale much shorter than the scattering length scale, the spectrum will presumably be well described by a purely one dimensional system, with the three dimensional potential simply restricted to a line. A crucial difference in this case, is that the one dimensional scattering length arising from such a confinement may be positive, as opposed to the effective Lieb-Liniger model in which the one dimensional scattering length always is negative.


The one dimensional result was similarly obtained in \cite{agerskov2022ground} and will enter as part of this thesis in Chapter \ref{ChapterTheGroundStateEnergyOfTheOneDimensionalDiluteBoseGas}.

The bosonic ground state energy expansion 

One dimension: No previous results (even sparse in the physics literature) Hohenberg-Mermin-Wagner. absence of condensation. Universality. Magnetism....
