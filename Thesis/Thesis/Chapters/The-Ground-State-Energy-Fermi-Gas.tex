\chapter{The ground state energy of the one dimensional dilute spin--$ \frac{1}{2} $ Fermi gas}
In the paper of Chapter \ref{ChapterTheGroundStateEnergyOfTheOneDimensionalDiluteBoseGas}, we proved an upper and a lower bound for the ground state energy of a dilute Bose gas in one dimension. It was also shown that, as a corollary, the ground state energy of a one dimensional dilute spin polarized Fermi gas admitted similar bounds. In this chapter we seek to analyse instead the full spin--1/2 Fermi gas. Due to an important theorem of Lieb and Mattis, \cite{lieb1962theory}, it is known that the ground state of a repulsively interacting spin--1/2 Fermi gas (with an even number of particles), will have vanishing total spin. Thus we will focus on the total spin $ 0 $ sector of the one dimensional dilute spin--1/2 Fermi gas.
\section{The model}
We consider a gas of fermions, each with spin--1/2, interacting through a repulsive pair potential $ v\geq 0 $. The assumptions on $ v $ will be similar to those in Chapter \ref{ChapterTheGroundStateEnergyOfTheOneDimensionalDiluteBoseGas}, \ie $ v $ has compact support, say in the ball $ B_{R_0} $, and can be decomposed in $ v=v_{\text{reg}}+v_{\text{h.c.}} $ , where $ v_{\text{reg}} $ is a finite measure and $ v_{\text{h.c.}} $ is a positive linear combination of hard cores. Formally, we write the Hamiltonian \begin{equation}\label{EqFermi1/2Hamiltonian}
H=-\sum_{i=1}^{N}\partial_i^2+\sum_{1\leq i<j\leq N} v(x_i-x_j),
\end{equation}
and with a domain contained in the Hilbert space $ L^2_{\text{as}}\left(\left([0,L]\times \{0,1\}\right)^N\right)\cong \left(L^2([0,L])\otimes \C^2\right)^{\wedge N} $.
We recap here the conjecture, from Chapter \ref{ChapterTheGroundStateEnergyOfTheOneDimensionalDiluteBoseGas}, about the ground state energy for such a system. \begin{conjecture}\label{ConjectureEqCSpin1/2FermiGroundStateEnergy}
	Let $ v\geq0 $ satisfy the assumption from above, then the ground state energy of the dilute spin--$ 1/2 $ Fermi gas satisfies\begin{equation}\label{EqConjectureEqCSpin1/2FermiGroundStateEnergy}
	E=N\frac{\pi^2}{3}\rho^2\left(1+2\rho \left(\ln(2) a_e+(1-\ln(2))a_o\right)+\mathcal{O}(\rho^2\max(\abs{a_e},a_o)^2)\right).
	\end{equation}
\end{conjecture}
\section{Upper bound}
In this section, we prove an upper bound for the ground state energy of the model \eqref{EqFermi1/2Hamiltonian}. The upper bound match, to next to leading order, Conjecture \ref{ConjectureEqCSpin1/2FermiGroundStateEnergy}.
In order to prove the desired upper bound, some prerequisites are needed. We have already covered the definition of the scattering length and scattering wave function in Chapter...., and the free Fermi ground state was found in Chapter \eqref{ChapterTheGroundStateEnergyOfTheOneDimensionalDiluteBoseGas}. For the spin--$ 1/2 $ gas, we furthermore need knowledge about how to handle the spin degrees of freedom. For this purpose we give some heuristics based on physical intuition, and utilize this intuition in constructing a trial state giving the correct upper bound. 
The main result of this section is the following theorem.
\begin{theorem}\label{TheoremUpperBoundSpin1/2Fermi}
	Let $ v\geq0 $ satisfy the assumption from above, then the ground state energy of the dilute spin--$ 1/2 $ Fermi gas satisfies\begin{equation}\label{EqUpperBoundSpin1/2Fermi}
	E\leq N\frac{\pi^2}{3}\rho^2\left(1+2\rho \left(\ln(2) a_e+(1-\ln(2))a_o\right)+\mathcal{O}\left((\rho R)^{6/5}+N^{-1}\right)\right),
	\end{equation}
	with $ R=\max(\abs{a_e}, a_o, R_0) $.
\end{theorem}
\subsection{Constructing a trial state}
In constructing a trial state for the dilute Fermi gas, we may restrict to a sector of the form $ \{\sigma\}=\{\sigma_1,\sigma_2,\ldots,\sigma_N\}=\{0<x_{\sigma_1}<x_{\sigma_2}<\ldots<x_{\sigma_N}<L\} $, then the full trial state is given by anti-symmetrically extending to other sectors. Of course this means that certain boundary conditions needs to be satisfied at the boundary $ \{x_{\sigma_i}=x_{\sigma_{i+1}}\} $ in order for this extension to be in the relevant domain. This boundary condition is exactly that $ \operatorname{P}_t^{i,i+1} \Psi\rvert_{\{x_{\sigma_i}=x_{\sigma_{i+1}}\}}=0 $. Here $ \operatorname{P}_t^{i,j} $ denotes the spin projection onto the triplet of particles $ i $ and $ j $, and equivalently we will denote the spin projection onto the singlet of particles $ i $ and $ j $ by $ \operatorname{P}_s^{i,j} $. We recall from Chapter \ref{ChapterTheGroundStateEnergyOfTheOneDimensionalDiluteBoseGas} that the ground state energy (of the Bose gas or spin polarized Fermi gas) may be well approximated in the dilute limit, by a state that resembles a free Fermi state when particles are far apart, and resembles the two-particle scattering solution when a pair is close. With this in mind, we may construct a variational state trial state on a sector $ \{1,2,\ldots,N\} $ as follows\begin{equation}\label{EqTrial StateSpin1/2Fermi}
\Psi_\chi=\begin{cases}
\frac{\Psi_F}{\mathcal{R}}\left(\left(\eta\omega^{\mathcal{R}}_e+(1-\eta)\omega^{\mathcal{R}}_o\right)\operatorname{P}_s^{\mathcal{R}}+\omega_o^{\mathcal{R}}\operatorname{P}_t^{\mathcal{R}}\right)\chi,&\mathcal{R}(x)<b\\
\Psi_F,&\mathcal{R}(x)\geq b
\end{cases},
\end{equation}
where $ \chi $ is some spin state, $ b>R_0 $, $ \omega^\mathcal{R}_{s/o}(x)\coloneqq \omega_{s/o}(\mathcal{R}(x)) $, and $ \operatorname{P}_{s/t}^{\mathcal{R}(x)}:=\operatorname{P}_{s/t}^{i,j} $ when $ \mathcal{R}(x)=\abs{x_i-x_j} $, and $ \eta $ is a continuous and almost everywhere differentiable function with the property $ \eta(x)=0 $ when $ \mathcal{R}_2(x)=b $, where $ \mathcal{R}_2(x)=\min_{(i,j)\neq (k,l)}\max(\abs{x_i-x_j},\abs{x_k-x_l}) $ is the distance between the second closest pair. More precisely we define
\begin{equation}\label{key}
\eta(x)\coloneqq\begin{cases}
0,&\text{ if } \mathcal{R}_2(x)\leq b\\
\left(\frac{\mathcal{R}_2(x)}{b}-1\right), &\text{ if } b<\mathcal{R}_2(x)<2b\\
1, &\text{ if } \mathcal{R}_2(x)\geq 2b.
\end{cases}
\end{equation}
In this case, we see that $ \operatorname{P}_t^{i,j}\Psi\rvert_{x_i=x_j}=0 $ due to the boundary condition satisfied by $ \omega_o $. We notice that a potential discontinuity could arise from $ \operatorname{P}_{s/t}^{\mathcal{R}(x)} $, since these projection are discontinuous at points where $ \mathcal{R}_2(x)=\mathcal{R}(x) $. However, since $ \operatorname{P}_{s}^{\mathcal{R}(x)}+\operatorname{P}_{t}^{\mathcal{R}(x)}=1 $, we see that $ \Psi $ is continuous due to the inclusion of $ \eta $. The extension of $ \Psi $ to other sectors $ \{\sigma\} $ is then defined by anti-symmetry in the space-spin variables. In this case, due to symmetry of the Hamiltonian/energy form, the energy is determined completely by the energy on the sector $ \{1,2,\ldots,N\} $.\\
We saw in Chapter \ref{ChapterTheGroundStateEnergyOfTheOneDimensionalDiluteBoseGas} that the scattering solution, when particles are close, leads to correction to the free Fermi energy that are of order $ 2\rho a_{e/o} E_F $. Since $ \operatorname{P}_s^{i,j}=1/4-S_i\cdot S_j $ and $ \operatorname{P}_t^{i,j}=3/4+S_i\cdot S_j $, we expect (ignoring the effect of $ \eta $) that the correction we obtain from the variational state $ \Psi_\chi $ is of the order $$
2\rho\left( (a_o-a_e) \braket{\chi \left\vert \frac{1}{N}\sum_{i}S_i\cdot S_{i+1} \right\vert \chi} +\frac{1}{4}a_e+\frac{3}{4}a_o \right)E_F.
$$ 
The minimizer (in $ \chi $), $ \chi_0 $, is known, and in this case since $ a_o\geq a_e $, it is given by the ground state of the periodic antiferromagnetic Heisenberg chain $ \chi_0=\ket{\text{GS}_{\text{HAF}}} $. This ground is known, as it is of Bethe ansatz form \cite{bethe1931theorie}. Furthermore, the ground state energy of the antiferromagnetic Heisenberg chain is known to be \cite{hult1938,mattis2012theory} \begin{equation}
\braket{\text{GS}_{\text{HAF}} \left \lvert \frac{1}{N}\sum_{i}S_i\cdot S_{i+1} \right\rvert \text{GS}_{\text{HAF}}}=\frac{1}{4}-\ln(2)+\mathcal{O}(1/N).
\end{equation}
Hence we find the correction $ 2\rho \left(\ln(2)a_e+(1-\ln(2))a_o\right)E_F $ as desired.
\subsection{Proof of Theorem \ref{TheoremUpperBoundSpin1/2Fermi}}
In this section, we give the formal proof of Theorem \ref{TheoremUpperBoundSpin1/2Fermi}. The idea was already sketched in the previous section, and the goal is thus to make the statements in the previous section rigorous. An important, though completely trivial fact is the following lemma. \begin{lemma}\label{LemmaEtaDerivative}
	Let $ \eta $ be defined as above, then we have \begin{equation}
	\abs{\nabla\eta}\leq \frac{\sqrt{2}}{b}, \textnormal{ a.e}.
	\end{equation}
\end{lemma}
The quantity of interest in the following, will be the energy of the trial state\begin{equation}
\mathcal{E}(\Psi_\chi)=\int_{[0,L]^N}\sum_{i=1}^{N} \abs{\partial_i\Psi_\chi}^2+\sum_{1\leq i<j\leq N}v_{ij}\abs{\Psi_\chi}^2.
\end{equation}
We will henceforth assume $ \chi $ to be translation invariant. In fact, this assumption is not needed, when we have periodic boundary conditions, see Appendix \ref{AppendixPeriodicBCSpin1/2}.
As was done in Chapter \ref{ChapterTheGroundStateEnergyOfTheOneDimensionalDiluteBoseGas}, we rewrite this by use of the diamagnetic inequality\begin{equation}\label{EqEnergy1Spin1/2}
\begin{aligned}
\mathcal{E}(\Psi_\chi)&\leq E_F + \int_{B}\sum_{i=1}^{N}\abs{\partial_i\Psi_{\chi}}^2+\sum_{1\leq i<j\leq N}v_{ij}\abs{\Psi_{\chi}}^2-\sum_{i=1}^{N}\abs{\partial_i \Psi_F}^2\\
&=E_F+\binom{N}{2}\int_{B_{12}}\sum_{i=1}^{N}\abs{\partial_i\Psi_{\chi}}^2+\sum_{1\leq i<j\leq N}v_{ij}\abs{\Psi_{\chi}}^2-\sum_{i=1}^{N}\abs{\partial_i \Psi_F}^2,
\end{aligned}
\end{equation}
where $ B=\{x\in [0,L]^N \vert \mathcal{R}(x)<b\} $, and $ B_{12}=\{x\in [0,L]^N\vert \mathcal{R}(x)=\abs{x_1-x_2}<b\} $. Now due to the presence of $ \eta $ in the trial state, we need to further divide the integration domain. We list here different domains of integration that will be relevant in this section \begin{equation}
\begin{aligned}
B^{\geq}_{12}&=B_{12}\cap \{\mathcal{R}_2(x)\geq 2b\},\\
B^{23}_{12}&=B_{12}\cap\{\mathcal{R}_2(x)=\abs{x_2-x_3}<2b\},\\
B^{34}_{12}&=B_{12}\cap\{\mathcal{R}_2(x)=\abs{x_3-x_4}<2b\},\\
A_{12}&=\left\{x\in [0,L]^N\ \big\vert \abs{x_1-x_2}<b\right\},\\
A_{12}^{23}&=A_{12}\cap \{\abs{x_2-x_3}<2b\},\\
A_{12}^{34}&=A_{12}\cap \{\abs{x_3-x_4}<2b\}.
\end{aligned}
\end{equation} 
In \eqref{EqEnergy1Spin1/2} the last term is dealt with in the same way as in Chapter \ref{ChapterTheGroundStateEnergyOfTheOneDimensionalDiluteBoseGas}. It is also obvious that we may replace $ v $ by $ v_{\text{reg}} $, as the trial state vanishes whenever a pair is inside the outermost hard core. Now due to the anti-symmetry, we conclude from \eqref{EqEnergy1Spin1/2}
\begin{equation}\label{EqEnergy2Spin1/2}
\begin{aligned}
\mathcal{E}(\Psi_{\chi})\leq E_F&+\binom{N}{2}\int_{B_{12}^{\geq}}\sum_{i=1}^{N}\abs{\partial_i\Psi_{\chi}}^2+2(N-2)\binom{N}{2}\int_{B_{12}^{23}}\abs{\partial_i\Psi_{\chi}}^2\\
&+\binom{N}{2}\binom{N-2}{2}\int_{B_{12}^{34}}\sum_{i=1}^{N}\abs{\partial_i\Psi_{\chi}}^2\\
&+\binom{N}{2}\int_{B_{12}}\sum_{1\leq i<j\leq N}(v_{\text{reg}})_{ij}\abs{\Psi_{\chi}}^2-\binom{N}{2}\int_{B_{12}}\sum_{i=1}^{N}\abs{\partial_i \Psi_F}^2.
\end{aligned}
\end{equation}
We now define on $ \{1,2,\ldots,N\}\cap \{\abs{x_1-x_2}<b\} $ \begin{equation}\label{EqTrialState12}
(\tilde{\Psi}_\chi)_{12}\coloneqq \frac{\Psi_F}{x_2-x_1}\left(\omega^{12}_e \operatorname{P}_s^{1,2}+\omega_o^{12}\operatorname{P}_t^{1,2}\right)\chi,
\end{equation}
and extend $ (\tilde{\Psi}_\chi)_{12} $ to all of $ A_{12} $ by anti-symmetry. We may then notice that $ \Psi_{\chi}=(\tilde{\Psi}_\chi)_{12} $ on $ B_{12}^{\geq}\subset A_{12} $. 
Defining \begin{equation}
\begin{aligned}
(\Psi_e)_{12}\coloneqq\frac{\Psi_F}{\abs{x_2-x_1}}\omega_e^{12}\text{ and } (\Psi_o)_{12}\coloneqq\frac{\Psi_F}{\abs{x_2-x_1}}\omega_o^{12},
\end{aligned}
\end{equation}
we find, by the translation invariance of $ \chi $, \begin{equation}
\begin{aligned}
\int_{B_{12}^{\geq}}\abs{\partial_i\Psi_{\chi}}^2\leq&\int_{A_{12}}\abs{\partial_i(\Psi_e)_{12}}^2\frac{1}{N}\sum_{k=1}^{N}\braket{\chi \left\lvert \operatorname{P}^{k,k+1}_s  \right\rvert \chi}\\
&+\int_{A_{12}}\abs{\partial_i(\Psi_o)_{12}}^2\frac{1}{N}\sum_{k=1}^{N}\braket{\chi \left\lvert \operatorname{P}^{k,k+1}_t  \right\rvert \chi},
\end{aligned}
\end{equation}
where $ \operatorname{P}_{s/t}^{N,N+1}\coloneqq \operatorname{P}_{s/t}^{N,1} $.
Considering \eqref{EqEnergy2Spin1/2} again, we see from the trivial relation $$
\frac{1}{N}\left(\sum_{k=1}^{N} \braket{\chi \left\lvert \operatorname{P}^{k,k+1}_s  \right\rvert \chi}+\sum_{k=1}^{N}\braket{\chi \left\lvert \operatorname{P}^{k,k+1}_t  \right\rvert \chi}\right)=1,
$$
and from the fact that $ B_{12}\subset A_{12} $ and the observation that $ \abs{\Psi_{\chi}}^2\leq \abs{\left(\tilde{\Psi}_{\chi}\right)_{12}}^2 $ on $ B_{12} $ that we have the following upper bound for the energy\begin{equation}\label{EqTrialStateEnergyUpperBound1}
	\begin{aligned}
	\mathcal{E}(\Psi_{\chi})\leq E_F&+\binom{N}{2}\frac{1}{N}\sum_{k=1}^{N}\braket{\chi \left\lvert \operatorname{P}^{k,k+1}_s  \right\rvert \chi}\Bigg(\int_{A_{12}}\sum_{i=1}^{N}\abs{\partial_i(\Psi_e)_{12}}^2\\
	&+\int_{A_{12}}\sum_{1\leq i<j\leq N}(v_{\text{reg}})_{ij}\abs{(\Psi_e)_{12}}^2-\int_{B_{12}}\sum_{i=1}^{N}\abs{\partial_i \Psi_F}^2\Bigg)\\
	&+\binom{N}{2}\frac{1}{N}\sum_{k=1}^{N}\braket{\chi \left\lvert \operatorname{P}^{k,k+1}_t  \right\rvert \chi}\Bigg(\int_{A_{12}}\sum_{i=1}^{N}\abs{\partial_i(\Psi_o)_{12}}^2\\
	&+\int_{A_{12}}\sum_{1\leq i<j\leq N}(v_{\text{reg}})_{ij}\abs{(\Psi_o)_{12}}^2-\int_{B_{12}}\sum_{i=1}^{N}\abs{\partial_i \Psi_F}^2\Bigg)\\
	&+\binom{N}{2}\binom{N-2}{2}\int_{B_{12}^{34}}\sum_{i=1}^{N}\abs{\partial_i\Psi_{\chi}}^2\\
	&+2(N-2)\binom{N}{2}\int_{B_{12}^{23}}\sum_{i=1}^{N}\abs{\partial_i\Psi_{\chi}}^2.
	\end{aligned}
\end{equation}
We see that this reduces proving an upper bound to a case we have already analyzed in Chapter \ref{ChapterTheGroundStateEnergyOfTheOneDimensionalDiluteBoseGas}, expect for the last two term, which we then need to estimate. Let us denote the two quantities by \begin{equation}
\begin{aligned}
E_{12}^{34}&\coloneqq\binom{N}{2}\binom{N-2}{2}\int_{B_{12}^{34}}\sum_{i=1}^{N}\abs{\partial_i\Psi_{\chi}}^2,\\
E_{12}^{23}&\coloneqq 2(N-2)\binom{N}{2}\int_{B_{12}^{23}}\sum_{i=1}^{N}\abs{\partial_i\Psi_{\chi}}^2.
\end{aligned}
\end{equation} 
The following lemmas, which we prove below, provide estimates of these quantities.
\begin{lemma}\label{LemmaSpin1/2EtaContribution1}
	Let $ E_{12}^{34} $ and $ \Psi_{\chi} $ be defined as above, then we have the following bound:\begin{equation}
	E_{12}^{34}\leq \text{ const. } E_F\left(N\left(\rho b\right)^4+N^2\left(\rho b\right)^6 \right).
	\end{equation}
	where $ E_F $ denotes the free spin polarized (spinless) Fermi energy.
\end{lemma}

\begin{lemma}\label{LemmaSpin1/2EtaContribution2}
	Let $ E_{12}^{23} $ and $ \Psi_{\chi} $ be defined as above, then we have the following bound:\begin{equation}
	E_{12}^{23}\leq \text{ const. } E_F\left(\left(\rho b\right)^4+N\left(\rho b\right)^6 \right).
	\end{equation}
	where $ E_F $ denotes the free spin polarized (spinless) Fermi energy.
\end{lemma}
Using Lemmas \ref{LemmaSpin1/2EtaContribution1} and \ref{LemmaSpin1/2EtaContribution2}, we deduce, from \eqref{EqTrialStateEnergyUpperBound1} the following bound upper bound on the trial state energy
\begin{equation}
\begin{aligned}
\mathcal{E}(\Psi_{\chi})\leq E_F&+\binom{N}{2}\frac{1}{N}\sum_{k=1}^{N}\braket{\chi \left\lvert \operatorname{P}^{k,k+1}_s  \right\rvert \chi}\Bigg(\int_{A_{12}}\sum_{i=1}^{N}\abs{\partial_i(\Psi_e)_{12}}^2\\
&+\int_{A_{12}}\sum_{1\leq i<j\leq N}(v_{\text{reg}})_{ij}\abs{(\Psi_e)_{12}}^2-\int_{B_{12}}\sum_{i=1}^{N}\abs{\partial_i \Psi_F}^2\Bigg)\\
&+\binom{N}{2}\frac{1}{N}\sum_{k=1}^{N}\braket{\chi \left\lvert \operatorname{P}^{k,k+1}_t  \right\rvert \chi}\Bigg(\int_{A_{12}}\sum_{i=1}^{N}\abs{\partial_i(\Psi_o)_{12}}^2\\
&+\int_{A_{12}}\sum_{1\leq i<j\leq N}(v_{\text{reg}})_{ij}\abs{(\Psi_o)_{12}}^2-\int_{B_{12}}\sum_{i=1}^{N}\abs{\partial_i \Psi_F}^2\Bigg)\\
&+E_F\left(N(\rho b)^4+ N^2 (\rho b)^6\right).
\end{aligned}
\end{equation}
Defining the quantities \begin{equation}
\begin{aligned}
E_{1,e}\coloneqq\binom{N}{2}\Bigg(\int_{A_{12}}\sum_{i=1}^{N}\abs{\partial_i(\Psi_e)_{12}}^2
&+\sum_{1\leq i<j\leq N}(v_{\text{reg}})_{ij}\abs{(\Psi_e)_{12}}^2-\sum_{i=1}^{N}\abs{\partial_i \Psi_F}^2\Bigg),\\
E_{1,o}\coloneqq\binom{N}{2}\Bigg(\int_{A_{12}}\sum_{i=1}^{N}\abs{\partial_i(\Psi_o)_{12}}^2
&+\sum_{1\leq i<j\leq N}(v_{\text{reg}})_{ij}\abs{(\Psi_o)_{12}}^2-\sum_{i=1}^{N}\abs{\partial_i \Psi_F}^2\Bigg),
\end{aligned}
\end{equation}
and the quantities from Chapter \ref{ChapterTheGroundStateEnergyOfTheOneDimensionalDiluteBoseGas}:\begin{equation}
\begin{aligned}
E_2^{(1)}&\coloneqq\binom{N}{2}2N\int_{A_{12}\cap A_{13}}\sum_{i=1}^{N}\abs{\partial_i\Psi_F}^2,\\ E_2^{(2)}&\coloneqq\binom{N}{2}\binom{N-2}{2}\int_{A_{12}\cap A_{34}}\sum_{i=1}^{N}\abs{\partial_i\Psi_F}^2,
\end{aligned}
\end{equation}
we see from an inclusion/exclusion argument identical to the one in Chapter \ref{ChapterTheGroundStateEnergyOfTheOneDimensionalDiluteBoseGas} that \begin{equation}
\begin{aligned}
\mathcal{E}(\Psi_{\chi})\leq E_F &+\frac{1}{N}\sum_{k=1}^{N}\braket{\chi \left\lvert \operatorname{P}^{k,k+1}_s  \right\rvert \chi} \left(E_{1,e}+E_2^{(1)}+E_2^{(2)}\right)\\
&+\frac{1}{N}\sum_{k=1}^{N}\braket{\chi \left\lvert \operatorname{P}^{k,k+1}_t  \right\rvert \chi} \left(E_{1,o}+E_2^{(1)}+E_2^{(2)}\right)\\
&+E_F\left(N(\rho b)^4+ N^2 (\rho b)^6\right)
\end{aligned}
\end{equation}
We see that $ E_{1,e/o} $ corresponds to the quantity $ E_1 $ in Chapter \ref{ChapterTheGroundStateEnergyOfTheOneDimensionalDiluteBoseGas} with the even/odd wave scattering solution in the trial state. The proving equivalent bound for the $ E_{1,e/o} $ amounts to following the same proof strategy and we have the equivalent lemma:
\begin{lemma}[Lemma 14 of Chapter \ref{ChapterTheGroundStateEnergyOfTheOneDimensionalDiluteBoseGas}]\label{LemmaE1BoundSpin1/2}
	Let $ E_{1,e/o} $ be defined as above. For $ N(\rho b)^3\leq 1 $ we have \begin{equation}
	E_{1,e/o}\leq E_F \left(2\rho a_{e/o}\frac{b}{b-a_{e/o}}+ \textnormal{const.}\ N(\rho b)^3\left[ 1+ \rho b^2\int v_{\textnormal{reg}}\right]\right).
	\end{equation}
\end{lemma}
We also recall the lemma\begin{lemma}[Lemma 15 of Chapter \ref{ChapterTheGroundStateEnergyOfTheOneDimensionalDiluteBoseGas}]\label{LemmaE2BoundSpin1/2}
	\begin{equation}
	E_2^{(1)}+E_2^{(2)}\leq E_F\left(N (\rho b)^4+ N^2(\rho b)^6\right).
	\end{equation}
\end{lemma}
Using Lemmas \ref{LemmaE1BoundSpin1/2} and $ \ref{LemmaE2BoundSpin1/2} $ we find the result 
\begin{lemma}
	For $ N(\rho b)^3\leq  1 $ and $ b>2 a_o $ we have \begin{equation}\label{EqTrialStateEnergyBound1}
	\begin{aligned}
	\mathcal{E}(\Psi_{\chi})\leq E_F\Bigg( 1&+2\rho\left[\frac{1}{4}a_e+\frac{3}{4}a_o+(a_o-a_e)\frac{1}{N}\braket{\chi \left\vert \sum_{k=1}^N S_k\cdot S_{k+1}\right\vert \chi}\right]\\ &+\textnormal{ const. } N (\rho b)^3\left(1+\rho b^2\int v_{\textnormal{reg}}\right) \Bigg).
	\end{aligned}
	\end{equation}
\end{lemma}
It is then immediately clear that on the righthand side of \eqref{EqTrialStateEnergyBound1}, given that $ a_o>a_e $, the optimal choice for $ \chi $ is the ground state of the periodic antiferromagnetic Heisenberg chain, which due to the Marshall-Lieb-Mattis \cite{lieb1962ordering,marshall1955antiferromagnetism} theorem is translation invariant. Of course, if $ a_o=a_e $, the choice of $ \chi $ is irrelevant for the righthand side of \eqref{EqTrialStateEnergyBound1}. 

\subsubsection{Estimating $ E_{12}^{34} $ (proof of lemma \ref{LemmaSpin1/2EtaContribution1})}
\begin{proof}[Proof of Lemma \ref{LemmaSpin1/2EtaContribution1}]
Estimating $ E_{12}^{34} $ is a straightforward computation that goes as follows:\\ 
Defining $ \xi_{12}^{34}\coloneqq \left(\left(\eta\omega^{12}_e+(1-\eta)\omega^{12}_o\right)\operatorname{P}_s^{1,2}+\omega_o^{12}\operatorname{P}_t^{1,2}\right)\chi $ on $ B_{12}^{34}\cap \{1,2,\ldots\} $ and (space-spin)-symmetrically extended to $ B_{12}^{34} $, we see that $\Psi_{\chi}=\xi_{12}^{34} \frac{\Psi_F}{\abs{x_2-x_1}}$ on $ B_{12}^{34} $: Hence we find 
\begin{equation}
\begin{aligned}
E_{12}^{34}=&\binom{N}{2}\binom{N-2}{2}\int_{B_{12}^{34}}\sum_{i=1}^{N}\abs{\partial_i\left(\xi_{12}^{34} \frac{\Psi_F}{\abs{x_2-x_1}}\right)}^2\\
\leq& \binom{N}{2}\binom{N-2}{2}\Bigg[\int_{A_{12}^{34}}\sum_{i=1}^{4}\abs{\partial_i\left(\xi_{12}^{34} \frac{\Psi_F}{\abs{x_2-x_1}}\right)}^2 \\&\qquad+\int_{A_{12}^{34}}\sum_{i=5}^{N}\overline{\xi_{12}^{34} \frac{\Psi_F}{\abs{x_2-x_1}}}\left(\xi_{12}^{34} \frac{(-\partial_i^2\Psi_F)}{\abs{x_2-x_1}}\right)\Bigg]\\
=& \binom{N}{2}\binom{N-2}{2}\Bigg[\int_{A_{12}^{34}}\sum_{i=1}^{4}\abs{\partial_i\left(\xi_{12}^{34} \frac{\Psi_F}{\abs{x_2-x_1}}\right)}^2 \\&\qquad-\int_{A_{12}^{34}}\sum_{i=1}^{4}\overline{\xi_{12}^{34} \frac{\Psi_F}{\abs{x_2-x_1}}}\left(\xi_{12}^{34} \frac{(-\partial_i^2\Psi_F)}{\abs{x_2-x_1}}\right)\\
&\qquad + E_F\int_{A_{12}^{34}} \abs{\xi_{12}^{34} \frac{\Psi_F}{\abs{x_2-x_1}}}\Bigg],
\end{aligned}
\end{equation}
where we used $ B_{12}^{34}\subset A_{12}^{34} $, integration by parts, and the fact that $ \Psi_F $ is an eigenfunction of $ (-\Delta) $, with eigenvalue $ E_F $.\\ Thus, using $ \abs{\xi_{12}^{34}}^2\leq b^2 $ and restricting to $ b\geq 2a_0\geq 2a_e $, we find \begin{equation}
\begin{aligned}
E_{12}^{34}\leq& 4\! \int_{A_{12}^{34}}\Bigg(\! \sum_{i=1}^{4}\partial_{y_i}\partial_{x_i}\overline{\frac{\xi_{12}^{34}(y)}{\abs{y_2-y_1}}}\frac{\xi_{12}^{34}(x)}{\abs{x_2-x_1}}\gamma^{(4)}(y_1,y_2,y_3,y_4;x_1,x_2,x_3,x_4)\Bigg\rvert_{y=x}\\
&\qquad\quad+\abs{\frac{\xi_{12}^{34}(x)}{\abs{x_2-x_1}}}^2\abs{\sum_{i=1}^{4}  \partial_{y_i}^2\gamma^{(4)}(y_1,y_2,y_3,y_4;x_1,x_2,x_3,x_4)\Bigg\rvert_{y=x}}\\
&\qquad\quad+E_F\abs{\frac{\xi_{12}^{34}(x)}{\abs{x_2-x_1}}}^2\rho^{(4)}(x_1,x_2,x_3,x_4)\Bigg)\\
\leq& \text{ const. } E_F\left(N\left(\rho b\right)^4+N^2\left(\rho b\right)^6 \right)
\end{aligned}
\end{equation}
where we used the following bounds \begin{equation}
\begin{aligned}
\partial_{y_i}\partial_{x_i}\frac{\gamma^{(4)}(y_1,y_2,y_3,y_4;x_1,x_2,x_3,x_4)}{\abs{x_2-x_1}\abs{y_2-y_1}}\Bigg \lvert_{y=x}&\leq \text{const.} \rho^{8},\\
\partial_{y_i}\frac{\gamma^{(4)}(y_1,y_2,y_3,y_4;x_1,x_2,x_3,x_4)}{\abs{x_2-x_1}\abs{y_2-y_1}}\Bigg \lvert_{y=x}&\leq \text{const.} \rho^{8}\abs{x_3-x_4},\\
\frac{\gamma^{(4)}(y_1,y_2,y_3,y_4;x_1,x_2,x_3,x_4)}{\abs{x_2-x_1}\abs{y_2-y_1}}\Bigg \lvert_{y=x}&\leq \text{const.} \rho^{8}\abs{x_3-x_4}^2,\\
\abs{\sum_{i=1}^{4}\partial_{y_i}^2\gamma^{(4)}(y_1,y_2,y_3,y_4;x_1,x_2,x_3,x_4) \Bigg \lvert_{y=x}}&\leq \text{const.} \rho^{10} \abs{x_1-x_2}^2\abs{x_3-x_4}^2,
\end{aligned}
\end{equation}
and \begin{equation}
\rho^{(4)}(x_1,x_2,x_3,x_4)\leq\text{const.} \rho^8 \abs{x_1-x_2}^2\abs{x_3-x_4}^2,
\end{equation}
which all follows from Taylor expansion of the free Fermi reduced density (matrices). Furthermore, we used the bounds \begin{equation}
\sqrt{\abs{\partial_i\xi_{12}^{34}}^2}\leq b \max\left(\frac{\sqrt{2}}{b},\frac{1}{b-a_o}\right)\leq 2,\qquad \sqrt{\abs{\xi_{12}^{34}}^2}\leq b
\end{equation}
which follows from properties of the scattering solution, monotonicity of its derivative, and Lemma \ref{LemmaEtaDerivative}.
\end{proof}
\subsubsection{Estimating $ E_{12}^{23} $ (proof of Lemma \ref{LemmaSpin1/2EtaContribution2})}
\begin{proof}[Proof of Lemma \ref{LemmaSpin1/2EtaContribution2}]
Estimating $ E_{12}^{23} $ is, similarly to the estimation of $ E_{12}^{34} $, a straightforward computation. We retrace the steps of the previous calculation, suitably modified for $ E_{12}^{23} $, in the following:\\
Defining $ \xi_{12}^{23}\coloneqq \left(\left(\eta\omega^{12}_e+(1-\eta)\omega^{12}_o\right)\operatorname{P}_s^{1,2}+\omega_o^{12}\operatorname{P}_t^{1,2}\right)\chi $ on $ B_{12}^{23}\cap \{1,2,\ldots\} $ and (space-spin)-symmetrically extended to $ B_{12}^{23} $, we see that $\Psi_{\chi}=\xi_{12}^{23} \frac{\Psi_F}{\abs{x_2-x_1}}$ on $ B_{12}^{23} $: Hence we find 
\begin{equation}
\begin{aligned}
E_{12}^{23}=&\binom{N}{2}2(N-2)\int_{B_{12}^{23}}\sum_{i=1}^{N}\abs{\partial_i\left(\xi_{12}^{23} \frac{\Psi_F}{\abs{x_2-x_1}}\right)}^2\\
\leq& \binom{N}{2}2(N-1)\Bigg[\int_{A_{12}^{23}}\sum_{i=1}^{3}\abs{\partial_i\left(\xi_{12}^{23} \frac{\Psi_F}{\abs{x_2-x_1}}\right)}^2 \\&\qquad+\int_{A_{12}^{23}}\sum_{i=4}^{N}\overline{\xi_{12}^{23} \frac{\Psi_F}{\abs{x_2-x_1}}}\left(\xi_{12}^{23} \frac{(-\partial_i^2\Psi_F)}{\abs{x_2-x_1}}\right)\Bigg]\\
=& \binom{N}{2}2(N-2)\Bigg[\int_{A_{12}^{23}}\sum_{i=1}^{3}\abs{\partial_i\left(\xi_{12}^{23} \frac{\Psi_F}{\abs{x_2-x_1}}\right)}^2 \\&\qquad-\int_{A_{12}^{23}}\sum_{i=1}^{3}\overline{\xi_{12}^{23} \frac{\Psi_F}{\abs{x_2-x_1}}}\left(\xi_{12}^{23} \frac{(-\partial_i^2\Psi_F)}{\abs{x_2-x_1}}\right)\\
&\qquad + E_F\int_{A_{12}^{23}} \abs{\xi_{12}^{23} \frac{\Psi_F}{\abs{x_2-x_1}}}\Bigg],
\end{aligned}
\end{equation}
where we used $ B_{12}^{23}\subset A_{12}^{23} $, integration by parts, and the fact that $ \Psi_F $ is an eigenfunction of $ (-\Delta) $, with eigenvalue $ E_F $.\\ Thus, using $ \abs{\xi_{12}^{23}}^2\leq b^2 $ and restricting to $ b\geq 2a_o\geq2a_e $, we find \begin{equation}
\begin{aligned}
E_{12}^{23}\leq& 4\! \int_{A_{12}^{23}}\Bigg(\! \sum_{i=1}^{3}\partial_{y_i}\partial_{x_i}\overline{\frac{\xi_{12}^{23}(y)}{\abs{y_2-y_1}}}\frac{\xi_{12}^{23}(x)}{\abs{x_2-x_1}}\gamma^{(3)}(y_1,y_2,y_3;x_1,x_2,x_3)\Bigg\rvert_{y=x}\\
&\qquad\quad+\abs{\frac{\xi_{12}^{23}(x)}{\abs{x_2-x_1}}}^2\abs{\sum_{i=1}^{3}  \partial_{y_i}^2\gamma^{(4)}(y_1,y_2,y_3;x_1,x_2,x_3)\Bigg\rvert_{y=x}}\\
&\qquad\quad+E_F\abs{\frac{\xi_{12}^{23}(x)}{\abs{x_2-x_1}}}^2\rho^{(3)}(x_1,x_2,x_3)\Bigg)\\
\leq& \text{ const. } E_F\left(\left(\rho b\right)^4+N\left(\rho b\right)^6 \right)
\end{aligned}
\end{equation}
where we used the following bounds \begin{equation}
\begin{aligned}
\partial_{y_i}\partial_{x_i}\frac{\gamma^{(3)}(y_1,y_2,y_3;x_1,x_2,x_3)}{\abs{x_2-x_1}\abs{y_2-y_1}}\Bigg \lvert_{y=x}&\leq \text{const.} \rho^{7},\\
\partial_{y_i}\frac{\gamma^{(3)}(y_1,y_2,y_3;x_1,x_2,x_3)}{\abs{x_2-x_1}\abs{y_2-y_1}}\Bigg \lvert_{y=x}&\leq \text{const.} \rho^{7}\abs{x_3-x_4},\\
\frac{\gamma^{(3)}(y_1,y_2,y_3;x_1,x_2,x_3)}{\abs{x_2-x_1}\abs{y_2-y_1}}\Bigg \lvert_{y=x}&\leq \text{const.} \rho^{7}\abs{x_3-x_4}^2,\\
\abs{\sum_{i=1}^{3}\partial_{y_i}^2\gamma^{(3)}(y_1,y_2,y_3;x_1,x_2,x_3) \Bigg \lvert_{y=x}}&\leq \text{const.} \rho^{11} \abs{x_1-x_2}^2\abs{x_2-x_3}^2\abs{x_1-x_3}^2,
\end{aligned}
\end{equation}
and \begin{equation}
\rho^{(3)}(x_1,x_2,x_3)\leq\text{const.} \rho^9 \abs{x_1-x_2}^2\abs{x_2-x_3}^2\abs{x_1-x_3}^2,
\end{equation}
which all follows from Taylor expansion of the free Fermi reduced density (matrices). Furthermore, we used the bounds \begin{equation}
\sqrt{\abs{\partial_i\xi_{12}^{23}}^2}\leq b \max\left(\frac{\sqrt{2}}{b},\frac{1}{b-a_o}\right)\leq 2,\qquad \sqrt{\abs{\xi_{12}^{23}}^2}\leq b
\end{equation}
which follows from properties of the scattering solution, monotonicity of its derivative, and Lemma \ref{LemmaEtaDerivative}.
\end{proof}



\appendix
\chapter{Periodic boundary conditions}
\label{AppendixPeriodicBCSpin1/2}
If we consider the case with periodic boundary conditions in the box, one may actually show that the antiferromagnetic Heisenberg ground state, is the optimal spin state in the trial state. Starting from \eqref{EqTrialState12}, where no properties of $ \chi $ have been used, we find \begin{equation}\label{EqPeriodicUpperBound1}
\int_{B_{12}^{\geq}}\abs{\partial_i\Psi_{\chi}}^2\leq \int_{A_{12}}\abs{\partial_i(\Psi_e)_{12}}^2\frac{1}{N}\sum_k\braket{\chi \left\lvert \operatorname{P}^{k,k+1}_s  \right\rvert \chi}+\abs{\partial_i(\Psi_o)_{12}}^2\frac{1}{N}\sum_k\braket{\chi \left\lvert \operatorname{P}^{k,k+1}_t  \right\rvert \chi}
\end{equation}
where we defined\begin{equation}
\begin{aligned}
(\Psi_e)_{12}\coloneqq\frac{\Psi_F}{\abs{x_2-x_1}}\omega_e\text{ and } (\Psi_o)_{12}\coloneqq\frac{\Psi_F}{\abs{x_2-x_1}}\omega_o,
\end{aligned}
\end{equation}
and we used that \begin{equation}
\int_{A_{12}\cap \{\ldots,1,2,\ldots\}}\abs{\partial_i(\Psi_{e/o})_{12}}^2=\frac{1}{N}\int_{A_{12}}\abs{\partial_i(\Psi_{e/o})_{12}}^2.
\end{equation}
But from \eqref{EqPeriodicUpperBound1} it is clear that the antiferromagnetic Heisenberg ground state is optimal.


