\chapter{The ground state energy of the one dimensional dilute spin--$ \frac{1}{2} $ Fermi gas}
\label{ChapterTheGroundStateEnergyOfTheOneDimensionalDiluteSpin1/2FermiGas}
In the paper of Chapter \ref{ChapterTheGroundStateEnergyOfTheOneDimensionalDiluteBoseGas}, we proved an upper and a lower bound for the ground state energy of a dilute Bose gas in one dimension. It was also shown that, as a corollary, the ground state energy of a one dimensional dilute spin polarized Fermi gas admitted similar bounds. In this chapter we seek to analyse instead the full spin--1/2 Fermi gas. Due to an important theorem of Lieb and Mattis, \cite{lieb1962theory}, it is known that the ground state of a repulsively interacting spin--1/2 Fermi gas (with an even number of particles), will have vanishing total spin. Thus we will focus on the total spin $ 0 $ sector of the one dimensional dilute spin--1/2 Fermi gas.
\section{The model}
We consider a gas of fermions, each with spin--1/2, interacting through a repulsive pair potential $ v\geq 0 $. The assumptions on $ v $ will be similar to those in Chapter \ref{ChapterTheGroundStateEnergyOfTheOneDimensionalDiluteBoseGas}, \ie $ v $ has compact support, say in the ball $ B_{R_0} $, and can be decomposed in $ v=v_{\text{reg}}+v_{\text{h.c.}} $ , where $ v_{\text{reg}} $ is a finite measure and $ v_{\text{h.c.}} $ is a positive linear combination of hard cores. Formally, we write the Hamiltonian \begin{equation}\label{EqFermi1/2Hamiltonian}
H=-\sum_{i=1}^{N}\partial_i^2+\sum_{1\leq i<j\leq N} v(x_i-x_j),
\end{equation}
and with a domain contained in the Hilbert space $ L^2_{\text{as}}\left(\left([0,L]\times \{0,1\}\right)^N\right)\cong \left(L^2([0,L])\otimes \C^2\right)^{\wedge N} $.
We recap here the conjecture, from Chapter \ref{ChapterTheGroundStateEnergyOfTheOneDimensionalDiluteBoseGas}, about the ground state energy for such a system. \begin{conjecture}\label{ConjectureEqCSpin1/2FermiGroundStateEnergy}
	Let $ v\geq0 $ satisfy the assumption from above, then the ground state energy of the dilute spin--$ 1/2 $ Fermi gas satisfies\begin{equation}\label{EqConjectureEqCSpin1/2FermiGroundStateEnergy}
	E=N\frac{\pi^2}{3}\rho^2\left(1+2\rho \left(\ln(2) a_e+(1-\ln(2))a_o\right)+\mathcal{O}(\rho^2\max(\abs{a_e},a_o)^2)\right).
	\end{equation}
\end{conjecture}
\section{Upper bound}
In this section, we prove an upper bound for the ground state energy of the model \eqref{EqFermi1/2Hamiltonian}. The upper bound match, to next to leading order, Conjecture \ref{ConjectureEqCSpin1/2FermiGroundStateEnergy}.
In order to prove the desired upper bound, some prerequisites are needed. We have already covered the definition of the scattering length and scattering wave function in Chapter...., and the free Fermi ground state was found in Chapter \eqref{ChapterTheGroundStateEnergyOfTheOneDimensionalDiluteBoseGas}. For the spin--$ 1/2 $ gas, we furthermore need knowledge about how to handle the spin degrees of freedom. For this purpose we give some heuristics based on physical intuition, and utilize this intuition in constructing a trial state giving the correct upper bound. 
The main result of this section is the following theorem.
\begin{theorem}\label{TheoremUpperBoundSpin1/2Fermi}
	Let $ v\geq0 $ satisfy the assumption from above, then the ground state energy of the dilute spin--$ 1/2 $ Fermi gas satisfies\begin{equation}\label{EqUpperBoundSpin1/2Fermi}
	E\leq N\frac{\pi^2}{3}\rho^2\left(1+2\rho \left(\ln(2) a_e+(1-\ln(2))a_o\right)+\mathcal{O}\left((\rho R)^{6/5}+N^{-1}\right)\right),
	\end{equation}
	with $ R=\max(\abs{a_e}, a_o, R_0) $.
\end{theorem}
\subsection{Constructing a trial state}
In constructing a trial state for the dilute Fermi gas, we may restrict to a sector of the form $ \{\sigma\}=\{\sigma_1,\sigma_2,\ldots,\sigma_N\}=\{0<x_{\sigma_1}<x_{\sigma_2}<\ldots<x_{\sigma_N}<L\} $, then the full trial state is given by anti-symmetrically extending to other sectors. Of course this means that certain boundary conditions needs to be satisfied at the boundary $ \{x_{\sigma_i}=x_{\sigma_{i+1}}\} $ in order for this extension to be in the relevant domain. This boundary condition is exactly that $ \operatorname{P}_t^{i,i+1} \Psi\rvert_{\{x_{\sigma_i}=x_{\sigma_{i+1}}\}}=0 $. Here $ \operatorname{P}_t^{i,j} $ denotes the spin projection onto the triplet of particles $ i $ and $ j $, and equivalently we will denote the spin projection onto the singlet of particles $ i $ and $ j $ by $ \operatorname{P}_s^{i,j} $. We recall from Chapter \ref{ChapterTheGroundStateEnergyOfTheOneDimensionalDiluteBoseGas} that the ground state energy (of the Bose gas or spin polarized Fermi gas) may be well approximated in the dilute limit, by a state that resembles a free Fermi state when particles are far apart, and resembles the two-particle scattering solution when a pair is close. With this in mind, we may construct a variational state trial state on a sector $ \{1,2,\ldots,N\} $ as follows\begin{equation}\label{EqTrial StateSpin1/2Fermi}
\Psi_\chi=\begin{cases}
\frac{\Psi_F}{\mathcal{R}}\left(\left(\eta\omega^{\mathcal{R}}_e+(1-\eta)\omega^{\mathcal{R}}_o\right)\operatorname{P}_s^{\mathcal{R}}+\omega_o^{\mathcal{R}}\operatorname{P}_t^{\mathcal{R}}\right)\chi,&\mathcal{R}(x)<b\\
\Psi_F,&\mathcal{R}(x)\geq b
\end{cases},
\end{equation}
where $ \chi $ is some spin state, $ b>R_0 $, $ \omega^\mathcal{R}_{s/o}(x)\coloneqq \omega_{s/o}(\mathcal{R}(x)) $, and $ \operatorname{P}_{s/t}^{\mathcal{R}(x)}:=\operatorname{P}_{s/t}^{i,j} $ when $ \mathcal{R}(x)=\abs{x_i-x_j} $, and $ \eta $ is a continuous and almost everywhere differentiable function with the property $ \eta(x)=0 $ when $ \mathcal{R}_2(x)=b $, where $ \mathcal{R}_2(x)=\min_{(i,j)\neq (k,l)}\max(\abs{x_i-x_j},\abs{x_k-x_l}) $ is the distance between the second closest pair. More precisely we define
\begin{equation}\label{key}
\eta(x)\coloneqq\begin{cases}
0,&\text{ if } \mathcal{R}_2(x)\leq b\\
\left(\frac{\mathcal{R}_2(x)}{b}-1\right), &\text{ if } b<\mathcal{R}_2(x)<2b\\
1, &\text{ if } \mathcal{R}_2(x)\geq 2b.
\end{cases}
\end{equation}
In this case, we see that $ \operatorname{P}_t^{i,j}\Psi\rvert_{x_i=x_j}=0 $ due to the boundary condition satisfied by $ \omega_o $. We notice that a potential discontinuity could arise from $ \operatorname{P}_{s/t}^{\mathcal{R}(x)} $, since these projection are discontinuous at points where $ \mathcal{R}_2(x)=\mathcal{R}(x) $. However, since $ \operatorname{P}_{s}^{\mathcal{R}(x)}+\operatorname{P}_{t}^{\mathcal{R}(x)}=1 $, we see that $ \Psi $ is continuous due to the inclusion of $ \eta $. The extension of $ \Psi $ to other sectors $ \{\sigma\} $ is then defined by anti-symmetry in the space-spin variables. In this case, due to symmetry of the Hamiltonian/energy form, the energy is determined completely by the energy on the sector $ \{1,2,\ldots,N\} $.\\
As was the case in Chapter \ref{ChapterTheGroundStateEnergyOfTheOneDimensionalDiluteBoseGas}, the trial state given in \eqref{EqTrial StateSpin1/2Fermi} produces error that grow with the particle number. This is undesirable for proving Theorem \ref{TheoremUpperBoundSpin1/2Fermi}. However, as before, we may construct the full trial state by localizing to smaller intervals. This is done by splitting the interval $ [0,L] $ into smaller intervals $ I_m\coloneqq[m(\ell+b),(m+1)\ell+mb] $ $ m=0,1,2,\ldots M-1 $, where $ \ell=L/M-b $. We then consider the trial state given by a product \begin{equation}\label{EqFullTrialStateSpin1/2}
\Psi_{\chi,\text{full}}(x_1,\ldots,x_N)=\prod_{m=0}^{M-1}\Psi^{I_m}_{\chi}(x_1^m,\ldots,x_{\tilde{N}}^m),
\end{equation}
where $ \tilde{N}=N/M $ and $ x_i^m\coloneqq x_{m\tilde{N}+i} $ and the superscribt $ I_m $ in $ \Psi^{I_m}_{\chi} $ means that we take the state $ \Psi_{\chi} $ constructed on $ I_m $ instead of $ [0,L] $. Notice that there are no interactions between boxes since $ b>R_0 $.\\
We saw in Chapter \ref{ChapterTheGroundStateEnergyOfTheOneDimensionalDiluteBoseGas} that the scattering solution, when particles are close, leads to correction to the free Fermi energy that are of order $ 2\rho a_{e/o} E_F $. Since $ \operatorname{P}_s^{i,j}=1/4-S_i\cdot S_j $ and $ \operatorname{P}_t^{i,j}=3/4+S_i\cdot S_j $, we expect (ignoring the effect of $ \eta $) that the correction we obtain from the variational state $ \Psi_\chi $ is of the order $$
2\rho\left( (a_o-a_e) \braket{\chi \left\vert \frac{1}{N}\sum_{i}S_i\cdot S_{i+1} \right\vert \chi} +\frac{1}{4}a_e+\frac{3}{4}a_o \right)E_F.
$$ 
The minimizer (in $ \chi $), $ \chi_0 $, is known, and in this case since $ a_o\geq a_e $, it is given by the ground state of the periodic antiferromagnetic Heisenberg chain $ \chi_0=\ket{\text{GS}_{\text{HAF}}} $. This ground is known, as it is of Bethe ansatz form \cite{bethe1931theorie}. Furthermore, the ground state energy of the antiferromagnetic Heisenberg chain is known to be \cite{hult1938,mattis2012theory} (See lemma \ref{LemmaHeisenbergChainFiniteNEstimate} below) \begin{equation}
\braket{\text{GS}_{\text{HAF}} \left \lvert \frac{1}{N}\sum_{i}S_i\cdot S_{i+1} \right\rvert \text{GS}_{\text{HAF}}}=\frac{1}{4}-\ln(2)+\mathcal{O}(1/N).
\end{equation}
Hence we find the correction $ 2\rho \left(\ln(2)a_e+(1-\ln(2))a_o\right)E_F $ as desired.
\subsection{Proof of Theorem \ref{TheoremUpperBoundSpin1/2Fermi}}
In this section, we give the formal proof of Theorem \ref{TheoremUpperBoundSpin1/2Fermi}. The idea was already sketched in the previous section, and the goal is thus to make the statements in the previous section rigorous. An important, though completely trivial fact is the following lemma. \begin{lemma}\label{LemmaEtaDerivative}
	Let $ \eta $ be defined as above, then we have \begin{equation}
	\abs{\nabla\eta}\leq \frac{\sqrt{2}}{b}, \textnormal{ a.e}.
	\end{equation}
\end{lemma}
The quantity of interest in the following, will be the energy of the trial state\begin{equation}
\mathcal{E}(\Psi_\chi)=\int_{[0,L]^N}\sum_{i=1}^{N} \abs{\partial_i\Psi_\chi}^2+\sum_{1\leq i<j\leq N}v_{ij}\abs{\Psi_\chi}^2.
\end{equation}
We will henceforth assume $ \chi $ to be translation invariant. In fact, this assumption is not needed, when we have periodic boundary conditions, see Appendix \ref{AppendixPeriodicBCSpin1/2}.
As was done in Chapter \ref{ChapterTheGroundStateEnergyOfTheOneDimensionalDiluteBoseGas}, we rewrite this by use of the diamagnetic inequality\begin{equation}\label{EqEnergy1Spin1/2}
\begin{aligned}
\mathcal{E}(\Psi_\chi)&\leq E_F + \int_{B}\sum_{i=1}^{N}\abs{\partial_i\Psi_{\chi}}^2+\sum_{1\leq i<j\leq N}v_{ij}\abs{\Psi_{\chi}}^2-\sum_{i=1}^{N}\abs{\partial_i \Psi_F}^2\\
&=E_F+\binom{N}{2}\int_{B_{12}}\sum_{i=1}^{N}\abs{\partial_i\Psi_{\chi}}^2+\sum_{1\leq i<j\leq N}v_{ij}\abs{\Psi_{\chi}}^2-\sum_{i=1}^{N}\abs{\partial_i \Psi_F}^2,
\end{aligned}
\end{equation}
where $ B=\{x\in [0,L]^N \vert \mathcal{R}(x)<b\} $, and $ B_{12}=\{x\in [0,L]^N\vert \mathcal{R}(x)=\abs{x_1-x_2}<b\} $. Now due to the presence of $ \eta $ in the trial state, we need to further divide the integration domain. We list here different domains of integration that will be relevant in this section \begin{equation}
\begin{aligned}
B^{\geq}_{12}&=B_{12}\cap \{\mathcal{R}_2(x)\geq 2b\},\\
B^{23}_{12}&=B_{12}\cap\{\mathcal{R}_2(x)=\abs{x_2-x_3}<2b\},\\
B^{34}_{12}&=B_{12}\cap\{\mathcal{R}_2(x)=\abs{x_3-x_4}<2b\},\\
A_{12}&=\left\{x\in [0,L]^N\ \big\vert \abs{x_1-x_2}<b\right\},\\
A_{12}^{23}&=A_{12}\cap \{\abs{x_2-x_3}<2b\},\\
A_{12}^{34}&=A_{12}\cap \{\abs{x_3-x_4}<2b\}.
\end{aligned}
\end{equation} 
In \eqref{EqEnergy1Spin1/2} the last term is dealt with in the same way as in Chapter \ref{ChapterTheGroundStateEnergyOfTheOneDimensionalDiluteBoseGas}. It is also obvious that we may replace $ v $ by $ v_{\text{reg}} $, as the trial state vanishes whenever a pair is inside the outermost hard core. Now due to the anti-symmetry, we conclude from \eqref{EqEnergy1Spin1/2}
\begin{equation}\label{EqEnergy2Spin1/2}
\begin{aligned}
\mathcal{E}(\Psi_{\chi})\leq E_F&+\binom{N}{2}\int_{B_{12}^{\geq}}\sum_{i=1}^{N}\abs{\partial_i\Psi_{\chi}}^2+2(N-2)\binom{N}{2}\int_{B_{12}^{23}}\abs{\partial_i\Psi_{\chi}}^2\\
&+\binom{N}{2}\binom{N-2}{2}\int_{B_{12}^{34}}\sum_{i=1}^{N}\abs{\partial_i\Psi_{\chi}}^2\\
&+\binom{N}{2}\int_{B_{12}}\sum_{1\leq i<j\leq N}(v_{\text{reg}})_{ij}\abs{\Psi_{\chi}}^2-\binom{N}{2}\int_{B_{12}}\sum_{i=1}^{N}\abs{\partial_i \Psi_F}^2.
\end{aligned}
\end{equation}
%Now we define 
%\begin{equation}\label{EqTrialState12}
%(\tilde{\Psi}_\chi)_{12}\coloneqq \Psi_{\chi}\big\rvert_{B_{12}^\geq},
%\end{equation}
%\begin{equation}\label{EqTrialState12}
%(\tilde{\Psi}_\chi)_{12}\coloneqq \frac{\Psi_F}{x_2-x_1}\left(\omega^{12}_e \operatorname{P}_s^{1,2}+\omega_o^{12}\operatorname{P}_t^{1,2}\right)\chi,
%\end{equation}
%and extend $ (\tilde{\Psi}_\chi)_{12} $ to all of $ A_{12} $ by anti-symmetry. By definition we have $ \Psi_{\chi}=(\tilde{\Psi}_\chi)_{12} $ on $ B_{12}^{\geq}\subset A_{12} $, and on $ \{1,2,\ldots,N\}\cap A_{12} $ we have 
%\begin{equation}
%(\tilde{\Psi}_\chi)_{12}\coloneqq \frac{\Psi_F}{x_2-x_1}\left(\omega^{12}_e \operatorname{P}_s^{1,2}+\omega_o^{12}\operatorname{P}_t^{1,2}\right)\chi,
%\end{equation}
Defining \begin{equation}
\begin{aligned}
(\Psi_e)_{12}\coloneqq\frac{\Psi_F}{\abs{x_2-x_1}}\omega_e^{12}\text{ and } (\Psi_o)_{12}\coloneqq\frac{\Psi_F}{\abs{x_2-x_1}}\omega_o^{12},
\end{aligned}
\end{equation}
we find by the fact that $ B_{12}^\geq\subset A_{12} $, \begin{equation}\label{EqEnergy3Spin1/2}
\begin{aligned}
\int_{B_{12}^{\geq}}\abs{\partial_i\Psi_{\chi}}^2\leq&\sum_{\{\sigma\}\in S_{12}}\left(\int_{A_{12}\cap\{\sigma\}}\abs{\partial_i(\Psi_e)_{12}}^2\right)\braket{\chi_{\sigma} \left\lvert \operatorname{P}^{1,2}_s  \right\rvert \chi_\sigma}\\
&+\sum_{\{\sigma\}\in S_{12}}\left(\int_{A_{12}\cap \{\sigma\}}\abs{\partial_i(\Psi_o)_{12}}^2\right)\braket{\chi_{\sigma} \left\lvert \operatorname{P}^{1,2}_t  \right\rvert \chi_\sigma},
\end{aligned}
\end{equation}
where $ \operatorname{P}_{s/t}^{N,N+1}\coloneqq \operatorname{P}_{s/t}^{N,1} $ and $ \chi_{\sigma} $ is the spin state $ \chi $ with spins permuted by $ (1,\ldots,N)\mapsto(\sigma_1,\ldots,\sigma_N) $ and 
$$ S_{12}=\{\text{sectors } \{\sigma\}\ \vert\ (\sigma_k,\sigma_{k+1})=(1,2)\text{ or }(\sigma_k,\sigma_{k+1})=(2,1)\text{ for some }k\}. $$
Using translation invariance of $ \chi $ we see that 
$$ \braket{\chi_{\sigma} \left\lvert \operatorname{P}^{1,2}_{s/t}  \right\rvert \chi_\sigma}=\frac{1}{N}\sum_{k=1}^{N}\braket{\chi_\sigma\left\lvert \operatorname{P}^{\sigma_k,\sigma_{k+1}}_{s/t}  \right\rvert \chi_\sigma}=\frac{1}{N}\sum_{k=1}^{N}\braket{\chi\left\lvert \operatorname{P}^{k,k+1}_{s/t}  \right\rvert \chi} $$
is independent of $ \sigma\in S_{12} $ and that
\begin{equation}\label{EqEnergy4Spin1/2}
\begin{aligned}
\int_{B_{12}^{\geq}}\abs{\partial_i\Psi_{\chi}}^2\leq&\left(\int_{A_{12}}\abs{\partial_i(\Psi_e)_{12}}^2\right)\frac{1}{N}\sum_{k=1}^{N}\braket{\chi \left\lvert \operatorname{P}^{k,k+1}_s  \right\rvert \chi}\\
&+\left(\int_{A_{12}}\abs{\partial_i(\Psi_o)_{12}}^2\right)\frac{1}{N}\sum_{k=1}^{N}\braket{\chi\left\lvert \operatorname{P}^{k,k+1}_t  \right\rvert \chi},
\end{aligned}
\end{equation}


Considering \eqref{EqEnergy2Spin1/2} again, we see from the trivial relation $$
\frac{1}{N}\left(\sum_{k=1}^{N} \braket{\chi \left\lvert \operatorname{P}^{k,k+1}_s  \right\rvert \chi}+\sum_{k=1}^{N}\braket{\chi \left\lvert \operatorname{P}^{k,k+1}_t  \right\rvert \chi}\right)=1,
$$
and from the fact that $ B_{12}\subset A_{12} $ and the observation that $ \abs{\Psi_{\chi}}^2\leq \abs{\left(\tilde{\Psi}_{\chi}\right)_{12}}^2 $ on $ B_{12} $ that we have the following upper bound for the energy\begin{equation}\label{EqTrialStateEnergyUpperBound1}
	\begin{aligned}
	\mathcal{E}(\Psi_{\chi})\leq E_F&+\binom{N}{2}\frac{1}{N}\sum_{k=1}^{N}\braket{\chi \left\lvert \operatorname{P}^{k,k+1}_s  \right\rvert \chi}\Bigg(\int_{A_{12}}\sum_{i=1}^{N}\abs{\partial_i(\Psi_e)_{12}}^2\\
	&+\int_{A_{12}}\sum_{1\leq i<j\leq N}(v_{\text{reg}})_{ij}\abs{(\Psi_e)_{12}}^2-\int_{B_{12}}\sum_{i=1}^{N}\abs{\partial_i \Psi_F}^2\Bigg)\\
	&+\binom{N}{2}\frac{1}{N}\sum_{k=1}^{N}\braket{\chi \left\lvert \operatorname{P}^{k,k+1}_t  \right\rvert \chi}\Bigg(\int_{A_{12}}\sum_{i=1}^{N}\abs{\partial_i(\Psi_o)_{12}}^2\\
	&+\int_{A_{12}}\sum_{1\leq i<j\leq N}(v_{\text{reg}})_{ij}\abs{(\Psi_o)_{12}}^2-\int_{B_{12}}\sum_{i=1}^{N}\abs{\partial_i \Psi_F}^2\Bigg)\\
	&+\binom{N}{2}\binom{N-2}{2}\int_{B_{12}^{34}}\sum_{i=1}^{N}\abs{\partial_i\Psi_{\chi}}^2\\
	&+2(N-2)\binom{N}{2}\int_{B_{12}^{23}}\sum_{i=1}^{N}\abs{\partial_i\Psi_{\chi}}^2.
	\end{aligned}
\end{equation}
We see that this reduces proving an upper bound to a case we have already analyzed in Chapter \ref{ChapterTheGroundStateEnergyOfTheOneDimensionalDiluteBoseGas}, expect for the last two term, which we then need to estimate. Let us denote the two quantities by \begin{equation}
\begin{aligned}
E_{12}^{34}&\coloneqq\binom{N}{2}\binom{N-2}{2}\int_{B_{12}^{34}}\sum_{i=1}^{N}\abs{\partial_i\Psi_{\chi}}^2,\\
E_{12}^{23}&\coloneqq 2(N-2)\binom{N}{2}\int_{B_{12}^{23}}\sum_{i=1}^{N}\abs{\partial_i\Psi_{\chi}}^2.
\end{aligned}
\end{equation} 
The following lemmas, which we prove below, provide estimates of these quantities.
\begin{lemma}\label{LemmaSpin1/2EtaContribution1}
	Let $ E_{12}^{34} $ and $ \Psi_{\chi} $ be defined as above, then we have the following bound:\begin{equation}
	E_{12}^{34}\leq \textnormal{ const. } E_F\left(N\left(\rho b\right)^4+N^2\left(\rho b\right)^6 \right).
	\end{equation}
	where $ E_F $ denotes the free spin polarized (spinless) Fermi energy.
\end{lemma}

\begin{lemma}\label{LemmaSpin1/2EtaContribution2}
	Let $ E_{12}^{23} $ and $ \Psi_{\chi} $ be defined as above, then we have the following bound:\begin{equation}
	E_{12}^{23}\leq \textnormal{ const. } E_F\left(\left(\rho b\right)^4+N\left(\rho b\right)^6 \right).
	\end{equation}
	where $ E_F $ denotes the free spin polarized (spinless) Fermi energy.
\end{lemma}
Using Lemmas \ref{LemmaSpin1/2EtaContribution1} and \ref{LemmaSpin1/2EtaContribution2}, we deduce, from \eqref{EqTrialStateEnergyUpperBound1} the following bound upper bound on the trial state energy
\begin{equation}
\begin{aligned}
\mathcal{E}(\Psi_{\chi})\leq E_F&+\binom{N}{2}\frac{1}{N}\sum_{k=1}^{N}\braket{\chi \left\lvert \operatorname{P}^{k,k+1}_s  \right\rvert \chi}\Bigg(\int_{A_{12}}\sum_{i=1}^{N}\abs{\partial_i(\Psi_e)_{12}}^2\\
&+\int_{A_{12}}\sum_{1\leq i<j\leq N}(v_{\text{reg}})_{ij}\abs{(\Psi_e)_{12}}^2-\int_{B_{12}}\sum_{i=1}^{N}\abs{\partial_i \Psi_F}^2\Bigg)\\
&+\binom{N}{2}\frac{1}{N}\sum_{k=1}^{N}\braket{\chi \left\lvert \operatorname{P}^{k,k+1}_t  \right\rvert \chi}\Bigg(\int_{A_{12}}\sum_{i=1}^{N}\abs{\partial_i(\Psi_o)_{12}}^2\\
&+\int_{A_{12}}\sum_{1\leq i<j\leq N}(v_{\text{reg}})_{ij}\abs{(\Psi_o)_{12}}^2-\int_{B_{12}}\sum_{i=1}^{N}\abs{\partial_i \Psi_F}^2\Bigg)\\
&+E_F\left(N(\rho b)^4+ N^2 (\rho b)^6\right).
\end{aligned}
\end{equation}
Defining the quantities \begin{equation}
\begin{aligned}
E_{1,e}\coloneqq\binom{N}{2}\Bigg(\int_{A_{12}}\sum_{i=1}^{N}\abs{\partial_i(\Psi_e)_{12}}^2
&+\sum_{1\leq i<j\leq N}(v_{\text{reg}})_{ij}\abs{(\Psi_e)_{12}}^2-\sum_{i=1}^{N}\abs{\partial_i \Psi_F}^2\Bigg),\\
E_{1,o}\coloneqq\binom{N}{2}\Bigg(\int_{A_{12}}\sum_{i=1}^{N}\abs{\partial_i(\Psi_o)_{12}}^2
&+\sum_{1\leq i<j\leq N}(v_{\text{reg}})_{ij}\abs{(\Psi_o)_{12}}^2-\sum_{i=1}^{N}\abs{\partial_i \Psi_F}^2\Bigg),
\end{aligned}
\end{equation}
and the quantities from Chapter \ref{ChapterTheGroundStateEnergyOfTheOneDimensionalDiluteBoseGas}:\begin{equation}
\begin{aligned}
E_2^{(1)}&\coloneqq\binom{N}{2}2N\int_{A_{12}\cap A_{13}}\sum_{i=1}^{N}\abs{\partial_i\Psi_F}^2,\\ E_2^{(2)}&\coloneqq\binom{N}{2}\binom{N-2}{2}\int_{A_{12}\cap A_{34}}\sum_{i=1}^{N}\abs{\partial_i\Psi_F}^2,
\end{aligned}
\end{equation}
we see from an inclusion/exclusion argument identical to the one in Chapter \ref{ChapterTheGroundStateEnergyOfTheOneDimensionalDiluteBoseGas} that \begin{equation}\label{EqTrialStateEnergyUpperBound2}
\begin{aligned}
\mathcal{E}(\Psi_{\chi})\leq E_F &+\frac{1}{N}\sum_{k=1}^{N}\braket{\chi \left\lvert \operatorname{P}^{k,k+1}_s  \right\rvert \chi} \left(E_{1,e}+E_2^{(1)}+E_2^{(2)}\right)\\
&+\frac{1}{N}\sum_{k=1}^{N}\braket{\chi \left\lvert \operatorname{P}^{k,k+1}_t  \right\rvert \chi} \left(E_{1,o}+E_2^{(1)}+E_2^{(2)}\right)\\
&+E_F\left(N(\rho b)^4+ N^2 (\rho b)^6\right)
\end{aligned}
\end{equation}
We see that $ E_{1,e/o} $ corresponds to the quantity $ E_1 $ in Chapter \ref{ChapterTheGroundStateEnergyOfTheOneDimensionalDiluteBoseGas} with the even/odd wave scattering solution in the trial state. The proving equivalent bound for the $ E_{1,e/o} $ amounts to following the same proof strategy and we have the equivalent lemma:
\begin{lemma}[Lemma 14 of Chapter \ref{ChapterTheGroundStateEnergyOfTheOneDimensionalDiluteBoseGas}]\label{LemmaE1BoundSpin1/2}
	Let $ E_{1,e/o} $ be defined as above. For $ N(\rho b)^3\leq 1 $ we have \begin{equation}
	E_{1,e/o}\leq E_F \left(2\rho a_{e/o}\frac{b}{b-a_{e/o}}+ \textnormal{const.}\ N(\rho b)^3\left[ 1+ \rho b^2\int v_{\textnormal{reg}}\right]\right).
	\end{equation}
\end{lemma}
We also recall the lemma\begin{lemma}[Lemma 15 of Chapter \ref{ChapterTheGroundStateEnergyOfTheOneDimensionalDiluteBoseGas}]\label{LemmaE2BoundSpin1/2}
	\begin{equation}
	E_2^{(1)}+E_2^{(2)}\leq E_F\left(N (\rho b)^4+ N^2(\rho b)^6\right).
	\end{equation}
\end{lemma}
Using Lemmas \ref{LemmaE1BoundSpin1/2} and $ \ref{LemmaE2BoundSpin1/2} $ we find the result 
\begin{lemma}\label{LemmaTrialStateEnergySpin1/2}
	For $ N(\rho b)^3\leq  1 $ and $ b>2 a_o $ we have \begin{equation}\label{EqTrialStateEnergyBound1}
	\begin{aligned}
	\mathcal{E}(\Psi_{\chi})\leq E_F\Bigg( 1&+2\rho\left[\frac{1}{4}\tilde{a}_e+\frac{3}{4}\tilde{a}_o+(\tilde{a}_o-\tilde{a}_e)\frac{1}{N}\braket{\chi \left\vert \sum_{k=1}^N S_k\cdot S_{k+1}\right\vert \chi}\right]\\ &+\textnormal{ const. } N (\rho b)^3\left(1+\rho b^2\int v_{\textnormal{reg}}\right) \Bigg),
	\end{aligned}
	\end{equation}
	where $ \tilde{a}_{e/o}\coloneqq a_{e/o}\frac{b}{b-a_{e/o}} $.
\end{lemma}
\begin{proof}
	This lemma follows directly by combining \eqref{EqTrialStateEnergyUpperBound2} with Lemmas \ref{LemmaE1BoundSpin1/2} and \ref{LemmaE2BoundSpin1/2}.
\end{proof}
It is then immediately clear that on the right-hand side of \eqref{EqTrialStateEnergyBound1}, given that $ a_o>a_e $, the optimal choice for $ \chi $ is the ground state of the periodic antiferromagnetic Heisenberg chain, which due to the Marshall-Lieb-Mattis theorem, \cite{lieb1962ordering,marshall1955antiferromagnetism}, is translation invariant. Of course, if $ a_o=a_e $, the choice of $ \chi $ is irrelevant for the right-hand side of \eqref{EqTrialStateEnergyBound1}. \\
We thus conclude that the ground state energy of the antiferromagnetic Heisenberg chain is of importance. Fortunately, this model is exactly solvable, as shown by Bethe \cite{bethe1931theorie}, and the ground state energy can be found in the thermodynamic limit, as shown by Hulthén \cite{hult1938}: \begin{lemma}[\cite{mattis2012theory}, Eq. (5.171)]\label{LemmaHeisenbergChainThermodynamicGSEnergy}
Let $ \ket{\textnormal{GS}_{\textnormal{HAF}}} $ denote the ground state of the periodic antiferromagnetic Heisenberg chain. Then\begin{equation}
\lim\limits_{N\to\infty}\braket{\textnormal{GS}_{\textnormal{HAF}}\Bigg\vert\frac{1}{N}\sum_{k=1}^N S_k\cdot S_{k+1} \Bigg\vert \textnormal{GS}_{\textnormal{HAF}}}=\frac{1}{4}-\ln(2) 
\end{equation}
\end{lemma}
This lemma, gives the ground state energy of the Heisenberg chain in the thermodynamic limit, however, we need an estimate for the finite chain. This is given by the following lemma:
\begin{lemma}\label{LemmaHeisenbergChainFiniteNEstimate}
	Let $ \ket{\textnormal{GS}_{\textnormal{HAF}}} $ denote the ground state of the periodic antiferromagnetic Heisenberg chain. Then\begin{equation}
	\braket{\textnormal{GS}_{\textnormal{HAF}}\Bigg\vert\frac{1}{N}\sum_{k=1}^N S_k\cdot S_{k+1} \Bigg\vert \textnormal{GS}_{\textnormal{HAF}}}=\frac{1}{4}-\ln(2) +\mathcal{O}(N^{-1})
	\end{equation}
\end{lemma}
\begin{proof}
	Denoting the Dirichlet (edge spin down) energy of the spin chain $ E_D^N $ with $ N $ sites and the periodic energy $ E_P^N $, we have $ E_P^N\leq E_D^N $. 
	This follows directly from the variational principle.
%	To see this notice that the Dirichlet ground state is given as the ground state of the Hamiltoninan $ H_D=\sum_{k=1}^{N} S_{k}\cdot S_{k+1}+ B_e(1+S_1^z+S_N^z) $ for large enough $ B_e>0 $. Since $ H_D\geq H_{\text{HAF}} $ and since $ B_e(1+S_1^z+S_N^z) $ vanishes in the Dirichlet ground state, we see that \begin{equation}
%	E_D^N=\braket{\sum_{k=1}^{N} S_{k}\cdot S_{k+1}}_D\geq \braket{\sum_{k=1}^{N} S_{k}\cdot S_{k+1}}_P=E_P^N ,
%	\end{equation}
%	where the subscribts refer to Dirichlet and periodic ground state expectations.
On the other hand we have the following bound \begin{equation}\label{EqDirichletPeriodicSpinChainBound}
E_D^{N+2}\leq E_P^N+\frac{3}{4}.
\end{equation}
To see this, consider a periodic chain of length $ N $ in its ground state. Add a spin-down at each edge, making the chain of length $ N+2 $. The resulting state, is  now a trial state for the Dirichlet chain of energy at most $ E_P^N+\frac{3}{4} $ and \eqref{EqDirichletPeriodicSpinChainBound} follows. Furthermore, it is not hard to see that for any integer $ m\geq 1 $ we have $ E_D^{mN}\leq E_D^{mN-m+1}\leq mE_D^N $. The first inequality follows simply from the fact that extending a Dirichlet state by Néel ordering (alternating spin) to a larger chain, lowers the energy, hence ground state energy in the larger chain must also be lower. The second inequality follows by constructing a trial state for the Dirichlet chain of length $ mN-m+1 $ by gluing $ m $ ground states of the Dirichlet chain of length $ N $, such that they share a spin down at the gluing points. Collecting everything we have \begin{equation}
\frac{1}{mN}E_P^{mN}\leq \frac{1}{mN}E_D^{mN}\leq \frac{1}{N}E_D^N\leq \frac{1}{N}\left(E_P^{N-2}+3/4\right).
\end{equation}
It is clear that by a trial state argument and by translation invariance, which follows from the Marshall-Lieb-Mattis theorem (uniqueness of the ground state), we have $ E_P^N\leq \frac{N}{M+1}E_P^M+\frac{1}{4} $ for $ M>N $, simply take the ground state of chain length $ M $ and truncate it at length $ N $. Hence we get \begin{equation}
\frac{1}{mN}E_P^{mN}\leq\frac{N-2}{N}\frac{1}{N-2}\left(E_P^{N-2}+3/4\right)\leq \frac{N-2}{N} \left(\frac{1}{M}E^{M}_P +\frac{3}{2}\frac{1}{N-2}\right)
\end{equation}
taking the limits $ m\to \infty $ and $ M\to\infty $ we have \begin{equation}
\frac{N}{N-2}e_P-\frac{3}{4N}\leq\frac{1}{N-2}E_P^{N-2}\leq e_P+\frac{1}{4}\frac{1}{N-2},
\end{equation}
where $ e_P=\lim\limits_{N\to\infty}\frac{1}{N}E_P^N $. The desired result the follows from Lemma \ref{LemmaHeisenbergChainThermodynamicGSEnergy}.
\end{proof}
We are now ready to collect everything to give the proof of Theorem \ref{TheoremUpperBoundSpin1/2Fermi}:
\begin{proof}[Proof of Theorem \ref{TheoremUpperBoundSpin1/2Fermi}]
	Consider now the full trial state as given in \eqref{EqFullTrialStateSpin1/2}. Because of the spacing between intervals, $ I_m $, there no interacting between particles in different intervals. Hence the energy of such a state \begin{equation}\label{key}
	\mathcal{E}(\Psi_{\chi,\text{full}})/\norm{\Psi_{\chi,\text{full}}}=M\mathcal{E}(\Psi_{\chi}^{I_0})/\norm{\Psi^{I_0}_{\chi}}.
	\end{equation}
	Combining lemmas \ref{LemmaTrialStateEnergySpin1/2} and \ref{LemmaHeisenbergChainFiniteNEstimate}, we find \begin{equation}
	\begin{aligned}
	\mathcal{E}(\Psi_{\chi,\text{full}})\leq N\frac{\pi^2}{3}\tilde{\rho}^2\Bigg(1&+2\tilde{\rho}\left[\ln(2)\tilde{a}_e+(1-\ln(2))\tilde{a}_o\right]+\frac{M}{N}\\&+\text{ const. }\frac{N}{M}(\tilde{\rho} b)^3 \left(1+\tilde{\rho} b^2\int v_{\text{reg}}\right)\Bigg),
	\end{aligned}
	\end{equation}
	with $ \rho\leq \tilde{\rho}=\frac{N}{L-Mb}\leq\rho\left(1+2\frac{M}{N}\rho b\right) $ for $ \frac{M}{N}\rho b\leq 1/2 $. Choosing $ M/N=(\rho b)^{3/2}\left(1+\rho b^2\int v_{\text{reg}}\right)^{1/2} $ we find \begin{equation}
	\begin{aligned}
	\mathcal{E}(\Psi_{\chi,\text{full}})\leq N\frac{\pi^2}{3}&\rho^2\Bigg(1+2\rho\left[\ln(2)\tilde{a}_e+(1-\ln(2))\tilde{a}_o\right]+\\&+\text{ const. }(\rho b)^{3/2} \left(1+\rho b^2\int v_{\text{reg}}\right)^{1/2}\Bigg),
	\end{aligned}
	\end{equation}
	Furthermore choosing $ b=\max(\rho^{-1/5}\abs{a_e}^{4/5},\rho^{-1/5}a_o^{4/5},R_0) $ we see that $ a_{e/o}\leq \tilde{a}_{e/o}=a_{e/o}\frac{b}{b-a_{e/o}}\leq a_{e/o}\left(1+2(\rho R)^{1/5}\right) $ for $ (\rho R)^{1/5}\leq 1/2 $ and the desired result follows from the simple estimate on the norm\begin{equation}
	\begin{aligned}
	\norm{\Psi^{I_0}_{\chi}}&\geq 1-\int_{A_{12}} \rho^{(2))}(x_1,x_2)\geq 1-\text{ const. }\tilde{N}(\tilde{\rho}b)^3\\&\geq 1-\text{ const. }(\rho b)^{3/2}\geq 1-\text{ const. }(\rho R)^{6/5}.
	\end{aligned}
	\end{equation}
\end{proof}



\subsubsection{Estimating $ E_{12}^{34} $ (proof of lemma \ref{LemmaSpin1/2EtaContribution1})}
\begin{proof}[Proof of Lemma \ref{LemmaSpin1/2EtaContribution1}]
%Estimating $ E_{12}^{34} $ is a straightforward computation that goes as follows:\\ 
%Defining $ \xi_{12}^{34}\coloneqq \left(\left(\eta\omega^{12}_e+(1-\eta)\omega^{12}_o\right)\operatorname{P}_s^{1,2}+\omega_o^{12}\operatorname{P}_t^{1,2}\right)\chi_\sigma $ on $ B_{12}^{34}\cap \{\sigma\} $ for all sectors $ \{\sigma\}\in S_{12}^{34} $, with \begin{equation}
%\begin{aligned}
%S_{12}^{34}\coloneqq\big\{\text{sectors } \{\sigma\}\ \vert \ (1,2)=(\sigma_k,\sigma_{k+1}) \text{ or }\\ (2,1)=(\sigma_k,\sigma_{k+1})\text{ for some } k\\
%\text{ and } (3,4)=(\sigma_l,\sigma_{l+1}) \text{ or }\\ (4,3)=(\sigma_l,\sigma_{l+1})\text{ for some } l\big\}.
%\end{aligned}
%\end{equation}
% We then see that $\Psi_{\chi}=\xi_{12}^{34} \frac{\Psi_F}{\abs{x_2-x_1}}$ on $ B_{12}^{34} $: Hence we find 
%\begin{equation}
%\begin{aligned}
%E_{12}^{34}=&\binom{N}{2}\binom{N-2}{2}\int_{B_{12}^{34}}\sum_{i=1}^{N}\abs{\partial_i\left(\xi_{12}^{34} \frac{\Psi_F}{\abs{x_2-x_1}}\right)}^2\\
%\leq& \binom{N}{2}\binom{N-2}{2}\Bigg[\int_{A_{12}^{34}}\sum_{i=1}^{4}\abs{\partial_i\left(\xi_{12}^{34} \frac{\Psi_F}{\abs{x_2-x_1}}\right)}^2 \\&\qquad+\int_{A_{12}^{34}}\sum_{i=5}^{N}\overline{\xi_{12}^{34} \frac{\Psi_F}{\abs{x_2-x_1}}}\left(\xi_{12}^{34} \frac{(-\partial_i^2\Psi_F)}{\abs{x_2-x_1}}\right)\Bigg]\\
%=& \binom{N}{2}\binom{N-2}{2}\Bigg[\int_{A_{12}^{34}}\sum_{i=1}^{4}\abs{\partial_i\left(\xi_{12}^{34} \frac{\Psi_F}{\abs{x_2-x_1}}\right)}^2 \\&\qquad-\int_{A_{12}^{34}}\sum_{i=1}^{4}\overline{\xi_{12}^{34} \frac{\Psi_F}{\abs{x_2-x_1}}}\left(\xi_{12}^{34} \frac{(-\partial_i^2\Psi_F)}{\abs{x_2-x_1}}\right)\\
%&\qquad + E_F\int_{A_{12}^{34}} \abs{\xi_{12}^{34} \frac{\Psi_F}{\abs{x_2-x_1}}}\Bigg],
%\end{aligned}
%\end{equation}
%where we used $ B_{12}^{34}\subset A_{12}^{34} $, integration by parts, and the fact that $ \Psi_F $ is an eigenfunction of $ (-\Delta) $, with eigenvalue $ E_F $.\\ Thus, using $ \abs{\xi_{12}^{34}}^2\leq b^2 $ and restricting to $ b\geq 2a_0\geq 2a_e $, we find \begin{equation}
%\begin{aligned}
%E_{12}^{34}\leq& 4\! \int_{A_{12}^{34}}\Bigg(\! \sum_{i=1}^{4}\partial_{y_i}\partial_{x_i}\overline{\frac{\xi_{12}^{34}(y)}{\abs{y_2-y_1}}}\frac{\xi_{12}^{34}(x)}{\abs{x_2-x_1}}\gamma^{(4)}(y_1,y_2,y_3,y_4;x_1,x_2,x_3,x_4)\Bigg\rvert_{y=x}\\
%&\qquad\quad+\abs{\frac{\xi_{12}^{34}(x)}{\abs{x_2-x_1}}}^2\abs{\sum_{i=1}^{4}  \partial_{y_i}^2\gamma^{(4)}(y_1,y_2,y_3,y_4;x_1,x_2,x_3,x_4)\Bigg\rvert_{y=x}}\\
%&\qquad\quad+E_F\abs{\frac{\xi_{12}^{34}(x)}{\abs{x_2-x_1}}}^2\rho^{(4)}(x_1,x_2,x_3,x_4)\Bigg)\\
%\leq& \text{ const. } E_F\left(N\left(\rho b\right)^4+N^2\left(\rho b\right)^6 \right)
%\end{aligned}
%\end{equation}
%where we used the following bounds \begin{equation}
%\begin{aligned}
%\partial_{y_i}\partial_{x_i}\frac{\gamma^{(4)}(y_1,y_2,y_3,y_4;x_1,x_2,x_3,x_4)}{\abs{x_2-x_1}\abs{y_2-y_1}}\Bigg \lvert_{y=x}&\leq \text{const.} \rho^{8},\\
%\partial_{y_i}\frac{\gamma^{(4)}(y_1,y_2,y_3,y_4;x_1,x_2,x_3,x_4)}{\abs{x_2-x_1}\abs{y_2-y_1}}\Bigg \lvert_{y=x}&\leq \text{const.} \rho^{8}\abs{x_3-x_4},\\
%\frac{\gamma^{(4)}(y_1,y_2,y_3,y_4;x_1,x_2,x_3,x_4)}{\abs{x_2-x_1}\abs{y_2-y_1}}\Bigg \lvert_{y=x}&\leq \text{const.} \rho^{8}\abs{x_3-x_4}^2,\\
%\abs{\sum_{i=1}^{4}\partial_{y_i}^2\gamma^{(4)}(y_1,y_2,y_3,y_4;x_1,x_2,x_3,x_4) \Bigg \lvert_{y=x}}&\leq \text{const.} \rho^{10} \abs{x_1-x_2}^2\abs{x_3-x_4}^2,
%\end{aligned}
%\end{equation}
%and \begin{equation}
%\rho^{(4)}(x_1,x_2,x_3,x_4)\leq\text{const.} \rho^8 \abs{x_1-x_2}^2\abs{x_3-x_4}^2,
%\end{equation}
%which all follows from Taylor expansion of the free Fermi reduced density (matrices). Furthermore, we used the bounds \begin{equation}
%\sqrt{\abs{\partial_i\xi_{12}^{34}}^2}\leq b \max\left(\frac{\sqrt{2}}{b},\frac{1}{b-a_o}\right)\leq 2,\qquad \sqrt{\abs{\xi_{12}^{34}}^2}\leq b
%\end{equation}
%which follows from properties of the scattering solution, monotonicity of its derivative, and Lemma \ref{LemmaEtaDerivative}.


Estimating $ E_{12}^{34} $ is a straightforward computation that goes as follows:\\
Define 
\begin{equation}
\begin{aligned}
\xi_{12}^{34}\coloneqq \Big(\left(\eta(\abs{x_3-x_4})\omega^{12}_e(\abs{x_1-x_2})+(1-\eta(\abs{x_3-x_4}))\omega^{12}_o(\abs{x_1-x_2})\right)\operatorname{P}_s^{1,2}\\+\omega_o^{12}(\abs{x_1-x_2})\operatorname{P}_t^{1,2}\Big)\chi_\sigma 
\end{aligned}
\end{equation} 
on $ A_{12}^{34}\cap \{\sigma\}$, for all sectors $ \{\sigma\}\in S_{12}^{34}, $ 
with \begin{equation}
\begin{aligned}
S_{12}^{34}\coloneqq\big\{\text{sectors } \{\sigma\}\ \vert \ (1,2)=(\sigma_k,\sigma_{k+1}) \text{ or }\\ (2,1)=(\sigma_k,\sigma_{k+1})\text{ for some } k\\
\text{ and } (3,4)=(\sigma_l,\sigma_{l+1}) \text{ or }\\ (4,3)=(\sigma_l,\sigma_{l+1})\text{ for some } l\big\}.
\end{aligned}
\end{equation} We then see that $\Psi_{\chi}=\xi_{12}^{34} \frac{\Psi_F}{\abs{x_2-x_1}}$ on $ B_{12}^{34} $: Hence defining \begin{equation}
\begin{aligned}
\left(\xi_{12}^{34}\right)_s&\coloneqq \eta(\abs{x_2-x_3})\omega^{12}_e(\abs{x_1-x_2})+(1-\eta(\abs{x_2-x_3}))\omega^{12}_o(\abs{x_1-x_2}),\\
\left(\xi_{12}^{34}\right)_t&\coloneqq \omega_o^{12}(\abs{x_1-x_2}),
\end{aligned}
\end{equation}

we find using $ B_{12}^{34}\subset A_{12}^{34} $
\begin{equation}
\begin{aligned}
E_{12}^{34}=&\binom{N}{2}2(N-2)\int_{B_{12}^{34}}\sum_{i=1}^{N}\abs{\partial_i\left(\xi_{12}^{34} \frac{\Psi_F}{\abs{x_2-x_1}}\right)}^2\\
\leq& \binom{N}{2}2(N-1)\sum_{a\in\{s,t\}}\sum_{\{\sigma\}\in S_{12}^{34}}\braket{\chi_{\sigma}\left\lvert \operatorname{P}_a^{12}\right \rvert \chi_{\sigma}}\\&\qquad\qquad\times\Bigg[\int_{A_{12}^{34}\cap\{\sigma\}}\sum_{i=1}^{N}\abs{\partial_i\left(\left(\xi_{12}^{34}\right)_a \frac{\Psi_F}{\abs{x_2-x_1}}\right)}^2\Bigg].
\end{aligned}
\end{equation}
One may use that $ \braket{\chi_{\sigma}\left\lvert \operatorname{P}_a^{12}\right \rvert \chi_{\sigma}} $ is independent of $ \sigma $, however, since we are not interested in finding the optimal constant in Lemma \ref{LemmaSpin1/2EtaContribution1} we instead use the more crude bound, $ \braket{\chi_{\sigma}\left\lvert \operatorname{P}_a^{12}\right \rvert \chi_{\sigma}}\leq 1 $, to find
\begin{equation}
\begin{aligned}
E_{12}^{34}\leq& \binom{N}{2}2(N-1)\sum_{a\in\{s,t\}}\Bigg[\int_{A_{12}^{34}}\sum_{i=1}^{4}\abs{\partial_i\left(\left(\xi_{12}^{34}\right)_a \frac{\Psi_F}{\abs{x_2-x_1}}\right)}^2 \\&\qquad\qquad\quad+\int_{A_{12}^{34}}\sum_{i=5}^{N}\overline{\left(\xi_{12}^{34}\right)_a \frac{\Psi_F}{\abs{x_2-x_1}}}\left(\xi_{12}^{34} \frac{(-\partial_i^2\Psi_F)}{\abs{x_2-x_1}}\right)\Bigg].
\end{aligned}
\end{equation}
where we used integration by parts and $ \bigsqcup_{\{\sigma\}\in S_{12}^{34}}\left(A_{12}^{34}\cap \{\sigma\}\right)\subset A_{12}^{34} $.  Using that $ \Psi_F $ is an eigenfunction of $ (-\Delta) $, with eigenvalue $ E_F $ we further find
\begin{equation}
\begin{aligned}
E_{12}^{34}\leq& \binom{N}{2}2(N-2)\sum_{a\in\{s,t\}}\Bigg[\int_{A_{12}^{34}}\sum_{i=1}^{4}\abs{\partial_i\left(\left(\xi_{12}^{34}\right)_a \frac{\Psi_F}{\abs{x_2-x_1}}\right)}^2 \\&\qquad-\int_{A_{12}^{34}}\sum_{i=1}^{4}\overline{\left(\xi_{12}^{34}\right)_a \frac{\Psi_F}{\abs{x_2-x_1}}}\left(\left(\xi_{12}^{34}\right)_a \frac{(-\partial_i^2\Psi_F)}{\abs{x_2-x_1}}\right)\\
&\qquad + E_F\int_{A_{12}^{34}} \abs{\left(\xi_{12}^{34}\right)_a \frac{\Psi_F}{\abs{x_2-x_1}}}\Bigg].
\end{aligned}
\end{equation}
Thus, using $ \abs{\left(\xi_{12}^{34}\right)_a}^2\leq b^2 $ and restricting to $ b\geq 2a_o\geq2a_e $, we find \begin{equation}
\begin{aligned}
E_{12}^{34}\leq& 4\!\sum_{a\in\{s,t\}} \int_{A_{12}^{34}}\Bigg(\! \sum_{i=1}^{4}\partial_{y_i}\partial_{x_i}\overline{\frac{\left(\xi_{12}^{34}\right)_a(y)}{\abs{y_2-y_1}}}\frac{\left(\xi_{12}^{34}\right)_a(x)}{\abs{x_2-x_1}}\gamma^{(4)}(y_1,y_2,y_3,y_4;x_1,x_2,x_3,x_4)\Bigg\rvert_{y=x}\\
&\qquad\quad+\abs{\frac{\left(\xi_{12}^{34}\right)_a(x)}{\abs{x_2-x_1}}}^2\abs{\sum_{i=1}^{4}  \partial_{y_i}^2\gamma^{(4)}(y_1,y_2,y_3,y_4;x_1,x_2,x_3,x_4)\Bigg\rvert_{y=x}}\\
&\qquad\quad+E_F\abs{\frac{\left(\xi_{12}^{34}\right)_a(x)}{\abs{x_2-x_1}}}^2\rho^{(4)}(x_1,x_2,x_3,x_4)\Bigg)\\
\leq& \text{ const. } E_F\left(N\left(\rho b\right)^4+N^2\left(\rho b\right)^6 \right)
\end{aligned}
\end{equation}
where we used the following bounds \begin{equation}
\begin{aligned}
\partial_{y_i}\partial_{x_i}\frac{\gamma^{(4)}(y_1,y_2,y_3,y_4;x_1,x_2,x_3,x_4)}{\abs{x_2-x_1}\abs{y_2-y_1}}\Bigg \lvert_{y=x}&\leq \text{const.} \rho^{8},\\
\partial_{y_i}\frac{\gamma^{(4)}(y_1,y_2,y_3,y_4;x_1,x_2,x_3,x_4)}{\abs{x_2-x_1}\abs{y_2-y_1}}\Bigg \lvert_{y=x}&\leq \text{const.} \rho^{8}\abs{x_3-x_4},\\
\frac{\gamma^{(4)}(y_1,y_2,y_3,y_4;x_1,x_2,x_3,x_4)}{\abs{x_2-x_1}\abs{y_2-y_1}}\Bigg \lvert_{y=x}&\leq \text{const.} \rho^{8}\abs{x_3-x_4}^2,\\
\abs{\sum_{i=1}^{4}\partial_{y_i}^2\gamma^{(4)}(y_1,y_2,y_3,y_4;x_1,x_2,x_3,x_4) \Bigg \lvert_{y=x}}&\leq \text{const.} \rho^{10} \abs{x_1-x_2}^2\abs{x_3-x_4}^2,
\end{aligned}
\end{equation}
and \begin{equation}
\rho^{(4)}(x_1,x_2,x_3,x_4)\leq\text{const.} \rho^8 \abs{x_1-x_2}^2\abs{x_3-x_4}^2,
\end{equation}
which all follows from Taylor expansion of the free Fermi reduced density (matrices). Furthermore, we used the bounds \begin{equation}
\sqrt{\abs{\partial_i\left(\xi_{12}^{34}\right)_a}^2}\leq b \max\left(\frac{\sqrt{2}}{b},\frac{1}{b-a_o}\right)\leq 2,\qquad \sqrt{\abs{\left(\xi_{12}^{34}\right)_a}^2}\leq b
\end{equation}
which follows from properties of the scattering solution, monotonicity of its derivative, and Lemma \ref{LemmaEtaDerivative}.
\end{proof}
\subsubsection{Estimating $ E_{12}^{23} $ (proof of Lemma \ref{LemmaSpin1/2EtaContribution2})}
\begin{proof}[Proof of Lemma \ref{LemmaSpin1/2EtaContribution2}]
Estimating $ E_{12}^{23} $ is, similarly to the estimation of $ E_{12}^{34} $, a straightforward computation. We retrace the steps of the previous calculation, suitably modified for $ E_{12}^{23} $, in the following:\\
Defining 
\begin{equation}
	\begin{aligned}
	\xi_{12}^{23}\coloneqq \Big(\left(\eta(\abs{x_2-x_3})\omega^{12}_e(\abs{x_1-x_2})+(1-\eta(\abs{x_2-x_3}))\omega^{12}_o(\abs{x_1-x_2})\right)\operatorname{P}_s^{1,2}\\+\omega_o^{12}(\abs{x_1-x_2})\operatorname{P}_t^{1,2}\Big)\chi_\sigma 
	\end{aligned}
\end{equation} 
on $ A_{12}^{23}\cap \{\sigma\}$, for all sectors $ \{\sigma\}\in S_{12}^{23}, $ 
with \begin{equation}
	\begin{aligned}
	S_{12}^{23}\coloneqq\big\{\text{sectors } \{\sigma\}\ \vert \ (1,2,3)=(\sigma_k,\sigma_{k+1},\sigma_{k+2}) \text{ or }\\ (3,2,1)=(\sigma_k,\sigma_{k+1},\sigma_{k+2}) \text{ for some } k\big\}
	\end{aligned}
\end{equation}. We then see that $\Psi_{\chi}=\xi_{12}^{23} \frac{\Psi_F}{\abs{x_2-x_1}}$ on $ B_{12}^{23} $: Hence defining \begin{equation}
	\begin{aligned}
	\left(\xi_{12}^{23}\right)_s&\coloneqq \eta(\abs{x_2-x_3})\omega^{12}_e(\abs{x_1-x_2})+(1-\eta(\abs{x_2-x_3}))\omega^{12}_o(\abs{x_1-x_2}),\\
	\left(\xi_{12}^{23}\right)_t&\coloneqq \omega_o^{12}(\abs{x_1-x_2}),
	\end{aligned}
\end{equation}

 we find using $ B_{12}^{23}\subset A_{12}^{23} $
\begin{equation}
\begin{aligned}
E_{12}^{23}=&\binom{N}{2}2(N-2)\int_{B_{12}^{23}}\sum_{i=1}^{N}\abs{\partial_i\left(\xi_{12}^{23} \frac{\Psi_F}{\abs{x_2-x_1}}\right)}^2\\
\leq& \binom{N}{2}2(N-1)\sum_{a\in\{s,t\}}\sum_{\{\sigma\}\in S_{12}^{23}}\braket{\chi_{\sigma}\left\lvert \operatorname{P}_a^{12}\right \rvert \chi_{\sigma}}\\&\qquad\qquad\times\Bigg[\int_{A_{12}^{23}\cap\{\sigma\}}\sum_{i=1}^{N}\abs{\partial_i\left(\left(\xi_{12}^{23}\right)_a \frac{\Psi_F}{\abs{x_2-x_1}}\right)}^2\Bigg].
\end{aligned}
\end{equation}
One may use that $ \braket{\chi_{\sigma}\left\lvert \operatorname{P}_a^{12}\right \rvert \chi_{\sigma}} $ is independent of $ \sigma $, however, since we are not interested in finding the optimal constant in Lemma \ref{LemmaSpin1/2EtaContribution2} we instead use the more crude bound, $ \braket{\chi_{\sigma}\left\lvert \operatorname{P}_a^{12}\right \rvert \chi_{\sigma}}\leq 1 $, to find
\begin{equation}\label{key}
\begin{aligned}
E_{12}^{23}\leq& \binom{N}{2}2(N-1)\sum_{a\in\{s,t\}}\Bigg[\int_{A_{12}^{23}}\sum_{i=1}^{3}\abs{\partial_i\left(\left(\xi_{12}^{23}\right)_a \frac{\Psi_F}{\abs{x_2-x_1}}\right)}^2 \\&\qquad\qquad\quad+\int_{A_{12}^{23}}\sum_{i=4}^{N}\overline{\left(\xi_{12}^{23}\right)_a \frac{\Psi_F}{\abs{x_2-x_1}}}\left(\xi_{12}^{23} \frac{(-\partial_i^2\Psi_F)}{\abs{x_2-x_1}}\right)\Bigg].
\end{aligned}
\end{equation}
where we used integration by parts and $ \bigsqcup_{\{\sigma\}\in S_{12}^{23}}\left(A_{12}^{23}\cap \{\sigma\}\right)\subset A_{12}^{23} $.  Using that $ \Psi_F $ is an eigenfunction of $ (-\Delta) $, with eigenvalue $ E_F $ we further find
\begin{equation}
\begin{aligned}
E_{12}^{23}\leq& \binom{N}{2}2(N-2)\sum_{a\in\{s,t\}}\Bigg[\int_{A_{12}^{23}}\sum_{i=1}^{3}\abs{\partial_i\left(\left(\xi_{12}^{23}\right)_a \frac{\Psi_F}{\abs{x_2-x_1}}\right)}^2 \\&\qquad-\int_{A_{12}^{23}}\sum_{i=1}^{3}\overline{\left(\xi_{12}^{23}\right)_a \frac{\Psi_F}{\abs{x_2-x_1}}}\left(\left(\xi_{12}^{23}\right)_a \frac{(-\partial_i^2\Psi_F)}{\abs{x_2-x_1}}\right)\\
&\qquad + E_F\int_{A_{12}^{23}} \abs{\left(\xi_{12}^{23}\right)_a \frac{\Psi_F}{\abs{x_2-x_1}}}\Bigg].
\end{aligned}
\end{equation}
 Thus, using $ \abs{\left(\xi_{12}^{23}\right)_a}^2\leq b^2 $ and restricting to $ b\geq 2a_o\geq2a_e $, we find \begin{equation}
\begin{aligned}
E_{12}^{23}\leq& 4\!\sum_{a\in\{s,t\}} \int_{A_{12}^{23}}\Bigg(\! \sum_{i=1}^{3}\partial_{y_i}\partial_{x_i}\overline{\frac{\left(\xi_{12}^{23}\right)_a(y)}{\abs{y_2-y_1}}}\frac{\left(\xi_{12}^{23}\right)_a(x)}{\abs{x_2-x_1}}\gamma^{(3)}(y_1,y_2,y_3;x_1,x_2,x_3)\Bigg\rvert_{y=x}\\
&\qquad\quad+\abs{\frac{\left(\xi_{12}^{23}\right)_a(x)}{\abs{x_2-x_1}}}^2\abs{\sum_{i=1}^{3}  \partial_{y_i}^2\gamma^{(3)}(y_1,y_2,y_3;x_1,x_2,x_3)\Bigg\rvert_{y=x}}\\
&\qquad\quad+E_F\abs{\frac{\left(\xi_{12}^{23}\right)_a(x)}{\abs{x_2-x_1}}}^2\rho^{(3)}(x_1,x_2,x_3)\Bigg)\\
\leq& \text{ const. } E_F\left(\left(\rho b\right)^4+N\left(\rho b\right)^6 \right)
\end{aligned}
\end{equation}
where we used the following bounds \begin{equation}
\begin{aligned}
\partial_{y_i}\partial_{x_i}\frac{\gamma^{(3)}(y_1,y_2,y_3;x_1,x_2,x_3)}{\abs{x_2-x_1}\abs{y_2-y_1}}\Bigg \lvert_{y=x}&\leq \text{const.} \rho^{7},\\
\partial_{y_i}\frac{\gamma^{(3)}(y_1,y_2,y_3;x_1,x_2,x_3)}{\abs{x_2-x_1}\abs{y_2-y_1}}\Bigg \lvert_{y=x}&\leq \text{const.} \rho^{7}\abs{x_3-x_4},\\
\frac{\gamma^{(3)}(y_1,y_2,y_3;x_1,x_2,x_3)}{\abs{x_2-x_1}\abs{y_2-y_1}}\Bigg \lvert_{y=x}&\leq \text{const.} \rho^{7}\abs{x_3-x_4}^2,\\
\abs{\sum_{i=1}^{3}\partial_{y_i}^2\gamma^{(3)}(y_1,y_2,y_3;x_1,x_2,x_3) \Bigg \lvert_{y=x}}&\leq \text{const.} \rho^{11} \abs{x_1-x_2}^2\abs{x_2-x_3}^2\abs{x_1-x_3}^2,
\end{aligned}
\end{equation}
and \begin{equation}
\rho^{(3)}(x_1,x_2,x_3)\leq\text{const.} \rho^9 \abs{x_1-x_2}^2\abs{x_2-x_3}^2\abs{x_1-x_3}^2,
\end{equation}
which all follows from Taylor expansion of the free Fermi reduced density (matrices). Furthermore, we used the bounds \begin{equation}
\sqrt{\abs{\partial_i\left(\xi_{12}^{23}\right)_a}^2}\leq b \max\left(\frac{\sqrt{2}}{b},\frac{1}{b-a_o}\right)\leq 2,\qquad \sqrt{\abs{\left(\xi_{12}^{23}\right)_a}^2}\leq b
\end{equation}
which follows from properties of the scattering solution, monotonicity of its derivative, and Lemma \ref{LemmaEtaDerivative}.
\end{proof}
\section{Extending the upper bound to other symmetries and spin dependent potentials}
We present here corollaries that follows directly, \emph{mutatis mutandis}, from the proof of Theorem \ref{TheoremUpperBoundSpin1/2Fermi}. We also give an application of one the results, to a model where the new upper bound improves best up to now best known result.
\subsection{Spin-1/2 bosons}
Going through the proof of Theorem \ref{TheoremUpperBoundSpin1/2Fermi} (and the lemmas used), we obtain an immediate corollary. Changing spin-space anti-symmetry to spin-space symmetry, we obtain the equivalent result for bosons. The change of symmetry interchanges the even and odd condition in the singlet and triplet, hence constructing the trial state \eqref{EqTrial StateSpin1/2Fermi}, we must interchange $ \operatorname{P}_s $ and $ \operatorname{P}_t $. Thus we get \begin{equation}\label{EqTrial StateSpin1/2Bose}
\Psi_\chi=\begin{cases}
\frac{\Psi_F}{\mathcal{R}}\left(\left(\eta\omega^{\mathcal{R}}_e+(1-\eta)\omega^{\mathcal{R}}_o\right)\operatorname{P}_t^{\mathcal{R}}+\omega_o^{\mathcal{R}}\operatorname{P}_s^{\mathcal{R}}\right)\chi,&\mathcal{R}(x)<b\\
\Psi_F,&\mathcal{R}(x)\geq b
\end{cases}.
\end{equation}
The proof is unchanged except for the choice of $ \chi $. In this case, since $ a_o\geq a_e $ and the roles of $ a_o $ and $ a_e $ are exchanged, the optimal choice for $ \chi $ is a spin polarized state. Hence we get the following corollary:
\begin{corollary}[Bosonic version of Theorem \ref{TheoremUpperBoundSpin1/2Fermi}]\label{CorollaryUpperBoundSpin1/2Bose}
	Let $ v $ satisfy the assumption from above, then the ground state energy of the dilute spin--$ 1/2 $ Bose gas satisfies\begin{equation}
		E\leq N\frac{\pi^2}{3}\rho^2\left(1+2\rho a_e+\mathcal{O}\left((\rho R)^{6/5}+N^{-1}\right)\right)
	\end{equation}
	Here $ R=\max(\abs{a_e}, R_0) $.
\end{corollary}


\subsection{Spin dependent potentials}
Interestingly, the proof of Theorem \ref{TheoremUpperBoundSpin1/2Fermi} we gave in the last section, allows for a slight generalization to potentials that are of the form \begin{equation}
v(x_i-x_j)=v_e(x_i-x_j) \operatorname{P}^{i,j}_s+v_o(x_i-x_j)\operatorname{P}^{i,j}_t
\end{equation} 
with $  v_{e/o}=v_{e/o,\text{h.c.}}+v_{e/o,\text{reg}} $ each satisfying the assumptions on $ v $. In this case the $ E_{1,e/o} $ becomes
\begin{equation}
\begin{aligned}
E_{1,e}\coloneqq\binom{N}{2}\Bigg(\int_{A_{12}}\sum_{i=1}^{N}\abs{\partial_i(\Psi_e)_{12}}^2
&+\sum_{1\leq i<j\leq N}(v_{e,\text{reg}})_{ij}\abs{(\Psi_e)_{12}}^2-\sum_{i=1}^{N}\abs{\partial_i \Psi_F}^2\Bigg),\\
E_{1,o}\coloneqq\binom{N}{2}\Bigg(\int_{A_{12}}\sum_{i=1}^{N}\abs{\partial_i(\Psi_o)_{12}}^2
&+\sum_{1\leq i<j\leq N}(v_{o,\text{reg}})_{ij}\abs{(\Psi_o)_{12}}^2-\sum_{i=1}^{N}\abs{\partial_i \Psi_F}^2\Bigg),
\end{aligned}
\end{equation}
Consequently, Theorem \ref{TheoremUpperBoundSpin1/2Fermi} still holds, with $ a_e $ the even-wave scattering length of $ v_e $ and $ a_o $ the odd wave scattering length of $ v_o $. We summarize this observation in the following corollary
\begin{corollary}[Spin dependent version of Theorem \ref{TheoremUpperBoundSpin1/2Fermi}]\label{CorollaryUpperBoundSpin1/2FermiSpinDependent}
	Let $ v=v_e\operatorname{P}_s+v_o\operatorname{P}_t\geq0 $ satisfy the assumption from above, then the ground state energy of the dilute spin--$ 1/2 $ Fermi gas satisfies\begin{equation}
	E\leq N\frac{\pi^2}{3}\rho^2\left(1+2\rho \left(\ln(2) a_e+(1-\ln(2))a_o\right)+\mathcal{O}\left((\rho R)^{6/5}+N^{-1}\right)\right),
	\end{equation}
	if $ a_o\geq a_e $ and 
	\begin{equation}
	E\leq N\frac{\pi^2}{3}\rho^2\left(1+2\rho a_o+\mathcal{O}\left((\rho R)^{6/5}+N^{-1}\right)\right),
	\end{equation}
	if $ a_o\leq a_e $.\\
	Here $ R=\max(\abs{a_e}, a_o, R_0) $. Furthermore, $ a_e $ denotes the even-wave scattering length of $ v_e $ and $ a_o $ the odd wave scattering length of $ v_o $.
\end{corollary}
\begin{proof}
	Repeat the proof of Theorem \ref{TheoremUpperBoundSpin1/2Fermi} but change $ \omega_{e/o} $ to even/odd wave scattering solutions of $ v_{e/o} $. Notice that it is no longer clear that $ a_o>a_e $ and hence the choice of $ \chi $ is the periodic antiferromagnetic Heisenberg chain when $ a_o\geq a_e $ and a spin polarized state when $ a_e>a_o $, both of which are translation invariant.
\end{proof}
An interesting application of a version of Corollary \ref{CorollaryUpperBoundSpin1/2FermiSpinDependent} given below in Corollary \ref{CorollaryUpperBoundSpin1/2SymmetricSpinDependent} is the Lieb-Liniger-Heisenberg model introduced by Girardeau in \cite{girardeau2006ground}. In his paper, an upper bound is given by a trial state argument in the case $ c>c' $ \begin{equation}\label{EqGirardeauUpperBoundLLH}
E_{LLH}\leq E_{LL}(\ln(2)c'+(1-\ln(2))c),
\end{equation}
where $ E_{LL}(\cdot) $ is the ground state energy of the Lieb-Liniger model as a function of the coupling strength. The Lieb-Liniger-Heisenberg model is defined with the formal Hamiltonian\begin{equation}\label{EqHamiltonianLLH}
H_{LLH}=-\sum_i \partial_i^2 +2\sum_{i<j} \left(c'\operatorname{P}^{i,j}_s+c\operatorname{P}^{i,j}_t\right)\delta(x_i-x_j),
\end{equation}
However the domain is taken to be wave functions that are \emph{symmetric in the spacial coordinates} meaning that under combined spin-space coordinate exchange $ (x_i,\sigma_i)\leftrightarrow(x_j,\sigma_j) $ the $ (i,j) $-singlet part of the wave function is anti-symmetric and $ (i,j) $-triplet part is symmetric. This of course implies that Corollary \ref{CorollaryUpperBoundSpin1/2FermiSpinDependent} is not directly useful in this case. However, Going through the proof of Theorem \ref{TheoremUpperBoundSpin1/2Fermi}, we see that we may as well get the following corollary  
\begin{corollary}[Spatially symmetric, spin dependent version of Theorem \ref{TheoremUpperBoundSpin1/2Fermi}]\label{CorollaryUpperBoundSpin1/2SymmetricSpinDependent}
	Let $ v=v_s\operatorname{P}_s+v_t\operatorname{P}_t\geq0 $ satisfy the assumption from above, then the ground state energy of the dilute spin--$ 1/2 $ spacially symmetric gas satisfies\begin{equation}
	E\leq N\frac{\pi^2}{3}\rho^2\left(1+2\rho \left(\ln(2) a_s+(1-\ln(2))a_t\right)+\mathcal{O}\left((\rho R)^{6/5}+N^{-1}\right)\right),
	\end{equation}
	if $ a_t\geq a_s $ and 
	\begin{equation}
	E\leq N\frac{\pi^2}{3}\rho^2\left(1+2\rho a_t+\mathcal{O}\left((\rho R)^{6/5}+N^{-1}\right)\right),
	\end{equation}
	if $ a_t \leq a_s $.\\
	Here $ R=\max(\abs{a_s}, \abs{a_t}, R_0) $. Furthermore, $ a_s $ denotes the even wave scattering length of $ v_s $ and $ a_t $ the even wave scattering length of $ v_t $.
\end{corollary}
\begin{proof}
	Repeat the proof of Theorem \ref{TheoremUpperBoundSpin1/2Fermi} (including lemmas used) but change $ \omega_{e/o} $ to the even wave scattering solution of $ v_{s/t} $ and extend the trial state to all sectors, $ \{\sigma\} $, by spatial symmetry instead of spin-space anti-symmetry. The choice of $ \chi $ is the periodic antiferromagnetic Heisenberg chain when $ a_t\geq a_s $ and a spin polarized state when $ a_s\geq a_t $.
	Whenever anti-symmetry was used in the proof of Theorem \ref{TheoremUpperBoundSpin1/2Fermi} the same step may be justified by spacial symmetry. To see this, we note that \eqref{EqEnergy2Spin1/2} can be derived by use of only spatial symmetry. However, in \eqref{EqEnergy3Spin1/2} we find instead
	\begin{equation}
	\begin{aligned}
	\int_{B_{12}^{\geq}}\abs{\partial_i\Psi_{\chi}}^2\leq&\sum_{\{\sigma\}\in S_{12}}\left(\int_{A_{12}\cap\{\sigma\}}\abs{\partial_i(\Psi_e)_{12}}^2\right)\braket{\chi \left\lvert \operatorname{P}^{\sigma^{-1}(1),\sigma^{-1}(2)}_s  \right\rvert \chi}\\
	&+\sum_{\{\sigma\}\in S_{12}}\left(\int_{A_{12}\cap \{\sigma\}}\abs{\partial_i(\Psi_o)_{12}}^2\right)\braket{\chi \left\lvert \operatorname{P}^{\sigma^{-1}(1),\sigma^{-1}(2)}_t  \right\rvert \chi},
	\end{aligned}
	\end{equation}
	where $ \sigma^{-1}(i) $ is defined such that $ \sigma_{\sigma^{-1}(i)}=i $. This is a consequence of the fact that the spins are not permuted when defining the trial state using the spatial symmetry. A similar modification is made in the proofs of Lemmas \ref{LemmaSpin1/2EtaContribution1} and \ref{LemmaSpin1/2EtaContribution2}. From this point, the proof proceeds as before by noticing that 
	$$ \braket{\chi \left\lvert \operatorname{P}^{\sigma^{-1}(1),\sigma^{-1}(2)}_{s/t}  \right\rvert \chi}=\frac{1}{N}\sum_{k=1}^{N}\braket{\chi \left\lvert \operatorname{P}^{k,k+1}_{s/t}  \right\rvert \chi} $$
	is independent of $ \sigma\in S_{12} $ because of translation invariance of $ \chi $.
\end{proof}
We see that the upper bound given by Corollary \ref{CorollaryUpperBoundSpin1/2SymmetricSpinDependent} (up to a small error in the dilute limit) is \begin{equation}
E_{LLH}\leq E_{LL}\left(\left(\frac{\ln(2)}{c'}+\frac{1-\ln(2)}{c}\right)^{-1}\right),
\end{equation}
when $ c>c' $. By the weighted harmonic-arithmetic mean inequality it is clear that our bound improves \eqref{EqGirardeauUpperBoundLLH}. The two bounds agree in the limit $ \frac{c-c'}{c'}\to0 $. However, \eqref{EqGirardeauUpperBoundLLH} gives just the free Fermi energy on the right-hand side when $ c\to\infty $, whereas our bound reduces to the correct Yang-Gaudin energy, to leading order, in this limit.


\section{Lower bound}
In this section, we will further motivate the Conjecture \ref{ConjectureEqCSpin1/2FermiGroundStateEnergy}, however a complete proof of a lower bound matching the upper bound in Theorem \ref{TheoremUpperBoundSpin1/2Fermi} is still missing. One may try to apply the same technique as was used in Chapter \ref{ChapterTheGroundStateEnergyOfTheOneDimensionalDiluteBoseGas}, however, we will see that there are obstacles in this strategy.\\
\subsection{Solvable cases}
To begin with, we may analyze the solvable models at hand. We will see that these are in agreement with Conjecture \ref{ConjectureEqCSpin1/2FermiGroundStateEnergy}.\\

\textbf{The hard core model}: The first solvable case is the hard core model, with $ v=\infty \mathbbm{1}_{[-a,a]} $, with $ a_e=a_o=a $ by Example \ref{ExampleScatteringLengthHardCore}. In this case we have \begin{equation}
E=E_F\left(L\to \frac{1}{1-\rho a}L\right)=N\frac{\pi^2}{3}\rho^2\left(1-\rho a\right)^{-2}+\rho^2\mathcal{O}(1),
\end{equation} 
with $ E_F\left(L\to \frac{1}{1-\rho a}L\right) $ denoting the spin polarized free Fermi energy in a box of length $ \frac{1}{1-\rho a}L $. Of course since since $ a_e=a_o=a $ in this case we have \begin{equation}
E=N\frac{\pi^2}{3}\rho^2\left(1-\ln(2)\rho a_e-(1-\ln(2))\rho a_o\right)^{-2}+\rho^2\mathcal{O}(1),
\end{equation}
which match Conjecture \ref{ConjectureEqCSpin1/2FermiGroundStateEnergy}.\\

\textbf{The Yang-Gaudin model}:
This model was studied in Section \ref{SectionYG}. In this case we have $ a_e=-2/c $ and $ a_o=0 $ by Example \ref{ExampleScatteringLengthDelta} Of course the upper bound from Theorem \ref{TheoremUpperBoundSpin1/2Fermi} applies. Furthermore, we found in Proposition \ref{PropositionYGLowerBound} the bound\begin{equation}
e=``\lim\limits_{\substack{N,L\to\infty\\
N/L=\rho}}E/L"\geq \frac{\pi^2}{3}\rho^3\left[\left(1-\ln(2)\rho a_e\right)^{-2}\right].
\end{equation}
Here ``" is used to emphasize that $ e $ is strictly speaking not known to be the true ground state energy (see Subsection \ref{SubsectionYGCaveat}).
Hence we conclude $ e=\frac{\pi^2}{3}\rho^3\left(1+2\ln(2)\rho a_e+\mathcal{O}(\rho \abs{a_e})^{6/5}\right) $, which is in agreement with Conjecture \ref{ConjectureEqCSpin1/2FermiGroundStateEnergy}.

\subsection{The general case}
In the case of a general potential, $ v $, where the resulting model is not solvable, we migth attempt to mimic the proof from the bosonic/spin polarized case in Chapter \ref{ChapterTheGroundStateEnergyOfTheOneDimensionalDiluteBoseGas}. We will here follow this strategy. We note first that Lemmas 20 and 21 of Chapter \ref{ChapterTheGroundStateEnergyOfTheOneDimensionalDiluteBoseGas} does not depend on any symmetry of the wave function. Dyson's lemma (Lemma 22 of Chapter \ref{ChapterTheGroundStateEnergyOfTheOneDimensionalDiluteBoseGas}) is modifed slightly in the following way: Let $ H^1_{\text{even/odd}} $ denote even/odd $ H^1 $ functions, then we have the following lemma.
\begin{lemma}[Dyson's lemma spin--$ 1/2 $ fermions]
	\label{LemmaDysonSpin1/2Fermi}
	Let $ R>R_0=\textnormal{range}(v) $ and $ \varphi\in \left(H_{\textnormal{even}}^1(\R)\otimes\operatorname{P}_s \left(\C^2\right)^2\right)\oplus\left(H_{\textnormal{odd}}^1(\R)\otimes \operatorname{P}_t \left(\C^2\right)^2 \right) $, then for any interval $ \mathcal{I}\ni 0 $ 
	\begin{equation}
	\int_{\mathcal{I}} \abs{\partial \varphi}^2+\frac12 v\abs{\varphi}^2\geq \int_{\mathcal{I}}\overline{\varphi}\left(\frac{1}{R-a_e}\operatorname{P}_s+\frac{1}{R-a_o}\operatorname{P}_t\right)\left(\delta_R+\delta_{-R}\right)\varphi,
	\end{equation}
	where $ a $ is the s-wave scattering length.
\end{lemma}
\begin{proof}
	The lemma follows straigthforwardly from the Definitions \ref{DefinitionEvenScatteringLength} and \ref{DefinitionOddScatteringLenght}.
\end{proof}
Thus we may prove the equivalent of Lemma 23 of Chapter \ref{ChapterTheGroundStateEnergyOfTheOneDimensionalDiluteBoseGas}: In the following $ \Psi $ denotes the spin--$ 1/2 $ fermionic (Neumann) ground state of \begin{equation}
H=-\sum_{i=1}^{N}\partial_i^2+\sum_{1\leq i<j\leq N}v(x_i-x_j).
\end{equation}	
We shall also define the contionuous function $ \psi\in \left(L^2([0,L-(n-1)R])\otimes\C^2\right)^{\otimes N} $, with $ R\geq \max\left(R_0,2\abs{a_e},2a_o\right)  $, such that for $ 0\leq x_1\leq\dots\leq x_n\leq L-(n-1)R $
\begin{equation}
\label{Definition:psi}
\psi(x_1,x_2,\dots,x_n):=\Psi(x_1,R+x_2,\dots,(n-1)R+x_n), 
\end{equation}
 and extended by spacial symmtry.
	\begin{lemma}\label{LemmaNormBoundEpsilonSpin1/2Fermi}
	Let $R>\max\left(R_0,2\abs{a}\right) $ and $ \epsilon\in[0,1] $. For $ \psi $ defined in \eqref{Definition:psi},
	\begin{equation}
	\begin{aligned}
	\int \sum_{i}\abs{\partial_i\Psi}^2+\sum_{i\neq j} \frac{1}{2}v_{ij}\abs{\Psi}^2\geq E_{LLH}^N \left(N,\tilde{L},\frac{2\epsilon}{R-a_e},\frac{2\epsilon}{R-a_o}\right)\braket{\psi|\psi}\\+ \frac{(1-\epsilon)}{R^2}\textnormal{const. }(1-\braket{\psi|\psi}).
	\end{aligned}
	\end{equation}
	where $ \tilde{L}:=L-(n-1)R $, the superscript ``$ N $" denotes Neumann boundary condition, and $ E_{LLH}(N,L,c',c) $ is the ground state energy of the Lieb-Liniger-Heisenberg model in \eqref{EqHamiltonianLLH}.
\end{lemma}
\begin{proof}
	We mimic the proof of Chapter \ref{ChapterTheGroundStateEnergyOfTheOneDimensionalDiluteBoseGas} (\cite{agerskov2022ground}): Splitting the energy functional in two parts, and using Lemma 21 from Chapter \ref{ChapterTheGroundStateEnergyOfTheOneDimensionalDiluteBoseGas} on one term and Lemma \ref{LemmaDysonSpin1/2Fermi} on the other, we find 
	\begin{equation}
	\begin{aligned}
	&\int \sum_{i}\abs{\partial_i\Psi}^2+\sum_{i\neq j} \frac{1}{2}v_{ij}\abs{\Psi}^2\geq\\ &\int\sum_{i}\abs{\partial_i\Psi}^2\mathds{1}_{\mathfrak{r}_i(x)>R}+\overline{\Psi}\epsilon\sum_{i}\delta(\mathfrak{r}_i(x)-R)\left(\frac{1}{R-a_e}\operatorname{P}_s^{i,j_i}+\frac{1}{R-a_o}\operatorname{P}_t^{i,j_i}\right)\Psi\\&\qquad\qquad\qquad+ (1-\epsilon)\left(\sum_{i<j}\int_{D_{ij}}\abs{\partial_i \Psi}^2+\int\sum_{i<j} v_{ij} \abs{\Psi}^2\right),
	\end{aligned}
	\end{equation}
	where $ \mathfrak{r}_i(x)=\min_{j\neq i}(\abs{x_i-x_j}) $, $ j_i\coloneqq j $, with $ \mathfrak{r}_i(x)=\abs{x_i-x_j} $, is unique a.e., and the nearest neighbor delta interaction can be written $\delta(\mathfrak{r}_i(x)-R)=\left(\sum_{j\neq i}\left[\delta(x_i-x_j-R)+\delta(x_i-x_j+R)\right]\right)\mathbbm{1}_{\mathfrak{r_i(x)}\geq R}$. The nearest-neighbor interaction is obtained from Lemma \ref{LemmaDysonSpin1/2Fermi} in the following manner: For each term in the sum $ \sum_{i} $, fix all particles $ x_j\neq x_i $, then divide the integration domain in $ x_i $ into Voronoi cells around all remaining particles, and integrate over all Voronoi cells individually.\\
	With use of Lemma 21 of Chapter \ref{ChapterTheGroundStateEnergyOfTheOneDimensionalDiluteBoseGas} with $ R>2\abs{a} $ in the last term, and by realizing that the first two terms can be obtained by using $ \psi $ as a trial state in the Lieb-Liniger-Heisenberg model, we obtain\begin{equation*}
	\begin{aligned}
	\int \sum_{i}\abs{\partial_i\Psi}^2+\sum_{i\neq j} \frac{1}{2}v_{ij}\abs{\Psi}^2\geq E_{LLH}^N \left(N,\tilde{L},\frac{2\epsilon}{R-a_e},\frac{2\epsilon}{R-a_o}\right)\braket{\psi|\psi}\\+ \frac{(1-\epsilon)}{R^2}\textnormal{const. }(1-\braket{\psi|\psi}),
	\end{aligned}
	\end{equation*}
	which is the desired result.
\end{proof}
We may also prove the equivalent of Lemma 24 of Chapter \ref{ChapterTheGroundStateEnergyOfTheOneDimensionalDiluteBoseGas}, by using that $ E_{LLH}(N,\tilde{L},c',c)\geq E_{LL}(N.\tilde{L},c') $ when $ c>c' $.
	\begin{lemma}\label{LemmaImprovedMassBoundSpin1/2Fermi}
	For $ n(\rho R)^2\leq  \frac{3}{16\pi^2}\frac{1}{8} $, $ \rho R\leq \frac{1}{2} $ and $ R>2\max(\abs{a_e},a_o,R_0) $ we have
	\begin{equation}\label{EqImprovedMassBoundSpin1/2Fermi}
	\begin{aligned}
	\braket{\psi|\psi} \geq 1-\textnormal{const. }\left(n(\rho R)^3+n^{1/3}(\rho R)^2\right).
	\end{aligned}
	\end{equation}
\end{lemma}
\begin{proof}
	We mimic the proof of Lemma 24 in Chapter \ref{ChapterTheGroundStateEnergyOfTheOneDimensionalDiluteBoseGas} (\cite{agerskov2022ground}): From the known upper bound, \ie Theorem \ref{TheoremUpperBoundSpin1/2Fermi}, and by Lemma \ref{LemmaNormBoundEpsilonSpin1/2Fermi} with $ \epsilon=1/2 $, it follows that 
    \begin{equation}
    \begin{aligned}
    N\frac{\pi^2}{3}\rho^2\left(1+2\rho\left( \ln(2)a_e+(1-\ln(2)a_o) \right)+\text{const. }(\rho R)^{6/5}\right)\\\geq E_{LLH}^N \left(N,\tilde{L},\frac{1}{R-a_e},\frac{1}{R-a_o}\right)\braket{\psi|\psi}+ \frac{1}{16R^2}(1-\braket{\psi|\psi}).
    \end{aligned}
    \end{equation}
	Subtracting $ E_{LLH}^N \left(N,\tilde{L},\frac{1}{R-a_e},\frac{1}{R-a_o}\right) $ on both sides, and using $$ E_{LLH}(N,\tilde{L},c',c)\geq E_{LL}(N.\tilde{L},c'), $$ and Lemma 18 of Chapter \ref{ChapterTheGroundStateEnergyOfTheOneDimensionalDiluteBoseGas} on the left-hand side, we find\begin{equation}
	\begin{aligned}
	&n\frac{\pi^2}{3}\Bigg(\rho^2\left(1+2\rho\left( \ln(2)a_e+(1-\ln(2)a_o) \right)+\text{const. }(\rho R)^{6/5}\right)\\&\qquad \qquad\qquad\qquad\qquad-\tilde{\rho}^2\left(1-4\tilde{\rho} (R-a_e)-\text{const. }n^{-2/3}\right)\Bigg)\\
	&\geq  \left(\frac{1}{16R^2}-E_{LLH}^N \left(N,\tilde{L},\frac{1}{R-a_e},\frac{1}{R-a_o}\right)\right)(1-\braket{\psi|\psi}),
	\end{aligned}
	\end{equation}
	with $ \tilde{\rho}=n/\tilde{\ell}=\rho/(1-(\rho-1/\ell)R)$.\\
	Using the upper bound $ E_{LLH}^N \left(N,\tilde{L},\frac{1}{R-a_e},\frac{1}{R-a_o}\right)\leq n\frac{\pi^2}{3}\tilde{\rho}^2 $ on the right-hand side, as well as $ 2\rho \geq\tilde{\rho}\geq \rho(1+\rho R)$, we find
	\begin{equation}
	\begin{aligned}
	\text{const. }n\rho^2R^2\left(\rho R+(\rho R)^{6/5}+n^{-2/3}\right)&\geq \left(\frac{1}{16}-R^2n\frac{4\pi^2}{3}\rho^2\right)\left(1-\braket{\psi|\psi}\right).
	\end{aligned}
	\end{equation}
	It follows that we have \begin{equation}
	\braket{\psi|\psi}\geq 1-\text{const. }\left(n(\rho R)^3+n^{1/3}(\rho R)^2\right).
	\end{equation}
\end{proof}
We continue the generalizations from Chapter \ref{ChapterTheGroundStateEnergyOfTheOneDimensionalDiluteBoseGas} and prove the following equivalent of Proposition 25 of Chapter \ref{ChapterTheGroundStateEnergyOfTheOneDimensionalDiluteBoseGas}:
\begin{proposition}
	\label{PropositionLowerBoundSpecNSpin1/2Fermi}
	For $ n(\rho R)^2\leq  \frac{3}{16\pi^2}\frac{1}{8} $, $ \rho R\leq \frac{1}{2} $ and $ R>2\max(\abs{a_e},a_o,R_0) $ we have \begin{equation}
	\begin{aligned}
	E^N(N,L)\geq E_{LLH}^N&\left(N,\tilde{L},\frac{2}{R-a_e},\frac{2}{R-a_o}\right)\\&\times\left(1-\textnormal{const.}\left(n(\rho R)^3+n^{1/3}(\rho R)^2\right)\right).
	\end{aligned}
	\end{equation}
\end{proposition}
\begin{proof}
	This follow by using Lemma \ref{LemmaNormBoundEpsilonSpin1/2Fermi} with $ \epsilon=1 $ and \ref{LemmaImprovedMassBoundSpin1/2Fermi}.
\end{proof}
We now see that this is where the strategy of Chapter \ref{ChapterTheGroundStateEnergyOfTheOneDimensionalDiluteBoseGas} is obstructed. The obstruction lies with the Lieb-Liniger-Heisenberg model not being Yang-Baxter solvable, meaning that the Bethe ansatz approach no longer gives exact solutions for the eigenvalue problem, as the Yang-Baxter equation is no longer satisfied. This being said, there is still hope that one might obtain a tight lower bound for the ground state energy of the Lieb-Linger-Heisenberg model in the dilute limit. We may even conjecture such a lower bound:
\begin{conjecture}\label{ConjectureLLHGroundStateEnergy}
	Let $ E_{LLH} $ denote the ground state energy of the Lieb-Liniger-Heisenberg model \eqref{EqHamiltonianLLH}. Then we have \begin{equation}
	E_{LLH}\geq N\frac{\pi^2}{3}\rho^2\left(1-4\rho\left(\frac{\ln(2)}{c'}+\frac{1-\ln(2)}{c}\right) \right).
	\end{equation}
\end{conjecture}
We that this conjecture is in line with both the Lieb-Liniger scenario, $ c=c' $, and the Yang-Gaudin scenario, $ c=\infty $. The validity of Conjecture \ref{ConjectureLLHGroundStateEnergy}, would give us the desired lower bound in Proposition \ref{PropositionLowerBoundSpecNSpin1/2Fermi}.
In the following subsection we will give some heuristics for such a lower bound.
\subsection{The Lieb-Liniger-Heisenberg ground state energy: Heuristics}
In this subsection, we give only heuristic arguments for a lower bound of the Lieb-Liniger-Heisenberg (LLH) ground state energy, $ E_{LLH} $. We do not claim the arguments given to be rigorous. For simplicity we restrict to having periodic boundary conditions. 

\textbf{Degenerate perturbation theory}\\
The first natural approach to estimating $ E_{LLH} $, is to do first order perturbation theory. To do this let is rewrite the LLH Hamiltonian \eqref{EqHamiltonianLLH} on the sector $ \{1,2,\ldots,N\} $ as follows\begin{equation}
H=-\sum_{i=1}^{N}\partial_i^2+2\sum_{1\leq i\leq N} \left(\frac{c'+3c}{4}+(c-c')S_i\cdot S_{i+1}\right)\delta(x_i-x_{i+1}),
\end{equation}
with $ c>c' $. We restrict to the regime $ c,c'\gg \rho $, $ \frac{c-c'}{c'}\ll 1 $, \ie the perturbative dilute regime.
We consider $ H_0=-\sum_{i=1}^{N}\partial_i^2+2\sum_{1\leq i\leq N} \frac{c'+3c}{4}\delta(x_i-x_{i+1}) $ the ``unperturbed" Hamiltonian, which is a Lieb-Liniger (LL) model with coupling $ \tilde{c}=\frac{c'+3c}{4} $. This model is of course, due to the presence of spin, degenerate (with finite multiplicity). The Perturbation is then 
$$ H'=2\sum_{1\leq i\leq N} \left((c-c')S_i\cdot S_{i+1}\right)\delta(x_i-x_{i+1}). $$
We restrict to analyzing the problem on the sector $ \{1,2,\ldots,N\} $, since all other sectors are related by symmetry.
In this case first order (finitely degenerate) perturbations theory, \cite{reed1978iv}, dictates that the perturbed eigenvalue is approximated by \begin{equation}
E_{LLH}\approx E_{LL}(\tilde{c})+\inf_{\chi\in\text{spin states}}\braket{\Psi^{\tilde{c}}_{LL}\chi\left\vert H'\right\vert \Psi^{\tilde{c}}_{LL}\chi},
\end{equation}
with $ \Psi_{LL}^{\tilde{c}} $ begin the LL ground state at coupling $ \tilde{c} $. For the Lieb-Liniger model, we have by the Feynman-Hellmann theorem and translation invariance \begin{equation}
\braket{\Psi^{\tilde{c}}_{LL}\left\vert\delta(x_i-x_{i+1})\right\vert \Psi^{\tilde{c}}_{LL}}=\frac{1}{N}\frac{\partial}{\partial \tilde{c}}E_{LL}(\tilde{c}).
\end{equation}
Therefore, it follows that we have \begin{equation}
E_{LLH}\approx E_{LL}(\tilde{c})+\frac{c-c'}{N}\frac{\partial}{\partial \tilde{c}}E_{LL}(\tilde{c})\inf_{\chi\in\text{spin states}}\braket{\chi\left\vert \sum_{i=1}^{N}S_i\cdot S_{i+1}\right\vert \chi}
\end{equation}
and by Lemma \ref{LemmaHeisenbergChainFiniteNEstimate}, we find \begin{equation}
\begin{aligned}
E_{LLH}&\approx E_{LL}(\tilde{c})+(c-c')\frac{\partial}{\partial \tilde{c}}E_{LL}(\tilde{c})\left(\frac{1}{4}-\ln(2)\right)\\
&\approx N\frac{\pi^2}{3}\rho^2\left(1-\frac{4\rho}{\tilde{c}}\left[1-\left(\frac{1}{4}-\ln(2)\right)\frac{c-c'}{\tilde{c}}\right]\right)\\
&\approx N\frac{\pi^2}{3}\rho^2\left(1-\frac{4\rho}{\tilde{c}^2}\left[\frac{1}{4}c'+\frac{3}{4}c-\left(\frac{1}{4}-\ln(2)\right)(c-c')\right]\right)\\
&\approx N\frac{\pi^2}{3}\rho^2\left(1-\frac{4\rho}{\tilde{c}^2}\left[\left(\frac{1}{2}-\ln(2)\right)c'+\left(\frac{1}{2}+\ln(2)\right)c\right]\right)
\end{aligned}
\end{equation}
Now by Taylor expanding in $ c' $ around $ c $ we find 
\begin{equation}
	E_{LLH}\approx N\frac{\pi^2}{3}\rho^2\left(1-4\rho\left(\frac{\ln(2)c'}{c^2}+\frac{1-\ln(2)}{c}\right)\right)
\end{equation}
which agrees with Conjecture \ref{ConjectureLLHGroundStateEnergy}.

\textbf{Lower bound by adding space}
Consider the two particle case. By translation invariance we expect the ground state to depend only on the distance between the two particle, $ x_1-x_2 $, and by spacial symmetry we may further restrict to $ \abs{x_1-x_2} $. Now we may define new coordinates on the sector $ \{1,2\} $ given by $ y_1=x_1 $, $ y_2=x_2+\mathfrak{r} $. Let $ \Psi(\abs{x_1-x_2}) $ denote the LLH ground state, then we may define the extended state $ \tilde{\Psi}(\abs{y_1-y_2})=\Psi(\abs{y_1-y_2+\mathfrak{r}}) $, when $ y_2>y_1+\mathfrak{r} $, and extended to the whole $ [0,L+\mathfrak{r}] $ by $ \operatorname{P}_t\tilde{\Psi}(r)=\operatorname{P}_t\Psi(0)\left(1-\frac{c}{2}r\right) $ and $ \operatorname{P}_s\tilde{\Psi}(r)=\operatorname{P}_s\Psi(0)\left(1-\frac{c'}{2}r\right) $ for $ r<\mathfrak{r} $. Notice that we defined $ \Psi $ such that we have the boundary conditions $ 2\operatorname{P}_t\Psi'(0_+)=(\partial_2-\partial_1)\operatorname{P}_t\Psi\vert_{x_2=x_{1+}}=c\operatorname{P}_t\Psi(0) $ and $ 2\operatorname{P}_s\Psi'(0_+)=(\partial_2-\partial_1)\operatorname{P}_s\Psi\vert_{x_2=x_{1+}}=c'\operatorname{P}_s\Psi(0) $. With this definition we see that \begin{equation}
2\int_{0}^{L+\mathfrak{r}}\abs{\operatorname{P}_{t}\tilde{\Psi}'(r)}^2\diff r+\tilde{c}\abs{\operatorname{P}_{t}\tilde{\Psi}(0)}^2=2\int_{0}^{L}\abs{\operatorname{P}_{t}\Psi'(r)}^2\diff r+c\abs{\operatorname{P}_{t}\Psi(0)}^2
\end{equation} 
If $  \frac{1}{2}c^2\mathfrak{r}\abs{\Psi(0)}^2+\tilde{c}\abs{\Psi(0)}^2\left(1-\frac{c}{2}\mathfrak{r}\right)^2=c\abs{\Psi(0)}^2 $, that is $ \tilde{c}=\frac{c}{1-\frac{c}{2}\mathfrak{r}} $. And similarly, we find \begin{equation}
2\int_{0}^{L+\mathfrak{r}}\abs{\operatorname{P}_{s}\tilde{\Psi}'(r)}^2\diff r+\tilde{c}'\abs{\operatorname{P}_{s}\tilde{\Psi}(0)}^2=2\int_{0}^{L}\abs{\operatorname{P}_{s}\Psi'(r)}^2\diff r+c'\abs{\operatorname{P}_{s}\Psi(0)}^2
\end{equation}
if $ \tilde{c}'=\frac{c'}{1-\frac{c'}{2}\mathfrak{r}} $.
 Now of course we have $ \norm{\tilde{\Psi}}>\norm{\Psi} $, so by using $ \tilde{\Psi} $ as a trial state for $ H^{\tilde{c}',\tilde{c}}_{LLH} $, \ie the LLH model with couplings $ \tilde{c}' $ and $\tilde{c}$, we find $ E_{LLH}(2,L,c',c)\geq E_{LLH}(2,L+\mathfrak{r},\tilde{c}',\tilde{c})=E_{LLH}\left(2,L+\mathfrak{r},\frac{2}{\frac{2}{c'}-\mathfrak{r}},\frac{2}{\frac{2}{c}-\mathfrak{r}}\right) $. In this case we may choose $ \mathfrak{r}=\frac{2}{c} $, in which case we find \begin{equation}
 \begin{aligned}
 E_{LLH}(2,L,c',c)&\geq E_{LLH}\left(2,L+\frac{2}{c},\left(\frac{1}{c'}-\frac{1}{c}\right)^{-1},\infty\right)\\&=E_{YG}\left(2,L+\frac{2}{c},\left(\frac{1}{c'}-\frac{1}{c}\right)^{-1}\right)\\&\geq2\frac{\pi^2}{3}\rho^2\left(1-2\rho \left[\ln(2)\left(\frac{1}{c'}-\frac{1}{c}\right)+\frac{1}{c}\right]\right)
 \end{aligned}
 \end{equation}
 where we used Proposition \ref{PropositionYGLowerBound}, and recall that the last inequality strictly speaking is conjecture (see Subsection \ref{SubsectionYGCaveat}).
\section{Lower bound for other symmetries}
Recall the Corollaries \ref{CorollaryUpperBoundSpin1/2Bose}, \ref{CorollaryUpperBoundSpin1/2FermiSpinDependent}, and \ref{CorollaryUpperBoundSpin1/2SymmetricSpinDependent}. For these we may, in some cases actually prove a lower bound. This is always the case for the bosonic case, and exactly the case when $ a_o\leq a_e $ in the spin dependent fermionic case, and when $ a_t\leq a_s $ in spin dependent spatially symmetric case. Hence we have 
\begin{theorem}
	\label{TheoremLowerBoundSpin1/2Bose}
	Consider a spin--$ 1/2 $ Bose gas with repulsive interaction  $v=v_{\textnormal{reg.}}+v_{\textnormal{h.c.}}$ as defined above. Then there exists a constant $C_\text{L}>0$ such that the ground state energy $E^N(N,L)$ satisfies
	\begin{equation}
	\label{eqlowerSpin1/2Bose}
	E^N(N,L)\geq N\frac{\pi^2}{3}\rho^2\left(1+2\rho a_e-C_\text{L}\left((\rho R)^{6/5}+N^{-2/3}\right)\right).
	\end{equation}
	$ R=\max(\abs{a_e}, R_0) $
\end{theorem}
	\begin{theorem}
	\label{TheoremLowerBoundSpinDependentSpin1/2Fermi}
	Consider a spin--$ 1/2 $ Fermi gas with repulsive interaction  $v=v_e\operatorname{P}_s+v_o\operatorname{P}_t$. Assume furthermore that $ a_o\leq a_e $. Then there exists a constant $C_\text{L}>0$ such that the ground state energy $E^N(N,L)$ satisfies
	\begin{equation}
	\label{eqlowerSpinDependtSpin1/2Fermi}
	E^N(N,L)\geq N\frac{\pi^2}{3}\rho^2\left(1+2\rho a_t-C_\text{L}\left((\rho R)^{6/5}+N^{-2/3}\right)\right).
	\end{equation}
	Here $ R=\max(\abs{a_e}, a_o, R_0) $
\end{theorem}
\begin{theorem}
	\label{TheoremLowerBoundSpinDependentSpin1/2SpatiallySymmetric}
	Consider a spin--$ 1/2 $ spatially symmetric gas with repulsive interaction  $v=v_s\operatorname{P}_s+v_t\operatorname{P}_t$. Assume furthermore that $ a_t\leq a_s $. Then there exists a constant $C_\text{L}>0$ such that the ground state energy $E^N(N,L)$ satisfies
	\begin{equation}
	\label{eqlower}
	E^N(N,L)\geq N\frac{\pi^2}{3}\rho^2\left(1+2\rho a_t-C_\text{L}\left((\rho R)^{6/5}+N^{-2/3}\right)\right).
	\end{equation}
	Here $ R=\max(\abs{a_s}, \abs{a_t}, R_0) $
\end{theorem}




\appendix
\chapter{Periodic boundary conditions}
\label{AppendixPeriodicBCSpin1/2}
If we consider the case with periodic boundary conditions in the box, one may actually show that the antiferromagnetic Heisenberg ground state, is the optimal spin state in the trial state. Starting from \eqref{EqEnergy3Spin1/2}, where no properties of $ \chi $ have been used, we find, using translation invariance of $ (\Psi_{e/o})_{12} $,
\begin{equation}\label{EqPeriodicUpperBound1}
\begin{aligned}
\int_{B_{12}^{\geq}}\abs{\partial_i\Psi_{\chi}}^2\leq&\left(\int_{A_{12}\cap\{1,2,\ldots,N\}}\abs{\partial_i(\Psi_e)_{12}}^2\right)\sum_{\{\sigma\}\in S_{12}}\braket{\chi_{\sigma} \left\lvert \operatorname{P}^{1,2}_s  \right\rvert \chi_\sigma}\\
&+\left(\int_{A_{12}\cap \{1,2,\ldots,N\}}\abs{\partial_i(\Psi_o)_{12}}^2\right)\sum_{\{\sigma\}\in S_{12}}\braket{\chi_{\sigma} \left\lvert \operatorname{P}^{1,2}_t  \right\rvert \chi_\sigma}\\
=&2(N-2)!\left(\int_{A_{12}\cap\{1,2,\ldots,N\}}\abs{\partial_i(\Psi_e)_{12}}^2\right)\sum_{k=1}^{N}\braket{\chi \left\lvert \operatorname{P}^{k,k+1}_s  \right\rvert \chi}\\
&+2(N-2)!\left(\int_{A_{12}\cap \{1,2,\ldots,N\}}\abs{\partial_i(\Psi_o)_{12}}^2\right)\sum_{k=1}^{N}\braket{\chi \left\lvert \operatorname{P}^{k,k+1}_t  \right\rvert \chi}.
\end{aligned}
\end{equation}
Using now that \begin{equation}
2(N-2)!N\int_{A_{12}\cap \{\ldots,1,2,\ldots\}}\abs{\partial_i(\Psi_{e/o})_{12}}^2\leq \int_{A_{12}}\abs{\partial_i(\Psi_{e/o})_{12}}^2,
\end{equation}
equation \eqref{EqEnergy4Spin1/2} follows.
But from \eqref{EqEnergy4Spin1/2} it is clear that the antiferromagnetic Heisenberg ground state is optimal. Thus we circumvented the use of translation invariance of $ \chi $.



