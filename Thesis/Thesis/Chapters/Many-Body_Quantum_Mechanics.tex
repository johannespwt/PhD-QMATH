
\chapter{Many-Body Quantum Mechanics} 
\label{ChapterMany-BodyQuantumMechanics}
	In this chapter, we give a brief introduction to many-body quantum mechanics. The chapter will serve to define relevant quantities, set up the mathematical framework, and state some preliminary results.

\section{Many-body Wave Functions}
	In quantum mechanics, a system is described by a \emph{state} or \emph{wave function} in an underlying Hilbert space. 
	\begin{definition}
		A quantum system at a fixed time is a pair \begin{equation*}
			(\Psi,\mathcal{H}),\text{ with } \Psi\in\mathcal{H} \text{ and } \norm{\Psi}=1,
		\end{equation*}
		where $ \mathcal{H} $ is a Hilbert space. Here $ \Psi $ is called the state or wave function of the system.
	\end{definition}
	In this thesis, we are mostly interested in a quantum system consisting of $ N $ particles in a region $ \Omega\subseteq \R^d $, possibly with spin degrees of freedom $ \{S_i\}_{i\in{1,\ldots,N}} $. We refer to $ d $ as the \emph{dimension} of the system. Such a system is described by having $$ \mathcal{H}\subset L^2\left(\prod_{i=1}^{N}\left(\Omega\times \{-S_i,...,S_i\}\right) \right)=\otimes_{i=1}^{N}L^2\left(\Omega;\C^{2S_i+1}\right), $$ where $ S_i $ is the \emph{spin} of the $ i $th particle. Since we are more specifically interested in identical particles we will further restrict the structure of the underlying Hilbert space below.

	\subsection{Identical particles: Bosons and fermions}
		In the case when the particles in question are identical, \ie indistinguishable, it turns out that one can restrict the underlying Hilbert space, to have certain symmetries. Considering $ N $ indistinguishable particles, we restrict to the physical configuration space $ C_{p,N}=C_N/S_N $, with $ C_N:=\{(x_1,\ldots,x_N)\in \Omega^N \vert x_i\neq x_j \text{ if }i\neq j\} $ on which the symmetric group act freely. For $ d\geq 2 $, we then require the wave function of the system to take values in a unitary irreducible representation of the fundamental group $\pi_1(C_{p,N})$, where we noted that the physical configuration space is path-connected in order for $ \pi_1(C_{p,N},x) $ to be independent of $ x\in C_{p,N} $.
		\begin{remark}
			For $ d\geq 3 $ we have $\pi_1(C_{p,N})=S_N$, for $ d=2 $ we have $\pi_1(C_{p,N})=B_N$ and for $d=1$ we have $\pi_1(C_{p,N})=\{1\}$. In the somewhat special case of $d=1$, $C_{p,N}=\{x_1<x_2<\ldots<x_N\}$. In this configuration space, one can never interchange particles without crossing the singular excluded incidence (hyper)planes. Thus the allowed particle statistics are determined by the possible permutation invariant dynamics on this space. In Chapter \ref{ChapterTheGroundStateEnergyOfTheOneDimensionalDiluteBoseGas} we will see examples of different particle statistics in one dimension. 
		\end{remark}
		\begin{remark}
			Adding spin to the above considerations amounts to having $C_N:=\{(z_1,\ldots,z_N)\in \left(\Omega\times\{-S,\ldots,S\}\right)^N \vert (z_i)_1\neq (z_j)_1 \text{ if }i\neq j\}$, and $C_{p,N}:=C_N/S_N$. In this case, $C_{p,N}$ is not path connected, however, for each configuration of spins $ \sigma=(\sigma_1,\ldots,\sigma_N)\in\{-S,\ldots,S\}^N $ the configuration spaces $ C_{p,N,\sigma}=\{((x_1,\sigma_1),\ldots,(x_N,\sigma_N))\in \left(\Omega\times\{-S,\ldots,S\}\right)^N \vert x_i\neq x_j \text{ if }i\neq j\}/S_N $ are path connected and their fundamental groups are isomorphic to the fundamental group in the spinless case independent of $ \sigma $.\\
			Alternatively, one can view the wave function as a $ (2S+1)^N $-dimensional vector bundle over the physical (spinless) configuration space.
		\end{remark}
		In the remaining part of this thesis, we will mainly be interested in the two irreducible representations that are the symmetric representation and the anti-symmetric representation, in which we refer to the particles as \emph{bosons} and \emph{fermions} respectively. It is an empirical fact that bosons and fermions are the only types of elementary particles that are encountered in nature. Hence for bosons, we restrict to wave functions in the symmetric (or bosonic) subspace $ L^2_{s}\left(\left(\Omega\times \{-S,\ldots,S\}\right)^N \right)\cong\vee_{i=1}^{N}L^2\left(\Omega; \C^{2S+1}\right)$ and for fermions, we restrict to wave-functions in the anti-symmetric (or fermionic) subspace $ L^2_{a}\left(\left(\Omega\times \{-S,\ldots,S\}\right)^N \right)\cong\wedge_{i=1}^{N}L^2\left(\Omega; \C^{2S+1}\right)$.\\
		To recap we list the following important definitions
		\begin{definition}
			A quantum system of $N$ spin--$ S $ bosons in $ \Omega\subseteq\R^d $ at fixed time is a pair
			\begin{equation*}
			(\Psi,\mathcal{H}),\text{ with } \Psi\in\mathcal{H}\text{ and }\norm{\Psi}=1,
			\end{equation*}
			where $ \mathcal{H} $ is a closed subspace of $ L^2_{s}\left(\left(\Omega\times \{-S,\ldots,S\}\right)^N \right)\cong\vee_{i=1}^{N}L^2\left(\Omega; \C^{2S+1}\right) $, and thus is a Hilbert space.
		\end{definition}
		\begin{definition}
			A quantum system of $N$ spin--$ S $ fermions in $ \Omega\subseteq\R^d $ at fixed time is a pair
			\begin{equation*}
			(\Psi,\mathcal{H}),\text{ with } \Psi\in\mathcal{H}\text{ and }\norm{\Psi}=1,
			\end{equation*}
			where $ \mathcal{H} $ is a closed subspace of $ L^2_{a}\left(\left(\Omega\times \{-S,\ldots,S\}\right)^N \right)\cong\wedge_{i=1}^{N}L^2\left(\Omega; \C^{2S+1}\right) $, and thus is a Hilbert space.
		\end{definition}
	
\section{Observables, Dynamics, and Energy}
In general, we call any self-adjoint operator on $ \mathcal{H} $ an \emph{observable}. Physically, observables represent quantities that, in principle, can be measured in an experiment. It is a postulate of quantum mechanics that given an observable $ \mathcal{O}=\int_{\sigma(\mathcal{O})}\lambda \diff P_\lambda $, where $\{P_\lambda\}_{\lambda\in\sigma(\mathcal{O})}$ is the projection valued measure associated to $ \mathcal{O} $ by the spectral theorem \cite{reed1981functional}, the probability of measurement of $ \mathcal{O} $ in the state $ \Psi\in\dom{\mathcal{O}} $ having any outcome $ \lambda $ such that $ \lambda\in M\subset \R $ is given by $ P\left((\mathcal{O},\Psi)\in M\right)=\int_{\lambda\in M} \braket{\Psi,P_\lambda\Psi} $. 
Furthermore, we defined the expected value of an observable.
\begin{definition}
	The \textbf{expectation value} of an observable $ \mathcal{O} $ in state $ \Psi\in\dom{\mathcal{O}} $ is 
	$$ \braket{\mathcal{O}}_\Psi:=\int_{\lambda\in\sigma(\mathcal{O})}\lambda\braket{\Psi,P_\lambda\Psi} $$
	where $\{P_\lambda\}_{\lambda\in\sigma(\mathcal{O})}$ is the projection valued measure associated to $ \mathcal{O} $ by the spectral theorem.
\end{definition}


In the previous section, we defined a quantum system at a fixed time. However, we are often interested in the dynamics of the system. In quantum mechanics, time evolution is modeled by the infinitesimal generator of time evolution, $ H $, also known as the \emph{Hamiltonian}. We will in this thesis take $ H $ to be a (time-independent) lower bounded self-adjoint operator on $ \mathcal{H} $. A state evolves in time according to the Schr\"odinger equation\begin{equation*}
\Psi(t)=\exp\left(-iH(t-t_0)\right)\Psi(t_0),
\end{equation*}
where we have set $ \hbar=1 $.
\begin{remark}
	By Stone's theorem \cite{reed1981functional}, the existence of a self-adjoint Hamiltonian, $ H $, is guaranteed for any time evolution described by $ \Psi(t)=U(t-t_0)\Psi(t_0) $, when $ U(t) $ is a strongly continuous one-parameter unitary group.
\end{remark}
Since the Hamiltonian, $ H $, is self-adjoint, it represents an observable which we call \emph{energy}. Since $ H $ is lower bounded, there is a natural notion of the lowest energy of $ H $.
\begin{definition}
	The \textbf{ground state energy} of $ H $ is defined by 
	$$
	E_0(H):=\inf(\sigma(H)),
	$$
	where $ \sigma(H) $ denotes the spectrum of $ H $.
\end{definition}
Furthermore, we define the notion of a \emph{ground state} of $ H $ as
\begin{definition}
	We say that a (normalized) state $ \Psi\in \dom{H}\subset \mathcal{H} $ is a \textbf{ground state} of $ H $ if $$ \braket{H}_\Psi=E_0(H). $$
\end{definition}
When studying ground states and ground state energies it is useful to have the following variational characterization.
\begin{remark}\label{RemarkVariationalPrinciple1}
	It follows from the spectral theorem (see \cite{reed1981functional}) that the ground state energy is given by \begin{equation}\label{EqVariationalGroundStateEnergyOperator}
		E_0(H)=\inf_{\Psi\in \dom{\mathcal{H}}}\frac{\braket{\Psi,H\Psi}}{\norm{\Psi}^2}.
	\end{equation}
\end{remark}
\begin{remark}
	It is straightforward to show that the quadratic form $ \dom{H}\ni\Psi\mapsto\braket{\Psi,H\Psi} $ is lower bounded and closable since $ H $ is lower bounded and self-adjoint.
\end{remark}
\begin{definition}
	Given a Hamiltonian, $ H $, we define the \textbf{associated energy quadratic form}, $ \mathcal{E}_H:\dom{\mathcal{E}_H}\to \R $, as the closure of the quadratic form $ \dom{H}\ni\Psi\mapsto\braket{\Psi,H\Psi} $. When $ H $ is given from the context, we will often write $ \mathcal{E} $ as short for $ \mathcal{E}_H $.
\end{definition}
\begin{remark}\label{RemarkVariationalPrinciple2}
	From the definition of $\mathcal{E}_H$ and from Remark \ref{RemarkVariationalPrinciple1} it follows straightforwardly that we have \begin{equation}\label{EqVariationalGroundStateEnergyForm}
		E_0(H)=\inf_{\Psi\in \dom{\mathcal{E}_H}}\frac{\mathcal{E}_H(\Psi)}{\abs{\Psi}^2}=\inf_{\substack{\Psi\in \dom{\mathcal{E}_H},\\\norm{\Psi}=1}}\mathcal{E}_H(\Psi), 
	\end{equation}
	as $ \dom{H} $ is form core for $\mathcal{E}_H$.
\end{remark}
We refer to both \eqref{EqVariationalGroundStateEnergyOperator} and \eqref{EqVariationalGroundStateEnergyForm} as \emph{the variational principle}.
We will often, in the remaining part of this thesis, take \eqref{EqVariationalGroundStateEnergyForm} as the very definition of the ground state energy. Furthermore, one can also define the dynamics of a quantum system by specifying an energy quadratic form in the following sense
\begin{remark}[\cite{reed1981functional} Theorem VIII.15]\label{RemarkOperatorFromQuadraticForm}
	Given a densely defined, lower bounded, closable, quadratic form $ \mathcal{E}:\dom{\mathcal{E}}\to \R $ there exists a \textbf{unique} lower bounded, self-adjoint operator $ H_\mathcal{E} $, such that $ \mathcal{E}(\Psi)=\braket{\Psi,H_\mathcal{E}\Psi} $ for all $ \Psi\in \dom{H_\mathcal{E}} $, and $ \dom{H_\mathcal{E}} $ is form core for $\overline{\mathcal{E}}$, \ie the form closure of $ \braket{\cdot,H_\mathcal{E}\cdot} $ is equal to the form closure of $\mathcal{E}$. 
\end{remark}
Thus we will frequently switch between the two equivalent formulations of the dynamics of a quantum system that are the operator, $ H $, formulation and the quadratic form, $ \mathcal{E} $, formulation
\subsection{Many-body Hamiltonians}
Until this point, we have not specified the class of Hamiltonians that we will be interested in. We have stated that we will care mainly about Hamiltonians defined on the bosonic or fermionic subspace, however, no specification has been made about the dynamics on these subspaces. We are interested in modeling $ N $ particles in some region $ \Omega\subseteq\R^d $ that interact locally with each other. For the remainder of this subsection, we will ignore spin, knowing that including spin degrees of freedom is completely analogous. In practice, and for suitably mild interactions, this means that the Hamiltonian \emph{formally} (meaning restricted to the fermionic or bosonic subspace of $ C^\infty_0(\Omega^N) $) takes the form \begin{equation}
H=\sum_{i=1}^{N}T_i+U(x_1,\ldots,x_N)
\end{equation}
where $ T_i $ is the \emph{kinetic energy operator} for particle $ i $ and the \emph{potential} $ U $ is a multiplication operator which models the local interaction among the particles. The kinetic energy operator is taken to be\footnote{This is usually justified by going through a canonical quantization procedure for the classical Hamiltonian function of the system we are interested in modeling} \begin{equation}
T_i=-\frac{1}{2m_i}\Delta_i\qquad (\hbar=1)
\end{equation} 
since we are interested in identical particles, we will from this point onward choose $ m_i=1/2 $. As for the potential, $ V $, we of course immediately restrict to permutation-invariant function, $ U $, for identical particles. However, in the following, we will further restrict to a combination of having a trapping potential and radial pair potentials, which model pairwise interactions that only depend on the distances between particles. Such potentials take the form \begin{equation}
U(x_1,\ldots,x_N)=\sum_{i<j} v(x_i-x_j) + \sum_{i=1}^{N}V(x_i)
\end{equation}
where we take $ v $ to be a radial function and $ V $ is called the \emph{trapping potential}. We will generally take $ v $ to be repulsive, meaning $ v\geq 0 $, with compact support. The trapping potential we will disregard \ie $ V=0 $. We will then in general take the true Hamiltonian to be a self-adjoint extension of the symmetric \emph{formal} Hamiltonian. Now some models of stronger interactions, \eg the hard core interaction, require a more delicate construction with respect to the initial definition of the formal Hamiltonian. However, the construction of the Hamiltonian can be done in a more unified manner when constructing the energy quadratic form. 
\begin{definition}
	For a system of $ N $ bosons/fermions in region $ \Omega\in\R^d $, we define for $ \sigma\in[0,\infty] $ \textbf{the energy quadratic forms}
\begin{equation}\label{key}
\mathcal{E}_{(v,\sigma)}(\Psi)=\int_{\Omega^N} \sum_{i=1}^{N}\abs{\nabla_i\Psi}^2+\sum_{i<j} v(x_i-x_j)\abs{\Psi}^2+\sigma\int_{\dd (\Omega^N)}\abs{\Psi}^2,
\end{equation}
with domain $ \dom{\mathcal{E}_{(v,\sigma)}}=\{\Psi\in (C^\infty_0(\Omega^N))_{\text{b/f}}\vert \mathcal{E}_{(v,\sigma)}(\Psi)<\infty\} $. With $ (C^\infty_0(\Omega^N))_{\text{b/f}} $ meaning the bosonic/fermionic subspace of $ C^\infty_0(\Omega^N) $. $ \sigma=\infty $ is taken to mean Dirichlet boundary conditions.
\end{definition}
Of course $ \mathcal{E}_{(v,\sigma)}\geq 0 $ for any $ \sigma\in[0,\infty] $ and $ v\geq 0 $. However, the closability of $ \mathcal{E}_{(v,\sigma)} $ is not evident. In fact for general $ v $, $ \mathcal{E}_{(v,\sigma)} $ will not be neither densely defined nor closable on $ L^2_{s/a}(\Omega^N) $. However, it will be densely defined on a closed subspace $ \mathcal{H}_{(v,\sigma)}:=\overline{\dom{\mathcal{E}_{(v,\sigma)}}}^{\norm{\cdot}_2} $ of $ L^2_{s/a}(\Omega^N) $, hence we take $ \mathcal{H}_{(v,\sigma)} $ to be the Hilbert space of the system when this is the case. Closability of $ \mathcal{E}_{(v,\sigma)} $ on $ \mathcal{H}_{(v,\sigma)} $ is not necessarily satisfied. Thus we make the following definition
\begin{definition}
	We say a potential $ v\geq 0 $ is \textbf{allowed} in dimension $ d $, if $ \mathcal{E}_{(v,\sigma)} $ is closable on $ \mathcal{H}_{(v,\sigma)}:=\overline{\dom{\mathcal{E}_{(v,\sigma)}}}^{\norm{\cdot}_2}\subset L^2_{s/a}(\Omega^N)$ for any $ \sigma\in[0,\infty] $.
\end{definition}
\begin{remark}
	There are plenty of allowed potentials, but the notion does depend on the dimension, $ d $. For example is $ v=\delta_0 $, \ie the delta function potential, allowed in dimension $ d=1 $, but not in dimension $ d\geq 2 $. This can be seen from the fact that for $ d=1 $ the incidence planes are co-dimension $ 1 $, and hence the trace theorem gives closability, but for $ d\geq 2 $ where the incidence planes are of co-dimension $ \geq 2 $ it is known that the trace of $ H^1 $ is not contained in $ L^2 $.
\end{remark}

\begin{remark} \label{RemarkdDimPotentialAllowed}
	For any radial $ v:\R\to [0,\infty] $ measurable (note this implies that $ x\mapsto v(x_i-x_j) $ is measurable since $ \R^{Nd} \ni x\mapsto x_i-x_j\in \R^d $ is Lebesgue-Lebesgue measurable),  $\mathcal{E}_{(v,\sigma)} $ is the quadratic form associated with a self-adjoint operator on some Hilbert space $ \mathcal{H}_{(v,\sigma)}\subset L^2_{s/a}(\Omega^N) $.\\
	It is well known that $\mathcal{E}_{(0,\sigma)}$ is closable on $ \mathcal{H}_{(0,\sigma)}\supseteq \mathcal{H}_{(v,\sigma)}    $. Hence $ \mathcal{E}_{(0,\sigma)}\lvert_{\dom{\mathcal{E}_{(v,\sigma)}}} $ is closable on $  \mathcal{H}_{(v,\sigma)} $. Thus closability of $ \mathcal{E}_{(v,\sigma)} $ amount to showing that $ \psi_n \xrightarrow{\norm{\cdot}_2}0 $ as $ n\to\infty $ and $ (\psi_n)_{n\in\mathbb{N}}\subset L^2\left(\Omega^N,\underbrace{\sum_{i<j}v(x_i-x_j)\diff\lambda^{N}}_{\coloneqq \diff\mu_v}\right) $ Cauchy, implies $ \psi_n\xrightarrow{\norm{\cdot}_{L^2(\Omega^N,\diff\mu_v)}}0  $. \\
	This is evident from the fact that $\psi_n\xrightarrow{\norm{\cdot}_{L^2(\Omega^N,\diff\mu_v)}}f$ for some $ f\in L^2(\Omega^N,\diff\mu_v) $ by completeness. Now $ \psi_n $ has a subsequence that converges $ \lambda^{N} $--almost everywhere to $ 0 $, and this subsequence further has a subsequence that converges $ \mu_v $--almost everywhere to $ f $. Hence $ f=0 $ $ \mu_v $--almost everywhere, as $ \mu_v\ll \lambda^N $. \\
	Thus there is a corresponding self-adjoint operator $ H_{(v,\sigma)} $ to $ \mathcal{E}_{(v,\sigma)} $ on $ \mathcal{H}_{(v,\sigma)} $, which we shall formally write as $ H_{(v,\sigma)}=-\sum_{i=1}^{N}\Delta_i+\sum_{1\leq i<j\leq N}v(x_i-x_j) $.
\end{remark}
The argument from the previous remark may be generalized slightly in the case of $ d=1 $, in order to show that any $ \sigma $--finite symmetric measure $ v(x_i-x_j)\diff\lambda(x_i-x_j):=\diff\mu_{v_{ij}} $ is allowed as potential. Notice that we slightly abuse notation and write $  v(x_i-x_j)\diff\lambda(x_i-x_j) $ even when $ v $ is a singular measure and thus has no density. However, we do think of $ v $ a being a one-dimensional measure in the sense that 
$$ v(x_i-x_j)\diff\lambda^N:=\diff\mu_{v_{ij}}\times \diff \lambda^{N-1}_{(x_i-x_j)=\text{ fixed}},  $$ 
where we defined $ \lambda^{N-1}_{(x_i-x_j)=\text{ fixed}} $ to be the measure such that $ \diff\lambda^{N}=\diff(x_i-x_j)\times \diff\lambda^{N-1}_{(x_i-x_j)=\text{ fixed}} $. The uniqueness of the product measure is guaranteed by $ \sigma $-finiteness of $ v $ ($ \mu_v $).
%\begin{lemma}
%	Let $ (f_n)_{n\in\mathbb{N}}\subset H^1(\Omega) $ be a sequence such that $ \norm{f_n}_{H^1}\to0 $ as $ n\to\infty $. Then $ f_n $ has a subsequence that converges pointwise to 0 on $ \lambda^{N-1} $--almost all lines.
%\end{lemma}
%\begin{proof}
%	We pass to a subsequence, which we also denote $ f_n$, such that $ f_n $ converges pointwise $ \lambda^N $--a.e. to $ 0 $. 
%	Since $ f_n\in H^1(\Omega) $, we know that $ f_n $ are absolutely continuous on $ \lambda^{N-1} $--almost all lines. Now consider the $ H^1(\R) $ norms of $ f_n $ along each line. Clearly, these constitute $ L^2 $ functions, with norms converging to $ 0 $. Hence there exists a subsequence that converges pointwise $ \lambda^{N-1} $--almost everywhere to $ 0 $. Thus there is a subsequence such that for almost all lines, the $ H^1 $ norm along the line converges to zero. But then $ f_n $ converges, by Morrey's inequality, uniformly to 0 on almost all lines.
%\end{proof}
%precise statement. 
We will need the following essential lemma, where we use the notation  $ \lambda_k^{N-1}\coloneqq \prod_{i\neq k}\lambda(x_i) $.
\begin{lemma}\label{LemmaH1PointwiseConvergence}
	Let $ (f_n)_{n\in\mathbb{N}}\subset H^1(\Omega^N) $ be a sequence such that $ \norm{f_n}_{H^1}\to0 $ as $ n\to\infty $. Then defining $ f^k_n(t,\overline{x}^k)\coloneqq f_n(x_1,\ldots,x_{k-1},t,x_{k+1},\ldots,x_N) $ for any $ k=1,\ldots,N $, we have that $ (f^k_n)_{n\in\mathbb{N}} $ has a subsequence that converges pointwise (in $ t $) to $ 0 $, $ \lambda_k^{N-1} $--a.e. for all $ k=1,\ldots,N $.
\end{lemma}
\begin{proof}
	We pass first to a subsequence, which we also denote $ f_n$, such that $ f_n $ converges pointwise $ \lambda^N $--a.e. to $ 0 $. 
	Since $ f_n\in H^1(\Omega^N) $, we know for any $ k=1,\ldots,N $ that $ f^k_n(t,\overline{x}^k) $ are in $ H^1(\Omega) $ (as functions of t) $ \lambda_k^{N-1} $--a.e.
	[\cite{evans1991measure} Theorem 2 p. 164].
	Now consider the $ H^1(\Omega) $ norms $ g_n^k(\overline{x}^k)\coloneqq \norm{f^k_n(\cdot,\overline{x}^k)}_{H^1(\Omega)} $. Clearly $ g_n^k $ constitute $ L^2 $ functions, with norms converging to $ 0 $ as $ n\to\infty $. Hence there exists a subsequence that converges pointwise $ \lambda_k^{N-1} $--almost everywhere to $ 0 $. So a subsequence $ f_{n_i}^k $ exists, such that for $ \lambda_k^{N-1} $--a.e. $ \overline{x}^k $, $ f_{n_i}^k(\cdot,\overline{x}^k) $ converges to $ 0 $ in $ H^1(\Omega) $. But then $  f_{n_i}^k(\cdot,\overline{x}^k) $ converges, by Morrey's inequality, pointwise to $ 0 $, for $ \lambda_k^{N-1} $--a.e. $ \overline{x}^k $.
\end{proof}
Using this lemma, we may prove the following Proposition
\begin{proposition}\label{Lemma1dPotentialAllowed}
	Let $ d=1 $, then for any $ \sigma $-finite measure, $ v $, we have that $ \mathcal{E}_{(v,\sigma)} $ is the quadratic form associated to a unique lower bounded self adjoint operator $ H_{(v,\sigma)} $ on some Hilbert space $ \mathcal{H}_{(v,\sigma)} $.
\end{proposition}
\begin{proof}
	As previously, we define $\mathcal{H}_{(v,\sigma)}:=\overline{\dom{\mathcal{E}_{(v,\sigma)}}}^{\norm{\cdot}_2} $ and \\
	$ \diff\mu_v=\sum_{1\leq i<j\leq N}v(x_i-x_j)\diff\lambda^N $. Clearly $ \mathcal{E}_{(v,\sigma)} $ is lower bounded and densely defined in $ \mathcal{H}_{(v,\sigma)} $. Closability amounts to showing that $ \psi_n\xrightarrow{\norm{\cdot}_{L^2(\Omega^N,\diff\lambda^{N})}}0 $ and $ (\psi_n)_{n\in\mathbb{N}}\subset \dom{\mathcal{E}_{(v,\sigma)}}\subset L^2\left(\Omega^N,\diff\mu_v\right) $ Cauchy w.r.t the norm $ \norm{\cdot}_{\mathcal{E}_{(v,\sigma)}}=\sqrt{\mathcal{E}_{(v,\sigma)}(\cdot)+\norm{\cdot}_2^2} $, implies $ \psi_n\xrightarrow{\norm{\cdot}_{L^2(\Omega^N,\diff\mu_v)}}0  $. Now since $ (\psi_n)_{n\in\mathbb{N}}$ is a Cauchy sequence in $ L^2\left(\Omega^N,\diff\mu_v\right) $, it has a subsequence that converges $ \mu_v $--almost everywhere to some function $ f\in L^2\left(\Omega^N,\diff\mu_v\right)  $. Furthermore, this subsequence has a further subsequence that converges to $ 0 $, $ \lambda^{N} $--almost everywhere. However, since $ (\psi_n)_{n\in\mathbb{N}}$ converges to $ 0 $ in $ H^1(\Omega^N,\diff \lambda^N) $, by passing to a subsequence and after a linear coordinate transformation, Lemma \ref{LemmaH1PointwiseConvergence} implies that for $ (x_i-x_j) $ fixed $ (\psi_n)_{n\in\mathbb{N}} $ converges $ \lambda^{N-1}_{(x_i-x_j)=\text{ fixed}} $--a.e. to $ 0 $. 
%	Hence $ \left(\psi_n\right)_{n\in\mathbb{N}} $ converges pointwise to $ 0 $ on $ \lambda^{N-1} $--almost all lines. Now notice that $ \diff \mu_v=\sum_{1\leq i<j\leq N} \diff \mu_{v_{ij}}\times \diff\lambda^{N-1}_{(x_i-x_j)=\text{ fixed}} $. Thus for $ \lambda^{N-1}_{(x_i-x_j)=\text{ fixed}} $--almost all lines in $ \Omega^N $ with $ x_i+x_j $ and $ x_k $ fixed for all $ k\neq i,j $, by passing to a subsequence $ \psi_n $ converges pointwise to $ 0 $, by Lemma \ref{LemmaH1PointwiseConvergence}. 
	But $ \psi_n $ also converges, by Tonelli's theorem, for $ \mu_{v_{ij}} $--almost every $ (x_i-x_j) $ to $ f$,  $ \lambda^{N-1}_{(x_i-x_j)=\text{ fixed}} $--almost everywhere , and hence $ f=0 $ $ \mu_{v_{ij}} $--almost everywhere, $ \lambda^{N-1}_{(x_i-x_j)=\text{ fixed}} $--almost everywhere. Thus we conclude, again by Tonelli's theorem, that $ f=0 $ $ \mu_v $--almost everywhere. The proposition now follows from Remark \ref{RemarkOperatorFromQuadraticForm}.
\end{proof}
\begin{remark}\label{RemarkInifinteDeltaAllowed}
	By the very definition of the domain $ \dom{\mathcal{E}_{v,\sigma}} $, it is not hard to see that in one dimension, a potential of the form $ v=\infty\delta_0 $, \ie an infinite point mass, is allowed. This potential creates a Dirichlet boundary condition on the incidence (hyper)planes in the domain.
\end{remark}
\begin{remark}\label{RemarkAllowedPotentialsSum}
	It is clear that if $ v_1 $ and $ v_2 $ are allowed potentials, then $ v_1+v_2 $ is an allowed potential. Defining $ \norm{\cdot}_{\mathcal{E}_{(v,\sigma)}}\coloneqq\sqrt{\mathcal{E}_{(v,\sigma)}(\cdot)+\norm{\cdot}_2^2} $, this follows from the fact a Cauchy sequence w.r.t. the norm $ \norm{\cdot}_{\mathcal{E}_{(v_1+v_2,\sigma)}} $ is similarly Cauchy w.r.t. $ \norm{\cdot}_{\mathcal{E}_{(v_{1},\sigma)}} $ and $ \norm{\cdot}_{\mathcal{E}_{(v_2,\sigma)}} $. In fact we have $$ \max\left(\norm{\cdot}_{\mathcal{E}_{(v_1,\sigma)}},\norm{\cdot}_{\mathcal{E}_{(v_2,\sigma)}}\right)\leq \norm{\cdot}_{\mathcal{E}_{(v_1+v_2,\sigma)}}\leq \sqrt{\norm{\cdot}_{\mathcal{E}_{(v_1,\sigma)}}^2+\norm{\cdot}_{\mathcal{E}_{(v_2,\sigma)}}^2}. $$
\end{remark}
\begin{remark}
	Combining Proposition \ref{Lemma1dPotentialAllowed}, Remarks \ref{RemarkdDimPotentialAllowed}, \ref{RemarkInifinteDeltaAllowed}, and \ref{RemarkAllowedPotentialsSum} we conclude that potentials of the form $ v=v_{\sigma\text{--finite}}+v_{\text{meas.}}+c\delta_0 $, with $ c\in[0,\infty] $, are allowed in one dimension, $ d=1 $. Here $ v_{\sigma\text{--finite}} $ is a $ \sigma $--finite measure and $ v_{\text{meas.}}:\R \to[0,\infty] $ is a measurable function. Of course $ c\delta_0 $ may be absorbed in the $ \sigma $-finite measure when $ c<\infty $, so only the $ c=\infty $ case requires Remark \ref{RemarkInifinteDeltaAllowed}.
	 We will in Chapters \ref{ChapterTheGroundStateEnergyOfTheOneDimensionalDiluteBoseGas} and \ref{ChapterTheGroundStateEnergyOfTheOneDimensionalDiluteSpin1/2FermiGas} obtain results about the ground state energies of such systems.
\end{remark}
\begin{remark}
	We emphasize that one can construct dynamics of a quantum system that are not given by a pair potential in the sense of the discussion above. It is, for example, possible to study point interactions in $ d\geq 2 $, however, they cannot be seen as arising from a potential (\eg a $ \delta $-function potential). Instead, one studies in this case the self-adjoint extensions of the Laplacian on functions supported away from the incidence planes of the particles. \cite{albeverio2012solvable}.
\end{remark}


\section{The Scattering Length}
When analyzing the dynamics of a quantum system, it is natural to define certain length scales, on which different processes take place. These length scales often play important roles in understanding the physics of the system, and thus often appear naturally in expressions for the energies of the system. One such length scale that will be of particular importance throughout this thesis is the \emph{scattering length}. The intuition behind the name is that scattering occurs on this length scale. This intuition will be important throughout the thesis, especially when constructing low-energy trial states in order to estimate ground state energies by applying the variational principle. The scattering length has multiple equivalent definitions in the literature, but we shall here define it conveniently from a variational principle.\\
Consider the two-body problem in $ \Omega=\R^d $ with a spherically symmetric positive potential of compact support $ v\geq 0 $. We allow for the potential, $ v $, to be a measure when it makes sense, \ie when it is \emph{allowed}. Let $ R_0>0 $ be such that $ \supp(v)\subset B_{R_0} $. Many assumptions on $ v $ can be weakened, but these conditions are sufficient for the scope of this thesis. The formal Hamiltonian can be written \begin{equation}
H_2=-\frac{1}{2m_1}\Delta_1-\frac{1}{2m_2}\Delta_2+v(x_1-x_2),
\end{equation}
For now, we keep the masses, but we will be, for the most part, interested in the case $ m_1=m_2=1/2 $. Defining the center of mass coordinate $ X=(m_1x_1+m_2x_2)/(m_1+m_2) $ and the relative coordinate $ y=x_1-x_2 $, we see that the kinetic energy may be rewritten \begin{equation}
\begin{aligned}
-\frac{1}{2m_1}\Delta_1-\frac{1}{2m_2}\Delta_2&=-\sum_{i=1}^{d}\frac{1}{2m_1}\left(\frac{\partial y_i}{\partial (x_1)_i}\partial_{y_i}+\frac{\partial X_i}{\partial (x_1)_i}\partial_{X_i}\right)^2\\&\qquad\qquad+\frac{1}{2m_2}\left(\frac{\partial y_i}{\partial (x_2)_i}\partial_{y_i}+\frac{\partial X_i}{\partial (x_2)_i}\partial_{X_i}\right)^2\\
&=-\sum_{i=1}^{d}\frac{1}{2m_1}\left(\partial_{y_i}+\frac{m_1}{m_1+m_2}\partial_{X_i}\right)^2\\&\qquad\qquad+\frac{1}{2m_2}\left(-\partial_{y_i}+\frac{m_2}{m_1+m_2}\partial_{X_i}\right)^2\\
&=-\frac{1}{2\mu}\Delta_y-\frac{1}{2(m_1+m_2)}\Delta_X,
\end{aligned}
\end{equation}
where $ \mu\coloneqq \frac{m_1m_2}{m_1+m_2} $. Thus we have separated the center of mass motion and the Hamiltonian may be decomposed \begin{equation}
H=H_{\text{CM}}+H_{\text{rel}},
\end{equation}
with $ H_{\text{CM}}=-\frac{1}{2(m_1+m_2)}\Delta_X $ and $ H_{\text{rel}}=-\frac{1}{2\mu}\Delta_y+v(y) $. In scattering theory, we will generally be interested in the relative motion of particles. A natural question is whether we can locally minimize the relative energy of the two particles when they are nearby. The answer is affirmative, which can be seen by the following:\\
Consider the ($ R $-local, relative) energy functional \begin{equation}
\mathcal{E}_R(\psi)=\int_{B_R} \frac{1}{2\mu}\abs{\nabla\psi}^2+v\abs{\psi}^2,
\end{equation}
with $ R>R_0 $. Then we have 
\begin{theorem}[Theorem A.1 in \cite{lieb2001ground}]\label{TheoremScatteringLength}
	Let $ R>R_0 $, then in the class of functions 
	$$ \{\phi\in H^1(B_R)\ \vert\ \phi(x)=1,\text{ for }x\in S_R\}, $$ with $ S_R $ the sphere of radius $ R $,
	there is a unique $ \phi_0 $ that minimizes $ \mathcal{E}_R $. This function is non-negative and spherically symmetric, $ \phi_0(x)=f_0(\abs{x}) $ for some $ f\geq 0 $, and it satisfies the equation\begin{equation}
	-\frac{1}{2\mu}\Delta\phi_0+v\phi_0=0,
	\end{equation}
	in the sense of distributions on $ B_R $.
	
	For $ R_0<r<R $ we have \begin{equation}
	f_0(r)=\begin{cases}
	(r-a)/(R-a)&\text{ for }d=1\\
	\ln(r/a)/\ln(R/a)& \text{ for }d=2\\
	(1-ar^{2-n})/(1-aR^{2-n})& \text{ for }d=3
	\end{cases}
	\end{equation}
	for some length, $ a $, which we call \textbf{the (s-wave) scattering length}.
	
	The minimum value of $ \mathcal{E}_R $ is 
	\begin{equation}
	f_0(r)=\begin{cases}
	1/\mu(R-a)&\text{ for }d=1\\
	\pi/[\mu\ln(R/a)]& \text{ for }d=2\\
	\pi^{n/2}a/[\mu\Gamma(n/2)(1-aR^{2-n})]& \text{ for }d=3.
	\end{cases}
	\end{equation}
\end{theorem}
We note that in $ d>3 $, the scattering length is not actually a length in the sense of units. This is purely an artifact of the conventions used in the definition.\\
The definition above defined only the s-wave scattering length. One can proceed to define different kinds of scattering lengths depending on which asymptotic behavior (boundary condition) we demand of the minimizer of $ \mathcal{E}_R $. We will be, for the most part, interested in different kinds of scattering lengths in dimension $ d=1 $, with the masses $ m_1=m_2=1/2 $. Thus we define the scattering lengths of interest:
\begin{definition}\label{DefinitionEvenScatteringLength}
	Let $ f_e\in H^1(\R) $ be the unique solutions of the equation\begin{equation}\label{EqEvenScatteringEquation}
	-f_e''(x)+\frac{1}{2}v(x)f_e=0,
	\end{equation}
	in the sense of distributions on $ B_R $, with boundary conditions $ f_e(R)=1 $ and $ f_e(-R)=1 $. Then we have \begin{equation}
	\int_{B_R} 2\abs{f_e'}^2+v \abs{f_e}^2=\frac{4}{R-a_e},
	\end{equation} 
	for some length, $ a_e $, called the \textbf{even wave scattering length}.
\end{definition}
\begin{definition}\label{DefinitionOddScatteringLenght}
	Let $ f_o\in H^1(\R) $ be the unique solutions of the equation\begin{equation}\label{EqOddScatteringEquation}
	-f_o''(x)+\frac{1}{2}v(x)f_o=0,
	\end{equation}
	in the sense of distributions on $ B_R $, with boundary conditions $ f_o(R)=1 $ and $ f_o(-R)=-1 $. Then we have \begin{equation}
	\int_{B_R} 2\abs{f_o'}^2+v \abs{f_o}^2=\frac{4}{R-a_o},
	\end{equation} 
	for some length, $ a_o $, called the \textbf{odd wave scattering length}.
\end{definition}
\begin{remark}
	We did not prove the uniqueness of the solutions above. In Definition \ref{DefinitionEvenScatteringLength}, it follows from Theorem \ref{TheoremScatteringLength} by noting that any solution of \eqref{EqEvenScatteringEquation} is a minimizer of $ \mathcal{E}_R $. In Definition \ref{DefinitionOddScatteringLenght} it follows from the fact that by Theorem \ref{TheoremScatteringLength} there is a unique solution that vanishes at the origin (simply consider the solution of \eqref{EqEvenScatteringEquation} with potential $ v'=v+\infty \delta_0 $ and multiply by $ \text{sign}(x) $). Thus the odd part of $ f_o $ is unique. The even part of $ f_o $ vanishes at $ x=R $, and since \eqref{EqOddScatteringEquation} is the Euler-Lagrange equation for $ \mathcal{E}_R $, we see that $ (f_o)_{\text{even}}=0 $, since this is the only local extremum of $ \mathcal{E}_R $ with zero boundary conditions.
\end{remark}
\begin{remark}
	The even wave scattering length, $ a_e $, need not be non-negative as is the case for the $ s $-wave scattering length in $ d\geq 2 $. However, we do have $ a_o\geq 0 $. This is easily seen by noticing that the minimizer of \begin{equation}
	\int_{B_R} 2\abs{f_o'}^2,
	\end{equation}
	with boundary condition $ f(R)=-f(-R)=1 $, is $ f(x)=(1/R) x $ on $ B_R $, which has energy $ \frac{4}{R} $. Thus adding a positive potential must increase the energy. \\
	Alternatively, we may see this by noting that the odd wave scattering length is equivalent to the s-wave scattering length in $ d=3 $ with potential $ v(\abs{\cdot}) $ since 
	\eqref{EqOddScatteringEquation} is exactly the radial scattering equation in $ d=3 $ when restricted to $ [0,R] $.
\end{remark}
\begin{remark}
	We also have $ a_o\geq a_e $ by the fact that $ \abs{f_o} $ is a trial state for $ \mathcal{E}_R $ with even boundary conditions, and its energy is $ 4/(R-a_o)\geq 4/(R-a_e) $.
\end{remark}
We give two examples of the scattering length in the following:
\begin{example}\label{ExampleScatteringLengthDelta}
	Consider $ v=c\delta $. For the even wave scattering length, we solve, in this case, the equation \begin{equation}
	f_e''(x)=0,
	\end{equation}
	on the interval $ [0,R] $, with the boundary condition $ f'(0_+)=\frac{c}{2}f(0) $ and $ f(R)=1 $.
	The solution is $ f_e(x)=\frac{x+2/c}{R+2/c} $, for $ x\in[0,R] $. We conclude that $ a_e=-2/c $.\\
	For the odd wave scattering length, we notice that the $ v $, does not change the scattering solution from the $ v=0 $ case, and we have $ f_o(x)=\frac{x}{R} $ and we conclude $ a_o=0 $.
\end{example}
\begin{example}\label{ExampleScatteringLengthHardCore}
	Consider $ v=\infty\mathbbm{1}_{[-R_0,R_0]} $, \ie \emph{the hard core}. In this case \begin{equation}
	f_{e/o}''(x)=0,\text{ for } x\in(R_0,R]
	\end{equation}
	and $ f_{e/o}(x)=0 $ for $ x\in [0,R_0] $ constitutes scattering equation on $ [0,R] $. Thus find that \begin{equation}
	f_{e/o}(x)=\begin{cases}
	0& x\in[0,R_0]\\
	\frac{x-R_0}{R-R_0}& x\in(R_0,R]
	\end{cases}
	\end{equation}
	solves the scattering equation. We conclude that $ a_e=a_o=R_0 $.
\end{example}

\section{The Ground State Energy of Dilute Gases}
To put the results of this thesis into context, we here summarize the current known result about the ground state energies of dilute Bose gases. To begin with, we define what is meant by "dilute":
\begin{definition}
	For the $ d $-dimensional ($ d=1,2,3 $) system of bosons, with the formal Hamiltonian \begin{equation}
	H=-\sum_{i=1}^{N}\Delta_i+\sum_{1\leq i<j\leq N}v(x_i-x_j),
	\end{equation}
	we say that the system is in \textbf{ the dilute limit} or that the Bose gas is \textbf{dilute} if $ \rho^{1/d} \abs{a}\ll 1 $. Notice that the absolute value on $ a $ is only important when $ d=1 $, since only then can the $ s $-wave scattering length be negative.
\end{definition}
\begin{definition}
	For the one dimensional system of fermions with the formal Hamiltonian \begin{equation}
	H=-\sum_{i=1}^{N}\partial^2_i+\sum_{1\leq i<j\leq N}v(x_i-x_j),
	\end{equation}
	we say that the system is in \textbf{ the dilute limit} or that the Fermi gas is \textbf{dilute} if $ \rho \max(\abs{\abs{a_e},a_o})\ll 1 $.
\end{definition}
\begin{remark}
	For fermions in dimension $ d=2,3 $, one can similarly define diluteness. The diluteness parameter will in this case depend on the spin configuration. For example will the vanishing total spin-$ z $ gas have the same definition as for bosons, since the p-wave scattering contribution to the energy is sub-leading. However, for spin-polarized gases the p-wave scattering length will appear in the diluteness parameter.
\end{remark}
\begin{remark}
	For $ d=1 $ the free Bose gas, \ie with $ v=0 $, has $ \abs{a_e}=\infty $. Hence the free Bose gas cannot be considered dilute at any density.
\end{remark}
In the following, we will list some of the known results about dilute gases in dimensions $ d=2,3 $. We will then in the remainder of this thesis shed light on the corresponding results in one dimension.
\subsubsection{The dilute Bose gas in three dimensions}
The three dimensional dilute Bose gas is probably the most well-studied example of a dilute quantum gas. Historically the most famous result on the three dimensional dilute Bose gas is due to Lee, Huang, and Yang \cite{lee1957eigenvalues}. Interestingly, the energy was found, to second order, to depend on the potential only through the scattering length.
The mathematical literature is quite rich, and we refer to the papers \cite{dyson1957ground,lieb1998ground,lieb1999ground,lieb2001ground,yau2009second,basti2021new,fournais2020energy,fournais2021energy} for more details. The latest results are very recent, and it was only in 2020 that the Lee-Huang-Yang formula was rigorously established to second order when Fournais and Solovej proved a second order lower bound. Without giving details about assumptions needed on the potential, the Lee-Huang-Yang formula takes the form with \begin{equation}
e^{3D}(\rho)=4\pi \rho^2 a\left(1+\frac{128}{15\sqrt{\pi}}\sqrt{\rho a^3}+o\left(\sqrt{\rho a^3}\right)\right),
\end{equation}
where $ e^{3D}(\rho)=\lim\limits_{\substack{N,L\to\infty\\ N/L^3=\rho}}\frac{E^{3D}(N,L)}{L^3} $, and $ E^{3D}(N,L) $ is the ground state of the Bose gas with $ N $ bosons in $ \Omega=[0,L]^3 $ with dynamics given by the Hamiltonian $ H=-\sum_{i=1}^{N}\Delta_i+\sum_{1\leq i<j\leq N} v(\abs{x_i-x_j}) $.

\subsubsection{The dilute Bose gas in two dimensions}
In the two dimensional scenario, the ground state energy of the dilute Bose gas again possesses an expansion that, to second order, only depends on the potential through the scattering length, analogous to the Lee-Huang-Yang result in three dimensions. The first derivation of this expansion to leading order was given for the hard sphere case in \cite{schick1971two} and higher order terms were first given in \cite{hines1978hard}. To leading order, a rigorous understanding was only reached in 2001 by Lieb and Yngvason in \cite{lieb2001ground}, and very recently the full proof was given at next to leading order by Fournais \emph{et al.} in \cite{fournais2022ground}. Without giving details on assumptions on the potential, the formula takes the form
\begin{equation}
e^{2D}(\rho)=4\pi \rho^2 Y\left(1-Y\abs{\ln(Y)}+\left(2\gamma+\frac{1}{2}+\ln(\pi)\right)Y+o(Y)\right),
\end{equation}
with $ \gamma $ being the Euler-Mascheroni constant and $ Y\coloneqq\abs{\ln(\rho a^2)}^{-1} $. Above $ e^{2D}(\rho)=\lim\limits_{\substack{N,L\to\infty\\ N/L^2=\rho}}\frac{E^{2D}(N,L)}{L^2} $, and $ E^{2D}(N,L) $ is the ground state of the Bose gas with $ N $ bosons in $ \Omega=[0,L]^2 $ with dynamics given by the Hamiltonian $ H=-\sum_{i=1}^{N}\Delta_i+\sum_{1\leq i<j\leq N} v(\abs{x_i-x_j}) $.
\subsubsection{The dilute spin--$ S $ Fermi gas in three dimensions}
The establishment of expansions for the ground state energy of the dilute Bose gas in terms of the scattering length led to the natural question of whether a similar expansion exists for the spin--$ 1/2 $ Fermi gas. The asymptotics were first derived in \cite{huang1957quantum,lee1957many}, however, it was not until 2004 that the result was rigorously proven in \cite{lieb2005ground}. Recently the error has been improved to be almost optimal \cite{lauritsen2023almost}, \ie the order of magnitude is almost equal to that of the conjectured next term in the expansion. Furthermore, for smooth potentials, the error was recently improved in  \cite{falconi2021dilute} and a proof with optimal error was given in \cite{giacomelli2022optimal}. Without giving details on the assumptions on the potential, the formula takes the form
\begin{equation}
e^{3D}_{F,S}(\rho)=\frac{3}{5}\left(6\pi\right)^{2/3}\sum_{i=-S}^{S}\rho_i^{5/3}+8\pi a\sum_{-S\leq i<j\leq S}\rho_i\rho_j+\rho^{5/3}o\left(\rho^{1/3}a\right).
\end{equation}
where $ \rho_i $ denotes the density of particles with spin-$ z $ $ i $ and the index $ i $ runs over integers or half integers. Furthermore, $ e^{3D}_{F,S}(\rho)=\lim\limits_{\substack{N,L\to\infty\\ N/L^3=\rho}}\frac{E^{3D}_{F,S}(N,L)}{L^3} $, and $ E^{3D}_{F,S}(N,L) $ is the ground state of the spin--$ S $ Fermi gas with $ N $ spin--$ S $ fermions in $ \Omega=[0,L]^3 $ with dynamics given by the Hamiltonian $ H=-\sum_{i=1}^{N}\Delta_i+\sum_{1\leq i<j\leq N} v(\abs{x_i-x_j}) $.\\
We may note, that $ a $, denotes the s-wave scattering length. The p-wave scattering length is relevant when two Fermions of the same species/spin interact, however, this is lower order since fermions of the same spin tend to localize away from each other due to the Pauli exclusion principle. Recently an upper bound was proven in the spin-polarized case in \cite{lauritsen2023ground}, in which the relevant scattering length is the so-called \emph{p-wave scattering length}, analogous to the \emph{odd-wave scattering length} in one dimension defined above in Definition \ref{DefinitionOddScatteringLenght}.

\subsubsection{The dilute spin--$ S $ Fermi gas in two dimensions}
In \cite{lieb2005ground}, the two dimensional result was also proved. The intuition behind the two dimensional result is, in this case, understood by considering the bosonic result, where to first order one replaces the scattering length $ a $ with $ \ln(\abs{\rho a^2})^{-1} $. Furthermore, the kinetic energy term is of course replaced by the free Fermi energy in two dimensions. Without giving details on the assumptions on the potential, the formula takes the form
\begin{equation}
e^{2D}_{F,S}(\rho)=2\pi \sum_{i=-S}^{S}\rho_i^{2}+\frac{8\pi}{\ln(\abs{\rho a^2})} \sum_{-S\leq i<j\leq S}\rho_i\rho_j+\rho^2 o\left(\ln(\abs{\rho a^2})^{-1}\right).
\end{equation}
where $ \rho_i $ denotes the density of particles with spin $ i $ and the index $ i $ runs over integers or half integers. Furthermore, $ e^{2D}_{F,S}(\rho)=\lim\limits_{\substack{N,L\to\infty\\ N/L^2=\rho}}\frac{E^{2D}_{F,S}(N,L)}{L^2} $, and $ E^{2D}_{F,S}(N,L) $ is the ground state of the spin--$ S $ Fermi gas with $ N $ spin--$ S $ fermions in $ \Omega=[0,L]^2 $ with dynamics given by the Hamiltonian $ H=-\sum_{i=1}^{N}\Delta_i+\sum_{1\leq i<j\leq N} v(\abs{x_i-x_j}) $.\\


\section{The Lieb-Liniger Model: A Solvable Model in One Dimension} \label{SectionLiebLinigerModel}
In 1960 a one dimensional model of impenetrable bosons was solved by Girardeau \cite{girardeau1960relationship}. This initialized the study of solvable models of particles in the continuum in one dimension. The next major breakthrough was in this context made in 1963 by Lieb and Liniger, who posed and solved a model of one dimensional point interacting bosons \cite{lieb1963exact}. Their solution generalized the solution of the impenetrable bosons by Girardeau. The technique that was used is known as \emph{Bethe ansatz} or \emph{Bethe's hypothesis} after it was invented by Bethe to solve the one dimensional Heisenberg chain \cite{bethe1931theorie}. We will in this section, for self-containment, go through the solution of the Lieb-Liniger model, as the solution and more generally the ground state energy is of importance later in the thesis when studying the ground state energy of the dilute one dimensional Bose gas. We follow the steps given in \cite{lieb1963exact} and present a few more general results.\\
The Lieb-Liniger model is a model of bosons with dynamics given by the Hamiltonian \begin{equation}
H_{LL}=-\sum_{i=1}^{N}\Delta_i+2c \sum_{1\leq i<j\leq N}\delta(x_i-x_j),
\end{equation}
where the left-hand side is defined in the sense of quadratic forms. More precisely on a \emph{sector}, $\{\sigma\}=\{\sigma_1,\sigma_2,\ldots,\sigma_N\}:=\{0< x_{\sigma_1}< x_{\sigma_2} < \ldots < x_{\sigma_N}< L\} $, where $ \sigma\in S_N $ is a permutation of $ \{1,\ldots,N\} $, the Hamiltonian acts as $ -\sum_{i=1}^{N}\Delta_i $, and by elliptic regularity, (\cite{grisvard2011elliptic}, Theorem 3.2.3.1), the domain is given by \begin{equation*}
	\begin{aligned}
	\dom{H_{LL}}=\left\{\psi\in H_s^1([0,L]^N)\ \middle\vert\ \psi\big\rvert_{\sigma}\in H^2(\{\sigma\})\text{ for any }\sigma\in S_{N},\right.\\ \left. \text{ and } (\partial_i-\partial_j)\psi\rvert_{x_i=x_j^+}=c\psi\rvert_{x_i=x_j}\right\}.
	\end{aligned}
\end{equation*}
The Bethe ansatz then prescribes that we, on a sector $ \{1,2,\ldots,N\} $, seek solutions to the eigenvalue equation, $ H_{LL}\psi=E\psi $, of the form\begin{equation}\label{EqBetheAnsatz}
\psi(x)=\sum_{P\in S_N} a(P)\exp\left(i\sum_{i=1}^{N}k_{P_i}x_i\right),
\end{equation}
where $ a(P)\in\C $ are suitably chosen coefficients. The boundary conditions $$
(\partial_{j+1}-\partial_j)\psi\rvert_{x_{j+1}=x_j}=c\psi\rvert_{x_i=x_j},
$$
are satisfied if for $ P=(p_1,p_2,\ldots,p_j=\alpha,p_{j+1}=\beta,\ldots,p_N) $ and $ Q=(p_1,p_2,\ldots,q_j=\beta,q_{j+1}=\alpha,\ldots,p_N) $, we have $ i(k_\beta-k_\alpha)(a(P)-a(Q))=c (a(P)+a(Q)) $ implying\begin{equation}\label{EqCoefficientsLL}
	a(Q)=-\frac{c-i(k_\beta-k_\alpha)}{c+i(k_\beta-k_\alpha)}a(P):=-\exp(i\theta_{\beta,\alpha})a(P)
\end{equation} where we have defined\begin{equation}
\theta_{i,j}=-2 \arctan\left(\frac{k_i-k_j}{c}\right).
\end{equation}
We note that we will require $ k_i\neq k_j $ for $ i\neq j $ in order for $ \psi $ to be non-vanishing. Defining $ a(I)=1 $, it is simple to see that by the relations \eqref{EqCoefficientsLL}, all $ a(P) $ are fixed. In fact that $ a(P) $ is uniquely determined by \eqref{EqCoefficientsLL} follows from the fact that in going from the identity $ I $ to some permutation $ P $, the same elements are eventually transposed, by any path of transpositions.\\
The values of the pseudo momenta $ k_i $ are now determined by the periodic boundary conditions, which on the sector $ \{1,2,$\ldots$,N\} $ take the form \begin{equation}\begin{aligned}
\psi(0,x_2,x_3,\ldots,x_N)&=\psi(x_2,x_3,\ldots,x_N,L),\\
\left(\partial_x \psi(x,x_2,x_3,\ldots,x_N) \right)\big\rvert_{x=0}&=\left(\partial_x \psi(x_2,x_3,\ldots,x_N,x) \right)\big\rvert_{x=L}.
\end{aligned}
\end{equation}
With the ansatz state above, these equations correspond to the $ N $ equation\begin{equation}\label{EqLLPseudoMomenta}
	(-1)^{N-1}\exp(-i k_j L)=\exp\left(i\sum_{i=1}^{N}\theta_{i,j}\right),
\end{equation}
with the definition $ \theta_{i,i}:=0 $. Although the ``pseudo" momenta $ k_i $ cannot be regarded as being true momenta, one can construct the total momentum of a state. We notice that $ P\coloneqq\sum_{i=1}^{N}k_i $ is constant across different sectors, and hence it may be regarded as the true total momentum. Furthermore, we see that if the set $ (k_i)_{i\in\{1,\ldots,N\}} $ solves the equations \eqref{EqLLPseudoMomenta} then set $ (k_i'=k_i+2\pi n_0 /L)_{i\in\{1,\ldots,N\}} $ solves it as well. This corresponds to changing the total momentum by $ P'=P+ 2\pi n_0 \rho $, with $ \rho\coloneqq N/L $. Thus we may restrict to finding all solutions with $ -\pi \rho <P\leq \pi \rho $, then all other solutions are related by a constant change in ``pseudo" momenta. Ordering the ``pseudo" momenta such that $ k_1<k_2<\ldots<k_N $, another consequence of \eqref{EqLLPseudoMomenta} is that $ \sum_{i=1}^{N}k_i=2\pi n/L $ for some integer $ -N/2<n\leq N/2 $, since $ \theta_{i,j}=-\theta_{j,i} $.\\
Now we define \begin{equation}\label{EqDeltaDef}
\delta_i=(k_{i+1}-k_i)L=\sum_{s=1}^{N}(\theta_{s,i}-\theta_{s,i+1})+2\pi n_i,
\end{equation} where $ n_i $ are integers and the second equality follows from \eqref{EqLLPseudoMomenta}. Since $ \theta_{s,i} $ is strictly increasing in $ i $, we see that $ n_i\geq 1 $. Notice that $ k_j-k_i=\frac{1}{L}\sum_{s=i}^{j-1}\delta_i $ for $ j>i $, hence \eqref{EqDeltaDef} is a set of equations determining $ (\delta_i)_{i\in\{1,\ldots,N-1\}} $. Given a set of $ (n_i)_{i\in\{1,\ldots,N-1\}} $ and a solution of \eqref{EqDeltaDef}, $ (\delta_i)_{i\in\{1,\ldots,N-1\}} $, we merely choose $ k_1 $ to satisfy \eqref{EqLLPseudoMomenta} by having \begin{equation}\label{Eqk1}
	k_1=-\frac{1}{L}\sum_{i=1}^{N}\theta_{i,1}-\frac{2\pi m}{L}+\frac{\epsilon(N)}{L},
	\end{equation}
	where $ m $ is some integer determined by $ -\pi\rho <P\leq \pi\rho $ and\\ $ \epsilon(N)=\begin{cases}
	0&\text{ if } N\text{ is odd},\\
	\pi&\text{ if } N\text{ is even}
	\end{cases} $. The right-hand side of \eqref{Eqk1} depends only on the $ \delta $s. The proof of existence of solutions for \eqref{EqDeltaDef} that varies continuously with $ c $, was given in \cite{yang1969thermodynamics}.\\
	\subsection{The ground state}
	It is clear that within the set of ansatz states, the variational ground state must have $ n_i=1 $ for all $ i=1,\ldots, N-1 $. In this case, we have by symmetry and uniqueness of the ground state that $ k_i=-k_{N-i} $ and since $ P=\sum_{i=1}^{N}k_i=Nk_1+\frac{1}{L}\sum_{j=1}^{N-1}(N-j)\delta_j=0 $ we find $ k_1=-\frac{1}{NL}\sum_{j=1}^{N-1}(N-j)\delta_j=-k_N$.\\
	
	We may also ask whether the true ground state is attained among these ansatz states. This turns out to be the case, which may be seen by the following results
	\begin{lemma}\label{LemmaLLTrueGroundState}
		Let $ \Psi_c $ denote the (true) ground state and $ E_c $ denote the (true) ground state energy of $ H_{LL} $ with coupling $ c>0 $. Then $ \lim\limits_{c\to\infty}E_c=E_F=E_\infty $, where $ E_F $ is the free Fermi ground state energy and $ \Psi_c\to \Psi_{\infty} $ in $ L^2([0,L]^N) $ as $ c \to\infty $.
	\end{lemma}
	\begin{proof}
		Going to the quadratic form representation of $ H_{LL} $ is clear by a trial state argument that $ E_c\leq E_F $ for any $ c<\infty $. Now assume that $ E_c<\mathcal{E}<E_F $ for all $ c<\infty $ where $ \mathcal{E} $ is independent of $ c $. Then the ground state at coupling $ \Psi_c $ of $ H_{LL} $, is uniformly (in $ c $) bounded in $ H^1 $. Hence for any sequence $ (c_n)_{n\in\mathbb{N}}\subset\R_+ $, we find that $ \Psi_{c_n} $ is, by possibly passing to a subsequence, weakly convergent in $ H^1 $ to some $ \Psi\in H^1 $. By the Rellich--Kondrachov theorem $ \Psi_{c_n} $ converges in $ L^2 $ norm to the same limit. Now assuming $ c_n\to\infty $ as $ n\to\infty $ we have $ \Psi_{c_n}(x_i=x_j)\to 0 $ in $ L^2(\Omega^{N-1}) $ as $ n\to\infty $ for any $ i,j $ in order for the potential energy to stay finite. But then the limit $ \Psi $ also satisfies $ \Psi(x_i=x_j)=0 $ (in $ L^2(\Omega^{N-1}) $) for any $ i,j $. This follows from the fact that $ \delta(x_i-x_j)f(\overline{x^j})\in H^{-1}(\Omega^{N}) $ for any $ f\in L^2(\Omega^{N-1}) $ and from weak $ H^1 $ convergence of $ \Psi_{c_n} $. Notice that $ \Psi $ is a trial state for the impenetrable boson model ($ c=\infty $). However, clearly we have $ E_\Psi\leq \mathcal{E}<E_F $ by weak lower semi-continuity of the $ H^1 $--norm, which contradicts $ E_F $ being the ground state energy of the impenetrable boson model. Hence we conclude $ E_\Psi=E_F=E_\infty $, but then by uniqueness of the ground state in the impenetrable bosons model, $ \Psi=\Psi_\infty $. Since $ c_n\to \infty $ as $ n\to\infty $ was arbitrary, we conclude that any subsequence of $ \Psi_{c_n} $ has a further subsequence $ \Psi_{c_{n_i}} $ such that $ \Psi_{c_{n_i}}\to\Psi_\infty $ as $ i\to\infty $, and the proof is complete. 
	\end{proof}
%	\begin{proof}[Proof in the thermodynamic limit by Bethe ansatz see \eqref{EqLLEq1}--\eqref{EqLLEq3} below]
%		It follows from \eqref{EqLLEq3} that $ \lambda\to\infty $ as $ c\to\infty $. Then from \eqref{EqLLEq1} we see that $ g=\frac{1}{2\pi} $ so again by \eqref{EqLLEq3} $ \lambda=\frac{1}{\pi}\gamma $. Thus by $ \eqref{EqLLEq2} $ we have $ e(\gamma)=\frac{\pi^2}{3} $, which agrees with the free Fermi ground state energy 
%	\end{proof}
	\begin{proposition}\label{PropositionCInftyconvergenceLLEigenstates}
		Let $ \Psi_c $ denote the (true) ground state of $ H_{LL} $ with coupling $ c $. If $ (c_n>0)_{n\in\mathbb{N}} $ is a sequence of couplings then there exist a subsequence $ \Psi_{c_{n_i}} $, such that $ \Psi_{c_{n_i}} $ converges in $ C^{\infty}(\overline{\{1,2,\ldots,N\}}) $ as $ i\to\infty $.
	\end{proposition}
	\begin{proof}
		Since $ \Psi_{c_n} $ are ground states we know $ -\Delta\Psi_{c_n}=\lambda_n \Psi_{c_n} $, with $ \lambda_n\leq E_F $ for all $ n\in\mathbb{N} $. Since $ \overline{\{1,2,\ldots,N\}} $ is convex, we have by elliptic regularity (\cite{grisvard2011elliptic}, Theorem 3.2.3.1) that $ \norm{\Psi_{c_n}}_{H^{2m}(\{1,2,\ldots,N\})}\leq C_m \lambda_n^m \norm{\Psi_{c_n}}_{L^2(\{1,2,\ldots,N\})}\leq C_m E_F^m $. By the Rellich--Kondrachov theorem \cite{adams1975sobolev}, there exist for each $ m\in\mathbb{N} $ a subsequence $ \Psi^m_{c_{n_i}} $ such that $ \Psi^m_{c_{n_i}} $ converges in $ H^{2m-1}(\{1,2,\ldots,N\}) $. By a diagonal argument we find a subsequence, $ \Psi^i_{c_{n_i}} $, which converges in $ H^k(\{1,2,\ldots,N\}) $ for all $ k\in\mathbb{N} $. Hence, by the Sobolev embedding theorem (\cite{adams1975sobolev}, Theorem 5.4), $ \Psi^i_{c_{n_i}} $ converges to $ \Psi $ in $ C^\infty(\overline{\{1,2,\ldots,N\}}) $.
	\end{proof}
	\begin{proposition}\label{PropositionLLGroundState}
		Let $ \Psi_V(c) $ be the variational ground state (in the Bethe ansatz class, as given above) of $ H_{LL} $, then $ \Psi_V(c) $ is the true ground state.
	\end{proposition} 
\begin{proof}[Proof]
	Consider first the limit $ c\to \infty $. Here it is easily verified that $\Psi_V(c)\to \abs{\Psi_F}$ in $ L^2 $, where $ \Psi_F $ is the free Fermi ground state, \ie a Slater determinant state and that $ E_V(c)\to E_F $, where $ E_F $ is the free Fermi energy. This is the non-degenerate ground state energy at $ c=\infty $, \ie the impenetrable bosons. Now by the uniqueness of the bosonic ground state and continuity of the (true and variational) ground state energy in $ 1/c $, as well as the fact that $ \Psi_V(c) $ is an eigenstate, we conclude that the variational ground state must remain the true ground state, as $ 1/c $ varies. If this was not the case, there would be an orthogonal true ground state, implying a degeneracy either at $ c=\infty $ or at some $ c>0 $. \\
	Continuity of the true ground state energy, in $ c $, can be seen by perturbation theory \cite{reed1978iv}, or by a simple trial state argument, using $ \Psi_c $ (the ground state of $ H_{LL}(c) $) as a trial state for $ \mathcal{E}_{H_{LL}(c+\epsilon)} $.\\
%	Existence of a limit of the true ground state is proven by techniques similar to those used in the proof of Lemma \ref{LemmaLLTrueGroundState}.
\end{proof}
We note that while Proposition \ref{PropositionLLGroundState} holds for the ground state, its proof cannot be generalized to excited states, since there is no unique $ n $th excited state in the Bose gas. In this case, we refer to the more involved proof of completeness of the Bethe ansatz states by Dorlas \cite{cmp/1104252974}. Proposition \ref{PropositionLLGroundState} of course follows from this result as well.\\
Interestingly, it is possible to study the thermodynamic limit ($ N,L\to\infty $ with $ N/L=\rho $) of the system by the use of the Bethe ansatz solution. To do this, we define $ K(\gamma):=\lim_{\substack{N,L\to\infty\\
		N/L=\rho}}k_N $ where $ \gamma=c/\rho $. Of course, the energy will grow with the particle number, so we are, in this case, interested in the energy per volume (length)
\begin{equation}\label{}
e(\gamma):=\lim_{\substack{N,L\to\infty\\
N/L=\rho}} \frac{1}{L}E_N.
\end{equation}
Since we have $ k_{i+1}-k_i<2\pi/L $, we conclude\begin{equation}
\theta_{s,i}-\theta_{s,i+1}=-\frac{2c(k_{i+1}-k_i)}{c^2+(k_s-k_i)^2}+\mathcal{O}(1/(cL)^2).
\end{equation}
So by \eqref{EqDeltaDef} we see for the ground state ($ n_i=1 $) that \begin{equation}
k_{i+1}-k_i=\frac{2\pi}{L}-\frac{1}{L}\sum_{s=1}^{N}\frac{2c(k_{i+1}-k_i)}{c^2+(k_s-k_i)^2}+\rho O(1/(cL)^2).
\end{equation}
Now let $ f $ be such that $ k_{i+1}-k_i=1/(Lf(k_i)) $. Then by Poisson's summation formula, we have \begin{equation}
2\pi f(k)-1=2c\int_{-K}^{K}\frac{f(p)}{c^2+(p-k)^2}\diff p+o(1/(cL)).
\end{equation}
The very definition of $ f $ implies $ \int_{-K}^{K}f(p)\diff p=\rho $, with ground state energy\begin{equation}
E=\sum_{i}k_i^2=\int_{-K}^{K}k^2f(k)\diff k,
\end{equation} 
and it follows from the definition of $ f $ and $ k_i<k_{i+1} $ that $ f\geq 0 $.\\
It is now a matter of a simple coordinate transformation\begin{equation}
g(x)\coloneqq f(Kx),\quad  c\coloneqq K\lambda
\end{equation}
to find the equations for the ground state energy in the thermodynamic limit:
\begin{align}
2\pi g(x)-1&=2\lambda\int_{-1}^{1}\frac{g(y)}{\lambda^2+(y-x)^2}\diff y\label{EqLLEq1},\\
e(\gamma)&=\rho^3\frac{\gamma^3}{\lambda^3}\int_{-1}^{1}x^2 g(x)\diff x\label{EqLLEq2},\\
1&=\frac{\gamma}{\lambda}\int_{-1}^{1}g(x)\diff x.\label{EqLLEq3}
\end{align}
The first equation is an inhomogeneous Fredholm equation of the second kind which is solved by the Liouville-Neumann series. Notice that our equation for  $ e(\gamma) $ differ from those of Lieb and Liniger by a factor $ \rho^3 $, since we have absorbed this factor as part of $ e(\gamma) $. This difference is also present in chapter \ref{ChapterTheGroundStateEnergyOfTheOneDimensionalDiluteBoseGas}, where the convention of Lieb and Liniger is followed. In Lemma 17 of Chapter \ref{ChapterTheGroundStateEnergyOfTheOneDimensionalDiluteBoseGas} we prove the following lemma on the thermodynamic ground state energy of the Lieb-Liniger model. This lemma will be important in the proof of a lower bound on the ground state energy for the dilute Bose gas.
\begin{lemma}[Lemma 17 in Chapter \ref{ChapterTheGroundStateEnergyOfTheOneDimensionalDiluteBoseGas} (\cite{agerskov2022ground})]
	Let $ e(\gamma) $ be a solution of \eqref{EqLLEq1}--\eqref{EqLLEq3}. Then for $ \gamma>0 $ we have \begin{equation}
	e(\gamma)\geq \frac{\pi^2}{3}\rho^3\left(\frac{\gamma}{\gamma+2}\right)^2.
	\end{equation}
\end{lemma}

%\subsection{Lower bound in the large $ c $ limit}
%From the equations \eqref{EqLLEq1}-\eqref{EqLLEq3}, one can obtain an exact lower bound of the ground state energy in the thermodynamic limit, this is done in Chapter \ref{ChapterTheGroundStateEnergyOfTheOneDimensionalDiluteBoseGas}. However, since this lower bound is shown by the use of the exact solution of the Lieb-Liniger model, it is hard to generalize this lower bound more generic models such as perturbations of the Lieb-Liniger model. In this subsection, we seek to prove a weaker form of this lower bound by a more soft argument. For this purpose, we will use Proposition \ref{PropositionCInftyconvergenceLLEigenstates} to give an asymptotic (in $ c $) bound in a finite box. The strategy is as follows: Consider the ground state, $ \Psi_c $ of the Lieb-Liniger model in a box of size $ L $. We define $ \tilde{\Psi}_c:[0,L+(N-1)R]^N\to\C $ to satisfy $ \tilde{\Psi}(y_1,\ldots,y_N)=\Psi_c(y_1,y_2-R,y_3-2R,\ldots,y_N-(N-1)R) $ when $ y_{k+1}-kR>y_k $ for all $ k=1,\ldots,N $. We denote the set  $ \{y_{k+1}-kR>y_k\text{ for all }k=1,\ldots,N\}:=\Gamma $. Then $ \tilde{\Psi}_c\rvert_{\Gamma}=\Psi_c $. Now we define $ \tilde{\Psi}_c $ on all of $ [0,L+(N-1)R]^N $ by extending it to be an eigenfunction of the Laplacian $ -\Delta $ with the same eigenvalue as on $ \Gamma $. Indeed it is an eigenfunction on $ \Gamma $ with eigenvalue $ E_c $. Then we have \begin{equation}
%\int \abs{\nabla\tilde{\Psi}_c}^2=E_c \norm{\tilde{\Psi}_c}^2-\sum_{i<j}\int_{y_i=y_j} \overline{\tilde{\Psi}_c}\nabla_n \tilde{\Psi}_c
%\end{equation}
%where $ \nabla_n $ denotes the inward normal derivative at the boundary $ \{y_i=y_j\} $.\\
%Now to give a lower bound, we notice that the extension can be approximated by Taylor expanding from the original boundary, $ \partial \Gamma $, into the new region. Heuristically, denote a point on the boundary $ x_0\in \partial \Gamma\setminus \partial \Lambda_{L+(N-1)R} $, we have \begin{equation}
%\tilde{\Psi}_c(y_1=y_2,y_{i+1}>y_i+R \text{ for all } i\geq 2)=\tilde{\Psi}_c(x_0)-R\nabla_n \tilde{\Psi}_c(x_0)+\frac{1}{2}R^2\nabla_n^2\tilde{\Psi}_c(x_0)+\ldots
%\end{equation}
\section{The Yang-Gaudin Model}
\label{SectionYG}
Similarly to the Lieb-Liniger model, the Yang-Gaudin model is exactly solvable, by use of a generalized Bethe ansatz. This was originally done in \cite{yang1967some}, and we shall briefly review the methods in this section.
The model of interest describes $ N $ spin--$ 1/2 $ fermions and is given using the same formal Hamiltonian as for the Lieb-Liniger model\begin{equation}\label{EqYGHamiltonian}
H_{YG}=-\sum_{i=1}^{N}\partial_i^2+2c\sum_{1\leq i<j\leq N}\delta(x_i-x_j),
\end{equation}
however, the domain is not, for the moment being, taken to have any given spatial symmetry. 
\subsection{Labeling the symmetries}
To analyze the problem, Yang considers the possible spatial symmetries that may appear in the problem. Having combined spin-space anti-symmetry requires that any irreducible representation of $ S_N$ determining the spatial symmetry must have a corresponding conjugate spin symmetry. As an example consider the two particle case where the wave function is either symmetric and the spin state is the singlet, \emph{or} the wave function is anti-symmetric and the spin state is in the triplet. If you have more particles, the picture is more complicated, although similar. Notice that one cannot have $ 3 $ spin--$ 1/2 $ particles that are mutually in the singlet state with each other. It turns out, that one way to label the symmetry of a spin state is by Young tableaux, \ie a diagram of boxes with numbers obeying the rule that numbers increase along all rows and columns. A tableau labels a subspace of spin states. To construct the subspace consider all states that are symmetrized in particle labels in the same rows. Next anti-symmetrize, in these states, all particle labels in the same columns. For example: \begin{equation}
\begin{ytableau}
1 & 2 \\
3 &  \none 
\end{ytableau}=\Span{\ket{\uparrow \uparrow \downarrow }-\ket{\downarrow \uparrow \uparrow},\ket{\downarrow\downarrow\uparrow}-\ket{\uparrow\downarrow\downarrow}},
\end{equation}

\begin{equation}
\begin{ytableau}
1 & 3 \\
2 &  \none 
\end{ytableau}=\Span{\ket{\uparrow \downarrow \uparrow }-\ket{\downarrow \uparrow \uparrow},\ket{\downarrow \uparrow \downarrow }-\ket{\uparrow \downarrow \downarrow}},
\end{equation}


\begin{equation}
\begin{ytableau}
1 & 2& 3 
\end{ytableau}=\Span{\ket{\uparrow \uparrow \uparrow },\ket{\downarrow\downarrow\downarrow},\ket{\uparrow\downarrow\uparrow}+\ket{\downarrow\uparrow\uparrow}+\ket{\uparrow\uparrow\downarrow},\ket{\uparrow\downarrow\downarrow}+\ket{\downarrow\downarrow\uparrow}+\ket{\downarrow\uparrow\downarrow}}.
\end{equation}
By the before mentioned fact that one cannot anti-symmetrize three $ 1/2 $--spins, Young tableaux of spin--$1/2$ states have at most two rows. An interesting fact with this labeling of spin states is that the structure of a given tableau is related to the total spin of the state. To see this, notice that all columns of lengths two carry vanishing total spin because they form a singlet state. On the other hand, all columns of length one are symmetrized with each other. Hence it is well known that they carry maximal total spin. In the subspace labeled by a tableau with $ M $ columns of length $ 2 $ and $ N-2M $ columns of length $ 1 $, all states are of the form $ \ket{S_0}\otimes \ket{S_{(N-2M)/2}} $, where $ \ket{S_0} $ is some spin state of total spin $ 0 $ and $ \ket{S_{(N-2M)/2}} $ is some spin state of total spin $ (N-2M)/2 $. Remembering that irreducible representations of $ SU(2) $ are labeled by the total spin, we conclude that a Young diagram, which is just a Young tableau with blank entries, labels the irreducible $ SU(2) $ representations.\\
Remember that we may label the irreducible representation of $ S_N $ determining the spatial symmetry also by Young diagrams, \cite{william1991representation}. Recall that for irreducible representations of $ S_N $ we have the relation\begin{equation}
\begin{aligned}
 \{\lambda'\}\ &=\qquad\{\lambda\}& \otimes& \ \ \text{sgn}\\
\tiny\begin{ytableau}
\phantom{1} & \phantom{1} \\
\phantom{1} & \phantom{1} \\
\phantom{1} & \phantom{1} \\
\phantom{1} &  \none \\
\phantom{1} &  \none\\
\end{ytableau}&= \tiny{\begin{ytableau}
\phantom{1}& \phantom{1}& \phantom{1} &\phantom{1}& \phantom{1} \\
\phantom{1} &\phantom{1} &\phantom{1} &\none & \none 
\end{ytableau}}  &\otimes& \ \ \tiny\begin{ytableau}
\phantom{1}  \\
\phantom{1}  \\
\phantom{1}  \\
\phantom{1} \\
\phantom{1} \\
\phantom{1} \\
\phantom{1} \\
\phantom{1}  
\end{ytableau} \quad.
\end{aligned}
\end{equation} Thus we see that a wave function, which is anti-symmetric under (spin-space) permutations, and which spatially transforms in the irreducible representation
\begin{equation*}
\begin{ytableau}
\phantom{1} & \phantom{1} \\
\phantom{1} & \phantom{1} \\
\phantom{1} & \phantom{1} \\
\phantom{1} &  \none \\
\phantom{1} &  \none\\
\phantom{1} &  \none
\end{ytableau}_{\text{space}}\quad ,
\end{equation*}
must be defined in the spin subspace 
\begin{equation*}
\begin{ytableau}
 \phantom{1}& \phantom{1}& \phantom{1} &\phantom{1}& \phantom{1} &\phantom{1} \\
\phantom{1} &\phantom{1} &\phantom{1} &\none & \none & \none 
\end{ytableau}_{\text{spin}}.
\end{equation*}
We notice that this restricts the spatial symmetries that spin--$ 1/2 $ fermions can possess, since the spin diagrams have at most two rows. In the following, we will denote the diagram consisting of a row with $ N-M $ boxes and a row with $ M $ boxes by $ [N-M,M] $, and diagrams consisting of a column of $ N-M $ boxes and a column of $ M $ boxes by $ [2^{M},1^{N-2M}] $. 
\subsection{Recap of the findings of Yang: Solution by Bethe-Yang ansatz}
The solution found by Yang in \cite{yang1967some}, relies on a generalization of the Bethe ansatz, which we saw in the previous section solved the Lieb-Liniger model. The generalized Bethe ansatz is also known as the Yang-Bethe hypothesis or Yang-Bethe ansatz. We recap here, without proof, the findings of Yang. For references on these results, we point to \cite{gaudin1967systeme,yang1967some,sutherland1968further,fung1981validity,gaudin2014bethe}. \\
The model is solved by applying a standard Bethe ansatz state: One the sector $ \{\sigma\} $ define \begin{equation}
\psi=\sum_{P\in S_N} \xi_{P,\sigma} \exp\left(k_{P_1} x_{\sigma_1}+\ldots+k_{P_N}x_{\sigma_N}\right),
\end{equation}
with energy $ E=\sum_{i=1}^{N}k_i^2 $.
Similarly to in the Lieb-Liniger case, in order to satisfy the right boundary condition, we have\begin{equation}\label{EqYGCoeffecientsRelation}
	\xi_{P,\sigma}=Y^{1,2}_{ij}\xi_{Q,\sigma},
\end{equation} when $ Q=(P_1,\ldots,\underbrace{2}_{i},\ldots,\underbrace{1}_{j},\ldots,P_N) $ and\\ $ P=(P_1,\ldots,\underbrace{1}_{i},\ldots,\underbrace{2}_{j},\ldots,P_N) $, where we defined \begin{equation}
Y_{ij}^{12}=\frac{(k_i-k_j)(12)-ic}{(k_i-k_j)+ic},
\end{equation}
with $ (12) $ acting by interchanging $ \sigma_1 $ and $ \sigma_2 $. We see that we recover the Lieb-Liniger result if $ \psi $ is symmetric and a Slater determinant if $ \psi $ is anti-symmetric.\\
A crucial observation by Yang is that we have the following identities, of which the second is famously known as the Yang-Baxter equation. \begin{equation}
\begin{aligned}
Y_{ij}^{ab}Y_{ji}^{ab}&=1\\
Y_{jk}^{ab}Y_{ik}^{bc}Y_{ij}^{ab}&=Y_{ij}^{bc}Y_{ik}^{ab}Y_{jk}^{bc}.
\end{aligned}
\end{equation}
These make the equations \eqref{EqYGCoeffecientsRelation} mutually consistent.\\
The condition of periodic boundary conditions may now be written \begin{equation}\label{EqYGPeriodicBoundaryCondition}
\lambda_j\xi_{I,\sigma}=X_{(j+1)j}X_{(j+2)j}\ldots X_{Nj}X_{1j},\ldots X_{(j-1)j}\xi_{I,\sigma},
\end{equation}
with $ \lambda_j=\exp(ik_j L) $ and $ X_{ij}=P_{ij}Y_{ij}^{ij} $.\\
Now, restricting to $ \psi $ in some irreducible representation $ R=[2^{M},1^{N-2M}] $, one easily sees that, using $ X_{ij}=(1-P_{ij}x_{ij})/(1+x_{ij}) $,  we may equivalently consider a spin state, $ \Phi $, of total spin $ N-2M $, satisfying the equation \begin{equation}
\mu_j\Phi=X'_{(j+1)j}X'_{(j+2)j}\ldots X'_{Nj}X'_{1j},\ldots X'_{(j-1)j} \Phi,
\end{equation}
with $ X_{ij}'=(1+P^{\tilde{R}}_{ij}x_{ij})/(1+x_{ij}) $, where $ \tilde{R} $ denotes the conjugate representation, so $ P^{\tilde{R}}_{ij} $ is acting on the spins \ie $ P^{\tilde{R}}_{ij}=-P_{ij} $.\\
Now considering instead a spin chain of total $ z $--spin $ (N-2M)/2 $, we know that this chain can have components with total spin $ N/2,\ (N-1)/2,\ldots (N-2M)/2 $. Notice that $ P_{ij}=1/2+2S_i\cdot S_j $, for spin--$ 1/2 $ particles, which commute with the total spin operator. Hence we may find eigenvalues, $ \mu_j $, in each total spin sector separately. However, since these eigenvalues correspond to eigenvalues of \eqref{EqYGHamiltonian}, the theorem of Lieb and Mattis \cite{lieb1962theory}\footnote{\textbf{A detail often left out in the literature:} This theorem is only proved in the paper \cite{lieb1962theory} for Dirichlet or Neumann boundary conditions. One may prove that the absolute ground state is in the total spin $ S=0 $ subspace even with periodic boundary conditions when $ N/2 $ is an odd integer. The proof requires $ N/2 $ to be an odd integer in order to have a positive periodic ground state on an ordered sector $$ \{x_1<x_2<\ldots<x_{N/2}\text{ and } x_{N/2+1}<x_{N/2+2}<\ldots<x_N\}. $$
Furthermore, in this case, the ground state is unique.\\
The exact statement of the theorem is then: Denote by $ E(S) $ the lowest energy of any state with total spin $ S $. Then the following theorem holds:
\begin{theorem}[Lieb and Mattis, \cite{lieb1962theory}, for periodic boundary conditions, $ d=1 $]\label{TheoremLiebMattis}
	If $ N/2 $ is an odd integer and $ S>2n $ for some integer $ n $ then $ E(S)> E(2n) $, unless the potential, $ V $, is pathological, in which $ E(S)\geq E(2n) $. Furthermore, when $ V $ is not pathological, the ground state with energy $ E(0) $ is unique.
\end{theorem}
\begin{proof}
	The proof follows the proof of Theorem I in \cite{lieb1962theory} with $ M=2n $ and $ N/2 $ odd, in order for $ \abs{\varphi_0(x_1,\ldots,x_{N/2-2n}\vert x_{N/2-2n+1},\ldots,x_N)} $ to be a \emph{continuous} anti-symmetric periodic wave function on the above mentioned sector.
\end{proof}
For more details on the notion of ``pathological" and the proof, we refer to the original paper by Lieb and Mattis.} tells us that the eigenvalue $ \mu_j $ yielding the smallest eigenvalue of \eqref{EqYGHamiltonian} must come from the total spin sector $ N-2M $, \ie minimal total spin in the case when $ N/2 $ is an odd integer.\\
The Bethe-Yang hypothesis states that \begin{equation}\label{EqBetheYangHypothesis}
\Phi(y_1,\ldots,y_M)=\sum_{P\in S_{N}}A_P \prod_{i=1}^{M}F(\Lambda_{P_i},y_i),
\end{equation}
where $ y_i $ denotes the positions of the spin downs, and with \begin{equation}
F(\Lambda,y)=\prod_{j=1}^{y-1}\frac{ik_j-i\Lambda-c/2}{ik_{j+1}-i\Lambda+c/2},
\end{equation}
and \begin{equation}\label{EqYGSpinRapidityRelation}
-\prod_{j=1}^{N}\frac{ik_j-i\Lambda_\alpha-c/2}{ik_{j}-i\Lambda_\alpha+c/2}=\prod_{\beta=1}^{M}\frac{-i\Lambda_\beta+i\Lambda_\alpha-c}{-i\Lambda_\beta+i\Lambda_\alpha+c}.
\end{equation}
One may verify that $ \Phi $ has total spin $ N-2M $.
Yang then find \begin{equation}
\mu_j(k,c,[N-M,M])=\prod_{\beta=1}^{M}\frac{ik_j-i\Lambda_\beta+-c/2}{ik_j-i\Lambda_\beta+c/2}.
\end{equation}
Thus the energy is determined by the equation\begin{equation}\label{EqYGPeriodicSpectrumCondition}
\exp(ik_jL)=\prod_{\beta=1}^{M}\frac{ik_j-i\Lambda_\beta+-c/2}{ik_j-i\Lambda_\beta+c/2}.
\end{equation}
Taking the logarithm of \eqref{EqYGSpinRapidityRelation} and \eqref{EqYGPeriodicSpectrumCondition} adding certain integers to get a well defined $ c\to\infty $ limit, as we did in Section \ref{SectionLiebLinigerModel}, one finds \begin{equation}\label{EqYGintegers}
\begin{aligned}
-\sum_{k\in\{k_j\}_j}\theta(2\Lambda-2k)&=2\pi J_\Lambda-\sum_{\Lambda'\in(\Lambda_\alpha)_\alpha}\theta(\Lambda-\Lambda'),\\
kL&=2\pi I_k-\sum_{\Lambda'\in\{\Lambda_\alpha\}_\alpha}\theta(2k-2\Lambda'),
\end{aligned}
\end{equation}
with the usual $ \theta(x)\coloneqq-2\arctan(x/c) $, and where for $ N $ even and $ M $ odd we have for ground state (among the ansatz states)\begin{equation}
\begin{aligned}
J_\Lambda&\in\{-(M-1)/2,\ldots,(M-1)/2\},\\
I_k&\in \{1-N/2,\ldots,N/2\}.
\end{aligned}
\end{equation}
Going to the thermodynamic limit, \ie $ N,M,L\to\infty $ proportionally, one then find the equations for the energy \begin{align}
2\pi\sigma(\Lambda)&=-\int_{-B}^{B}\frac{2c\sigma(\Lambda')\diff \Lambda'}{c^2+(\Lambda-\Lambda')^2}+\int_{-Q}^{Q}\frac{4c f(k)\diff k}{c^2+4(k-\Lambda)^2}\label{EqYG1}\\
2\pi f(k) &= 1+\int_{-B}^{B}\frac{4c\sigma(\Lambda')\diff \Lambda'}{c^2+4(k-\Lambda')^2}\label{EqYG2}\\
\rho=N/L&=\int_{-Q}^{Q} f(k) \diff k,\quad M/L=\int_{-B}^{B}\sigma(\Lambda)\diff\Lambda,\label{EqYG3}\\
e=E/L&=\int_{-Q}^{Q}k^2 f(k)\diff k \label{EqYG4},
\end{align}
with $  f,\sigma\geq 0 $. We see that taking $ B=\infty $, and integrating over \eqref{EqYG1}, one finds by interchanging the order of integration \begin{equation}
2\pi M/L=-\int_{-\infty}^{\infty} 2\pi \sigma(\Lambda')\diff \Lambda'+ 2\pi \int_{-Q}^{Q} f(k)\diff k,
\end{equation}
where we used $ \int_{-\infty}^{\infty}\frac{\diff x}{1+x^2}=\pi $. So using $ \eqref{EqYG3} $ we find $ 2M=N $, and thus the total spin is $ S_{\text{tot.}}=0 $. By a theorem of Lieb and Mattis \cite{lieb1962theory}, this is then the total ground state.
\subsection{Lower bound of the Yang-Gaudin model}
Now the following lemma will prove useful in obtaining a lower bound for the thermodynamic ``ground state energy" (in the sense that it comes from a solution of integral equations \eqref{EqYG1}--\eqref{EqYG4}) of the Yang-Gaudin model.
\begin{lemma}\label{LemmaYGrho}
	For any $ m\in \mathbb{N}_+ $, the equations \eqref{EqYG1}--\eqref{EqYG4} imply that \begin{equation}\label{EqYG5}
	\begin{aligned}
	2\pi  f(k)=1+(-1)^{m+1}4\int_{-\infty}^{\infty}\frac{ (2m-1)c\sigma(\Lambda'')}{((2m-1)^2 c^2+4(k-\Lambda'')^2)}\diff\Lambda''\\
	+2\sum_{n=0}^{m-1}(-1)^{n+1}\int_{-Q}^{Q}\frac{2c(2n) f(k')}{((2n)^2c^2+4(k-k')^2)}\diff k',
	\end{aligned}
	\end{equation},
\end{lemma}

\begin{proof}
	We give an induction proof: For the induction start, we notice that the $ m=1 $ statement is simply \eqref{EqYG2}. For the induction step, assume that \eqref{EqYG5} hold for $ m=m_0 $, we may plug the right-hand side of \eqref{EqYG1} into \eqref{EqYG5}. By Tonelli's theorem, we may interchange the order of integration and we find\begin{equation}
	\begin{aligned}
	&2\pi  f(k)-1=\\
	&\frac{(-1)^{m_0+2}}{2\pi}\int_{-\infty}^{\infty}\int_{-\infty}^{\infty}\frac{8c^2(2m_0-1)\sigma(\Lambda'')}{(c^2+(\Lambda'-\Lambda'')^2)((2m_0-1)^2c^2+4(k-\Lambda')^2)}\diff\Lambda'\diff\Lambda''\\
	&+\frac{(-1)^{m_0+1}}{2\pi}\int_{-Q}^{Q}\int_{-\infty}^{\infty}\frac{4^2c^2(2m_0-1) f(k')}{(c^2+4(k'-\Lambda')^2)((2m_0-1)^2c^2+4(k-\Lambda')^2)}\diff\Lambda'\diff k'\\
	&+2\sum_{n=0}^{m_0-1}(-1)^{n+1}\int_{-Q}^{Q}\frac{2c(2n) f(k')}{((2n)^2c^2+4(k-k')^2)}\diff k',
	\end{aligned}
	\end{equation}
	
	Using the formulas
	\begin{align}
	\int_{-\infty}^{\infty}\frac{m}{(1+(x'-x'')^2)(m^2+4(y-x'))}\diff x'&=\frac{(m+2)\pi}{(2+m)^2+4(y-x'')^2},\\
	\int_{-\infty}^{\infty}\frac{m}{(1+4(y'-x')^2)(m^2+4(y-x'))}\diff x'&=\frac{(m+1)\pi}{2((m+1)^2+4(y-y')^2)},
	\end{align}
	for any $ x'',y,y'\in\R $ and $ m\in\mathbb{N}_+ $,
	We find 
	
	\begin{equation}
	\begin{aligned}
	2\pi  f(k)=1+(-1)^{m_0+2}4\int_{-\infty}^{\infty}\frac{ (2(m_0+1)-1)c\sigma(\Lambda'')}{((2(m_0+1)-1)^2 c^2+4(k-\Lambda'')^2)}\diff\Lambda''\\
	+2\sum_{n=0}^{m_0}(-1)^{n+1}\int_{-Q}^{Q}\frac{2c(2n) f(k')}{((2n)^2c^2+4(k-k')^2)}\diff k',
	\end{aligned}
	\end{equation}
	which proves the required result.
\end{proof}
We will aim at proving a lower bound. To do this, notice that in Lemma \ref{LemmaYGrho}, the second term in \eqref{EqYG5} vanish in the limit $ m\to\infty $ by the estimate \begin{equation}\label{EqYGLemma1bound}
	\begin{aligned}
	\int_{-\infty}^{\infty}\frac{ (2m-1)c\sigma(\Lambda'')}{((2m-1)^2 c^2+4(k-\Lambda'')^2)}\diff\Lambda''\leq \frac{1}{(2m-1)c}\int_{-\infty}^{\infty}\sigma(\Lambda'')\diff \Lambda''\\=\frac{M/L}{(2m-1)c}.
	\end{aligned}
	\end{equation}
For the third term in \eqref{EqYG5}, we need the estimate of the following lemma:
\begin{lemma} \label{LemmaYGbound:f}
	For any $ m_0\in\mathbb{N}_+ $ we have 
	\begin{equation}\label{EqYGLemma2bound}
		\sum_{n=0}^{m_0}(-1)^{n+1}\int_{-Q}^{Q}\frac{2c(2n) f(k')}{((2n)^2c^2+4(k-k')^2)}\diff k'\leq \sum_{n=0}^{m_0}(-1)^{n+1}\int_{-Q}^{Q}\frac{2 f(k')}{2nc}\diff k'.
	\end{equation}
\end{lemma}
\begin{proof}
Essentially we want to throw away the $ (k-k')^2 $ in the denominator on the left-hand side of \eqref{EqYGLemma2bound} to get an upper bound. For all terms with positive coefficients, this can be done by the inequality
\begin{equation}\label{EqYGLemma2bound2}
\int_{-Q}^{Q}\frac{2 f(k')}{2nc}\diff k'\geq \int_{-Q}^{Q}\frac{2c(2n) f(k')}{((2n)^2c^2+4(k-k')^2)}\diff k'.
\end{equation}
However, for the terms with a negative sign, this estimate cannot be used. Thus we use the following strategy instead:
 In order to deal with the signs we estimate the differences
\begin{equation}
	\begin{aligned}
	\Delta_n=&\left(\int_{-Q}^{Q}\frac{2 f(k')}{2nc}\diff k'-\int_{-Q}^{Q}\frac{2c(2n) f(k')}{((2n)^2c^2+4(k-k')^2)}\diff k'\right)\\
	&\quad-\left(\int_{-Q}^{Q}\frac{2 f(k')}{2(n+1)c}\diff k'-\int_{-Q}^{Q}\frac{2c(2(n+1)) f(k')}{((2(n+1))^2c^2+4(k-k')^2)}\diff k'\right).
	\end{aligned}
\end{equation}
A straightforward computation shows\begin{equation}
\begin{aligned}
&\Delta_n=\\
&\int_{-Q}^{Q} \frac{2 f(k')}{2n(n+1)c}\\&-\Bigg(\frac{2c(2n)\left[(2(n+1))^2c^2+4(k-k')^2\right]}{\left[(2n)^2c^2+4(k-k')^2\right]\left[(2(n+1))^2c^2+4(k-k')^2\right]}\\
&\qquad \frac{-2c(2(n+1))\left[(2n)^2c^2+4(k-k')^2\right]}{\left[(2n)^2c^2+4(k-k')^2\right]\left[(2(n+1))^2c^2+4(k-k')^2\right]}\Bigg) f(k')\diff k'\\
&=\int_{-Q}^{Q}\frac{2 f(k')}{2n(n+1)c}\\&\quad -\frac{2c\cdot  8n (n+1)c^2-4c\cdot 4(k-k')^2}{\left[(2n)^2c^2+4(k-k')^2\right]\left[(2(n+1))^2c^2+4(k-k')^2\right]} f(k')\diff k'\\
&\geq \int_{-Q}^{Q}\frac{4c\cdot 4(k-k')^2}{\left[(2n)^2c^2+4(k-k')^2\right]\left[(2(n+1))^2c^2+4(k-k')^2\right]} f(k')\diff k'\\
&\geq 0
\end{aligned}
\end{equation}
It follows for any $ m_0 $ that \begin{equation}
\begin{aligned}
\sum_{n=0}^{m_0}(-1)^{n+1}&\int_{-Q}^{Q}\frac{2c(2n) f(k')}{((2n)^2c^2+4(k-k')^2)}\diff k'\\
&\leq\sum_{n=1}^{m_0}(-1)^{n+1}\int_{-Q}^{Q}\frac{2 f(k')}{2nc}\diff k'-\sum_{l=1}^{\floor{m_0/2}}\Delta_{(2l-1)}
\\
&\leq \sum_{n=0}^{m_0}(-1)^{n+1}\int_{-Q}^{Q}\frac{2 f(k')}{2nc}\diff k'.
\end{aligned}
\end{equation}
Here the first inequality is an \emph{equality} if $ m_0 $ is even, and the inequality when $ m_0 $ is odd follows from \eqref{EqYGLemma2bound2} with $ n=m_0 $.
\end{proof}
 We notice that we may upper bound $ f $:
 \begin{lemma}\label{LemmaYG:fUpperBound}
 	Let $ f $ be the solution of \eqref{EqYG1}--\eqref{EqYG3}, then
 	\begin{equation}
 	\begin{aligned}
 	2\pi  f(k) \leq 1+2\sum_{n=1}^{\infty}\frac{(1)^{n+1}}{n}\int_{-Q}^{Q}\frac{ f(k')}{c}\diff k'=1+\frac{2\ln(2)}{c} \rho.
 	\end{aligned}
 	\end{equation}
 \end{lemma}
\begin{proof}
	By Lemma \ref{LemmaYGrho} with $ m\to\infty $ using \eqref{EqYGLemma1bound} and Lemma \ref{LemmaYGbound:f} the result follows. 
\end{proof}
We are ready to give a lower bound for the ``ground state" energy of the Yang-Gaudin model. 
\begin{proposition}\label{PropositionYGLowerBound}
	Let $ e $ be the solution of \eqref{EqYG1}--\eqref{EqYG4}, then\begin{equation}
	e\geq \frac{\pi^2}{3}\rho ^3\left(\frac{1}{1+\frac{2\ln(2)}{c}\rho}\right)^2.
	\end{equation}
\end{proposition}
\begin{proof}
	We notice that the expression for $ e=\int_{-Q}^{Q}f(k)k^2\diff k $, given $ \int_{-Q}^{Q}f(k)\diff k=\rho $ \emph{and} $ f\leq K $, is minimized by having $ f=K\mathbbm{1}_{[-\rho/(2K),\rho /(2K)]} $, in which case $ \int_{-Q}^{Q}f(k)k^2\diff k=\frac{2}{3}K\left(\frac{\rho}{2K}\right)^3 $. That $ \rho/(2K)\leq Q $ follows straight away from $ \rho =\int_{-Q}^{Q}f(k)\diff k \leq 2KQ $.
	By Lemma \ref{LemmaYG:fUpperBound}, we find $ f\leq \frac{1}{2\pi}\left(1+\frac{2\ln(2)}{c}\rho\right) $, so it follows that $ e\geq \frac{\pi^2}{3}\rho ^3\left(\frac{1}{1+\frac{2\ln(2)}{c}\rho}\right)^2 $.
\end{proof}
We will, in Chapter \ref{ChapterTheGroundStateEnergyOfTheOneDimensionalDiluteSpin1/2FermiGas}, find a matching upper bound for the Yang-Gaudin ground state energy in the dilute limit.
\subsection{A small caveat}\label{SubsectionYGCaveat}
There is an issue in the analysis of the Yang-Gaudin model: It is safe to say that in the physics/integrability literature, the ``ground state" of \eqref{EqYGHamiltonian} is widely believed to be the one found above. However, there is, to the best of our knowledge, no rigorous proof in the literature that the true ground state of \eqref{EqYGHamiltonian} is among the Yang-Bethe ansatz states. In fact, there seems to be no proof of the existence of a solution to the equations \eqref{EqYGintegers} given two sets of integers $ (I_j)_{j=1}^{N} $ and $ (J_a)_{a=1}^{M} $. This is in contrast to the analysis of the Lieb-Liniger model, in which both the existence of solutions as well as the completeness of the Bethe ansatz states is known, \cite{cmp/1104252974}. Since we will not use the results from this section for any rigorous analysis in the remainder of the thesis, we leave the establishment of these facts for future work. We will in Chapter \ref{ChapterTheGroundStateEnergyOfTheOneDimensionalDiluteSpin1/2FermiGas} refer to the $ e $ coming from a solution of \eqref{EqYG1}--\eqref{EqYG4} as the ground state energy of the Yang--Gaudin model, however this non-rigorous use of the terminology is never used in any rigorous setting.\\
We may state for good measure what is needed to make statements about the ground state rigorous:
\begin{itemize}
	\item Establish existence of solutions of \eqref{EqYGintegers} for any two sets of integers $ (I_j,J_a)_{j,a} $, at any $ c>0 $ such that $ k_j,\Lambda_a $ varies continuously with $ c $.
	\item Either of the two: \begin{enumerate}
		\item Establish that Yang finds full multiplicity of solution converging to the ground state in the limit $ c\to\infty $. (In this case the theorem of Lieb and Mattis [Theorem \ref{TheoremLiebMattis}]) implies that no extra ground state can exist.
		\item Justify rigorously Gaudin's findings in the $ c\to 0 $ limit, where the ground state is unique \cite{gaudin1967systeme}. In this case, the ground state is of Bethe-Yang ansatz form in this limit. It is then implied that this is the case for all $ c>0 $ again by the theorem of Lieb and Mattis.
	\end{enumerate}
\end{itemize}










