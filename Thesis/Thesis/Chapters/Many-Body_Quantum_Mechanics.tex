\chapter{Many-Body Quantum Mechanics}
	In this chapter we give a brief introduction to many-body quantum mechanics. The chapter will serve to define relevant quantities, to set up the mathematical framework, and to state some preliminary results.

\section{Many-body Wave Functions}
	We consider a system of $N$ particles confined to some region $ \Omega\subseteq \R^d $. Here we refer to $ d $ as the \emph{dimension} of the system. In quantum mechanics such a system is described by a \emph{state} or \emph{wave function} in an underlying Hilbert space. In the present general scenario, we describe these particles by a wave function $ \Psi\in L^2\left(\prod_{i=1}^{N}\left(\Omega\times \{-S_i,...,S_i\}\right) \right)=\otimes_{i=1}^{N}L^2\left(\Omega;\C^{2S_i+1}\right) $, where $ S_i $ is \emph{spin} of the $ i $th particle. Here the underlying Hilbert space is taken to be $ \otimes_{i=1}^{N}L^2\left(\Omega;\C^{2S_i+1}\right) $.
	\subsection{Identical Particles: Bosons and Fermions}
		For the moment, let us ignore spin. In the case when the particles in question are identical, \ie indistinguishable, it turn out that one can restrict the underlying Hilbert space, to have certain symmetries. Considering $ N $ indistiguishable particles, we restrict to the physical configuration space to $ C_{p,N}=C_N/S_N $, with $ C_N:=\{(x_1,...,x_N)\in \Omega^N \vert x_i\neq x_j \text{ if }i\neq j\} $ on which the symmetric group act freely. We then require the underlying Hilbert space of the system to form an irreducible representation of the fundamental group $\pi_1(C_{p,N})$, where we noted that the physical configuration space is path-connected.
		\begin{remark}
			For $ d\geq 3 $ we have $\pi_1(C_{p,N})=S_N$, for $ d=2 $ we have $\pi_1(C_{p,N})=B_N$ and for $d=1$ we have $\pi_1(C_{p,N})=\{1\}$.
		\end{remark}
		In the remaining part of this thesis, we will mainly be interested in the two irreducible representations that are the symmetric representation and the antisymmetric representation, in which we refer to the particles as \emph{bosons} and \emph{fermions} respectively. Hence for bosons we restrict to wave-functions in the symmetric (or bosonic) subspace $ L^2_{s}\left(\prod_{i=1}^{N}\left(\Omega\times \{-S_i,\ldots,S_i\}\right) \right)=\vee_{i=1}^{N}L^2\left(\Omega; \C^{2S+1}\right)$ and for fermions we restrict to wave-functions in the antisymmetric (or fermionic) subspace $ L^2_{a}\left(\prod_{i=1}^{N}\left(\Omega\times \{-S_i,\ldots,S_i\}\right) \right)=\wedge_{i=1}^{N}L^2\left(\Omega; \C^{2S+1}\right)$.