 \chapter{Introduction}
Since the seminal work of Lee, Huang, and Yang in 1957 \cite{lee1957eigenvalues, lee1957many, huang1957quantum}, there has been a tremendous interest in dilute quantum gases and their ground state energy expansions. Finding good approximations for the bosonic ground state energy, at least in two and three dimensions, is intimately related to understanding the formation of Bose-Einstein condensates. Furthermore, such ground state energy expansions often exhibit universality. More specifically, the ground state energy of dilute systems tends to depend on the interaction potential only through the scattering length. This interest has in the mathematical physics literature grown during the last decades culminating in the recent completion of a rigorous proof of the Lee-Huang-Yang formula in 2019 \cite{yau2009second,fournais2020energy}\footnote{While the lower bound was made fully general in terms of assumptions on the interaction potential in 2021 \cite{fournais2021energy}, weakening the assumptions under which the upper bound can be proven is still an active field of research. }. With the problem essentially solved for the three dimensional Bose gas, it is natural to seek similar ground state energy expansions in other dimensions or with different particle statistics. Recently, the two dimensional bosonic ground state energy expansion was proven to analogous precision in \cite{fournais2022ground}, and previously the fermionic ground state energy expansions have been studied in both two and three dimensions \cite{lieb2005ground}.

The general one-dimensional dilute Bose gas, or quantum gas in general, has been surprisingly little studied both in the physics and mathematics literature. This may be partly due to the presence of solvable models in one dimension. In 1963 Lieb and Liniger showed that the one-dimensional Bose gas with point (delta-function) interactions is solvable by Bethe ansatz \cite{lieb1963exact}. In practice, this means that one may obtain algebraic equations for the ground state and excited energies, by realizing the eigenstates to be superpositions of plane waves with suitable scattering boundary conditions. Similarly, in 1967, the one-dimensional spin--$ 1/2 $ Fermi gas with point interactions was shown, in the physics literature, to be solvable by means of a generalized Bethe ansatz \cite{yang1967some}. This argument was one year later further generalized to accommodate any symmetry of the domain and hence any spin \cite{sutherland1968further}. Some effort has since then gone into arguing that various confined three dimensional systems may be well approximated by such point interacting systems in one dimension, leaving the analysis of the spectrum already complete \cite{olshanii1998atomic,petrov2000regimes,dunjko2001bosons,lieb2003one,lieb2004one,seiringer2008lieb}. In \cite{lieb2003one,lieb2004one,seiringer2008lieb} it was shown that such an approximation indeed is valid in certain confinement regimes. We call this regime the \emph{weak confinement regime}, and it is described by having the trapping length scale, in the transverse direction much longer than the three dimensional scattering length scale. This means that transverse excitations cannot be neglected. On the other hand, one may instead consider the \emph{strong confinement regime}, described by having the transverse trapping length scale much shorter than the scattering length scale. In this regime, the spectrum will presumably be well described by a purely one-dimensional system, with the three dimensional potential simply restricted to a line. A crucial difference in this case, is that the one-dimensional scattering length arising from such confinements may be positive, as opposed to the effective Lieb-Liniger model in which the one-dimensional scattering length always is negative.\\
In this thesis, we analyze ground state energies of general one-dimensional dilute gasses. This covers the strongly point interacting models but further extends the result to models with positive scattering lengths. The ground state energy expansion for one-dimensional dilute bosons and spin polarized fermions was recently obtained in \cite{agerskov2022ground}, which appears, in a revised edition, as Chapter \ref{ChapterTheGroundStateEnergyOfTheOneDimensionalDiluteBoseGas} of this thesis. The expansion obtained will exhibit similar universality as in the three and two dimensional cases. However, one major difference is apparent in the analysis and phenomenology of the one-dimensional gas: There is no Bose-Einstein condensation. This fact may be traced back to the celebrated theorem of Hohenberg, Mermin, and Wagner \cite{hohenberg1967existence,mermin1966absence}, which excludes longe-range order for one-dimensional interacting systems. Thus the formation of a condensate is broken by the interaction in one dimension. This famous result is in agreement with the results found in this thesis, where we explicitly verify that the ground state energy shows greater similarity to energies arising from Slater determinant states than to energies arising from a condensate.\\
The proof of a ground state energy expansion for the one-dimensional dilute Bose gas and spin polarized Fermi gas leaves the question of whether there is a similar expansion for the total ground state of the spin--$ 1/2 $ fermionic system. Such an expansion is conjectured in Chapter \ref{ChapterTheGroundStateEnergyOfTheOneDimensionalDiluteBoseGas} (\cite{agerskov2022ground}), based on the solvable models at hand for such a system. We present in Chapter \ref{ChapterTheGroundStateEnergyOfTheOneDimensionalDiluteSpin1/2FermiGas} a proof of an upper bound matching this conjecture. In the proof, we define a trial state in which the spin part is determined variationally. Interestingly, the variational problem determining the spin part is that of the one-dimensional Heisenberg chain. In the case of the usual spin--$ 1/2 $ fermions, we get the antiferromagnetic Heisenberg chain. However, we will show that for models of a different symmetry or with spin-dependent potentials, the spin chain may be both ferro- or antiferromagnetic. Furthermore, we will present an idea of how to prove a corresponding lower bound. We do this by proving results that are analogous to findings of Chapter \ref{ChapterTheGroundStateEnergyOfTheOneDimensionalDiluteBoseGas} (\cite{agerskov2022ground}). However, it will be apparent that certain results do not generalize for the spin--$ 1/2 $ Fermi system straightforwardly. We then present a conjecture which, if proven true, allows us to complete the generalization of the Chapter \ref{ChapterTheGroundStateEnergyOfTheOneDimensionalDiluteBoseGas} results. We give heuristic arguments for the validity of this conjecture, but also highlight where these arguments are lacking in mathematical rigor. Finally, we notice that the result of Chapter \ref{ChapterTheGroundStateEnergyOfTheOneDimensionalDiluteBoseGas} do generalize for spin--$ 1/2 $ systems with other symmetries or spin-dependent potentials exactly when the system is in a ferromagnetic phase.\\

We summarize here overall the structure of this thesis: In Chapter \ref{ChapterMany-BodyQuantumMechanics}, we review relevant concepts in many-body quantum mechanics. Furthermore, since we will allow for quite general interactions in the later analysis, we review under which conditions on the interaction potential the dynamics of quantum systems can be defined in terms of a lower bounded self-adjoint Hamiltonian. We prove a result stating that in one dimension this is possible for any interaction potential that is the sum of a $ \sigma $-finite measure and an absolutely continuous measure. After this we review the concept of diluteness and known results about dilute quantum gases. Finally, we both review and prove certain result about two solvable models in one dimension. In Chapter \ref{ChapterTheGroundStateEnergyOfTheOneDimensionalDiluteBoseGas}, we find and prove ground state energy expansions for both the one-dimensional Bose and spin polarized Fermi gas. In Chapter \ref{ChapterTheGroundStateEnergyOfTheOneDimensionalDiluteSpin1/2FermiGas}, we generalize some results from Chapter \ref{ChapterTheGroundStateEnergyOfTheOneDimensionalDiluteBoseGas} in order to prove an upper bound on the ground state energy of the one-dimensional dilute spin--$ 1/2 $ Fermi gas. Furthermore, we generalize certain results related to the lower bound in Chapter \ref{ChapterTheGroundStateEnergyOfTheOneDimensionalDiluteBoseGas}. Finally, we notice that completing the proof of a lower bound for the spin--$ 1/2 $ Fermi gas, is possible by proving a conjecture on the ground state energy of a model known as the Lieb-Liniger-Heisenberg model in its antiferromagnetic phase. We also note, in the ferromagnetic phase, that a tight lower bound on the Lieb-Liniger-Heisenberg model is trivially valid. Thus for certain other symmetries or spin-dependent potential, we find a tight lower bound exactly when they are in a ferromagnetic phase in this sense.
In Chapter \ref{ChapterConclusionAndOutlook}, we give a final summary of our findings and discuss open problems.

