\documentclass{beamer}[10]
\usepackage[english]{babel}
\usepackage[utf8]{inputenc}
%\usepackage{beamerthemesplit}
\usepackage{graphics,epsfig, subfigure}
\usepackage{url}
\usepackage{srcltx}
\usepackage{mathrsfs}
\usepackage{amsfonts}
\usepackage{amsmath}
\DeclareMathOperator\arctanh{arctanh}
\usepackage{amssymb}
\usepackage{bbm}
\usepackage{amsthm}
\usepackage{graphicx}
\usepackage{centernot}
\usepackage{caption}
\usepackage{braket}
\usepackage{lastpage}
\usepackage{setspace}
\usepackage{xcolor}
\usepackage{mathtools}
\usepackage{cancel}
\usepackage{pdfpages}



\newcommand{\euler}[1]{\text{e}^{#1}}
\newcommand{\Real}{\text{Re}}
\newcommand{\Imag}{\text{Im}}
\newcommand{\supp}{\text{supp}}
\newcommand{\norm}[1]{\left\lVert #1 \right\rVert}
\newcommand{\abs}[1]{\left\lvert #1 \right\rvert}
\newcommand{\floor}[1]{\left\lfloor #1 \right\rfloor}
\newcommand{\Span}[1]{\text{span}\left(#1\right)}
\newcommand{\dom}[1]{\mathcal D\left(#1\right)}
\newcommand{\Ran}[1]{\text{Ran}\left(#1\right)}
\newcommand{\conv}[1]{\text{co}\left\{#1\right\}}
\newcommand{\Ext}[1]{\text{Ext}\left\{#1\right\}}
\newcommand{\vin}{\rotatebox[origin=c]{-90}{$\in$}}
\newcommand{\interior}[1]{%
	{\kern0pt#1}^{\mathrm{o}}%
}
\renewcommand{\braket}[1]{\left\langle#1\right\rangle}
\newcommand*\diff{\mathop{}\!\mathrm{d}}
\newcommand{\ie}{\emph{i.e.} }
\newcommand{\eg}{\emph{e.g.} }
\newcommand{\dd}{\partial }
\newcommand{\R}{\mathbb{R}}
\newcommand{\C}{\mathbb{C}}
\newcommand{\w}{\mathsf{w}}
\newcommand{\rr}{\mathcal{R}}


\newcommand{\Gliminf}{\Gamma\text{-}\liminf}
\newcommand{\Glimsup}{\Gamma\text{-}\limsup}
\newcommand{\Glim}{\Gamma\text{-}\lim}

\newtheorem{mtheorem}{Theorem}
\newtheorem{mconjecture}{Conjecture}
\newtheorem{mdefinition}{Definition}
\newtheorem{mproposition}{Proposition}
\newtheorem{mlemma}{Lemma}
\newtheorem{mcorollary}{Corollary}
\newtheorem{mremark}{Remark}


\definecolor{qmathblue}{RGB}{61,131,131}
\definecolor{kugreen1}{RGB}{50,93,61}
\definecolor{kugreen}{RGB}{70,116,60}
\definecolor{qmathbluelys}{RGB}{119,168,168}
\definecolor{kugreenlyslys}{RGB}{173,190,177}
\definecolor{kugreenlyslyslys}{RGB}{239,249,240}
\definecolor{qmathbluelyslyslys}{RGB}{236,243,243}
\setbeamercovered{transparent}
\mode<presentation>
%\usetheme[numbers,totalnumber,compress,sidebarshades]{PaloAlto}
\setbeamertemplate{footline}[frame number]
\setbeamertemplate{theorems}[numbered]

\usecolortheme[named=qmathblue]{structure}
\useinnertheme{circles}
\usefonttheme[onlymath]{serif}
\setbeamercovered{transparent}
\setbeamertemplate{blocks}[rounded][shadow=false]
\setbeamercolor{block body}{bg=qmathbluelyslyslys,fg=black}
\setbeamercolor{block title}{bg=qmathbluelyslyslys,fg=black}


\logo{\includegraphics[width=1.2cm]{qmath.jpg}}
%\useoutertheme{infolines} 
\title{One-dimensional Dilute Quantum Gases and Their Ground State Energies}
\subtitle{}
\author{Johannes Agerskov}
\institute{Department of Mathematical Sciences \\ University of Copenhagen}
\date{\scriptsize{\textbf{PhD defense}}\\ \scriptsize{June 6, 2023}}

\setbeamercovered{invisible}

\begin{document}
\frame{\titlepage \vspace{-0.5cm}
}

\frame
{
\frametitle{Overview}
\tableofcontents%[pausesection]
}

\section{Motivation}

\begin{frame}
	\frametitle{Motivation (Bosons)}
	\begin{itemize}
		\item 2D and 3D dilute Bose gases are well-studied in the mathematical physics literature.
		\item 2D and 3D results are related to Bose-Einstein condensation (BEC).
		\item No BEC is expected in 1D.
		\item In 1D the hard core and Lieb-Liniger models are solvable.
		\item Our result is consistent with the absence of BEC in 1D.
		\item On the contrary it suggests that the 1D dilute bose gas shares features with the Fermi gas.
		
	\end{itemize}
	
\end{frame}

\begin{frame}
	\frametitle{Motivation (Fermions)}
	\begin{itemize}
		\item 2D and 3D dilute Fermi gases are well-studied in the mathematical physics literature.
		\item In 1D the hard core and Yang-Gaudin models are solvable.
		\item  In 1962 E. H. Lieb and D. C. Mattis showed that one-dimensional Fermi gases are antiferromagnetic (contradicting standard perturbative tight-binding methods).
		\item Hence the standard justification of the Heisenberg model of magnetism is too simple.
		\item Our result will break ground in rigorously justifying the Heisenberg antiferromagnet as an effective model in 1D.
	\end{itemize}
	
\end{frame}

%\begin{frame}
%	\frametitle{Motivation (bosons)}
%	\begin{itemize}
%		\item 1924: S. N. Bose and A. Einstein predict BEC.
%		\item 1947: N. N. Bogoliubov develops theory of superfluidity based on BEC.
%		\item  1957: Lee, Huang, and Yang derive formula for ground state energies of certain dilute Bose gases in 3D.
%		\item 1963: E. Lieb and W. Liniger solve one-dimensional boson problem.
%		\item 1995: E. Cornell and C. Wieman experimentally construct a BEC.
%	\end{itemize}
%
%\end{frame}
%
%\begin{frame}
%	\frametitle{Motivation (fermions)}
%	\begin{itemize}
%		\item  1928: W. Heisenberg develops model of magnetism.
%		\item 1962: E. Lieb and D. Mattis shows that one-dimensional Fermi gases are antiferromagnetic. 
%		\item 1967: C. N Yang solves the point interacting one-dimensional fermion problem
%	\end{itemize}
%	
%\end{frame}

\section{Many-Body Quantum Mechanics}
%\begin{frame}{Many-Body Quantum Mechanics}
%	\small
%	\begin{definition}
%		A quantum system of $N$ spin--$ S $ bosons/fermions in $ \Omega\subseteq\R^d $ at fixed time is a pair
%		\begin{equation*}
%			(\Psi,\mathcal{H}),\text{ with } \Psi\in\mathcal{H}\text{ and }\norm{\Psi}=1,
%		\end{equation*}
%		where $ \mathcal{H} $ is a closed subspace of $ L^2_{s/a}\left(\left(\Omega\times \{-S,\ldots,S\}\right)^N \right)\cong\left(\vee/\wedge\right)_{i=1}^{N}L^2\left(\Omega; \C^{2S+1}\right) $, and thus a Hilbert space. $\Psi$ is called \emph{the state} of the system.
%	\end{definition}
%	\begin{definition}
%		the probability of measurement of $ \mathcal{O} $ in the state $ \Psi\in\dom{\mathcal{O}} $ having any outcome $ \lambda $ such that $ \lambda\in M\subset \R $ is given by $ P\left((\mathcal{O},\Psi)\in M\right)=\int_{\lambda\in M} \braket{\Psi,P_\lambda\Psi} $
%		where $\{P_\lambda\}_{\lambda\in\sigma(\mathcal{O})}$ is the projection valued measure associated with $ \mathcal{O} $ by the spectral theorem.
%	\end{definition}
%\end{frame}
%\begin{frame}
%	\small
%	\begin{definition}\label{DefinitionGroundStateEnergy}
%		The \textbf{ground state energy} of $ H $ is defined by 
%		\begin{equation}\label{EqVariationalGroundStateEnergyOperator}
%			E_0(H)\coloneqq\inf_{\Psi\in \dom{\mathcal{H}}}\frac{\braket{\Psi,H\Psi}}{\norm{\Psi}^2},
%		\end{equation}
%		with $H$ being the Hamiltonian (infinitesimal generator of time evolution).
%	\end{definition}
%\begin{definition}
%	We say that a (normalized) state $ \Psi\in \dom{H}\subset \mathcal{H} $ is a \textbf{ground state} of $ H $ if $$ \braket{H}_\Psi=E_0(H). $$
%\end{definition}
%\begin{definition}
%	Given a Hamiltonian, $ H $, we define the \textbf{associated energy quadratic form}, $ \mathcal{E}_H:\dom{\mathcal{E}_H}\to \R $, as the closure of the quadratic form $ \dom{H}\ni\Psi\mapsto\braket{\Psi,H\Psi} $. When $ H $ is given from the context, we will often write $ \mathcal{E} $ as short for $ \mathcal{E}_H $.
%\end{definition}
%
%\end{frame}
%\begin{frame}
%	\small
%	\begin{mremark}\label{RemarkVariationalPrinciple2}
%		From the definition of $\mathcal{E}_H$ and from Definition \ref{DefinitionGroundStateEnergy} it follows straightforwardly that we have \begin{equation}\label{EqVariationalGroundStateEnergyForm}
%			E_0(H)=\inf_{\Psi\in \dom{\mathcal{E}_H}}\frac{\mathcal{E}_H(\Psi)}{\norm{\Psi}^2}=\inf_{\substack{\Psi\in \dom{\mathcal{E}_H},\\\norm{\Psi}=1}}\mathcal{E}_H(\Psi), 
%		\end{equation}
%		as $ \dom{H} $ is form core for $\mathcal{E}_H$.
%	\end{mremark}
%\begin{mremark}[\cite{reed1981functional} Theorem VIII.15]\label{RemarkOperatorFromQuadraticForm}
%	Given a densely defined, lower bounded, closable, quadratic form $ \mathcal{E}:\dom{\mathcal{E}}\to \R $ there exists a \textbf{unique} lower bounded, self-adjoint operator $ H_\mathcal{E} $, such that $ \mathcal{E}(\Psi)=\braket{\Psi,H_\mathcal{E}\Psi} $ for all $ \Psi\in \dom{H_\mathcal{E}} $, and $ \dom{H_\mathcal{E}} $ is form core for $\overline{\mathcal{E}}$, \ie the form closure of $ \braket{\cdot,H_\mathcal{E}\cdot} $ is equal to the form closure of $\mathcal{E}$. 
%\end{mremark}
%\end{frame}


\begin{frame}{Many-Body Quantum Mechanics}
	\small
		For a system of $ N $ bosons/fermions in region $ \Omega\in\R^d $, we define for $ \sigma\in[0,\infty] $ \textbf{the energy quadratic forms}
		\begin{equation}\label{key}
			\mathcal{E}_{(v,\sigma)}(\Psi)=\int_{\Omega^N} \sum_{i=1}^{N}\abs{\nabla_i\Psi}^2+\sum_{i<j} v(x_i-x_j)\abs{\Psi}^2+\sigma\int_{\dd (\Omega^N)}\abs{\Psi}^2,
		\end{equation}
		with domain $ \dom{\mathcal{E}_{(v,\sigma)}}=\{\Psi\in (C^\infty(\Omega^N))_{\text{b/f}}\vert \mathcal{E}_{(v,\sigma)}(\Psi)<\infty\} $. 
%		With $ (C^\infty(\Omega^N))_{\text{b/f}} $ meaning the bosonic/fermionic subspace of $ C^\infty(\Omega^N) $. $ \sigma=\infty $ is taken to mean Dirichlet boundary conditions.
\begin{definition}
	We say a potential $ v\geq 0 $ is \textbf{allowed} in dimension $ d $, if $ \mathcal{E}_{(v,\sigma)} $ is closable on $ \mathcal{H}_{(v,\sigma)}:=\overline{\dom{\mathcal{E}_{(v,\sigma)}}}^{\norm{\cdot}_2}\subset L^2_{s/a}(\Omega^N)$ for any $ \sigma\in[0,\infty] $.
\end{definition}
\begin{mproposition}\label{Lemma1dPotentialAllowed}
	Let $ d=1 $, then any potential of the form $v=v_{\sigma\text{--finite}}+v_{\text{meas.}}+c\delta_0 $, with $ c\in\{0,\infty\} $, is allowed.
\end{mproposition}
\end{frame}

\section{The Scattering Length}

\begin{frame}
\frametitle{The Scattering Length}
	\small\begin{theorem}
		For $ B_R=\{0\leq\abs{x}<R\}\subset \R^d $ with $ R>R_0\coloneqq\text{range}(v) $, let $ \phi\in H^1(B_{R}) $ satisfy
	 \begin{equation}
		 -\Delta \phi +\frac12 v\phi=0,\qquad \text{on }B_R,
		 \end{equation}
		 with boundary condition $ \phi(x)=1 $ for $ \abs{x}=R$.
		 Then $ \phi(x)=f(\abs{x}) $ for some $ f:(0,R]\to [0,\infty) $, and for $ \text{range}(v)<r<R $, we have \begin{equation}
		 f(r)=\begin{cases}
		 (r-a)/(R-a) &\text{for }d=1\\
		 \ln(r/a)/\ln(R/a) &\text{for }d=2\\
		 (1-ar^{2-d})/(1-aR^{2-d})&\text{for }d\geq 3,
		 \end{cases}
		 \end{equation}
		 with some constant $ a $ called the (s-wave) \textbf{scattering length}.
	\end{theorem}
\end{frame}

%\begin{frame}
%	\frametitle{Model}
%	We consider a many-body system of bosons that interacts via a repulsive pair potential $ v_{ij}=v(\abs{x_i-x_j}) $, with $ v=v_{\text{reg}}+v_{\text{h.c.}} $\begin{equation}
%	\mathcal{E}(\psi)=\int_{\Lambda_L}\left(\sum_{i=1}^{N}\abs{\nabla_i\psi}^2+\sum_{i<j} v_{ij}\abs{\psi}^2\right)\quad \text{on } L^2(\Lambda_L)^{\otimes_{\text{sym}} N}.
%	\end{equation}
%	Recall the ground state energy is
%	$$
%	E(N,L)\coloneqq\inf_{\psi\in\mathcal{D}(\mathcal{E}),\ \norm{\psi}^2=1}\mathcal{E}(\psi).
%	$$
%\end{frame}


\begin{frame}
	\frametitle{2D and 3D}
	\small For $ \Lambda_L=[0,L]^d $, let $ e(\rho)\coloneqq\lim\limits_{\substack{L\to\infty\\ N/L^{d}\to\rho}}E(N,L)/L^{d} $.
	\vspace*{-0.3cm}
	\begin{block}{}
		\small
		\vspace*{-0.2cm}
		\begin{theorem}[\small $ d=3 $ result, Lee-Huang-Yang 1957\footnotemark]
			\begin{equation}
				e(\rho)=4\pi\rho^2 a\left(1+\frac{128}{15\sqrt{\pi}}\sqrt{\rho a^3}+o(\sqrt{\rho a^3})\right).
			\end{equation}
		\end{theorem}
		\vspace*{-0.3cm}
		\begin{theorem}[\small $ d=2 $ result\footnotemark]
			\begin{equation}
				e(\rho)=4 \pi \rho^2 Y\left(1-Y|\log Y|+\left(2 \Gamma+\frac{1}{2}+\ln (\pi)\right) Y\right)+o\left(\rho^2 Y^2\right),
			\end{equation}
			$Y=\abs{\ln(\rho a^2)}^{-1}$.
		\end{theorem}
		\footnotetext[1]{\tiny Upper bound: Yau-Yin 2009, Basti-Cenatiempo-Schlein 2021. Lower bound: Fournais-Solovej 2021}
		\footnotetext[2]{\tiny Fournais-Girardot-Junge-Morin-Olivieri 2022}
	\end{block}
\end{frame}


\section{Bosons Main Result}

\begin{frame}
	\frametitle{Bosons Main Result}
	\small
	For the remainder of the presentation, $ d=1 $.
	\begin{block}{}
		\begin{theorem}[A., R. Reuvers, J. P. Solovej, 2022]
			\label{TheoremMain}
			Consider a Bose gas with repulsive interaction  $v=v_{\text{reg}}+v_{\text{h.c.}}$. Define the density $\rho=N/L$. For $\rho|a|$ and $\rho R_0$ sufficiently small, the ground state energy can be expanded as 
			\begin{equation}
			\label{result}
			E(N,L)=N\frac{\pi^2}{3}\rho^2\left(1+2\rho a+
			\mathcal{O}
			\left((\rho|a|)^{6/5}+(\rho R_0)^{6/5}+N^{-2/3}\right)\right),
			\end{equation}
			where $a$ is the scattering length of $v$.
		\end{theorem}
	\end{block}	
\end{frame}

\begin{frame}
	\frametitle{Examples}
	\small
	\begin{block}{The hard core gas}
		Energy behaves like free Fermi energy in volume $ L-NR $, \ie \begin{equation}
		\begin{aligned}
		E_{\text{hard core}}(N,L)&=N\frac{\pi^2}{3}\rho^2 (1-NR/L)^{-2}\\&= E_0\left(1+2\rho R+\mathcal{O}\left((\rho R)^2\right)\right).
		\end{aligned}
		\end{equation}
		Scattering length is $ a=R $.
	\end{block}
	\begin{block}{Lieb-Liniger model}
		Energy behaves asymptotically like
		\begin{equation}
		E_{LL}(N,L,c)=N\frac{\pi^2}{3}\rho^2\left(1-4\rho/c+\mathcal{O}\left((\rho/c)^2\right)\right),
		\end{equation}
		with scattering length $ a=-\frac{2}{c} $.
	\end{block}
\end{frame}

\section{Upper Bound}

\begin{frame}
	\frametitle{Variational Principle}
	\begin{block}{}
		To obtain an upper bound, we use the variational principle, \ie
		$$
		E(N,L)\leq \frac{\mathcal{E}(\Psi)}{\norm{\Psi}^2},\quad \text{for any }  \Psi\in \mathcal{D}(\mathcal{E}) .
		$$
	\end{block}	
\end{frame}

\begin{frame}
	\frametitle{Trial State}
	\begin{block}{}
		Trial state has to capture free Fermi energy, as well as corrections due to scattering processes. Hence we consider $$
		\Psi(x)=\begin{cases}
		\omega(\rr(x))\frac{\abs{\Psi_F(x)}}{\rr(x)}& \text{if }\rr(x)<b\\
		\abs{\Psi_F(x)}&\text{if }\rr(x)\geq b,
		\end{cases}
		$$
		where $ \omega $ is the suitably normalized solution to the two-body scattering equation,  $\Psi_F$ is the free Fermi ground state, and $ \rr(x)\coloneqq \min_{i<j}(\abs{x_i-x_j}) $ is uniquely defined a.e.
	\end{block}	
\end{frame}

\begin{frame}
	\frametitle{One-particle Reduced Density Matrix}
	\begin{block}{}
		For the free Fermi gas we have
	\begin{equation}
	\begin{aligned}
	\gamma^{(1)}(x,y)&=\frac{2}{L}\sum_{j=1}^{N}\sin\left(\frac{\pi}{L}jx\right)\sin\left(\frac{\pi}{L} jy\right)\\
	&=\frac{\pi}{L}\left(D_{N}\left(\pi\frac{x-y}{L}\right)+D_{N}\left(\pi\frac{x+y}{L}\right)\right),
	\end{aligned}
	\end{equation}
	where $ D_N(x)=\frac{1}{2\pi}\sum_{k=-N}^{N}\euler{ikx}=\frac{\sin((N+1/2)x)}{2\pi\sin(x/2)} $ is the Dirichlet kernel.\\
	By Wick's theorem all derivatives of reduced density matrices are bounded by a constant times an appropriate power of $ \rho $.
		\end{block}	
	\end{frame}


\begin{frame}
	\frametitle{Some Useful Bounds}
	\begin{block}{}
		\vspace{-0.5cm}
		\footnotesize{\begin{mlemma}\label{Lemma rho2 bound}
			$$ \rho^{(2)}(x_1,x_2)\leq\left(\frac{\pi^2}{3}\rho^4+f(x_2)\right)(x_1-x_2)^2+\mathcal{O}(\rho^6(x_1-x_2)^4), $$ 
			with $ \int f(x_2)\diff x_2\leq \textnormal{ const. }\rho^3\log(N). $
		\end{mlemma}
			\begin{mlemma}\label{LemmaDensityBounds}
				We have the following bounds\begin{equation*}
				\begin{aligned}
				\rho^{(3)}(x_1,x_2,x_3)&\leq \textnormal{const. }\rho^9(x_1-x_2)^2(x_2-x_3)^2(x_1-x_3)^2,\\
				\rho^{(4)}(x_1,x_2,x_3,x_4)&\leq \textnormal{const. }\rho^8(x_1-x_2)^2(x_3-x_4)^2,\\
				\abs{\sum_{i=1}^{2}\partial_{y_i}^2\gamma^{(2)}(x_1,x_2;y_1,y_2)\big\rvert_{y=x}}&\leq \textnormal{const. } \rho^{6}(x_1-x_2)^2,\\[-3.2ex]
				&\ \ \vdots
				\end{aligned}
				\end{equation*}
			\end{mlemma}	}
	\end{block}	
\end{frame}

\begin{frame}
	\frametitle{Collecting Everything}
	\begin{block}{Upper bound}
	\small\begin{equation}
	E\leq N\frac{\pi^2}{3}\rho^2\frac{\left(1+2\rho a\frac{b}{b-a} +\text{const. } \left[\frac{1}{N}+ N (b\rho)^3\left(1+\rho b^2\int v_{\text{reg}}\right)\right]\right)}{\norm{\Psi}^2},
	\end{equation}	
	where  the finite measure $ v_{\text{reg}} $ is $v$ with any hard core removed.
		By lemma \ref{Lemma rho2 bound} we know $ \norm{\Psi}^2\geq 1-\text{const. }N(\rho b)^3 $.
	\end{block}	
	\begin{block}{Localization}
		Divide into $ M $ smaller boxes with $ \tilde{N}=N/M $ particles in each, and make distance $ b $ between boxes (no interaction between boxes), and choose $ M $ such that $ \tilde{N}=(\rho b)^{-3/2}\gg 1 $.
	\end{block}

\end{frame}

\begin{frame}
	\frametitle{Upper Bound}
	\begin{block}{After localization}
\footnotesize\begin{equation}
\begin{aligned}
E(N,L)\leq N\frac{\pi^2}{3}\rho^2\frac{\left(1+2\rho a\frac{b}{b-a}+\text{const. }\frac{M}{N}+\text{const. }\tilde{N}(b\rho)^3\left(1+\rho b^2\int v_{\text{reg}}\right)\right)}{1-\tilde{N}(\tilde{\rho} b)^3}
\end{aligned}
\end{equation}

Choosing $ b=\max(\rho^{-1/5}\abs{a}^{4/5}, R_0) $ we find

\begin{mproposition}[Upper bound Theorem \ref{TheoremMain}]
	\label{PropositionUpperBound}
 There exists a constant $C_\text{U}>0$ such that for $\rho|a|$, $\rho R_0\leq C_U^{-1}$, the ground state energy $E^D(N,L)$ satisfies
	\small{\begin{equation}
	\label{equpper}
	E^D(N,L)\leq N\frac{\pi^2}{3}\rho^2\left(1+2\rho a + C_\text{U}\left((\rho\abs{a})^{6/5}+(\rho R_0)^{3/2}+N^{-1}\right)\right).
	\end{equation}}
\end{mproposition}

	\end{block}
	
\end{frame}



\section{Lower Bound}

\begin{frame}
	\frametitle{Lower Bound}
	Proof of lower bound consists of the following steps:
	\begin{enumerate}
		\item Use Dyson's lemma to reduce to a nearest neighbor double delta-barrier potential.
		\item Reduce to the Lieb Liniger model by discarding \textbf{a small part} of the wave function.
		\item Use a known lower bound for the Lieb Liniger model.
	\end{enumerate}	
\end{frame}

\begin{frame}
	\frametitle{The Lieb-Liniger (LL) model}
	\begin{block}{}
	\begin{equation}
	H_{LL}=-\sum_{i=1}^{n}\partial_i^2+2c\sum_{i<j}\delta(x_i-x_j).
	\end{equation}
	Behavior in thermodynamic limit: $ \lim\limits_{\substack{\ell\to\infty,\\ n/\ell\to \rho}}E_{LL}(n,\ell,c)/\ell=\rho^3 e(\gamma) $ with $ \gamma=c/\rho  $.
		\begin{mlemma}[Lieb-Liniger lower bound] \label{LemmaLL-LowerBound}
			Let $ \gamma>0 $, then
			\begin{equation}
			e(\gamma)\geq \frac{\pi^2}{3}\left(\frac{\gamma}{\gamma+2}\right)^2\geq \frac{\pi^2}{3}\left(1-\frac{4}{\gamma}\right).
			\end{equation}
		\end{mlemma}
	\end{block}	
\end{frame}

\begin{frame}
	\frametitle{Reducing to the LL Model}
	\begin{block}{}
		\vspace{-0.5cm}
	\begin{mlemma}[Dyson]\label{LemmaDyson} Let $ R>R_0=\textnormal{range}(v) $ and $ \varphi\in H^1(\R) $, then for any interval $ \mathcal{I}\ni 0 $ 
		\begin{equation}
		\int_{\mathcal{I}} \abs{\partial \varphi}^2+\frac12 v\abs{\varphi}^2\geq \int_{\mathcal{I}}\frac{1}{R-a}\left(\delta_R+\delta_{-R}\right)\abs{\varphi}^2,
		\end{equation}
		where $ a $ is the s-wave scattering length.
	\end{mlemma}
	Hence we have, denoting $ \mathfrak{r}_{i}(x)=\min_j(\abs{x_i-x_j}) $ \begin{equation}
	\begin{aligned}
	&\int \sum_{i}\abs{\partial_i\Psi}^2+\sum_{i\neq j} \frac{1}{2}v_{ij}\abs{\Psi}^2\geq\\ &\int\sum_{i}\abs{\partial_i\Psi}^2\chi_{\mathfrak{r}_i(x)>R}+\sum_{i}\frac{1}{R-a}\delta(\mathfrak{r}_i(x)-R)\abs{\Psi}^2.
	\end{aligned}
	\end{equation}	
	\end{block}	
\end{frame}


\begin{frame}
	\frametitle{Reducing to the LL Model}
	\begin{block}{}
		Define $ \psi\in L^2([0,\ell-(n-1)R]^n) $ by 
			$$ \psi(x_1,x_2,...,x_n)=\Psi(x_1,R+x_2,...,(n-1)R+x_n), $$
			 for $ x_1\leq x_2\leq...\leq x_n $ and symmetrically extended.\\\vspace{0.2cm}
			 Then \begin{equation}
			 \begin{aligned}
			 \mathcal{E}(\Psi)&\geq E^N_{LL}(n,\ell-(n-1)R,2/(R-a))\braket{\psi|\psi}\\
			 &\geq n\frac{\pi^2}{3}\rho^2\left(1+2\rho(a-\cancel{R})+\cancel{2\rho R}-\text{const. }\frac{1}{n^{2/3}}\right)\braket{\psi|\psi}.
			 \end{aligned}
			 \end{equation}
	\end{block}	
\end{frame}

\begin{frame}
	\frametitle{Lower Bound for Mass of $ \psi $}
	\begin{block}{}\vspace{-0.5cm}
			\small\begin{mlemma}\label{LemmaNormLoss}
				Let $ \psi $ be defined as above, then \begin{equation}
				1-\braket{\psi|\psi}\leq 8 \left(R^2\sum_{i<j}\int_{B_{ij}}\abs{\partial_i \Psi}^2+R(R-a)\sum_{i<j}\int v_{ij} \abs{\Psi}^2\right),
				\end{equation}
			\end{mlemma}
			\
			Combining lemmas \ref{LemmaDyson} and \ref{LemmaNormLoss} we have the following lemma:
			\begin{mlemma}\label{LemmaImprovedMassBound}
			 For $ n(\rho R)^2\leq  \frac{3}{16\pi^2}\frac{1}{8} $, $ \rho R\ll 1 $ and $ R>2\abs{a} $ we have
				\begin{equation}\label{EqImprovedMassBound}
				\begin{aligned}
				\braket{\psi|\psi} \geq 1-\textnormal{const. }\left(n(\rho R)^3+n^{1/3}(\rho R)^2\right).
				\end{aligned}
				\end{equation}
			\end{mlemma}
	\end{block}	
\end{frame}

\begin{frame}
	\frametitle{Lower Bound}
	\begin{block}{}
	\small	By the reduction to the LL model we find 
		\begin{mproposition}\label{PropositionLowerBoundSpecN}
			For assumptions as in lemma \ref{LemmaImprovedMassBound} we have \begin{equation}
			E^N(n,\ell)\geq n\frac{\pi^2}{3}\rho^2\left(1+2\rho a+\textnormal{const. }\left(\frac{1}{n^{2/3}}+n(\rho R)^3+n^{1/3}(\rho R)^2\right)\right).
			\end{equation}
		\end{mproposition}
		\begin{mcorollary} \label{CorollaryLowerBoundSpecN}
			For $ n=\textnormal{const. } (\rho R)^{-9/5} $ we have 
			\begin{equation}
			E^N(n,\ell)\geq n\frac{\pi^2}{3}\rho^2\left(1+2\rho a-\textnormal{const. }\left((\rho R)^{6/5}+(\rho R)^{7/5}\right)\right).
			\end{equation}
		\end{mcorollary}
	\end{block}	
\end{frame}

\begin{frame}
	\frametitle{Lower Bound Localization}
	\begin{block}{}
	\small	To prove the lower bound, we localize, as in the upper bound, to smaller boxes.
		\begin{mlemma}\label{LemmaLocalizationFbound}
			Let $ \Xi\geq 4 $ be fixed and let $ n=m\Xi \rho \ell+n_0 $ with $ n_0\in[0,\Xi\rho \ell) $ for some $ m\in\mathbb{N} $ with $ n^{\ast}:=\rho\ell=\mathcal{O}(\rho R)^{-9/5} $. Furthermore, assume that $ \rho R\ll 1 $ and let $ \mu=\pi^2\rho^2\left(1+\frac{8}{3}\rho a\right) $, then \begin{equation}
			E^{N}(n,\ell)-\mu n \geq E^{N}(n_0,\ell)-\mu n_0.
			\end{equation}
		\end{mlemma}
		\begin{mproposition}[Lower bound Theorem \ref{TheoremMain}]
			\label{PropositionLowerBound}
			There exists a constant $C_\text{L}>0$ such that the ground state energy $E^N(N,L)$ satisfies
			\begin{equation}
			\label{eqlower}
			E^N(N,L)\geq N\frac{\pi^2}{3}\rho^2\left(1+2\rho a-C_\text{L}\left((\rho\abs{a})^{6/5}+(\rho R_0)^{6/5}+N^{-2/3}\right)\right).
			\end{equation}
		\end{mproposition}
	\end{block}	
\end{frame}
\begin{frame}
	\frametitle{Spinless/Spin-Polarized Fermions}
	Spinless fermions are unitarily equivalent to bosons with a zero b.c. at all planes of intersection, \ie with an infinite delta potential. As a consequence we have the following corollary. 
	\begin{theorem}[Spin-polarized fermions]\label{TheoremFermion}
		Consider a spin-polarized Fermi gas with repulsive interaction  $v=v_{\text{reg}}+v_{\text{h.c.}}$ as defined before. Let $ E_F(N,L)$ be its associated ground state energy. Write $\rho=N/L$. For $\rho a_o$ and $\rho R_0$ sufficiently small, the ground state energy can be expanded as 
		\begin{equation}
		E_F(N,L)=N\frac{\pi^2}{3}\rho^2\left(1+2\rho a_o+\mathcal{O}\left(\left(\rho R_0\right)^{6/5}+N^{-2/3}\right)\right),
		\end{equation}
		where $ a_o\geq0 $ is the odd wave scattering length of $v$. 
	\end{theorem} 
	This is consistent with lower bound $ E_F(N,L)\geq E_0 $, since $ a_o\geq 0 $.
\end{frame}
\section{Spin--$1/2$ Fermions}
\begin{frame}
	\frametitle{Two solvable model for spin-1/2 fermions}
	\vspace*{-0.4cm}
	\begin{block}{The hard core gas}
		Ground state energy is independent of spin
		so \begin{equation}
		E_{\text{hard core}}(N,L)=N\frac{\pi^2}{3}\rho^2 (1-NR/L)^{-2}\approx E_0(1+2\rho R).
		\end{equation}
		Scattering length is $ a_e=a_o=R $.
	\end{block}
	\begin{block}{Yang-Gaudin model}
		Is the spin-1/2 version of the LL model, \ie $ H_{YG}=H_{LL} $.
		 Behaves asymptotically like
		 \begin{equation}
		 E_{YG}(N,L,c)=N\frac{\pi^2}{3}\rho^2\left(1-4\rho\ln(2)/c+\mathcal{O}\left((\rho/c)^2\right)\right),
		 \end{equation}
		 with scattering length $ a_e=-\frac{2}{c} $, $ a_o=0 $.
		\end{block}
\end{frame}

\begin{frame}
	\frametitle{A Conjecture for Spin-1/2 Fermions}
	\small
	Based on the two solvable cases, we expect
	\begin{mconjecture}\label{ConjectureEqCSpin1/2FermiGroundStateEnergy}
		Let $ v\geq0 $ satisfy the assumption from above, then the ground state energy of the dilute spin--$ 1/2 $ Fermi gas satisfies\begin{equation}\label{EqConjectureEqCSpin1/2FermiGroundStateEnergy}
			E=N\frac{\pi^2}{3}\rho^2\left(1+2\rho \left(\ln(2) a_e+(1-\ln(2))a_o\right)+\mathcal{O}(\rho^2\max(\abs{a_e},a_o)^2)\right).
		\end{equation}
	\end{mconjecture}
	
	 \small\begin{equation}
	\begin{aligned}
	E(N,L)=N\frac{\pi^2}{3}\rho^2\Big(1+2\ln(2)\rho a_e+2(1-\ln(2))\rho a_o\\+\mathcal{O}\left((\rho\max(\abs{a_e},a_o))^2\right)\Big)
	\end{aligned}
	\end{equation}
\end{frame}

\begin{frame}
	\frametitle{Spin--$1/2$ Fermions Main Result (Upper Bound)}
	\begin{block}{}
		\small
		\begin{theorem}\label{TheoremUpperBoundSpin1/2Fermi}
			Let $ v\geq0 $ satisfy the assumption from above, then the ground state energy of the dilute spin--$ 1/2 $ Fermi gas satisfies\begin{equation}\label{EqUpperBoundSpin1/2Fermi}
				E\leq N\frac{\pi^2}{3}\rho^2\left(1+2\rho \left(\ln(2) a_e+(1-\ln(2))a_o\right)+\mathcal{O}\left((\rho R)^{6/5}+N^{-1}\right)\right),
			\end{equation}
			with $ R=\max(\abs{a_e}, a_o, R_0) $.
		\end{theorem}
	\end{block}	
\end{frame}

\begin{frame}
	\frametitle{Trial State}
	\small One the sector
	$$ \{1,2,...,N\}=\{0<x_{1}<x_{2}<\ldots<x_{N}<L\} $$
	we define the trial state by 
	\begin{equation}\label{EqTrial StateSpin1/2Fermi}
		\Psi_\chi=\begin{cases}
			\frac{\Psi_F}{\mathcal{R}}\left(\left(\eta\omega^{\mathcal{R}}_e+(1-\eta)\omega^{\mathcal{R}}_o\right)\operatorname{P}_s^{\mathcal{R}}+\omega_o^{\mathcal{R}}\operatorname{P}_t^{\mathcal{R}}\right)\chi,&\mathcal{R}(x)<b\\
			\Psi_F\chi ,&\mathcal{R}(x)\geq b
		\end{cases},
	\end{equation}
where $ \chi $ is some spin state, $ b>R_0 $, $ \mathcal{R}(x)=\min_{i<j}\abs{x_i-x_j} $, $ \omega^\mathcal{R}_{e/o}(x)\coloneqq \omega_{e/o}(\mathcal{R}(x))=bf_{e/o}(\mathcal{R}(x)) $ and 
\begin{equation}
	\eta(x)\coloneqq\begin{cases}
		0,&\text{ if } \mathcal{R}_2(x)\leq b\\
		\left(\frac{\mathcal{R}_2(x)}{b}-1\right), &\text{ if } b<\mathcal{R}_2(x)<2b\\
		1, &\text{ if } \mathcal{R}_2(x)\geq 2b.
	\end{cases}
\end{equation}
with $ \mathcal{R}_2(x)=\min_{(i,j)\neq (k,l)}\max(\abs{x_i-x_j},\abs{x_k-x_l}) $.
\end{frame}

\begin{frame}
	\frametitle{}
	\vspace*{1cm}
		\small Trial state energy is the free Fermi energy with a correction of the form
	$$
	2\rho\left( (a_o-a_e) \braket{\chi \left\vert \frac{1}{N}\sum_{i}S_i\cdot S_{i+1} \right\vert \chi} +\frac{1}{4}a_e+\frac{3}{4}a_o \right)E_F.
	$$ 
	\emph{Proof:}\vspace*{-0.1cm}
	\begin{figure}[h!]
		\includegraphics[scale=0.35]{/home/johannes/Downloads/Thesis_maincomputationpages.pdf}
	\end{figure}
\end{frame}

\begin{frame}
	\frametitle{Antiferromagnetic Heisenberg Chain}
	The (periodic) antiferromagnetic Heisenberg chain 
	$$
	H=\sum_{i=1}^{N}S_i\cdot S_{i+1}, \text{ with } S_{N+1}\coloneqq S_1
	$$
 Ground state energy per site of the infinite chain is know due to Hulthen
 \begin{lemma} \label{LemmaHeisenbergChainThermodynamicGSEnergy}
 	Let $ \ket{\textnormal{GS}_{\textnormal{HAF}}} $ denote the ground state of the periodic antiferromagnetic Heisenberg chain. Then\begin{equation}
 		\lim\limits_{N\to\infty}\braket{\textnormal{GS}_{\textnormal{HAF}}\Bigg\vert\frac{1}{N}\sum_{k=1}^N S_k\cdot S_{k+1} \Bigg\vert \textnormal{GS}_{\textnormal{HAF}}}=\frac{1}{4}-\ln(2) 
 	\end{equation}
 \end{lemma}
\end{frame}

\begin{frame}
	Control of the error for a finite chain
	\begin{lemma}\label{LemmaHeisenbergChainFiniteNEstimate}
		Let $ \ket{\textnormal{GS}_{\textnormal{HAF}}} $ denote the ground state of the periodic antiferromagnetic Heisenberg chain. Then\begin{equation}
			\braket{\textnormal{GS}_{\textnormal{HAF}}\Bigg\vert\frac{1}{N}\sum_{k=1}^N S_k\cdot S_{k+1} \Bigg\vert \textnormal{GS}_{\textnormal{HAF}}}=\frac{1}{4}-\ln(2) +\mathcal{O}(N^{-1})
		\end{equation}
	\end{lemma}
\begin{proof}
	Upper bound: Truncate longer of length $M>N$ chain at length $N$.
	Lower bound: Construct trial state for longer chain of length $mN$ by $m$ copies of length $N$ chain. Use translation invariance and uniqueness of the ground state:
	$\frac{1}{mN}(E_{mN}-m)\leq\frac1N E_N\leq \frac{1}{M}E_M+1$.
\end{proof}
\end{frame}

\begin{frame}
	\frametitle{Lower Bound in Terms of LLH Model}
	\small
	\begin{lemma}[Dyson's lemma spin--$ 1/2 $ fermions]
		\label{LemmaDysonSpin1/2Fermi}
		Let $ R>R_0=\textnormal{range}(v) $ and $ \varphi\in \left(H_{\textnormal{even}}^1(\R)\otimes\operatorname{P}_s \left(\left(\C^2\right)^2\right)\right)\oplus\left(H_{\textnormal{odd}}^1(\R)\otimes \operatorname{P}_t \left(\left(\C^2\right)^2\right) \right) $, then for any interval $ \mathcal{I}\ni 0 $ 
		\begin{equation}
			\int_{\mathcal{I}} \abs{\partial \varphi}^2+\frac12 v\abs{\varphi}^2\geq \int_{\mathcal{I}}\overline{\varphi}\left(\frac{1}{R-a_e}\operatorname{P}_s+\frac{1}{R-a_o}\operatorname{P}_t\right)\left(\delta_R+\delta_{-R}\right)\varphi,
		\end{equation}
		where $ a_{e/o} $ is the even/odd-wave scattering length.
	\end{lemma}
The Lieb-Liniger-Heisenberg model: \begin{equation}\label{EqHamiltonianLLH}
	H_{LLH}=-\sum_i \partial_i^2 +2\sum_{i<j} \left(c'\operatorname{\tilde{P}}^{i,j}_s+c\operatorname{\tilde{P}}^{i,j}_t\right)\delta(x_i-x_j),
\end{equation}
where the spin projectors, $\operatorname{\tilde{P}}_{s/t}$ are defined on the sector $ \{\sigma\} $ to be  $$\operatorname{\tilde{P}}^{ij}_{s/t}=\operatorname{P}^{\sigma^{-1}(i)\sigma^{-1}(j)}_{s/t}.$$
\end{frame}
\begin{frame}
\begin{mproposition}
	\label{PropositionLowerBoundSpecNSpin1/2Fermi}
	For $ n(\rho R)^2\leq  \frac{3}{16\pi^2}\frac{1}{8} $, $ \rho R\leq \frac{1}{2} $ and $ R>2\max(\abs{a_e},a_o,R_0) $ we have \begin{equation}
		\begin{aligned}
			E^N(N,L)\geq E_{LLH}^N&\left(N,\tilde{L},\frac{2}{R-a_e},\frac{2}{R-a_o}\right)\\&\times\left(1-\textnormal{const.}\left(n(\rho R)^3+n^{1/3}(\rho R)^2\right)\right).
		\end{aligned}
	\end{equation}
\end{mproposition}
\begin{mremark}
	The Lieb-Liniger-Heisenberg model is not exactly solvable. Thus no available good lower bound.
\end{mremark}
\end{frame}

\section{Conclusion and Outlook}
\begin{frame}
	\frametitle{Conclusion and Outlook}
	\small
	\begin{block}{We have shown that:}
		\small
		\begin{itemize}
			\item Interaction pair-potential of the form $v=v_{\sigma\text{-finite}}+v_{\text{meas.}}+c\delta_0$ give rise to a unique Hamiltonian.
			\item The ground state energy of the dilute Bose (spin polarized Fermi or anyon) gas in one dimension can be expanded in terms of the diluteness parameter to next-to-leading order (universality).
			\item The solvable models of spin--$1/2$ fermionic systems are consistent with a similar expansion.
			\item The ground state energy of the dilute spin--$1/2$ Fermi gas can be upper bounded by this expansion.
			\item   The solvable models of spin--$1/2$ fermionic systems are consistent with a similar expansion.
			\item The ground state energy of the dilute spin--$1/2$ Fermi gas can be lower bounded by the ground state energy of the Lieb-Liniger-Heisenberg model.
		\end{itemize}
	\end{block}
	
\end{frame}
\begin{frame}
	\begin{block}{Interesting future problems:}
		\begin{itemize}
			\item Showing a ground state energy expansion to order $(\rho a)^2$ for the dilute Bose gas.
			\item  Proving that the given upper bound for the Fermi gas is tight, \ie matching lower bound. For example by showing lower bound for LLH model.
			\item Fully understanding the connection to antiferromagnetism.
			\item Understanding the momentum distribution of the dilute Bose gas (similar universality).
			\item Generalize spin--$1/2$ Fermi results to higher spin.
		\end{itemize}
	\end{block}
\end{frame}

\begin{frame}
	\centering{Thanks for your attention!}
\end{frame}


\end{document}
