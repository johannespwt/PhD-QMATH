\documentclass{beamer}[10]
\usepackage{pgf}
\usepackage[english]{babel}
\usepackage[utf8]{inputenc}
%\usepackage{beamerthemesplit}
\usepackage{graphics,epsfig, subfigure}
\usepackage{url}
\usepackage{srcltx}
\usepackage{pdfpages}
\usepackage{mathrsfs}
\usepackage{amsfonts}
\usepackage{amsmath}
\DeclareMathOperator\arctanh{arctanh}
\usepackage{amssymb}
\usepackage{bbm}
\usepackage{amsthm}
\usepackage{graphicx}
\usepackage{centernot}
\usepackage{caption}
\usepackage{braket}
\usepackage{lastpage}
\usepackage{enumitem}
\usepackage{setspace}
\usepackage{xcolor}


\newcommand{\euler}[1]{\text{e}^{#1}}
\newcommand{\Real}{\text{Re}}
\newcommand{\Imag}{\text{Im}}
\newcommand{\supp}{\text{supp}}
\newcommand{\norm}[1]{\left\lVert #1 \right\rVert}
\newcommand{\abs}[1]{\left\lvert #1 \right\rvert}
\newcommand{\floor}[1]{\left\lfloor #1 \right\rfloor}
\newcommand{\Span}[1]{\text{span}\left(#1\right)}
\newcommand{\dom}[1]{\mathscr D\left(#1\right)}
\newcommand{\Ran}[1]{\text{Ran}\left(#1\right)}
\newcommand{\conv}[1]{\text{co}\left\{#1\right\}}
\newcommand{\Ext}[1]{\text{Ext}\left\{#1\right\}}
\newcommand{\vin}{\rotatebox[origin=c]{-90}{$\in$}}
\newcommand{\interior}[1]{%
	{\kern0pt#1}^{\mathrm{o}}%
}
\renewcommand{\braket}[1]{\left\langle#1\right\rangle}
\newcommand*\diff{\mathop{}\!\mathrm{d}}
\newcommand{\ie}{\emph{i.e.} }
\newcommand{\eg}{\emph{e.g.} }
\newcommand{\dd}{\partial }
\newcommand{\R}{\mathbb{R}}
\newcommand{\C}{\mathbb{C}}
\newcommand{\w}{\mathsf{w}}
\newcommand{\rr}{\mathcal{R}}


\newcommand{\Gliminf}{\Gamma\text{-}\liminf}
\newcommand{\Glimsup}{\Gamma\text{-}\limsup}
\newcommand{\Glim}{\Gamma\text{-}\lim}

%\newtheorem{theorem}{Theorem}
%\newtheorem{definition}{Definition}
%\newtheorem{proposition}{Proposition}
%\newtheorem{lemma}{Lemma}
%\newtheorem{corollary}{Corollary}
%\newtheorem{remark}{Remark}


\definecolor{kugreen}{RGB}{50,93,61}
\definecolor{kugreenlys}{RGB}{132,158,139}
\definecolor{kugreenlyslys}{RGB}{173,190,177}
\definecolor{kugreenlyslyslys}{RGB}{214,223,216}
\setbeamercovered{transparent}
\mode<presentation>
%\usetheme[numbers,totalnumber,compress,sidebarshades]{PaloAlto}
\setbeamertemplate{footline}[frame number]

  \usecolortheme[named=kugreen]{structure}
  \useinnertheme{circles}
  \usefonttheme[onlymath]{serif}
  \setbeamercovered{transparent}
  \setbeamertemplate{blocks}[rounded][shadow=true]

\logo{\includegraphics[width=0.8cm]{kuscience-logo}}
%\useoutertheme{infolines} 
\title{QLunch: The ground state energy of dilute 1d quantum systems}
\subtitle{}
\author{Johannes Agerskov\\\vspace{0.2 cm}
	\scriptsize{in collaboration with Robin Reuvers and Jan Philip Solovej}}
\institute{QMATH \\ University of Copenhagen}
\date{January 12th 2022}

\setbeamercovered{invisible}

\begin{document}
\frame{\titlepage \vspace{-0.5cm}
}

\frame
{
\frametitle{Overview}
\tableofcontents%[pausesection]
}

\section{Background}

\begin{frame}
\frametitle{Background}
\begin{block}{}
	
\end{block}	
\end{frame}


\section{Main result}

\begin{frame}
	\frametitle{Main result}
	\begin{block}{}
		
	\end{block}	
\end{frame}



\section{Upper bound}

\begin{frame}
	\frametitle{Variational principle}
	\begin{block}{}
		
	\end{block}	
\end{frame}

\begin{frame}
	\frametitle{Trial state}
	\begin{block}{}
		
	\end{block}	
\end{frame}

\begin{frame}
	\frametitle{Some useful bounds}
	\begin{block}{}
		
	\end{block}	
\end{frame}

\begin{frame}
	\frametitle{Collecting everything}
	\begin{block}{}
		
	\end{block}	
\end{frame}



\section{Lower bound}

\begin{frame}
	\frametitle{Lower bound}
	\begin{block}{}
		
	\end{block}	
\end{frame}

\begin{frame}
	\frametitle{The Lieb-Liniger (LL) model}
	\begin{block}{}
		
	\end{block}	
\end{frame}

\begin{frame}
	\frametitle{Reducing to the LL model}
	\begin{block}{}
		
	\end{block}	
\end{frame}

\begin{frame}
	\frametitle{Lower bound for specific particle number}
	\begin{block}{}
		
	\end{block}	
\end{frame}

\begin{frame}
	\frametitle{General lower bound}
	\begin{block}{}
		
	\end{block}	
\end{frame}


\end{document}
