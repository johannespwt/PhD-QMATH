\documentclass[a4paper,11pt]{article}
\usepackage[utf8]{inputenc}
\usepackage[margin=1in]{geometry}
\usepackage{pdfpages}
\usepackage{mathrsfs}
\usepackage{amsfonts}
\usepackage{amsmath}
\DeclareMathOperator\arctanh{arctanh}
\usepackage{amssymb}
\usepackage{bbm}
\usepackage{amsthm}
\usepackage{graphicx}
\usepackage{centernot}
\usepackage{caption}
\usepackage{subcaption}
\usepackage{braket}
\usepackage{pgfplots}
\usepackage{lastpage}
\usepackage{enumitem}
\usepackage{setspace}
\usepackage{xcolor}
\usepackage[english]{babel} 

\usepackage[square,sort,comma,numbers]{natbib}
\usepackage[colorlinks=true,linkcolor=blue]{hyperref}

\usepackage{fancyhdr}
\newcommand{\euler}[1]{\text{e}^{#1}}
\newcommand{\Real}{\text{Re}}
\newcommand{\Imag}{\text{Im}}
\newcommand{\supp}{\text{supp}}
\newcommand{\norm}[1]{\left\lVert #1 \right\rVert}
\newcommand{\abs}[1]{\left\lvert #1 \right\rvert}
\newcommand{\floor}[1]{\left\lfloor #1 \right\rfloor}
\newcommand{\Span}[1]{\text{span}\left(#1\right)}
\newcommand{\dom}[1]{\mathscr D\left(#1\right)}
\newcommand{\Ran}[1]{\text{Ran}\left(#1\right)}
\newcommand{\conv}[1]{\text{co}\left\{#1\right\}}
\newcommand{\Ext}[1]{\text{Ext}\left\{#1\right\}}
\newcommand{\vin}{\rotatebox[origin=c]{-90}{$\in$}}
\newcommand{\interior}[1]{%
	{\kern0pt#1}^{\mathrm{o}}%
}
\renewcommand{\braket}[1]{\left\langle#1\right\rangle}
\newcommand*\diff{\mathop{}\!\mathrm{d}}
\newcommand{\ie}{\emph{i.e.} }
\newcommand{\eg}{\emph{e.g.} }
\newcommand{\dd}{\partial }
\newcommand{\R}{\mathbb{R}}
\newcommand{\C}{\mathbb{C}}
\newcommand{\w}{\mathsf{w}}

\newcommand{\Gliminf}{\Gamma\text{-}\liminf}
\newcommand{\Glimsup}{\Gamma\text{-}\limsup}
\newcommand{\Glim}{\Gamma\text{-}\lim}

\newtheorem{theorem}{Theorem}
\newtheorem{definition}{Definition}
\newtheorem{proposition}{Proposition}
\newtheorem{lemma}{Lemma}
\newtheorem{corollary}{Corollary}

\numberwithin{equation}{section}
\linespread{1.3}

\pagestyle{fancy}
\fancyhf{}
\rhead{Notes on the scattering length of rank one perturbations}
\lhead{Johannes Agerskov}
\rfoot{\thepage}
\lfoot{Dated: \today}
\author{Johannes Agerskov}
\date{Dated: \today}
\title{Notes on the scattering length of rank one perturbation}
\begin{document}

	\maketitle
\vspace{1cm}
We are in these notes going to study a bosonic system consisting of bosons of equal mass, $ m $, interacting via rank one perturbations. For simplicity we assume $ m=1 $. The $ N $-body Hamiltonian act in the following way\begin{equation}
H_N\psi(x)=-\frac{1}{2}\sum_{i=1}^{N}\Delta_i\psi(x)+\gamma_R\sum_{i<j}\mathbbm{1}_{B_R}(x_i-x_j)\int_{B_R}\psi(x_1,...,(x_{ij}+\bar{x}_{ij})/2,...,(-x_{ij}+\bar{x}_{ij})/2,...,x_N)\diff x_{ij}.
\end{equation}
The two-body $ s $-wave scattering length is found by the usual variational. principle. Notice first, that for the two-body problem, we may split the Hamiltonian in a centre of mass part, and a relative motion part obtaining \begin{equation}
H_2=H_{CM}+H_{\text{rel}}
\end{equation}
with $ H_{CM}=-\frac{1}{4}\Delta_{X_{CM}} $ and $ H_{\text{rel}}=-\Delta_y+\gamma_R \ket{\phi_R}\bra{\phi_R} $, with $ \phi_R=\mathbbm{1}_{B_R} $ and $ y=x_1-x_2 $ being the relative coordinate. The scattering legth is defined by the assymtotics of the radial zero energy solution to the scattering equation \begin{equation}
H_{\text{rel}}\psi=0
\end{equation}
It is clear that for $ r>R $ we have $ \psi(x)=1-a/\abs{x} $, for some number $ a $, called the scattering length, since this is just the usual Laplace's equation.\\
writing $ \psi=1-\omega $, we have $ -\Delta\omega=A\mathbbm{1}_{B_R} $ with $ \omega(x)\to0 $ as $ \abs{x}\to\infty $, and we see by Gauss' law that $ a=\frac{R^3}{3}A $ and that $ \omega(x)=-\frac{ \abs{x}^2}{6}A+k $ for $ \abs{x}<R $. Continuity gives implies $ k=AR^2/2$. Now, self-consistency demands $ 4\pi\gamma_R\int_{0}^{R}(1-A(-\frac{1}{6}r^2+\frac{1}{2}R^2))r^2\diff r=A $, or equivalently $ A=4\pi\gamma_R\left(\frac{1}{3}R^3+\frac{A}{30}R^5-\frac{A}{6}R^5\right) $. Hence we find $ A\left(1+\frac{2}{15}4\pi\gamma_RR^5\right)=\frac{4\pi\gamma_RR^3}{3} $. Combining with the above we find \begin{equation}
\frac{a}{R}=\frac{4\pi\gamma_RR^5}{3^2(1+\frac{2}{15}4\pi\gamma_R R^5)}
\end{equation}

\end{document}