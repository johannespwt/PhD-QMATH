\documentclass[a4paper,11pt]{article}
\usepackage[utf8]{inputenc}
\usepackage[margin=1in]{geometry}
\usepackage{pdfpages}
\usepackage{mathrsfs}
\usepackage{amsfonts}
\usepackage{amsmath}
\DeclareMathOperator\arctanh{arctanh}
\usepackage{amssymb}
\usepackage{bbm}
\usepackage{amsthm}
\usepackage{graphicx}
\usepackage{centernot}
\usepackage{caption}
\usepackage{subcaption}
\usepackage{braket}
\usepackage{lastpage}
\usepackage{enumitem}
\usepackage{setspace}
\usepackage{xcolor}
\usepackage[english]{babel} 

\usepackage[square,sort,comma,numbers]{natbib}
\usepackage[colorlinks=true,linkcolor=blue]{hyperref}

\usepackage{fancyhdr}
\newcommand{\euler}[1]{\text{e}^{#1}}
\newcommand{\Real}{\text{Re}}
\newcommand{\Imag}{\text{Im}}
\newcommand{\supp}{\text{supp}}
\newcommand{\norm}[1]{\left\lVert #1 \right\rVert}
\newcommand{\abs}[1]{\left\lvert #1 \right\rvert}
\newcommand{\floor}[1]{\left\lfloor #1 \right\rfloor}
\newcommand{\Span}[1]{\text{span}\left(#1\right)}
\newcommand{\dom}[1]{\mathscr D\left(#1\right)}
\newcommand{\Ran}[1]{\text{Ran}\left(#1\right)}
\newcommand{\conv}[1]{\text{co}\left\{#1\right\}}
\newcommand{\Ext}[1]{\text{Ext}\left\{#1\right\}}
\newcommand{\vin}{\rotatebox[origin=c]{-90}{$\in$}}
\newcommand{\interior}[1]{%
	{\kern0pt#1}^{\mathrm{o}}%
}
\renewcommand{\braket}[1]{\left\langle#1\right\rangle}
\newcommand*\diff{\mathop{}\!\mathrm{d}}
\newcommand{\ie}{\emph{i.e.} }
\newcommand{\eg}{\emph{e.g.} }
\newcommand{\dd}{\partial }
\newcommand{\R}{\mathbb{R}}
\newcommand{\C}{\mathbb{C}}
\newcommand{\w}{\mathsf{w}}
\newcommand{\rr}{\mathcal{R}}


\newcommand{\Gliminf}{\Gamma\text{-}\liminf}
\newcommand{\Glimsup}{\Gamma\text{-}\limsup}
\newcommand{\Glim}{\Gamma\text{-}\lim}

\newtheorem{theorem}{Theorem}
\newtheorem{definition}{Definition}
\newtheorem{proposition}{Proposition}
\newtheorem{lemma}{Lemma}
\newtheorem{corollary}{Corollary}
\newtheorem{remark}{Remark}

\numberwithin{equation}{section}
\linespread{1.3}

\pagestyle{fancy}
\fancyhf{}
\rhead{Notes on 1D bosons}
\lhead{}
\rfoot{\thepage}
\lfoot{Dated: \today}
\author{}
\date{Dated: \today}
\title{Notes on 1D bosons}
\begin{document}
	\maketitle
	We consider the dilute Bose gas in one dimension, where we seek to prove the formula for the ground state energy \begin{equation}\label{EqGroundstateE}
	\frac{E}{L}=\frac{\pi^2}{3}\rho^3\left(1+2\rho a+ o\left(\rho \right)\right).
	\end{equation} 
	We assume that the interaction potential $ v $ has compact support, say in the ball of radius $ R_0 $, $ B_{R_0} $.
	\section{Upper bound}
%	\subsection{Periodic boundary conditions}
%	We provide the upper bound for \eqref{EqGroundstateE}, by using the variational principle with a suitable trial state. We assume for simplicity periodic boundary conditions to begin with.
%	 Consider the trial state\begin{equation}
%	\Psi(x)=\begin{cases}
%	\omega(\rr(x))\frac{\tilde{\Psi}_F(x)}{\sin\left(\frac{\pi}{L}\rr(x)\right)}& \text{if }\rr(x)<b,\\
%	\tilde{\Psi}_F(x)&\text{if }\rr(x)\geq b,
%	\end{cases}
%	\end{equation}
%	where $ \omega $ is the suitably normalized solution to the two-body scattering equation, \ie $ \omega(x)=f(x)\frac{\sin\left(\frac{\pi}{L}b\right)}{f(b)} $ where $ f $ is any solution of the two-body scattering equation and $ b\geq R_0 $.  $ \tilde{\Psi}_F(x)=\mathcal{N}^{1/2}\prod_{i<j}^{N}\sin\left(\frac{\pi}{L}\abs{x_i-x_j}\right) $ is the absolute value of the free fermionic ground state, and $ \rr(x)=\min_{i<j}(\abs{x_i-x_j}) $ is uniquely defined a.e.\\
%	The energy of this trial state is then\begin{equation}
%	\mathcal{E}(\Psi)=\int \sum_{i=1}^{N}\abs{\nabla_i\Psi}^2+\sum_{i<j}^{N}v_{ij}\abs{\Psi}^2,
%	\end{equation}
%	where $ v_{ij}(x)=v(x_i-x_j) $. Since $ v $ is supported in $ B_b $ and $ \Psi=\tilde{\Psi}_F $ except in the region $ B=\{x\in\R^N \vert \rr(x)<b \} $, we may rewrite this as \begin{equation}
%	\mathcal{E}(\Psi)=E_0+\int_B \sum_{i=1}^{N}\abs{\nabla_i\Psi}^2+\sum_{i<j}^{N}v_{ij}\abs{\Psi}^2-\sum_{i=1}^{N}\abs{\nabla_i\tilde{\Psi}_F}^2,
%	\end{equation}
%	where $ E_0=N\frac{\pi^2}{3}\rho^2 $ is the ground state energy of the free Fermi gas. Using that $ v\geq0 $, symmetry of exchange of particles, and defining the set $ B_{12}=\{x\in\R^N \vert \rr(x)<b,\ \rr(x)=\abs{x_1-x_2} \}\subset A_{12}=\{x\in\R^N\vert \abs{x_1-x_2}<b\} $ which up to a set of measure zero is the intersection of $ B $ and $ \{\text{1 and 2 are closest}\} $, we find \begin{equation}
%	\begin{aligned}
%	\mathcal{E}(\Psi)&=E_0+\binom{N}{2}\int_{B_{12}} \sum_{i=1}^{N}\abs{\nabla_i\Psi}^2+\sum_{i<j}^{N}v_{ij}\abs{\Psi}^2-\sum_{i=1}^{N}\abs{\nabla_i\tilde{\Psi}_F}^2\\&
%	=E_0+\binom{N}{2}\int_{B_{12}} \sum_{i=1}^{N}\abs{\nabla_i\tilde{\Psi}}^2+\sum_{i<j}^{N}v_{ij}\abs{\tilde{\Psi}}^2-\sum_{i=1}^{N}\abs{\nabla_i\tilde{\Psi}_F}^2\\&
%	=E_0+\binom{N}{2}\int_{A_{12}} \sum_{i=1}^{N}\abs{\nabla_i\tilde{\Psi}}^2+\sum_{i<j}^{N}v_{ij}\abs{\tilde{\Psi}}^2-\sum_{i=1}^{N}\abs{\nabla_i\tilde{\Psi}_F}^2\\&\qquad
%	-\binom{N}{2}\int_{A_{12}\setminus B_{12}} \sum_{i=1}^{N}\abs{\nabla_i\tilde{\Psi}}^2+\sum_{i<j}^{N}v_{ij}\abs{\tilde{\Psi}}^2-\sum_{i=1}^{N}\abs{\nabla_i\tilde{\Psi}_F}^2\\&
%	\leq E_0+E_1+\binom{N}{2}\int_{A_{12}\setminus B_{12}}\sum_{i=1}^{N}\abs{\nabla_i\tilde{\Psi}_F}^2
%	\end{aligned}
%	\end{equation}
%	where we have defined \begin{equation*}
%		\tilde{\Psi}=\begin{cases}
%		\omega(x_1-x_2)\frac{\tilde{\Psi}_F(x)}{\sin\left(\frac{\pi}{L}\abs{x_1-x_2}\right)}& \text{if }\abs{x_1-x_2}<b,\\
%		\tilde{\Psi}_F(x)&\text{if }\abs{x_1-x_2}\geq b,
%		\end{cases}
%	\end{equation*} and $ E_1=\binom{N}{2}\int_{A_{12}} \sum_{i=1}^{N}\abs{\nabla_i\tilde{\Psi}}^2+\sum_{i<j}^{N}v_{ij}\abs{\tilde{\Psi}}^2-\sum_{i=1}^{N}\abs{\nabla_i\tilde{\Psi}_F}^2 $.\\
%	We may estimate \begin{equation}
%	\begin{aligned}
%	\binom{N}{2}\int_{A_{12}\setminus B_{12}}\sum_{i=1}^{N}\abs{\nabla_i\tilde{\Psi}_F}^2&=\binom{N}{2}\left(2N\left[\int_{A_{12}\cap A_{13}}\sum_{i=1}^{N}\abs{\nabla_i\tilde{\Psi}_F}^2-\int_{B_{12}\cap A_{13}}\sum_{i=1}^{N}\abs{\nabla_i\tilde{\Psi}_F}^2\right]\right.\\
%	&\qquad\qquad\left.+\binom{N-2}{2}\left[\int_{A_{12}\cap A_{34}}\sum_{i=1}^{N}\abs{\nabla_i\tilde{\Psi}_F}^2-\int_{B_{12}\cap A_{34}}\sum_{i=1}^{N}\abs{\nabla_i\tilde{\Psi}_F}^2\right]\right)\\
%	&\leq \binom{N}{2}\left[2N\int_{A_{12}\cap A_{13}}\sum_{i=1}^{N}\abs{\nabla_i\tilde{\Psi}_F}^2+\binom{N-2}{2}\int_{A_{12}\cap A_{34}}\sum_{i=1}^{N}\abs{\nabla_i\tilde{\Psi}_F}^2\right]
%	\end{aligned}
%	\end{equation}
%	Thus we find \begin{equation}\label{EqBound1}
%	\mathcal{E}(\Psi)\leq E_0+E_1+E_2^{(1)}+E_2^{(2)}
%	\end{equation}
%	with $ E_2^{(1)}=\binom{N}{2}2N\int_{A_{12}\cap A_{13}}\sum_{i=1}^{N}\abs{\nabla_i\tilde{\Psi}_F}^2 $ and $ E_2^{(2)}=\binom{N}{2}\binom{N-2}{2}\int_{A_{12}\cap A_{34}}\sum_{i=1}^{N}\abs{\nabla_i\tilde{\Psi}_F}^2 $.\\
%	We notice that since $ \tilde{\Psi}_F=\abs{\Psi_F} $ so by the diamagnetic inequality we have $ \abs{\nabla_i\tilde{\Psi}_F}^2\leq \abs{\nabla_i\Psi_F}^2 $, which implies that $ \tilde{\Psi}_F $ is in $ H^{1}(\Lambda_L) $. Furthermore, $ \Psi_F $ is $ C^{1}(\Lambda_L) $ with a zero set $ \{\Psi_F=0\} $ of measure zero, $ \abs{\nabla_i\tilde{\Psi}_F}^2 $ and $ \abs{\nabla_i\Psi_F}^2 $ are equal a.e. But then $ \abs{\nabla_i\tilde{\Psi}_F}=\abs{\nabla_i\Psi_F} $ as $ L^{2}(\Lambda_L) $ functions. Hence we may replace $ \tilde{\Psi}_F $ with $ \Psi_F $ in all integrals above.
%	
%	\subsubsection{Reduced density matrices}
%	We recall briefly the definition of the reduced density matrices, as we will use some fact about these frequently in the subsequent calculations.
%	For an $ N $-(identical )particle state $ \Psi $, the $ n $-particle reduced density matrix is defined by \begin{equation}
%	\gamma^{(n)}(x_1,...,x_n;y_1,...,y_n)=\frac{N!}{(N-n)!}\int \overline{\Psi(x_1,...,x_N)}\Psi(y_1,...,y_N)\diff x_{n+1}...\diff x_N.
%	\end{equation}
%	For a determinant state $ \Psi=\det\left(u_i(x_j)\right) $ with $ u_i $ orthonormal states, the one-particle reduced density matrix is given by \begin{equation}
%	\gamma^{(1)}(x,y)=\sum_{i=1}^{N} \overline{u_i(x)}u_i(y).
%	\end{equation} 
%	The $ n $-particle reduced density matrix may be expressed in terms of creation and annihilation operators as \begin{equation}
%	\gamma^{(n)}(x_1,...,x_n;y_1,...,y_n)=\braket{a^\dagger_{x_1}...a^\dagger_{x_n}a_{y_n}...a_{y_1}}.
%	\end{equation}
%	For the ground state of a free Hamiltonian (or any quasi free state), Wick's theorem applies and $ n $-particle reduced density matrix of the Fermi ground state may be computed recusively by \begin{equation}
%	\braket{c^\dagger_{x_1}...c^\dagger_{x_n}c_{y_n}...c_{y_1}}=\sum_{i=1}^{n}(-1)^{i-1}\braket{c^\dagger_{x_1}c_{y_i}}\braket{c^\dagger_{x_2}...c^\dagger_{x_n}c_{y_n}...c_{y_{i+1}}c_{y_{i-1}}...c_{y_1}}.
%	\end{equation}
%	For the Fermi ground state with periodic boundary conditions, we also have \begin{equation}
%	\begin{aligned}
%	\gamma^{(1)}(x,y)=\braket{c^\dagger_{x}c_{y}}=\frac{1}{L}\sum_{j=-(N-1)/2}^{(N-1)/2}\euler{i2\pi(x-y)j/L}=\frac{1}{L}\euler{-i\pi(x-y)(\rho-1/L)}\sum_{j=0}^{N-1}\left(\euler{i2\pi(x-y)/L}\right){j}\\=\frac{1}{L}\euler{-i\pi(x-y)(N-1)/L}\frac{1-\euler{2\pi(x-y)\rho}}{1-\euler{2\pi(x-y)/L}}=\frac{1}{L}\frac{\euler{-i\pi\rho(x-y)}-\euler{i\pi\rho(x-y)}}{\euler{-i\pi(x-y)/L}-\euler{i\pi(x-y)/L}}=\frac{1}{L}\frac{\sin(\pi\rho(x-y))}{\sin\left(\frac{\pi}{L}(x-y)\right)}.
%	\end{aligned}
%	\end{equation}
%	For $ x-y\ll\rho^{-1} $ we may use the relation \begin{equation}
%	\gamma^{(1)}(x,y)=\rho+\frac{\pi^2}{6}\left(\frac{\rho}{L^2}-\rho^3\right)(x-y)^2+\mathcal{O}((x-y)^3).
%	\end{equation}
%	\subsubsection{Calculating $ E_1 $}
%	Recall the definition \begin{equation}
%	E_1=\binom{N}{2}\int_{A_{12}} \sum_{i=1}^{N}\abs{\nabla_i\tilde{\Psi}}^2+\sum_{i<j}^{N}v_{ij}\abs{\tilde{\Psi}}^2-\sum_{i=1}^{N}\abs{\nabla_i\tilde{\Psi}_F}^2
%	\end{equation}
%	We estimate $ E_1 $ by splitting it in three terms. First we have \begin{equation}
%	\begin{aligned}
%	E_1^{(1)}&=2\binom{N}{2}\int_{A_{12}}\abs{\nabla_1\tilde{\Psi}}^2\\&
%	=2\binom{N}{2}\int_{A_{12}}\overline{\tilde{\Psi}}\left( -\Delta_1 \tilde{\Psi} \right)+2\binom{N}{2}\int\left[\overline{\tilde{\Psi}}\nabla_1\tilde{\Psi}\right]_{x_1=x_2-b}^{x_1=x_2+b}.
%	\end{aligned}
%	\end{equation}
%	The boundary term can be explicitly calculated, and to lowest order in $ b $ we find \begin{equation}
%	\begin{aligned}
%	2\binom{N}{2}\int\left[\overline{\tilde{\Psi}}\nabla_1\tilde{\Psi}\right]_{x_1=x_2-b}^{x_1=x_2+b}=L\left[\frac{\omega(x)}{\sin(\pi x/L)}\partial_x\left(\frac{\omega(x)}{\sin(\pi x/L)}\right)\gamma^{(2)}(x,0)\right]_{-b}^{b}\\+L\left[\left(\frac{\omega(x)}{\sin(\pi x/L)}\right)^2\partial_x\left(\gamma^{(2)}(x,0;y,0)\right)\bigg\vert_{y=x}\right]_{-b}^{b}.
%	\end{aligned}
%	\end{equation}
%	Since the continuous function $ \frac{\omega(x)}{\sin(\pi x/L)}=\frac{x-a}{b-a}\frac{\sin(\pi b/L)}{\sin(\pi x/L)} $ for $ \abs{x}>b $, we see that \begin{equation}
%	\partial_x\left(\frac{\omega(x)}{\sin(\pi x/L)}\right)\bigg\vert_{x=\pm b}\approx\pm\pi b/L\frac{\frac{1}{b-a}-1}{(\pi b/L)}=\pm\frac{a}{b^2}
%	\end{equation}
%	and we know that $ \gamma^{(2)}(x,0)=\frac{\pi^2}{3}\rho^4 x^2 $. Furthermore, by Wick's theorem it is straightforward to show that \begin{equation}\label{EqGammaDeriv.}
%	\partial_x\left(\gamma^{(2)}(x,0;y,0)\right)\bigg\vert_{y=x}=\frac{\pi^2}{3}N\rho^3 x + \rho^2 o(\rho x)
%	\end{equation}
%	Thus we have \begin{equation}
%	E_1^{(1)}=\frac{\pi^2}{3}N\rho^3 (2a+2b)+2\binom{N}{2}\int_{A_{12}}\overline{\tilde{\Psi}}(-\Delta_1\tilde{\Psi})
%	\end{equation}
%	Another contribution to $ E_1 $ is \begin{equation}
%	\begin{aligned}
%	E_1^{(2)}&=-\binom{N}{2}\int_{A_{12}}2\abs{\nabla_1\Psi_F}^2+\sum_{i=3}^{N}\abs{\nabla_i\Psi_F}^2=\\&-\binom{N}{2}\int_{A_{12}}\sum_{i=1}^{N}\overline{\Psi_F}(-\Delta_i\Psi_F)-2\binom{N}{2}\int\left[\overline{\Psi_F}\nabla_1\Psi_F\right]_{x_1=x_2-b}^{x_1=x_2+b}\\
%	&=-E_0\binom{N}{2}\int_{A_{12}}\abs{\Psi_F}^2-L\left[\partial_y\gamma^{(2)}(x,0;y,0)\vert_{y=x}\right]_{-b}^{b}
%	\end{aligned}
%	\end{equation}
%	Again using \eqref{EqGammaDeriv.} and $ \gamma^{(2)} $ we find \begin{equation}
%	E_1^{(2)}=-E_0\frac{1}{2}\frac{\pi^2}{9}N\rho^3b^3-\frac{\pi^2}{3}N\rho^3 (2b).
%	\end{equation}
%	The last contributions are $ E^{(3)}_1=\binom{N}{2}\int_{A_{12}} \sum_{i<j}^{N}v_{ij}\abs{\tilde{\Psi}}^2=\binom{N}{2}\int_{A_{12}}v_{12}\abs{\tilde{\Psi}}^2+\binom{N}{2}\int_{A_{12}} \sum_{2\leq i<j}^{N}v_{ij}\abs{\tilde{\Psi}}^2 $ and $ E_1^{(4)}=\int_{A_{12}}\sum_{i=3}^{N}\abs{\nabla_i\tilde{\Psi}}^2 $.
%	First we notice that \begin{equation}
%	\begin{aligned}
%	&\binom{N}{2}\int_{A_{12}} \sum_{2\leq i<j}^{N}v_{ij}\abs{\tilde{\Psi}}^2\\&\quad\leq C'_1\int_{A_{12}\cap\supp(v_{34})}v(x_3-x_4)\gamma^{(4)}(x_1,x_2,x_3,x_4)+C_2'\int_{A_{12}\cap\supp(v_{23})}v(x_2-x_3)\gamma^{(3)}(x_1,x_2,x_3).
%	\end{aligned}
%	\end{equation}
%	To leading order in $ L $, $ \abs{x_3-x_4} $ and $ \abs{x_1-x_2} $ we find that \begin{equation}
%	\gamma^{(4)}(x_1,x_2,x_3,x_4)=\frac{\pi^4}{9}\rho^8(x_1-x_2)^2(x_3-x_4)^2
%	\end{equation}
%	and to leading order in $ L $, $ \abs{x_1-x_2} $ and $ \abs{x_2-x_3} $ we find \begin{equation}
%	\gamma^{(3)}(x_1,x_2,x_3)=\frac{\pi^6}{135}\rho^9\underbrace{(x_1-x_3)^2}_{=[(x_1-x_2)+(x_2-x_3)]^2}(x_1-x_2)^2(x_2-x_3)^2.
%	\end{equation}
%	Therefore we have
%	\begin{equation}
%	\begin{aligned}
%	&\binom{N}{2}\int_{A_{12}} \sum_{2\leq i<j}^{N}v_{ij}\abs{\tilde{\Psi}}^2\\&\quad\leq C' \left(N^2(\rho b)^3\rho^3\int x^2 v(x)\diff x+N(\rho b)^3 \rho^5 \int x^4 v(x)\diff x+N(\rho b)^4\rho^4 \int x^3 v(x)\diff x\right.\\
%	&\qquad \qquad \qquad \qquad\hspace{6cm}\left.+N(\rho b)^5 \rho^3 \int x^2 v(x)\diff x\right)\\
%	&\quad \leq C' N^2(\rho b)^5\rho \int v=\text{const. }E_0 N (\rho b)^3 \left(b\int v\right)
%	\end{aligned}
%	\end{equation}
%	 and then we find that \begin{equation}
%	\begin{aligned}
%	E_1&=E_1^{(1)}+E_1^{(2)}+E_1^{(3)}+E_1^{(4)}\\&\leq \frac{2\pi^2}{3}N\rho^3 a+2\binom{N}{2}\int_{A_{12}}\left(\overline{\tilde{\Psi}}(-\Delta_1)\tilde{\Psi}+\frac{1}{2}\sum_{i=3}^{N}\abs{\nabla_i\tilde{\Psi}}^2+\frac{1}{2}v_{12}\abs{\tilde{\Psi}}^2\right)+E_0N(\rho b)^3\left(-\frac{1}{2}\frac{\pi^2}{9}+\frac{1}{2}\rho \int v\right)
%	\end{aligned}
%	\end{equation}
%	Using the two body scattering equation this implies \begin{equation}
%	\begin{aligned}
%	E_1&\leq \frac{2\pi^2}{3}N\rho^3 a+2\binom{N}{2}\int_{A_{12}}\overline{\tilde{\Psi}}\omega(-\Delta_1)\frac{\Psi_F}{\sin(\pi(x_1-x_2)/L)}\\&\quad+2\binom{N}{2}\int_{A_{12}}\overline{\tilde{\Psi}}(\nabla_1\omega)\nabla_1\frac{\Psi_F}{\sin(\pi(x_1-x_2)/L)}\\
%	&\quad +\binom{N}{2}\int_{A_{12}}\sum_{i=3}^{N} \overline{\tilde{\Psi}}\frac{\omega}{\sin(\pi(x_1-x_2)/L)}(-\Delta_i)\Psi_F
%	\\&\quad-E_0\frac{1}{2}\frac{\pi^2}{9}N\rho^3b^3+\text{const. }E_0 N (\rho b)^3 \left(b\int v\right)
%	\end{aligned}
%	\end{equation}
%	Now using that \begin{equation}
%	\begin{aligned}
%	&\binom{N}{2}\int_{A_{12}}\sum_{i=3}^{N} \overline{\tilde{\Psi}}\frac{\omega}{\sin(\pi(x_1-x_2)/L)}(-\Delta_i)\Psi_F\\&\quad=E_0\binom{N}{2}\int_{A_{12}}\left\lvert\frac{\omega}{\sin(\pi(x_1-x_2)/L)}\tilde{\Psi}\right\rvert^2-2\binom{N}{2}\int_{A_{12}} \overline{\tilde{\Psi}}\frac{\omega}{\sin(\pi(x_1-x_2)/L)}(-\Delta_1)\Psi_F,
%	\end{aligned}
%	\end{equation}
%	 \begin{equation}
%	\binom{N}{2}\int_{A_{12}}\left\lvert\frac{\omega}{\sin(\pi(x_1-x_2)/L)}\Psi_F\right\rvert^2\leq C_1 \left(\frac{b}{L}\right)^2\pi^2\rho^4 \left(\frac{L}{\pi}\right)^2 L b=C_1N\rho^3 b^3
%	\end{equation}
%	and that \begin{equation}
%	\begin{aligned}
%	2\binom{N}{2}\int_{A_{12}} \overline{\tilde{\Psi}}\frac{\omega}{\sin(\pi(x_1-x_2)/L)}(-\Delta_1)\Psi_F&\leq C_2 E_0 N\rho^3b^3
%	\end{aligned}
%	\end{equation}
%	we find that \begin{equation}
%	\binom{N}{2}\int_{A_{12}}\sum_{i=3}^{N} \overline{\tilde{\Psi}}\frac{\omega}{\sin(\pi(x_1-x_2)/L)}(-\Delta_i)\Psi_F\leq C E_0 N(\rho b)^3.
%	\end{equation}
%	Furhtermore we find to leading order in $ N $ and $ \rho b $ that \begin{equation}
%	\begin{aligned}
%	2\binom{N}{2}\int_{A_{12}}\overline{\tilde{\Psi}}\omega(-\Delta_1)\frac{\Psi_F}{\sin(\pi(x_1-x_2)/L)}=\frac{\pi^2}{15}N\rho^2 (\rho b)^3,
%	\end{aligned}
%	\end{equation}
%	and that \begin{equation}
%	2\binom{N}{2}\int_{A_{12}}\overline{\tilde{\Psi}}(\nabla_1\omega)\nabla_1\frac{\Psi_F}{\sin(\pi(x_1-x_2)/L)}=\frac{\pi^2}{45}N\rho^2(\rho b)^3.
%	\end{equation}
%	Combining everything we find \begin{equation}
%	E_1\leq E_0 \left(2\rho a+ \text{const.}\ N(\rho b)^3\left[ 1+ b\int v\right]\right).
%	\end{equation}
%	\subsubsection{A remark about the hard core potential}
%	Notice that it appears that we cannot deal with the hard core case. However, in the above calculation we threw away the term $ \int_{A_{12}\setminus B_{12}} \sum_{2\leq i<j}^{N}v_{ij}\abs{\Psi}^2 $. Adding this back in, we get the error $ \binom{N}{2}\int_{B_{12}} \sum_{2\leq i<j}^{N}v_{ij}\abs{\tilde{\Psi}}^2 $ instead of $ \binom{N}{2}\int_{A_{12}} \sum_{2\leq i<j}^{N}v_{ij}\abs{\tilde{\Psi}}^2 $. In doing so, we immediately see that in the presence of a hard core potential wall, $ \tilde{\Psi} $ is zero, whenever two coordinates are within the hard core. Thus we may replace $ v_{ij} $ by $ \tilde{v}_{ij} $ which is zero whenever $ \abs{x_i-x_j} $ is within the range of the hard core. Thus our result, generalizes to the case of a hard core, plus an integrable potential.
%	\subsubsection{Calculating $ E_2 $}
%	Recall that $ E_2=E_2^{(1)}+E_2^{(2)} $ with \begin{equation}
%	\begin{aligned}
%	E_2^{(1)}&=\binom{N}{2}2N\int_{A_{12}\cap A_{13}}\sum_{i=1}^{N}\abs{\nabla_i\Psi_F}^2\\ E_2^{(2)}&=\binom{N}{2}\binom{N-2}{2}\int_{A_{12}\cap A_{34}}\sum_{i=1}^{N}\abs{\nabla_i\Psi_F}^2
%	\end{aligned}
%	\end{equation}
%	To estimate these, we first split them in two terms each and use partial integration. Consider first $ E_2^{(1)} $: 
%	\begin{equation}
%	\begin{aligned}
%	E_2^{(1)}&=\binom{N}{2}2N\int_{A_{12}\cap A_{13}}\sum_{i=1}^{N}\abs{\nabla_i\Psi_F}^2\\
%	&=\binom{N}{2}2N\left(\int_{A_{12}\cap A_{13}}\abs{\nabla_1\Psi_F}^2+2\int_{A_{12}\cap A_{13}}\abs{\nabla_2\Psi_F}^2\right)+\binom{N}{2}2N\int_{A_{12}\cap A_{13}}\sum_{i=4}^{N}\abs{\nabla_i\Psi_F}^2
%	\end{aligned}
%	\end{equation}
%	For the second term, we can perform partial integration directly, in order to obtain \begin{equation}
%	\begin{aligned}
%		\binom{N}{2}&2N\int_{A_{12}\cap A_{13}}\sum_{i=3}^{N}\abs{\nabla_i\Psi_F}^2=\binom{N}{2}2N\int_{A_{12}\cap A_{13}}\sum_{i=3}^{N}\overline{\Psi_F}(-\Delta_i\Psi_F)\\
%		&\leq E_0 N^3\int_{A_{12}\cap A_{23}}\abs{\Psi_F}^2-N^3\int_{A_{12}\cap A_{23}}\sum_{i=1}^{3}\overline{\Psi_F}(-\Delta_i\Psi_F)\\&\leq 2E_0\int_{[0,L]}\int_{[x_2-b,x_2+b]}\int_{x_2-b,x_2+b}\gamma^{(3)}(x_1,x_2,x_3)\diff x_3\diff x_1\diff x_2-N^3\int_{A_{12}\cap A_{23}}\sum_{i=1}^{3}\overline{\Psi_F}(-\Delta_i\Psi_F)
%	\end{aligned}
%	\end{equation}
%	Changing variable $ y_1=x_1-x_2 $, $ y_3=x_3-x_2 $ and using translational invariance, we find \begin{equation}
%	\begin{aligned}
%	2E_0 L \int_{[-b,b]}\int_{[-b,b]}\gamma^{(3)}(y_1,0,y_3)\diff y_1\diff y_3&\approx 8E_0 L\frac{\pi^6}{135}\rho^9  \int_{[-b,b]}\int_{[-b,b]}y_1^4y_3^2\diff y_1\diff y_3\\
%	&=E_0 N\frac{8\pi^6}{15\cdot 135}(b\rho)^8.
%	\end{aligned}
%	\end{equation}
%	Using Wick's theorem, we find that to leading order in $ L\rho $, $ \abs{x_1-x_2} $, $ \rho\abs{x_2-x_3} $, and $ \rho\abs{x_1-x_3} $ we have \begin{equation}
%	\left(\partial_{x_1}\partial_{y_1}\gamma^{(3)}(x_1,x_2,x_3;y_1,y_2,y_3)\right)\Bigg\vert_{y=x}=\frac{\pi^6}{135}\rho^9(x_2-x_3)^2\left((x_1-x_3)+(x_1-x_2)\right)^2
%	\end{equation}
%	and 
%	\begin{equation}
%	\left(\partial_{y_1}^2\gamma^{(3)}(x_1,x_2,x_3;y_1,y_2,y_3)\right)\Bigg\vert_{y=x}=\frac{2\pi^6}{135}\rho^9(x_1-x_2)(x_1-x_3)(x_2-x_3)^2
%	\end{equation}
%	Thus we find \begin{equation}
%	\binom{N}{2}2N\int_{A_{12}\cap A_{13}}\sum_{i=1}^{3}\left(\abs{\nabla_i\Psi_F}^2-\overline{\Psi_F}(-\Delta_i\Psi_F)\right)\leq \tilde{C}_1 \frac{\pi^2}{3}\rho^9 L b^6=\tilde{C}_1E_0 (b\rho)^6.
%	\end{equation}
%	Collecting everything we find \begin{equation}
%	E_2^{(1)}\leq \text{const. }E_0(\rho b)^6(1+N(\rho b)^2).
%	\end{equation}
%	To estimate $ E_2^{(2)} $ use integration by parts\begin{equation}
%	\begin{aligned}
%	E_2^{(2)}&=\binom{N}{2}\binom{N-2}{2}\int_{A_{12}\cap A_{34}} \left(4\abs{\nabla_1\Psi_F}^2+\sum_{i=5}^{N}\abs{\nabla_i\Psi_F}^2\right)\\
%	&=\binom{N}{2}\binom{N-2}{2}\left(4\int_{\abs{x_3-x_4}<b}\left[\overline{\Psi_F}\nabla_1\Psi_F\right]_{x_1=x_2-b}^{x_1=x_2+b} +\int_{A_{12}\cap A_{34}} \sum_{i=1}^{N}\overline{\Psi_F}(-\Delta_i\Psi_F)\right)\\
%	&=4\int_{x_2\in[0,L]}\int_{\abs{x_3-x_4}<b}\left[\partial_{y_1}\gamma^{(4)}(x_1,x_2,x_3,x_4;y_1,y_2,y_3,y_4)\bigg\vert_{y_1=x_1}\right]_{x_1=x_2-b}^{x_1=x_2+b}+E_0\int_{A_{12}\cap A_{34}}\gamma^{(4)}(x_1,..,x_4).
%	\end{aligned}
%	\end{equation}
%	Using translational invariance, we have \begin{equation}
%	\begin{aligned}
%		\left[\partial_{y_1}\gamma^{(4)}(x_1,x_2,x_3,x_4;y_1,y_2,y_3,y_4)\bigg\vert_{y_1=x_1}\right]_{x_1=x_2-b}^{x_1=x_2+b}&=\left[\partial_{y}\gamma^{(4)}(x,0,x_3,x_4;y,0,x_3,x_4)\bigg\vert_{y=x}\right]_{x=-b}^{x=b} \\
%		&=2b\partial_x\left[\partial_{y}\gamma^{(4)}(x,0,x_3,x_4;y,0,x_3,x_4)\bigg\vert_{y=x}\right]_{x=0}+o(b)
%	\end{aligned}
%	\end{equation}
%	We find by straigthforward computation that to leading order in $ \rho L $ and $ \abs{x_3-x_4} $ we have \begin{equation}
%	\partial_x\left[\partial_{y}\gamma^{(4)}(x,0,x_3,x_4;y,0,x_3,x_4)\bigg\vert_{y=x}\right]_{x=0}=\frac{1}{9}\pi^4\rho^8(x_3-x_4)^2
%	\end{equation}
%	and we therefore get \begin{equation}
%	4\int_{x_2\in[0,L]}\int_{\abs{x_3-x_4}<b}\left[\partial_{y_1}\gamma^{(4)}(x_1,x_2,x_3,x_4;y_1,y_2,y_3,y_4)\bigg\vert_{y_1=x_1}\right]_{x_1=x_2-b}^{x_1=x_2+b}=\frac{2}{9}\pi^2E_0 N (\rho b)^4
%	\end{equation}
%	Furthermore we find that to leading order in $ L $, $ \abs{x_1-x_2} $, and $ \abs{x_3-x_4} $ we have \begin{equation}
%	\gamma^{(4)}(x_1,x_2,x_3,x_4)\leq 9\hat{C}_2\rho^8 (x_1-x_2)^2(x_3-x_4)^2
%	\end{equation} from which it follows that \begin{equation}
%	E_0\int_{A_{12}\cap A_{34}}\gamma^{(4)}(x_1,..,x_4)\leq \hat{C}_2 E_0 N^2(\rho b)^6.
%	\end{equation}
%	Collecting all terms we find the upper bound \begin{equation}
%	E\leq E_0\left(1+2\rho a+ \text{const. }N(\rho b)^3\left(1+b\int v \right)\right)
%	\end{equation}
%	\subsubsection{Proof of bounds}
%	We will now provide more general proofs for the above performed expansions. We start by considering the $ \gamma^{(2)}(x_1,x_2;y_1,x_2) $, for which we show the following result \begin{proposition}
%		Let $\abs{x_1-x_2}<b\ll \rho^{-1}$ then \begin{equation}
%			\abs{\gamma^{(2)}(x_1,x_2;y_1,x_2)}\leq C\rho^4\abs{(x_1-x_2)(y_1-x_2)}+o(\rho^4\abs{(x_1-x_2)(y_1-x_2)}),
%		\end{equation}
%		where $ C $ is dimensionless and independt of $ \rho,\ L $ and $ N $.
%	\end{proposition}
%		\begin{proof}
%			By antisymmetry of the wave function, it is clear that $ \gamma{(2)}(x_2,x_2;x_2,x_2)=0 $. Thus by Taylor's theorem for multivariate functions we have \begin{equation}
%			\gamma^{(2)}(x_1,x_2;y_1,x_2)=c_1(x_2)\rho^3(x_1-x_2)+c_2\rho^3(x_2)(y_1-x_2)+C(x_2,N)\rho^4(x_1-x_2)(y_1-x_2)+\epsilon_T(x_1,x_2,y_1)
%			\end{equation}
%			where $ \epsilon_T(x_1,x_2,y_1) $ denotes the Taylor error term. Clearly, by antisymmetry we have $ c_1=c_2=0 $, since $ \gamma(x_2,x_2;y_1,x_2)=0 $ and vice versa. Now by dimensions and translational invariance $ C(x_2) $ can be choosen to only depend on $ N $. However, assume for contradiction that $ C(N) $ is unbounded as a function of $ N $. Then $ \partial_{x_1}\partial_{y_1}\gamma^{(2)}(x_1,0;y_1,0)\vert_{y_1,x_1=0} $ is unbounded as a function of $ N $. However, this contradicts the fact that $ \gamma^{(2)} $ by Wick's theorem is a finite sum of products of $ \gamma^{(1)} $ which all are finite and have finite derivatives. Thus we may conclude that there exist $ C>0 $ such that $ C>\abs{C(x_2,N)} $ and the result now follows from the triangle inequality.
%		\end{proof}
%		
%		Alternatively, we find the bounds by row and column expansions of the determinant in Wick's theorem. Consider \eg the following example\begin{proposition}
%			Let $ \abs{x_1-x_2}\leq b\ll \rho{-1} $ and $ \abs{x_3-x_3}\leq b\ll \rho{-1} $ then \begin{equation}
%			\rho^{(4)}(x_1,...,x_4)\leq C(x_1-x_2)^2(x_3-x_4)^2.
%			\end{equation}
%		\end{proposition}
%		\begin{proof}
%			Recall that Wick's theorem states that \begin{equation}
%			\rho^{(4)}(x_1,...,x_4)=\begin{vmatrix}
%			\gamma^{(1)}(x_1,x_1)&\gamma^{(1)}(x_1,x_2)&\gamma^{(1)}(x_1,x_3)&\gamma^{(1)}(x_1,x_4)\\
%			\gamma^{(1)}(x_2,x_1)&\gamma^{(1)}(x_2,x_2)&\gamma^{(1)}(x_2,x_3)&\gamma^{(1)}(x_2,x_4)\\
%			\gamma^{(1)}(x_3,x_1)&\gamma^{(1)}(x_3,x_2)&\gamma^{(1)}(x_3,x_3)&\gamma^{(1)}(x_3,x_4)\\
%			\gamma^{(1)}(x_4,x_1)&\gamma^{(1)}(x_4,x_2)&\gamma^{(1)}(x_4,x_3)&\gamma^{(1)}(x_4,x_4)
%			\end{vmatrix}
%			\end{equation}
%			where $ \gamma^{(1)}(x,x)=\lim\limits_{y\to x}\gamma^{(1)}(x,y)=\rho $. Now subtraction row 2 from 1 followed by subtracting column 2 from column 1 and then subtracting row 3 from row 4 followed by subtracting column 3 from column 4, we obtain denoting $ \gamma^{(1)}(x_i,x_j) $ by $ \gamma^{(1)}_{ij} $
%			\begin{equation}
%			\rho^{(4)}(x_1,...,x_4)=\begin{vmatrix}
%			\gamma^{(1)}_{11}-\gamma^{(1)}_{21}-\gamma^{(1)}_{12}+\gamma^{(1)}_{22}&\gamma^{(1)}_{12}-\gamma^{(1)}_{22}&\gamma^{(1)}_{13}-\gamma^{(1)}_{23}&\gamma^{(1)}_{14}-\gamma^{(1)}_{24}-\gamma^{(1)}_{13}+\gamma^{(1)}_{23}\\
%			\gamma^{(1)}_{21}-\gamma^{(1)}_{22}&\gamma^{(1)}_{22}&\gamma^{(1)}_{23}&\gamma^{(1)}_{24}-\gamma^{(1)}_{23}\\
%			\gamma^{(1)}_{31}-\gamma^{(1)}_{32}&\gamma^{(1)}_{32}&\gamma^{(1)}_{33}&\gamma^{(1)}_{34}-\gamma^{(1)}_{33}\\
%			\gamma^{(1)}_{41}-\gamma^{(1)}_{42}-\gamma^{(1)}_{31}+\gamma^{(1)}_{32}&\gamma^{(1)}_{42}-\gamma^{(1)}_{32}&\gamma^{(1)}_{43}-\gamma^{(1)}_{33}&\gamma^{(1)}_{44}-\gamma^{(1)}_{34}-\gamma^{(1)}_{43}+\gamma^{(1)}_{33}
%			\end{vmatrix}
%			\end{equation}
%			We then bound this by the corresponding permanent
%			\begin{equation}
%			\rho^{(4)}(x_1,...,x_4)\leq\begin{vmatrix}
%			\abs{\gamma^{(1)}_{11}-\gamma^{(1)}_{21}-\gamma^{(1)}_{12}+\gamma^{(1)}_{22}}&\abs{\gamma^{(1)}_{12}-\gamma^{(1)}_{22}}&\abs{\gamma^{(1)}_{13}-\gamma^{(1)}_{23}}&\abs{\gamma^{(1)}_{14}-\gamma^{(1)}_{24}-\gamma^{(1)}_{13}+\gamma^{(1)}_{23}}\\
%			\abs{\gamma^{(1)}_{21}-\gamma^{(1)}_{22}}& \abs{\gamma^{(1)}_{22}}&\abs{\gamma^{(1)}_{23}}&\abs{\gamma^{(1)}_{24}-\gamma^{(1)}_{23}}\\
%			\abs{\gamma^{(1)}_{31}-\gamma^{(1)}_{32}}&\abs{\gamma^{(1)}_{32}}&\abs{\gamma^{(1)}_{33}}&\abs{\gamma^{(1)}_{34}-\gamma^{(1)}_{33}}\\
%			\abs{\gamma^{(1)}_{41}-\gamma^{(1)}_{42}-\gamma^{(1)}_{31}+\gamma^{(1)}_{32}}&\abs{\gamma^{(1)}_{42}-\gamma^{(1)}_{32}}&\abs{\gamma^{(1)}_{43}-\gamma^{(1)}_{33}}&\abs{\gamma^{(1)}_{44}-\gamma^{(1)}_{34}-\gamma^{(1)}_{43}+\gamma^{(1)}_{33}}
%			\end{vmatrix}_+
%			\end{equation} Furhtermore, we use that $ \sup{\abs{\partial^k\gamma^{(1)}}}\leq \text{const. }\rho^{k+1} $, the mean value theorem, and translational invariance to obtain $ \gamma_{1j}-\gamma_{2j}\leq \text{const. } \rho^2(x_1-x_2) $ for $ j>2 $ and $ \gamma_{12}-\gamma_{22}\leq \text{const. }\rho^3(x_1-x_2)^2 $ (requires MVT twice). Notice also that for example if $ \abs{x_1-x_2}<\abs{x_3-x_4} $ we have by the MVT \begin{equation}
%			\gamma^{(1)}_{14}-\gamma^{(1)}_{24}-\gamma^{(1)}_{13}+\gamma^{(1)}_{23}= \gamma^{(1)\prime}(\xi_1-x_4)(x_1-x_2)-\gamma^{(1)\prime}(\xi_2-x_3)(x_1-x_2)=\gamma^{(1)\prime\prime}(\xi)(x_1-x_2)((x_3-x_4)+(\xi_2-\xi_1))
%			\end{equation}
%			such that $ \abs{\gamma^{(1)}_{14}-\gamma^{(1)}_{24}-\gamma^{(1)}_{13}+\gamma^{(1)}_{23}}\leq \text{const. } \rho^3 \abs{(x_1-x_2)(x_3-x_4)} $. Similarly for $ \abs{x_3-x_4}\leq \abs{x_1-x_2} $ we could have argued that $ \abs{\gamma^{(1)}_{14}-\gamma^{(1)}_{24}-\gamma^{(1)}_{13}+\gamma^{(1)}_{23}}\leq \text{const. } \rho^3 \abs{(x_1-x_2)(x_3-x_4)} $
%			Thus we get \begin{equation}
%			\rho^{(4)}\leq \text{const. } \rho^4 \begin{vmatrix}
%			\rho^2(x_1-x_2)^2&\rho^2(x_1-x_2)^2&\rho(x_1-x_2)&\rho^2(x_1-x_2)(x_3-x_4)\\
%			\rho^2(x_1-x_2)^2&1&1&\rho(x_3-x_4)\\
%			\rho(x_1-x_2)&1&1&\rho^2(x_3-x_4)^2\\
%			\rho^2(x_1-x_2)(x_3-x_4)&\rho(x_3-x_4)&\rho^2(x_3-x_4)^2&\rho^2(x_3-x_4)^2
%			\end{vmatrix}_+
%			\end{equation}
%			It is then straighforward to see that \begin{equation}
%			\rho^{(4)}\leq\text{const. } \rho^8 (x_1-x_2)^2(x_3-x_4)^2
%			\end{equation}
%		\end{proof}
%		\textbf{Remark: }Notice that for Dirichlet boundary conditions we do not have $ \gamma_{12}-\gamma_{22}\leq \text{const. }\rho^3(x_1-x_2)^2 $, nevertheless we do have \begin{equation}
%		\begin{aligned}
%			&\gamma^{(1)}_{11}-\gamma^{(1)}_{21}-\gamma^{(1)}_{12}+\gamma^{(1)}_{22}=\partial_{x_1}\gamma^{(1)}(\xi_1,x_1)(x_1-x_2)-\partial_{x_1}\gamma^{(1)}(\xi_2,x_2)(x_1-x_2)\\
%			&\qquad=\partial_{x_1}\gamma^{(1)}(\xi_2,x_1)(x_1-x_2)+\partial_{x_1}^2\gamma(\xi_3,x_1)(x_1-x_2)(\xi_1-\xi_2)-\partial_{x_1}\gamma^{(1)}(\xi_2,x_2)(x_1-x_2)\\
%			&\qquad=\partial_{x_2}\partial_{x_1}\gamma^{(1)}(\xi_2,\xi_4)(x_1-x_2)^2+\partial_{x_1}^2\gamma(\xi_3,x_1)(x_1-x_2)(\xi_1-\xi_2)
%		\end{aligned}
%		\end{equation}
%		and by triangle inequality we have \begin{equation}
%		\abs{\gamma^{(1)}_{11}-\gamma^{(1)}_{21}-\gamma^{(1)}_{12}+\gamma^{(1)}_{22}}\leq \text{const. }\rho^3(x_1-x_2)^2
%		\end{equation}
%		together with $ \gamma^{(1)}_{11}-\gamma^{(1)}_{21}\leq \text{const. } \rho^2 (x_1-x_2)$ it is clear that it holds even in the Dirichlet case that $ \rho^{(4)}\leq \text{const. }\rho^8(x_1-x_2)^2(x_3-x_4)^2 $
%		\begin{proposition}
%			Let $ \abs{x_1-x_2}<b\ll\rho^{-1} $ then \begin{equation}
%			\abs{\partial_{x_1}\partial_{y_1}\gamma^{4}(x_1,x_2,x_3,x_4,y_1,x_2,x_3,x_4)}\leq \textnormal{const. }\rho^8(x_3-x_4)^2
%			\end{equation}
%		\end{proposition}
%		\begin{proof}
%			By Wick's theorem we have \begin{equation}
%			\begin{aligned}
%			\partial_{x_1}\partial_{y_1}&\gamma^{4}(x_1,x_2,x_3,x_4,y_1,x_2,x_3,x_4)\\&=\begin{vmatrix}
%			\partial_{x_1}\partial_{y_1}\gamma^{(1)}(x_1,y_1)&\partial_{x_1}\gamma^{(1)}(x_1,x_2)&\partial_{x_1}\gamma^{(1)}(x_1,x_3)&\partial_{x_1}\gamma^{(1)}(x_1,x_4)\\
%			\partial_{y_1}\gamma^{(1)}(x_2,y_1)&\gamma^{(1)}(x_2,x_2)&\gamma^{(1)}(x_2,x_3)&\gamma^{(1)}(x_2,x_4)\\
%			\partial_{y_1}\gamma^{(1)}(x_3,y_1)&\gamma^{(1)}(x_3,x_2)&\gamma^{(1)}(x_3,x_3)&\gamma^{(1)}(x_3,x_4)\\
%			\partial_{y_1}\gamma^{(1)}(x_4,y_1)&\gamma^{(1)}(x_4,x_2)&\gamma^{(1)}(x_4,x_3)&\gamma^{(1)}(x_4,x_4)
%			\end{vmatrix}.
%			\end{aligned}
%			\end{equation}
%			Subtracting row 3 from row 4 followed by column 3 from column 4 we obtain \begin{equation}
%			\begin{aligned}
%			\partial_{x_1}\partial_{y_1}&\gamma^{4}(x_1,x_2,x_3,x_4,y_1,x_2,x_3,x_4)\\&=\begin{vmatrix}
%			\partial_{x_1}\partial_{y_1}\gamma^{(1)}_{11}&\partial_{x_1}\gamma^{(1)}_{12}&\partial_{x_1}\gamma^{(1)}_{13}&\partial_{x_1}\gamma^{(1)}_{14}-\partial_{x_1}\gamma^{(1)}_{13}\\
%			\partial_{y_1}\gamma^{(1)}_{21}&\gamma^{(1)}_{22}&\gamma^{(1)}_{23}&\gamma^{(1)}_{24}-\gamma^{(1)}_{23}\\
%			\partial_{y_1}\gamma^{(1)}_{31}&\gamma^{(1)}_{32}&\gamma^{(1)}_{33}&\gamma^{(1)}_{34}-\gamma^{(1)}_{33}\\
%			\partial_{y_1}\gamma^{(1)}_{41}-\partial_{y_1}\gamma^{(1)}_{31}&\gamma^{(1)}_{42}-\gamma^{(1)}_{32}&\gamma^{(1)}_{43}-\gamma^{(1)}_{33}&\gamma^{(1)}_{44}-\gamma^{(1)}_{34}-\gamma^{(1)}_{43}+\gamma^{(1)}_{33}
%			\end{vmatrix}.
%			\end{aligned}
%			\end{equation}
%			where $ \gamma^{(1)}_{j1} $ denotes $ \gamma^{(1)}(x_j,y_1) $ and $ \gamma^{(1)}_{ji}=\gamma^{(1)}(x_j,x_i) $ for $ i>1 $. By same arguments as above $ \abs{\gamma^{(1)}_{3j}-\gamma^{(1)}_{4j}}\leq \text{const. }\rho^2\abs{x_3-x_4} $ and $ \abs{\gamma^{(1)}_{44}-\gamma^{(1)}_{34}-\gamma^{(1)}_{43}+\gamma^{(1)}_{33}}\leq \text{const. }\rho^3(x_3-x_4)^2 $.
%			Furthermore, by the MVT we have $ \abs{\partial_{x_1}\gamma^{(1)}_{14}-\partial_{x_1}\gamma^{(1)}_{13}}\leq \text{const. }\rho^3\abs{x_3-x_4} $. Thus we obtain\begin{equation}
%			\begin{aligned}
%			&\abs{\partial_{x_1}\partial_{y_1}\gamma^{4}(x_1,x_2,x_3,x_4,y_1,x_2,x_3,x_4)}\\
%			&\qquad\leq\text{const. }\rho^4\begin{vmatrix}
%			\rho^2&\rho&\rho&\rho^2(x_3-x_4)\\
%			\rho&1&1&\rho(x_3-x_4)\\
%			\rho&1&1&\rho^2(x_3-x_4)^2\\
%			\rho^2(x_3-x_4)&\rho(x_3-x_4)&\rho^2(x_3-x_4)^2&\rho^2(x_3-x_4)^2
%			\end{vmatrix}_{+}
%			\end{aligned}
%			\end{equation}
%			from which it follows that $ \abs{\partial_{x_1}\partial_{y_1}\gamma^{4}(x_1,x_2,x_3,x_4,y_1,x_2,x_3,x_4)}\leq \text{const. }\rho^8(x_3-x_4)^2 $
%		\end{proof}
%		We present a more abstract argument for the following bounds. Notice that all deriviatives one $ \gamma^{(1)} $ are uniformly bounded by some power of $ \rho $. Thus this is also true, by Wick's theorem, for any $ \gamma^{(k)} $. Using this and the fact that $ \Psi_F $ is antisymmetric we prove\begin{proposition}
%			$ \gamma^{(4)}(x_1,x_2,x_3,x_4;y_1,x_2,x_3,x_4)\leq \textnormal{const. }\rho^8(x_1-x_2)(y_1-x_2)(x_3-x_4)^2 $
%		\end{proposition}
%		\begin{proof}
%			Recall that \begin{equation}
%			\gamma^{(4)}(x_1,x_2,x_3,x_4;y_1,x_2,x_3,x_4)=\frac{N!}{(N-4)!}\int\overline{\Psi_F(x_1,x_2,x_3,x_4,...,x_N)}\Psi_F(y_1,x_2,x_3,x_4,...,x_N)\diff x_5...\diff x_N 
%			\end{equation}
%			it is clear by antisymmetry of $ \Psi_F $ and symmetry when exchanging $ x_3 $ and $x_4 $ that by the mean value theorem used multiple times we have \begin{equation}
%			\gamma^{(4)}(x_1,x_2,x_3,x_4;y_1,x_2,x_3,x_4)\leq\text{const. }\rho^8(x_1-x_2)(y_1-x_2)(x_3-x_4)^2
%			\end{equation}
%		\end{proof}
	\subsection{Dirichlet boundary conditions}
	We provide the upper bound for \eqref{EqGroundstateE}, by using the variational principle with a suitable trial state. Since we are interested in an upper bound, we consider Dirichlet boundary conditions.
	 For $ b>R_0 $, consider the trial state	\begin{equation}
	 \Psi(x)=\begin{cases}
	 \omega(\rr(x))\frac{\tilde{\Psi}_F(x)}{\rr(x)}& \text{if }\rr(x)<b,\\
	 \tilde{\Psi}_F(x)&\text{if }\rr(x)\geq b,
	 \end{cases}
	 \end{equation}
	 where $ \omega $ is the suitably normalized solution to the two-body scattering equation, \ie $ \omega(x)=f(x)\frac{b}{f(b)} $ where $ f $ is any solution of the two-body scattering equation,  $ \tilde{\Psi}_F(x)=\abs{\Psi_F} $ is the absolute value of the free fermionic ground state, and $ \rr(x)=\min_{i<j}(\abs{x_i-x_j}) $ is uniquely defined a.e.\\
	The energy of this trial state is then\begin{equation}
	\mathcal{E}(\Psi)=\int \sum_{i=1}^{N}\abs{\nabla_i\Psi}^2+\sum_{i<j}^{N}v_{ij}\abs{\Psi}^2,
	\end{equation}
	where $ v_{ij}(x)=v(x_i-x_j) $. Since $ v $ is supported in $ B_b $ and $ \Psi=\tilde{\Psi}_F $ except in the region $ B=\{x\in\R^N \vert \rr(x)<b \} $, we may rewrite this as \begin{equation}
	\mathcal{E}(\Psi)=E_0+\int_B \sum_{i=1}^{N}\abs{\nabla_i\Psi}^2+\sum_{i<j}^{N}v_{ij}\abs{\Psi}^2-\sum_{i=1}^{N}\abs{\nabla_i\tilde{\Psi}_F}^2,
	\end{equation}
	where $ E_0=N\frac{\pi^2}{3}\rho^2(1+\mathcal{O}(1/N))\norm{\Psi}^2 $ is the ground state energy of the free Fermi gas. Using that $ v\geq0 $, symmetry of exchange of particles, and defining the set $ B_{12}=\{x\in\R^N \vert \rr(x)<b,\ \rr(x)=\abs{x_1-x_2} \}\subset A_{12}=\{x\in\R^N\vert \abs{x_1-x_2}<b\} $ which up to a set of measure zero is the intersection of $ B $ and the set $ \{\text{"1 and 2 are closest"}\} $, we find \begin{equation}
	\begin{aligned}
	\mathcal{E}(\Psi)&=E_0+\binom{N}{2}\int_{B_{12}} \sum_{i=1}^{N}\abs{\nabla_i\Psi}^2+\sum_{i<j}^{N}v_{ij}\abs{\Psi}^2-\sum_{i=1}^{N}\abs{\nabla_i\tilde{\Psi}_F}^2\\&
	=E_0+\binom{N}{2}\int_{B_{12}} \sum_{i=1}^{N}\abs{\nabla_i\tilde{\Psi}}^2+\sum_{i<j}^{N}v_{ij}\abs{\tilde{\Psi}}^2-\sum_{i=1}^{N}\abs{\nabla_i\tilde{\Psi}_F}^2\\&
	=E_0+\binom{N}{2}\int_{A_{12}} \sum_{i=1}^{N}\abs{\nabla_i\tilde{\Psi}}^2+\sum_{i<j}^{N}v_{ij}\abs{\tilde{\Psi}}^2-\sum_{i=1}^{N}\abs{\nabla_i\tilde{\Psi}_F}^2\\&\qquad
	-\binom{N}{2}\int_{A_{12}\setminus B_{12}} \sum_{i=1}^{N}\abs{\nabla_i\tilde{\Psi}}^2+\sum_{i<j}^{N}v_{ij}\abs{\tilde{\Psi}}^2-\sum_{i=1}^{N}\abs{\nabla_i\tilde{\Psi}_F}^2\\&
	\leq E_0+E_1+\binom{N}{2}\int_{A_{12}\setminus B_{12}}\sum_{i=1}^{N}\abs{\nabla_i\tilde{\Psi}_F}^2
	\end{aligned}
	\end{equation}
	where we have defined \begin{equation*}
		\tilde{\Psi}=\begin{cases}
		\omega(x_1-x_2)\frac{\tilde{\Psi}_F(x)}{\abs{x_1-x_2}}& \text{if }\abs{x_1-x_2}<b,\\
		\tilde{\Psi}_F(x)&\text{if }\abs{x_1-x_2}\geq b,
		\end{cases}
	\end{equation*} and $ E_1=\binom{N}{2}\int_{A_{12}} \sum_{i=1}^{N}\abs{\nabla_i\tilde{\Psi}}^2+\sum_{i<j}^{N}v_{ij}\abs{\tilde{\Psi}}^2-\sum_{i=1}^{N}\abs{\nabla_i\tilde{\Psi}_F}^2 $.\\
	We may estimate \begin{equation}
	\begin{aligned}
	\binom{N}{2}\int_{A_{12}\setminus B_{12}}\sum_{i=1}^{N}\abs{\nabla_i\tilde{\Psi}_F}^2&=\binom{N}{2}\left(2N\left[\int_{A_{12}\cap A_{13}}\sum_{i=1}^{N}\abs{\nabla_i\tilde{\Psi}_F}^2-\int_{B_{12}\cap A_{13}}\sum_{i=1}^{N}\abs{\nabla_i\tilde{\Psi}_F}^2\right]\right.\\
	&\qquad\qquad\left.+\binom{N-2}{2}\left[\int_{A_{12}\cap A_{34}}\sum_{i=1}^{N}\abs{\nabla_i\tilde{\Psi}_F}^2-\int_{B_{12}\cap A_{34}}\sum_{i=1}^{N}\abs{\nabla_i\tilde{\Psi}_F}^2\right]\right)\\
	&\leq \binom{N}{2}\left[2N\int_{A_{12}\cap A_{13}}\sum_{i=1}^{N}\abs{\nabla_i\tilde{\Psi}_F}^2+\binom{N-2}{2}\int_{A_{12}\cap A_{34}}\sum_{i=1}^{N}\abs{\nabla_i\tilde{\Psi}_F}^2\right]
	\end{aligned}
	\end{equation}
	Thus we find \begin{equation}\label{EqBound1}
	\mathcal{E}(\Psi)\leq E_0+E_1+E_2^{(1)}+E_2^{(2)}
	\end{equation}
	with $ E_2^{(1)}=\binom{N}{2}2N\int_{A_{12}\cap A_{13}}\sum_{i=1}^{N}\abs{\nabla_i\tilde{\Psi}_F}^2 $ and $ E_2^{(2)}=\binom{N}{2}\binom{N-2}{2}\int_{A_{12}\cap A_{34}}\sum_{i=1}^{N}\abs{\nabla_i\tilde{\Psi}_F}^2 $.\\
	We notice that since $ \tilde{\Psi}_F=\abs{\Psi_F} $ so by the diamagnetic inequality we have $ \abs{\nabla_i\tilde{\Psi}_F}^2\leq \abs{\nabla_i\Psi_F}^2 $, which implies that $ \tilde{\Psi}_F $ is in $ H^{1}(\Lambda_L) $. Furthermore, $ \Psi_F $ is $ C^{1}(\Lambda_L) $ with a zero set $ \{\Psi_F=0\} $ of measure zero, so $ \abs{\nabla_i\tilde{\Psi}_F}^2 $ and $ \abs{\nabla_i\Psi_F}^2 $ are equal a.e. But then $ \abs{\nabla_i\tilde{\Psi}_F}=\abs{\nabla_i\Psi_F} $ as $ L^{2}(\Lambda_L) $ functions. Hence we may replace $ \tilde{\Psi}_F $ with $ \Psi_F $ in all integrals above.
	\subsection{The free Fermi ground state}
	We now construct the free Fermi ground state. The Dirichlet eigenstates of the Laplacian are $ \phi_j(x)=\sqrt{2/L}\sin(\pi j x/L) $. Thus the free Fermi ground state is \begin{equation}
	\Psi_F(x)=\det\left(\phi_j(x_i)\right)_{i,j=1}^{N}=\sqrt{\frac{2}{L}}^N\left(\frac{1}{2i}\right)^N\begin{vmatrix}
	\euler{iy_1}-\euler{-iy_1}&\euler{i2y_1}-\euler{-i2y_1}&\ldots&\euler{iNy_1}-\euler{-iNy_1}\\
	\euler{iy_2}-\euler{-iy_2}&\euler{i2y_2}-\euler{-i2y_2}&\ldots&\euler{iNy_2}-\euler{-iNy_2}\\
	\vdots&\vdots&\ddots&\vdots\\
	\euler{iy_N}-\euler{-iy_N}&\euler{i2y_N}-\euler{-i2y_N}&\ldots&\euler{iNy_N}-\euler{-iNy_N}
	\end{vmatrix},
	\end{equation}
	where we defined $ y_i=\frac{\pi}{L}x_i $. Defining $ z=\euler{iy} $ and using the relation $ (x^n-y^n)/(x-y)=\sum_{k=0}^{n-1}x^ky^{n-1-k} $ we find\begin{equation}
	\Psi_F(x)=\sqrt{\frac{2}{L}}^N\left(\frac{1}{2i}\right)^N\prod_{i=1}^{N}(z_i-z_i^{-1})\begin{vmatrix}
	1&z_1+z_1^{-1}&\ldots&\sum_{k=0}^{N-1}z_1^{2k-N+1}\\
	1&z_2+z_2^{-1}&\ldots&\sum_{k=0}^{N-1}z_2^{2k-N+1}\\
	\vdots&\vdots&\ddots&\vdots\\
	1&z_N+z_N^{-1}&\ldots&\sum_{k=0}^{N-1}z_N^{2k-N+1}\\
	\end{vmatrix}.
	\end{equation}
	Notice now that $ (z+z^{-1})^n=\sum_{k=0}^{n}\binom{n}{k}z^{2k-n} $.
	Now for $ i $ from $ 1 $ to $ N-1 $ we add $ \left(\binom{N-1}{i}-\binom{N-1}{i-1}\right) $ times column $ N-i $ to column $ N $. This of course does not change the determinant, and we find \begin{equation}
	\Psi_F(x)=\sqrt{\frac{2}{L}}^N\left(\frac{1}{2i}\right)^N\prod_{i=1}^{N}(z_i-z_i^{-1})\begin{vmatrix}
	1&z_1+z_1^{-1}&\ldots&\sum_{k=0}^{N-2}z_1^{2k-N+1}&(z_1+z_1^{-1})^{N-1}\\
	1&z_2+z_2^{-1}&\ldots&\sum_{k=0}^{N-2}z_2^{2k-N+1}&(z_2+z_2^{-1})^{N-1}\\
	\vdots&\vdots&\ddots&\vdots&\vdots\\
	1&z_N+z_N^{-1}&\ldots&\sum_{k=0}^{N-2}z_N^{2k-N+1}&(z_N+z_N^{-1})^{N-1}\\
	\end{vmatrix}.
	\end{equation}
	Now for $ i=1 $ to $ N-2 $ we add $ \left(\binom{N-2}{i}-\binom{N-2}{i-1}\right) $ times column $ N-1-i $ to column $ N-1 $, continue this process, \ie for $ j=3 $ to $ N $: for $ i=1 $ to $ N-j $ add  $ \left(\binom{N-j}{i}-\binom{N-j}{i-1}\right) $ times column $ N-1-i $ to column $ N-j+1 $. Then we obtain \begin{equation}
	\Psi_F(x)=\sqrt{\frac{2}{L}}^N\left(\frac{1}{2i}\right)^N\prod_{i=1}^{N}(z_i-z_i^{-1})\begin{vmatrix}
	1&z_1+z_1^{-1}&(z_1+z_1^{-1})^2&\ldots&(z_1+z_1^{-1})^{N-1}\\
	1&z_2+z_2^{-1}&(z_2+z_2^{-1})^2&\ldots&(z_2+z_2^{-1})^{N-1}\\
	\vdots&\vdots&\vdots&\ddots&\vdots\\
	1&z_N+z_N^{-1}&(z_N+z_N^{-1})^2&\ldots&(z_N+z_N^{-1})^{N-1}\\
	\end{vmatrix}.
	\end{equation}
	The determinant is recognized as a Vandermonde determinant and thus we have \begin{equation}
	\begin{aligned}
	\Psi_F(x)&=\sqrt{\frac{2}{L}}^N\left(\frac{1}{2i}\right)^N\prod_{k=1}^{N}(z_k-z_k^{-1})\prod_{i<j}^{N}\left((z_i+z_i^{-1})-(z_j+z_j^{-1})\right)\\
	&=2^{\binom{N}{2}}\sqrt{\frac{2}{L}}^N\prod_{k=1}^{N}\sin\left(\frac{\pi}{L}x_k\right)\prod_{i<j}^{N}\left[\cos\left(\frac{\pi}{L}x_i\right)-\cos\left(\frac{\pi}{L}x_j\right)\right]\\
	&=-2^{\binom{N}{2}+1}\sqrt{\frac{2}{L}}^N\prod_{k=1}^{N}\sin\left(\frac{\pi}{L}x_k\right)\prod_{i<j}^{N}\sin\left(\frac{\pi(x_i-x_j)}{2L}\right)\sin\left(\frac{\pi(x_i+x_j)}{2L}\right)
	.
	\end{aligned}
	\end{equation}

	\subsubsection{Reduced density matrices}
	We compute the one-particle reduced density matrix of the free Fermi ground state with Dirichlet b.c. in the usual way\begin{equation}
	\begin{aligned}
	&\gamma^{(1)}(x,y)=\frac{2}{L}\sum_{j=1}^{N}\sin\left(\frac{\pi}{L}jx\right)\sin\left(\frac{\pi}{L} jy\right)\\
	&=\frac{\cos\left(\pi\left[ \left(\rho+\frac{1}{L}\right)x+\rho y\right]\right)-\cos\left(\pi \left[\left(\rho+\frac{1}{L}\right)x-\rho y\right]\right)-\cos\left(\pi \left[\left(\rho+\frac{1}{L}\right)y+\rho x\right]\right)+\cos\left(\pi \left[\left(\rho+\frac{1}{L}\right)y-\rho x\right]\right)}{4L\sin\left(\frac{\pi}{2L}(x-y)\right)\sin\left(\frac{\pi}{2L}(x+y)\right)}\\
	&=\frac{\sin\left(\pi\left(\rho+\frac{1}{2L}\right)(x-y)\right)}{2L\sin\left(\frac{\pi}{2L}(x-y)\right)}-\frac{\sin\left(\pi\left(\rho+\frac{1}{2L}\right)(x+y)\right)}{2L\sin\left(\frac{\pi}{2L}(x+y)\right)}.
	\end{aligned}
	\end{equation}
	Of course Wick's theorem applies to comupte a general $ n $-particle reduced matrix.
	\subsubsection{Taylor's theorem}
	We notice now that \begin{equation}
	\gamma^{(1)}(x,y)=\frac{\pi}{L}\left(D_{N}\left(\pi\frac{x-y}{L}\right)+D_{N}\left(\pi\frac{x+y}{L}\right)\right),
	\end{equation}
	where $ D_N(x)=\frac{1}{2\pi}\sum_{k=-N}^{N}\euler{ikx}=\frac{\sin((N+1/2)x)}{2\pi\sin(x/2)} $ is the Dirichlet kernel. One obvious consequence is that $ \abs{\partial_{x}^{k_1}\partial_{y}^{k_2}\gamma^{(1)}(x,y)}\leq \frac{1}{\pi}(2N)^{k_1+k_2+1}\left(\frac{\pi}{L}\right)^{k_1+k_2+1}=\pi^{k_1+k_2}(2\rho)^{k_1+k_2+1} $. This bound will allow us to Taylor expand any $ \gamma^{(k)} $, as all derivatives are uniformly bounded by a constant times some power of $ \rho $. In fact the relevant power of $ \rho $ can be directly obtained from dimensional analysis. Alternatively Taylor expanding may be thought of a using the mean value theorem mulitple times.
	\subsection{Some useful bounds}
	\begin{lemma}\label{Lemma rho2 bound}
		$ \rho^{(2)}(x_1,x_2)\leq\left(\frac{\pi^2}{3}\rho^4+f(x_2)\right)(x_1-x_2)^2+\mathcal{O}(\rho^6(x_1-x_2)^4) $ with $ \int f(x_2)\diff x_2\leq \text{ const. }\rho^3\log(N) $.
		\end{lemma}
		\begin{proof}
			Notice that with periodic b.c. we have by translation invariance $ \rho^{(2)}_{\text{per}}(x_1-x_2)=\frac{\pi^2}{3}\rho^4(x_1-x_2)^2+\mathcal{O}(\rho^4(x_1-x_2)^4) $. Furhtermore, we have $ \gamma^{(1)}(x_1,x_2)-\rho^{(1)}\left((x_1+x_2)/2\right)=\gamma_{\text{per},\ (\rho+1/(2L))}^{(1)}(x_1,x_2)-\rho $. Now by Wick's theorem we find \begin{equation}
			\rho^{(2)}(x_1,x_2)=\rho^{(1)}(x_1)\rho^{(1)}(x_2)-\gamma^{(1)}(x_1,x_2)\gamma^{(1)}(x_2,x_1).
			\end{equation}
			Using that $ \gamma^{(1)} $ is symmetric, and that \begin{equation}
			\begin{aligned}
			\rho^{(1)}(x_1)=\rho^{(1)}((x_1+x_2)/2)+\rho^{(1)\prime}((x_1+x_2)/2)\frac{x_1-x_2}{2}+\frac{1}{2}\rho^{(1)\prime\prime}((x_1+x_2)/2)\left(\frac{x_1-x_2}{2}\right)^2\\+\mathcal{O}(\rho^4(x_1-x_2)^3)\\
			\end{aligned}
			\end{equation}
			\begin{equation}
			\begin{aligned}
			\rho^{(1)}(x_2)=\rho^{(1)}((x_1+x_2)/2)+\rho^{(1)\prime}((x_1+x_2)/2)\frac{x_2-x_1}{2}+\frac{1}{2}\rho^{(1)\prime\prime}((x_1+x_2)/2)\left(\frac{x_1-x_2}{2}\right)^2\\+\mathcal{O}(\rho^4(x_1-x_2)^3)
			\end{aligned}
			\end{equation}
			we see that \begin{equation}
			\begin{aligned}
			\rho^{(2)}(x_1,x_2)\leq\rho^{(1)}((x_1+x_2)/2)^2-\gamma^{(1)}(x_1,x_2)^2-\left[\rho^{(1)\prime}((x_1+x_2)/2)\right]^2\left(\frac{x_1-x_2}{2}\right)^2\\+\rho^{(1)}((x_1+x_2)/2)\frac{1}{2}\rho^{(1)\prime\prime}((x_1+x_2)/2)\left(\frac{x_1-x_2}{2}\right)^2+\mathcal{O}(\rho^6(x_1-x_2)^4)
			\end{aligned}
			\end{equation}
			Notice that $ \mathcal{O}(\rho^5(x_1-x_2)^3) $ terms must cancel due to symmetry.\\
			Now use the fact that $ 0\leq\rho^{(1)}\leq 2\rho $, and that $ \rho^{(1)\prime}:[0,L]\to \R $, and that $\int_{[0,L]}\abs{\rho^{(1)\prime\prime}}\leq \text{const. }\rho^2\log(N) $, which follows from the bound on Dirichlet's kernel $ \norm{D_N^{(k)}}_{L^1([0,2\pi])}\leq \text{const. }N^{k}\log(N) $, to conclude that
			\begin{equation}
			\begin{aligned}
			\rho^{(2)}(x_1,x_2)\leq\rho^{(1)}((x_1+x_2)/2)^2-\gamma^{(1)}(x_1,x_2)^2+g_1(x_1+x_2)(x_1-x_2)^2+\mathcal{O}(\rho^6(x_1-x_2)^4),
			\end{aligned}
			\end{equation}
			for some function $ g_1 $ satisfying $ \int_{[0,L]}g_1\leq \text{const. }\rho^3\log(N)$.
			Furthermore, notice that 
			\begin{equation}
			\begin{aligned}
			\rho^{(1)}((x_1+x_2)/2)^2-\gamma^{(1)}(x_1,x_2)^2=(\rho^{(1)}((x_1+x_2)/2)-\gamma^{(1)}(x_1,x_2))(\rho^{(1)}((x_1+x_2)/2)+\gamma^{(1)}(x_1,x_2))\\
			=\left[\rho-\gamma_{\text{per},\ (\rho+1/(2L))}^{(1)}(x_1,x_2)\right]\left[\rho-\gamma_{\text{per},\ (\rho+1/(2L))}^{(1)}(x_1,x_2)+2\rho^{(1)}((x_1+x_2)/2)\right]\\
			=\left[\rho-\gamma_{\text{per},\ (\rho+1/(2L))}^{(1)}(x_1,x_2)\right]^2+2\left[\rho-\gamma_{\text{per},\ (\rho+1/(2L))}^{(1)}(x_1,x_2)\right]\rho^{(1)}((x_1+x_2)/2)\\
			=\mathcal{O}(\rho^6(x_1-x_2)^4)+2\left(\frac{\pi^2}{6}\rho^3(x_1-x_2)^2+\mathcal{O}(\rho^5(x_1-x_2)^4)\right)\left(\rho+\frac{1}{2L}-\frac{\pi}{L}D_{N}((x_1+x_2)/(2L))\right)\\
			\leq\frac{\pi^2}{3}\rho^4(x_1-x_2)^2+g_2(x_1-x_2)(x_1-x_2)^2+\mathcal{O}(\rho^6(x_1-x_2)^4)
			\end{aligned}
			\end{equation}
			where we have choosen $ g_2(x)=\frac{\pi^2}{3}\rho^3\left(\frac{1}{2L}+\abs{\frac{\pi}{L}D_N(x/(2L))} \right) $ which clearly satifies $  \int_{[0,L]} g_2\leq \text{const. } \rho^3 \log(N) $.
			Thus we conclude that \begin{equation}
			\rho^{(2)}(x_1,x_2)\leq\left(\frac{\pi^2}{3}\rho^4+f(x_2)\right)(x_1-x_2)^2+\mathcal{O}(\rho^6(x_1-x_2)^4)
			\end{equation}
			with $ f=g_1+g_2 $, satifying $ \int_{[0,L]} f\leq \text{const. } \rho^3 \log(N) $
			\end{proof}
			\begin{lemma}\label{LemmaDensityBounds}
				We have the following bounds\begin{equation}
				\begin{aligned}
				\rho^{(3)}(x_1,x_2,x_3)&\leq \textnormal{const. }\rho^9(x_1-x_2)^2(x_2-x_3)^2(x_1-x_3)^2,\quad &\textnormal{on }A_{12}\cap A_{23}.\\
				\rho^{(4)}(x_1,x_2,x_3,x_4)&\leq \textnormal{const. }\rho^8(x_1-x_2)^2(x_3-x_4)^2,\quad &\textnormal{on }A_{12}\cap A_{34}.\\
				\sum_{i=1}^{2}\partial_{y_i}^2\gamma^{(2)}(x_1,x_2,y_1,y_2)\rvert_{y=x}&\leq \textnormal{const. } \rho^{6}(x_1-x_2)^2, &\textnormal{on }A_{12}.\\
				\abs{\partial_{y_1}^2\left(\frac{\gamma^{(2)}(x_1,x_2,y_1,y_2)}{y_1-y_2}\right)\Bigg\rvert_{y=x}}&\leq \textnormal{const. } \rho^{6}\abs{x_1-x_2}, &\textnormal{on }A_{12}.\\
				\sum_{i=1}^{2}(-1)^{i-1}\partial_{y_i}\left(\frac{\gamma^{(2)}(x_1,x_2,y_1,y_2)}{y_1-y_2}\right)\Bigg\rvert_{y=x}&\leq \textnormal{const. } \rho^{6}(x_1-x_2)^2, &\textnormal{on }A_{12}.
				\end{aligned}
				\end{equation}
			\end{lemma}
			\begin{proof}
				The bounds follows straightforwardly from Taylor's theorem and symmetries of the left-hand sides. As an example, consider $ \sum_{i=1}^{2}\partial_{y_i}^2\gamma^{(2)}(x_1,x_2,y_1,y_2)\rvert_{y=x} $. Notice first that $ \sum_{i=1}^{2}\partial_{y_i}^2\gamma^{(2)}(x_1,x_2,y_1,y_2) $ is anti-symmetric in $ (x_1,x_2) $ and in $ (y_1,y_2) $. As we have previously argued that all derivatives of $ \gamma^{(n)} $ are bounded by a constant times $ \rho^k $ for some $ k\in\mathbb{N} $, we can clearly Taylor expand $ \gamma^{(2)} $. Taylor expanding $ x_1 $ around $ x_2 $ and similarly $ y_1 $ around $ y_2 $ we see by the anti-symmetry that $ \sum_{i=1}^{2}\partial_{y_i}^2\gamma^{(2)}(x_1,x_2,y_1,y_2)\leq \text{const. }\rho^6(x_1-x_2)(y_1-y_2) $, where the power of $ \rho $ can be found by simple dimensional analysis.
			\end{proof}
			\begin{lemma} \label{LemmaDensityBounds2}
				We have the following bounds
				\begin{equation}
				\begin{aligned}
				\sum_{i=1}^{3}\left(\partial_{x_i}\partial_{y_i}\gamma^{(3)}(x_1,x_2,x_3;y_1,y_2,y_3)\right)\Bigg\vert_{y=x}&\leq \textnormal{const. }\rho^9(x_2-x_3)^2(x_1-x_2)^2,\quad &\textnormal{on }A_{12}\cap A_{23}.\\
				\abs{\sum_{i=1}^{3}\left(\partial_{y_i}^2\gamma^{(3)}(x_1,x_2,x_3;y_1,y_2,y_3)\right)\Bigg\vert_{y=x}}&\leq\textnormal{const. }\rho^9(x_1-x_2)^2(x_2-x_3)^2,\quad &\textnormal{on }A_{12}\cap A_{23}.\\
				\abs{\left[\partial_{y}\gamma^{(4)}(x_1,x_2,x_3,x_4;y,x_2,x_3,x_4)\bigg\vert_{y=x_1}\right]_{x_1=x_2-b}^{x_1=x_2+b}}&\leq \textnormal{const. }\rho^8b(x_3-x_4)^2 ,\quad &\textnormal{on }A_{12}\cap A_{34}.
				\end{aligned}
				\end{equation}
			\end{lemma}
			\begin{proof}
				The proof follows straigthforwardly from Taylor's theorem and symmetries of the left-hand sides. 
			\end{proof}
			\begin{remark}
				Notice that the lemmas \ref{LemmaDensityBounds} and \ref{LemmaDensityBounds2} are not in any way optimal bounds, and many of them can indeed easily be improved by same proof technique using more symmetries. However, they are sufficient for our needs.
			\end{remark}
	\subsection{Estimating $ E_1 $}
		Recall the definition \begin{equation}
		E_1=\binom{N}{2}\int_{A_{12}} \sum_{i=1}^{N}\abs{\nabla_i\tilde{\Psi}}^2+\sum_{i<j}^{N}v_{ij}\abs{\tilde{\Psi}}^2-\sum_{i=1}^{N}\abs{\nabla_i\tilde{\Psi}_F}^2
		\end{equation}
		We now prove the result \begin{lemma}\label{LemmaE1Bound}
			\begin{equation}
			E_1\leq E_0 \left(2\rho a+ \textnormal{const.}\ N(\rho b)^3\left[ 1+ \rho b^2\int v\right]\right)
			\end{equation}
		\end{lemma}
		\begin{proof}
		We estimate $ E_1 $ by splitting it in four terms $ E_1=E_1^{(1)}+E_1^{(2)}+E_1^{(3)}+E_1^{(4)} $. First we have \begin{equation}
		\begin{aligned}
		E_1^{(1)}&=2\binom{N}{2}\int_{A_{12}}\abs{\nabla_1\tilde{\Psi}}^2\\&
		=2\binom{N}{2}\int_{A_{12}}\overline{\tilde{\Psi}}\left( -\Delta_1 \tilde{\Psi} \right)+2\binom{N}{2}\int\left[\overline{\tilde{\Psi}}\nabla_1\tilde{\Psi}\right]_{x_1=x_2-b}^{x_1=x_2+b}\diff \bar{x}^1.
		\end{aligned}
		\end{equation}
		The boundary term can be explicitly calculated and we find \begin{equation}
		\begin{aligned}
		2\binom{N}{2}\int\left[\overline{\tilde{\Psi}}\nabla_1\tilde{\Psi}\right]_{x_1=x_2-b}^{x_1=x_2+b}\diff \bar{x}^1=\int\left[\frac{\omega(x_1-x_2)}{(x_1-x_2)}\partial_{x_1}\left(\frac{\omega(x_1-x_2)}{(x_1-x_2)}\right)\rho^{(2)}(x_1,x_2)\right]_{x_2-b}^{x_2+b}\diff x_2\\+\int\left[\left(\frac{\omega(x_1-x_2)}{(x_1-x_2)}\right)^2\partial_{x_1}\left(\gamma^{(2)}(x_1,x_2;y,x_2)\right)\bigg\vert_{y=x_1}\right]_{x_2-b}^{x_2+b}\diff x_2.
		\end{aligned}
		\end{equation}
		Since the continuously differentiable function $ \frac{\omega(x)}{(x_1-x_2)}=\frac{\abs{x_1-x_2}-a}{b-a}\frac{b}{(x_1-x_2)} $ for $ \abs{x_1-x_2}>b $, we see that \begin{equation}
		\partial_{x_1}\left(\frac{\omega(x_1-x_2)}{(x_1-x_2)}\right)\bigg\vert_{x=x_2\pm b}=\pm b\frac{\frac{1}{b-a}-1}{ b}=\pm\frac{a}{b^2}
		\end{equation}
		Using lemma \ref{Lemma rho2 bound}, we see that \begin{equation}
		\int\left[\frac{\omega(x_1-x_2)}{(x_1-x_2)}\partial_{x_1}\left(\frac{\omega(x_1-x_2)}{(x_1-x_2)}\right)\gamma^{(2)}(x_1,x_2)\right]_{x_2-b}^{x_2+b}\diff x_2\leq 2a N\frac{\pi^2}{3}\rho^3\left(1+\text{const. }\frac{\log(N)}{N}\right)
		\end{equation}
		
		Furthermore, we denote \begin{equation}\label{EqGammaDeriv2.}
		\begin{aligned}
		&\int\left[\left(\frac{\omega(x_1-x_2)}{(x_1-x_2)}\right)^2\partial_{x_1}\left(\gamma^{(2)}(x_1,x_2;y,x_2)\right)\bigg\vert_{y=x_1}\right]_{x_2-b}^{x_2+b}\diff x_2\\
		&\quad=\int\left[\partial_{x_1}\left(\gamma^{(2)}(x_1,x_2;y,x_2)\right)\bigg\vert_{y=x_1}\right]_{x_2-b}^{x_2+b}\diff x_2=:\kappa_1
		\end{aligned}
		\end{equation}
		Thus we have \begin{equation}
		E_1^{(1)}=\frac{\pi^2}{3}N\rho^3 (2a)+\kappa_1+2\binom{N}{2}\int_{A_{12}}\overline{\tilde{\Psi}}(-\Delta_1\tilde{\Psi})
		\end{equation}
		Another contribution to $ E_1 $ is \begin{equation}
		\begin{aligned}
		E_1^{(2)}&=-\binom{N}{2}\int_{A_{12}}\left(2\abs{\nabla_1\Psi_F}^2+\sum_{i=3}^{N}\abs{\nabla_i\Psi_F}^2\right)\\&=-\binom{N}{2}\int_{A_{12}}\sum_{i=1}^{N}\overline{\Psi_F}(-\Delta_i\Psi_F)-2\binom{N}{2}\int\left[\overline{\Psi_F}\nabla_1\Psi_F\right]_{x_1=x_2-b}^{x_1=x_2+b}\\
		&=-E_0\binom{N}{2}\int_{A_{12}}\abs{\Psi_F}^2-\underbrace{\int\left[\partial_y\gamma^{(2)}(x_1,x_2;y,x_2)\vert_{y=x_1}\right]_{x_2-b}^{x_2+b} \diff x_2}_{\kappa_1}
		\end{aligned}
		\end{equation}
		Again using lemma \ref{Lemma rho2 bound} we find \begin{equation}
		E_1^{(2)}=-\text{const. }E_0 N\rho^3b^3-\kappa_1.
		\end{equation}
		The last contributions are $ E^{(3)}_1=\binom{N}{2}\int_{A_{12}} \sum_{i<j}^{N}v_{ij}\abs{\tilde{\Psi}}^2=\binom{N}{2}\int_{A_{12}}v_{12}\abs{\tilde{\Psi}}^2+\binom{N}{2}\int_{A_{12}} \sum_{2\leq i<j}^{N}v_{ij}\abs{\tilde{\Psi}}^2 $ and $ E_1^{(4)}=\int_{A_{12}}\sum_{i=3}^{N}\abs{\nabla_i\tilde{\Psi}}^2 $.
		First we notice that \begin{equation}
		\begin{aligned}
		&\binom{N}{2}\int_{A_{12}} \sum_{2\leq i<j}^{N}v_{ij}\abs{\tilde{\Psi}}^2\\&\quad\leq \text{const. } \left(\int_{\{\abs{x_1-x_2}<b\}\cap\supp(v_{34})}v(x_3-x_4)\rho^{(4)}(x_1,x_2,x_3,x_4)\right.\\
		&\qquad\qquad\qquad\left.+\int_{\{\abs{x_1-x_2}<b\}\cap\supp(v_{23})}v(x_2-x_3)\rho^{(3)}(x_1,x_2,x_3)\right).
		\end{aligned}
		\end{equation}
		By lemma \ref{LemmaDensityBounds} we have
		\begin{equation}
		\begin{aligned}
		&\binom{N}{2}\int_{A_{12}} \sum_{2\leq i<j}^{N}v_{ij}\abs{\tilde{\Psi}}^2\\&\quad\leq \text{const. } \left(N^2(\rho b)^3\rho^3\int x^2 v(x)\diff x+N(\rho b)^3 \rho^5 \int x^4 v(x)\diff x+N(\rho b)^4\rho^4 \int x^3 v(x)\diff x\right.\\
		&\qquad \qquad \qquad \qquad\hspace{6cm}\left.+N(\rho b)^5 \rho^3 \int x^2 v(x)\diff x\right)\\
		&\quad \leq \text{const. } N^2(\rho b)^5\rho \int v=\text{const. }E_0 N (\rho b)^3 \left(\rho b^2\int v\right)
		\end{aligned}
		\end{equation}
		and then we find that \begin{equation}
		\begin{aligned}
		E_1&=E_1^{(1)}+E_1^{(2)}+E_1^{(3)}+E_1^{(4)}\\&\leq \frac{2\pi^2}{3}N\rho^3 a+2\binom{N}{2}\int_{A_{12}}\left(\overline{\tilde{\Psi}}(-\Delta_1)\tilde{\Psi}+\frac{1}{2}\sum_{i=3}^{N}\abs{\nabla_i\tilde{\Psi}}^2+\frac{1}{2}v_{12}\abs{\tilde{\Psi}}^2\right)+E_0N(\rho b)^3\text{const. }\left(1+\rho b^2 \int v\right)
		\end{aligned}
		\end{equation}
		Using the two body scattering equation this implies \begin{equation}
		\begin{aligned}
		E_1&\leq \frac{2\pi^2}{3}N\rho^3 a+2\binom{N}{2}\int_{A_{12}}\overline{\tilde{\Psi}}\omega(-\Delta_1)\frac{\Psi_F}{(x_1-x_2)}\\&\quad+2\binom{N}{2}\int_{A_{12}}\overline{\tilde{\Psi}}(\nabla_1\omega)\nabla_1\frac{\Psi_F}{(x_1-x_2)}\\
		&\quad +\binom{N}{2}\int_{A_{12}}\sum_{i=3}^{N} \overline{\tilde{\Psi}}\frac{\omega}{(x_1-x_2)}(-\Delta_i)\Psi_F
		\\&\quad+\text{const. }E_0 N (\rho b)^3 \left(1+\rho b^2\int v\right).
		\end{aligned}
		\end{equation}
		Furhtermore we have \begin{equation}
		\begin{aligned}
		&\binom{N}{2}\int_{A_{12}}\sum_{i=3}^{N} \overline{\tilde{\Psi}}\frac{\omega}{(x_1-x_2)}(-\Delta_i)\Psi_F\\&\quad=E_0\binom{N}{2}\int_{A_{12}}\left\lvert\frac{\omega}{(x_1-x_2)}\Psi_F\right\rvert^2-2\binom{N}{2}\int_{A_{12}} \overline{\tilde{\Psi}}\frac{\omega}{(x_1-x_2)}(-\Delta_1)\Psi_F,
		\end{aligned}
		\end{equation}
		so by lemma \ref{Lemma rho2 bound} it follows that
		\begin{equation}
		\binom{N}{2}\int_{A_{12}}\left\lvert\frac{\omega}{(x_1-x_2)}\tilde{\Psi}\right\rvert^2\leq b^2\int_{\{\abs{x_1-x_2}<b\}} \frac{\rho^{(2)}(x_1,x_2)}{\abs{x_1-x_2}^2}\diff x_1\diff x_2\leq  \text{const. } b^2\rho^4  L b=\text{const. }N\rho^3 b^3,
		\end{equation}
		and by lemma \ref{LemmaDensityBounds} it follows that \begin{equation}
		\begin{aligned}
		2\binom{N}{2}\int_{A_{12}} \overline{\tilde{\Psi}}\frac{\omega}{(x_1-x_2)}(-\Delta_1)\Psi_F&=\frac12\sum_{i=1}^{2}\int_{A_{12}}\abs{\frac{\omega}{x_1-x_2}}^2\left[\partial^2_{y_i}\gamma^{(2)}(x_1,x_2,y_1,y_2)\right]\Big\rvert_{y=x}\\&\leq \text{const. } N\rho^2(\rho b)^3,
		\end{aligned}
		\end{equation}
		so we find \begin{equation}
		\binom{N}{2}\int_{A_{12}}\sum_{i=3}^{N} \overline{\tilde{\Psi}}\frac{\omega}{(x_1-x_2)}(-\Delta_i)\Psi_F\leq \text{const. } E_0 N(\rho b)^3.
		\end{equation}
		Finally, again by lemma \ref{LemmaDensityBounds} \begin{equation}
		\begin{aligned}
		2\binom{N}{2}\int_{A_{12}}\overline{\tilde{\Psi}}\omega(-\Delta_1)\frac{\Psi_F}{(x_1-x_2)}&=\int_{A_{12}}\abs{\frac{\omega^2}{x_1-x_2}}\left[\partial^2_{y_1}\left(\frac{\gamma^{(2)}(x_1,x_2,y_1,y_2)}{(y_1-y_2)}\right)\right]\Big\rvert_{y=x}\\&\leq\text{const. }N\rho^2 (\rho b)^3,
		\end{aligned}
		\end{equation}
		and by using $ \Delta\omega=\frac{1}{2}v\omega\geq0 $ which implies $ 0\leq\omega'(x)\leq \omega'(b)=\frac{b}{b-a} $ for $ \abs{x}<b $, we find that \begin{equation}
		\begin{aligned}
		2\binom{N}{2}\int_{A_{12}}\overline{\tilde{\Psi}}(\nabla_1\omega)\nabla_1\left(\frac{\Psi_F}{(x_1-x_2)}\right)&\leq\frac12\sum_{i=1}^{2}\int_{A_{12}}\abs{\frac{\omega}{x_1-x_2}}(-1)^{i-1}\omega'(x_1-x_2)\partial_{y_i}\left(\frac{\gamma^{(2)}(x_1,x_2,y_1,y_2)}{y_1-y_2}\right)\\
		&\leq \text{const. }\frac{b}{b-a}N\rho^2(\rho b)^3.
		\end{aligned}
		\end{equation}
		Combining everything we get \begin{equation}
		E_1\leq E_0 \left(2\rho a+ \text{const.}\ N(\rho b)^3\left[ 1+ \rho b^2\int v\right]\right)
		\end{equation}
	\end{proof}
		\subsubsection{A remark about the hard core potential}
		Notice that it appears that we cannot deal with the hard core case. However, in the above calculation we threw away the term $ \int_{A_{12}\setminus B_{12}} \sum_{2\leq i<j}^{N}v_{ij}\abs{\Psi}^2 $. Adding this back in, we get the error $ \binom{N}{2}\int_{B_{12}} \sum_{2\leq i<j}^{N}v_{ij}\abs{\tilde{\Psi}}^2 $ instead of $ \binom{N}{2}\int_{A_{12}} \sum_{2\leq i<j}^{N}v_{ij}\abs{\tilde{\Psi}}^2 $. In doing so, we immediately see that in the presence of a hard core potential wall, $ \tilde{\Psi} $ is zero, whenever two coordinates are within the hard core. Thus we may replace $ v_{ij} $ by $ \tilde{v}_{ij} $ which is zero whenever $ \abs{x_i-x_j} $ is within the range of the hard core. Thus our result, generalizes to the case of a hard core, plus an integrable potential.
		\subsection{Estimating $ E_2 $}
		Recall that $ E_2=E_2^{(1)}+E_2^{(2)} $ with \begin{equation}
		\begin{aligned}
		E_2^{(1)}&=\binom{N}{2}2N\int_{A_{12}\cap A_{13}}\sum_{i=1}^{N}\abs{\nabla_i\Psi_F}^2\\ E_2^{(2)}&=\binom{N}{2}\binom{N-2}{2}\int_{A_{12}\cap A_{34}}\sum_{i=1}^{N}\abs{\nabla_i\Psi_F}^2.
		\end{aligned}
		\end{equation}
		We now prove the result
		\begin{lemma}\label{LemmaE2Bound}
			\begin{equation}
			E_2\leq E_0(N(\rho b)^4+N^2(\rho b)^6)
			\end{equation}
		\end{lemma}
		\begin{proof}
		To estimate $ E_2^{(1)} $ and $ E_2^{(2)} $, we first split them in two terms each and use partial integration. Consider first $ E_2^{(1)} $: 
		\begin{equation}
		\begin{aligned}
		E_2^{(1)}&=\binom{N}{2}2N\int_{A_{12}\cap A_{13}}\sum_{i=1}^{N}\abs{\nabla_i\Psi_F}^2\\
		&=\binom{N}{2}2N\left(\int_{A_{12}\cap A_{13}}\abs{\nabla_1\Psi_F}^2+2\int_{A_{12}\cap A_{13}}\abs{\nabla_2\Psi_F}^2\right)+\binom{N}{2}2N\int_{A_{12}\cap A_{13}}\sum_{i=4}^{N}\abs{\nabla_i\Psi_F}^2
		\end{aligned}
		\end{equation}
		For the second term, we can perform partial integration directly, in order to obtain \begin{equation}
		\begin{aligned}
		\binom{N}{2}&2N\int_{A_{12}\cap A_{13}}\sum_{i=4}^{N}\abs{\nabla_i\Psi_F}^2=\binom{N}{2}2N\int_{A_{12}\cap A_{13}}\sum_{i=4}^{N}\overline{\Psi_F}(-\Delta_i\Psi_F)\\
		&\leq E_0 N^3\int_{A_{12}\cap A_{23}}\abs{\Psi_F}^2-N^3\int_{A_{12}\cap A_{23}}\sum_{i=1}^{3}\overline{\Psi_F}(-\Delta_i\Psi_F)\\&\leq 2E_0\int_{[0,L]}\int_{[x_2-b,x_2+b]}\int_{[x_2-b,x_2+b]}\rho^{(3)}(x_1,x_2,x_3)\diff x_3\diff x_1\diff x_2-N^3\int_{A_{12}\cap A_{23}}\sum_{i=1}^{3}\overline{\Psi_F}(-\Delta_i\Psi_F)
		\end{aligned}
		\end{equation}
		Using that lemma \ref{LemmaDensityBounds} we find \begin{equation}
		\begin{aligned}
		2E_0\int_{[0,L]}\int_{[x_2-b,x_2+b]}\int_{[x_2-b,x_2+b]}\rho^{(3)}(x_1,x_2,x_3)\diff x_3\diff x_1\diff x_2\leq NE_0(\rho b)^6
		\end{aligned}
		\end{equation}
		Furthermore, we find by lemma \ref{LemmaDensityBounds2} \begin{equation}
		\binom{N}{2}2N\int_{A_{12}\cap A_{13}}\sum_{i=1}^{3}\left(\abs{\nabla_i\Psi_F}^2-\overline{\Psi_F}(-\Delta_i\Psi_F)\right)\leq \text{const. }\rho^9 L b^6=\text{const. }E_0 (b\rho)^6.
		\end{equation}
		Collecting everything we find \begin{equation}
		E_2^{(1)}\leq \text{const. }NE_0(\rho b)^6.
		\end{equation}
		To estimate $ E_2^{(2)} $, we use integration by parts\begin{equation}
		\begin{aligned}
		E_2^{(2)}&=\binom{N}{2}\binom{N-2}{2}\int_{A_{12}\cap A_{34}} \left(4\abs{\nabla_1\Psi_F}^2+\sum_{i=5}^{N}\abs{\nabla_i\Psi_F}^2\right)\\
		&=\binom{N}{2}\binom{N-2}{2}\left(4\int_{\abs{x_3-x_4}<b}\left[\overline{\Psi_F}\nabla_1\Psi_F\right]_{x_1=x_2-b}^{x_1=x_2+b} +\int_{A_{12}\cap A_{34}} \sum_{i=1}^{N}\overline{\Psi_F}(-\Delta_i\Psi_F)\right)\\
		&=4\int_{x_2\in[0,L]}\int_{\abs{x_3-x_4}<b}\left[\partial_{y_1}\gamma^{(4)}(x_1,x_2,x_3,x_4;y_1,x_2,x_3,x_4)\bigg\vert_{y_1=x_1}\right]_{x_1=x_2-b}^{x_1=x_2+b}+E_0\int_{A_{12}\cap A_{34}}\rho^{(4)}(x_1,..,x_4).
		\end{aligned}
		\end{equation}
		By lemma \ref{LemmaDensityBounds2} we get \begin{equation}
		4\int_{x_2\in[0,L]}\int_{\abs{x_3-x_4}<b}\left[\partial_{y_1}\gamma^{(4)}(x_1,x_2,x_3,x_4;y_1,y_2,y_3,y_4)\bigg\vert_{y_1=x_1}\right]_{x_1=x_2-b}^{x_1=x_2+b}=\text{const. }E_0 N (\rho b)^4
		\end{equation}
		Furthermore, by lemma \ref{LemmaDensityBounds2} again, it follows that \begin{equation}
		E_0\int_{A_{12}\cap A_{34}}\rho^{(4)}(x_1,..,x_4)\leq \text{const. } E_0 N^2(\rho b)^6.
		\end{equation}
	\end{proof}
		
	\subsection{Localization with Dirichlet b.c.}
	We will in this section localize in smaller boxes, in order to have better control on the error. The localization is straigthforward with Dirichlet boundary conditions, as gluing the wavefunctions for each box together is simple, since the wavefunctions vanish at the boundaries. Thus we consider the state $ \Psi_{\text{full}}=\prod_{i=1}^{M}\Psi_{\ell}(x^i_1,...,x^i_{\tilde{N}}) $, where $ (x_1^i,...,x_{\tilde{N}}^i) $ are the particles in box $ i $ and $ \ell $ is the length of each box. Of course $ \cup_{i=1}^{M}\{x_1^i,...,x_{\tilde{N}}^i\}=\{x_1,...,x_N\} $ and $ \{x_1^i,...,x_{\tilde{N}}^i\}\cap\{x_1^j,...,x_{\tilde{N}}^j\}=\emptyset $ for $ i\neq j $, such that $ M\tilde{N}=N $. The boxes are of length $ \ell=L/M-b $, and are equally spaced through out $ [0,L] $ such that they are a distance of $ b $ from each other. This is to make sure that no particle interact between boxes. Combining lemmas \ref{LemmaE1Bound} and \ref{LemmaE2Bound}, the full energy is then bounded by \begin{equation}
	E\leq M e_0\left(1+2\tilde{\rho} a + \text{const. } \tilde{N} (b\tilde{\rho})^3\left(1+\rho b^2\int v_{\text{reg}}\right)\right)/\norm{\Psi}^2
	\end{equation}
	with $ e_0=\frac{\pi^2}{3}\tilde{N}\tilde{\rho}^2(1+\text{const. }\frac{1}{\tilde{N}}) $ and $ \tilde{\rho}=\tilde{N}/\ell=\rho/(1-\frac{bM}{L})\simeq\rho(1+bM/L) $. Clearly we have $ \norm{\Psi}^2\geq 1-\int_B\abs{\Psi_F}^2\geq 1-\int_{\abs{x_1-x_2}<b}\rho^{(2)}(x_1,x_2)\geq 1-\text{const. }\tilde{N}(\rho b)^3 $, where the last inequality follows from lemma \ref{Lemma rho2 bound}.
	Thus, choosing $ M $ such that $ bM/L\ll 1 $ we have \begin{equation}
	E\leq N\frac{\pi^2}{3}\rho^2\frac{\left(1+2\rho a+\text{const. }\frac{M}{N}+\text{const. }2\rho abM/L+\text{const. }\tilde{N}(b\rho)^3\left(1+\rho b^2\int v_{\text{reg}}\right)\right)}{1-\tilde{N}(\tilde{\rho} b)^3}.
	\end{equation}
	Now in fact, we would choose $ \tilde{N}=N/M=\rho L/M\gg 1 $, \ie $ M/L\ll \rho $. It is clear that we minimize the error, by choosing $ b=R_0 $ the range of the potential. Furthermore, setting $ x=M/N $ we see that the error is \begin{equation}
	\text{const. }\left[(1+2\rho^2 ab)x+x^{-1}(b\rho)^3\left(1+\rho b^2\int v_{\text{reg}}\right)\right],
	\end{equation}
	here we used that it will turn out to be the cases that $ \tilde{N}(\rho b)^3\leq 1/2 $ so that we have\\ $ 1/(1-\tilde{N}(\rho b)^3)\leq 1+2\tilde{N}(\rho b)^3 $.
	Optmizing in $ x $ we find $ x=M/N=\frac{(b\rho)^{3/2}\left(1+\rho b^2\int v_{\text{reg}}\right)^{1/2}}{1+2\rho^2 a b}\simeq(b\rho)^{3/2}\left(1+\rho b^2\int v_{\text{reg}}\right)^{1/2} $, which gives the error \begin{equation}
	\text{const. }(R_0\rho)^{3/2}\left(1+R_0 \int v_{\textnormal{reg}}\right)^{1/2}
	\end{equation}
	Thus we arrive at the following result
	\begin{theorem}
		Let the two-body potential $ v\in L^{1}([0,L])+\textnormal{h.c.p.} $ be fixed, with two-body s-wave scattering length $ a $. Then bosonic $ N $-body ground state energy satisfies the upper bound\begin{equation}
		E\leq E_0\left(1+2\rho a + \mathcal{O}\left((R\rho)^{3/2}\left(1+\rho R^2 \int v_{\textnormal{reg}}\right)^{1/2}\right)\right),
		\end{equation}
		where $ E_0 $ is the free fermionic ground state energy.
	\end{theorem}
	here $ \textnormal{h.c.p} $ denotes the space of hard core potentials.\\
	Notice that the result \begin{equation}
	E\leq N\frac{\pi^2}{3}\left(1+2\rho a + \mathcal{O}\left((R\rho)^{3/2}\left(1+\rho R^2 \int v_{\textnormal{reg}}\right)^{1/2}\right)\right)
	\end{equation}
	holds only for $ N\geq (\rho b)^{-3/2} $, but since the free Fermi, $ E_0 $ energy also contains a correction of order $ \frac{1}{N} $, \ie $ E_0=N\frac{\pi^2}{3}\rho^2(1+\text{const. }1/N) $, the result remains true for $ N<(\rho b)^{-3/2} $
	\subsection{Periodic boundary conditions}
	Another approach, would be to prove the result with periodic boundary conditions, and thus preserve translational invariance in all calculations above. Then one may use the bound [\cite{2020JMP....61f1901M}, lemma 4 or \cite{robinson2014thermodynamic} lemma 2.1.12] \begin{equation}
	\braket{\Psi_D\lvert H^D_L \rvert \Psi_D}\leq \braket{\Psi_P\lvert H^P_{L-2d}\rvert \Psi_P}+\frac{4N}{d^2}\norm{\Psi}^2.
	\end{equation}
	In this case, we get errors from the periodic b.c. calculation plus errors $ 4N/d^2+E_0\frac{d}{L} $, where the last one comes from change in the density assuming that $ d/L\ll 1 $. Optimizing in $ d $ this gives an error of $ \text{const. }E_0 \frac{1}{N^{2/3}} $. Thus the total result after localization is \begin{equation}
	E\leq E_0\left(1+2\rho a+\text{const. }\left(\frac{M}{N}\right)^{2/3}+\text{const. }2\rho abM/L+\text{const. }\tilde{N}(b\rho)^3\left(1+\rho b^2\int v_{\text{reg}}\right)\right).
	\end{equation}
	And optimizing in $ M $ and setting $ b=R_0 $ we find\begin{equation}
	E_0\left(1+2\rho a + \mathcal{O}\left((R\rho)^{6/5}\left(1+\rho R^2 \int v_{\textnormal{reg}}\right)^{1/2}\right)\right)
	\end{equation}
	which is not quite as good, as the bound computed directly with Dirichlet b.c.

	\section{Lower bound}
	We will in this section provide a lower bound for the one dimensional dilute Bose gas. The proof is based on a reduction to a Lieb Liniger model, and thus we will first recall some known features about this model
	\subsection{The Lieb Liniger model}
	Recall that the energy in thermodynamic limit of the the Lieb Liniger model (with periodic boundary conditions), is determined by the sytem of equation ((3.3) and (3.18)--(3.20) in \cite{PhysRev.130.1605})
	\begin{align}
	E^{\rho,\ell,c=\gamma\rho}_{LL}&=N\rho^2 e(\gamma),\label{Eq1}\\
	e(\gamma)&=\frac{\gamma^3}{\lambda^3}\int_{-1}^{1}g(x)x^2\diff x,\label{Eq2}\\
	2\pi g(y)&=1+2\lambda\int_{-1}^{1}\frac{g(x)\diff x}{\lambda^2+(x-y)^2},\label{Eq3}\\
	\lambda&=\gamma\int_{-1}^{1}g(x)\diff x.\label{Eq4}
	\end{align}
	The first lemma provides a rigorous lower bound for the thermodynamic Lieb Liniger energy density. 
	\begin{lemma}[Lieb Liniger lower bound] \label{LemmaLL-LowerBound}
		Let $ \gamma>0 $, then
		\begin{equation}
		e(\gamma)\geq \frac{\pi^2}{3}\left(\frac{\gamma}{\gamma+2}\right)^2\geq \frac{\pi^2}{3}\left(1-\frac{4}{\gamma}\right).
		\end{equation}
	\end{lemma}
	\begin{proof}
		Neglecting $ (x-y)^2 $ in the denominator of \eqref{Eq3}, we see that $ g\leq \frac{1}{2\pi}+2\frac{1}{\lambda}\int_{-1}^{1}g(x)\diff x $. On the other hand $ \eqref{Eq4} $ shows that $ e(\gamma)=\frac{\int_{-1}^{1}g(x)x^2\diff x}{\left(\int_{-1}^{1}g(x)\diff x\right)^3} $. Hence we denote $ \int_{-1}^{1}g(x)\diff x=M $, and notice that we have $ g\leq \frac{1}{2\pi}\left(1+\frac{2M}{\lambda}\right) $. It is now easily verified that, $ \int_{-1}^{1}g(x)x^2\diff x $ with $ M $ fixed and $ g\leq \frac{1}{2\pi}\left(1+\frac{2M}{\lambda}\right)=\frac{1}{2\pi}\left(1+2\gamma^{-1}\right) $ is mininmized by $ g=K\chi_{[-M/(2K),M/(2K)]} $, with $ K=\frac{1}{2\pi}\left(1+\frac{2}{\gamma}\right) $. This gives us $ \int_{-1}^{1}g(x)x^2\diff x=\frac{M^3}{3K^2}$ so that we have $ e(\gamma)\geq \frac{1}{3K^2}=\frac{\pi^2}{3}\left(\frac{\gamma}{\gamma+2}\right)^2\geq \frac{\pi^2}{3}(1-\frac{4}{\gamma})$ for $ \gamma>0 $.
	\end{proof}
	The next result concerns finite volume correction to the thermodynamic limit. Since we are interested in a lower bound, we consider the Neumann boundary conditions case denoted by a superscript "$ N $".
	\begin{lemma}[Finite volume corrections]\label{LemmaLiebLinigerNeumannLowerBound}
		\begin{equation}
		E_{LL}^{N}(n,\ell,c)\geq \frac{\pi^2}{3}n\rho^2\left(1-4\rho/c-\textnormal{const. }\frac{1}{n^{2/3}}\right)
		\end{equation}
	\end{lemma}
	\begin{proof}
		By Robinsons bound \cite{robinson2014thermodynamic}, we have for any $ b>0 $ \begin{equation}
		E_{LL}^{N}(n,\ell,c)\geq E_{LL}^D(n,\ell+b,c)-\text{const. }\frac{n}{b^2}.
		\end{equation}
		Since the range of the interaction in the Lieb-Liniger model is zero, we see that $ e^D_{LL}(2^mn,2^m\ell)=\frac{1}{2^m\ell}E_{LL}^{D}(2^mn,2^m\ell) $ is a decreasing sequence. To see this, simply split the box of size $ 2^m\ell $ in two boxes of size $ 2^{m-1}\ell $, now by ignoring interactions between the boxes and using the the product state of the two $ 2^{m-1}n $-particle ground states in each box as a trial state, we see that $ E^D_{LL}(2^{m}n,2^m\ell)\leq 2E^D_{LL}(2^{m-1}n,2^{m-1}\ell)  $. Since we also have $ e^D_{LL}(2^mn,2^m\ell)\geq e_{LL}(2^mn,2^m\ell)\to e_{LL}(n/\ell) $ as $ m\to\infty $ \cite{PhysRev.130.1605}, we see that \begin{equation}
		\begin{aligned}
		E_{LL}^{N}(n,\ell,c)\geq e_{LL}(n/(\ell+b),c)(\ell+b)-\text{const. }\frac{n}{b^2}\\\geq \frac{\pi^2}{3}n\rho^2\left(1-4\rho/c-\text{const. }\left(3b/\ell-\frac{1}{\rho^2b^2}\right)\right),
		\end{aligned}
		\end{equation}
		with $ \rho=n/\ell $, where the second inequality follows from lemma \ref{LemmaLL-LowerBound}. Optmizing in $ b $ we find \begin{equation}
		E_{LL}^{N}(n,\ell,c)\geq \frac{\pi^2}{3}n\rho^2\left(1-4\rho/c-\text{const. }\frac{1}{n^{2/3}}\right).
		\end{equation}
	\end{proof}
	\subsection{Lieb Liniger reduction}
	We will in this subsection lower bound the dilute bose gas by a Lieb Liniger energy. The reduction is optained by constructing a trial state for a Lieb Liniger model i a smaller volume from the true ground state of the Bose gas.\\
		Let $ \Psi $ be the ground state of $ \mathcal{E} $, we then define $ \psi\in L^2([0,\ell-(n-1)R]^n) $ by $ \psi(x_1,x_2,...,x_n)=\Psi(x_1,R+x_2,...,(n-1)R+x_n) $ for $ x_1\leq x_2\leq...\leq x_n $ and symmetrically extended. 
	\begin{lemma}
		For any function $ \psi\in H^1(\R) $ such that $ \psi(0)=0 $ then we have\begin{equation}\label{EqSobolevIneq}
		\int_{[0,R]}\abs{\partial\psi}^2\geq \max_{[0,R]}\abs{\psi}^2/R
		\end{equation}
	\end{lemma}
	\begin{proof}
		write $ \psi(x)=\int_{0}^{x}\psi'(t)\diff t $, and find that \begin{equation}
		\abs{\psi(x)}\leq \int_{0}^{x}\abs{\psi'(t)}\diff t.
		\end{equation}
		Hence $ \max_{x\in[0,R]}\abs{\psi(x)}\leq \int_{0}^{R}\abs{\psi'(t)}\diff t\leq \sqrt{R}\left(\int\abs{\psi'(t)}^2\diff t\right)^{1/2} $
	\end{proof}
	We can estimate the norm loss in the following way
	\begin{equation}\label{EqNormBoundBij}
	\begin{aligned}
	\braket{\psi|\psi}=1-\int_{B}\abs{\Psi}^2\geq 1-\sum_{i<j}\int_{B_{ij}}\abs{\Psi}^2
	\end{aligned}
	\end{equation}
	where $ B=\{x\in\R^n\vert \min_{i,j}\abs{x_i-x_j}<R \} $, and $ B_{ij}=\{x\in\R^n \vert \mathfrak{r}_i(x)=\abs{x_i-x_j}<R \} $. To give a good bound on the right-hand side, we need the following lemma
	\begin{lemma}\label{LemmaNormLoss}
		Let $ \psi $ be defined as above, then \begin{equation}
		1-\braket{\psi|\psi}\leq\textnormal{const. } \left(R^2\sum_{i<j}\int_{B_{ij}}\abs{\partial_i \Psi}^2+R(R-a)\sum_{i<j}\int v_{ij} \abs{\Psi}^2\right)
		\end{equation}
	\end{lemma}
	\begin{proof}
		Notice that by \eqref{EqSobolevIneq} we have for any $ \phi\in H^1 $, \begin{equation}
		\abs{\abs{\phi(x)}-\abs{\phi(x')}}^2\leq\abs{\phi(x)-\phi(x')}^2\leq R\left(\int_{[0,R]}\abs{\partial \phi}^2\right),
		\end{equation}
		for $ x,x'\in[0,R] $. Furhtermore, 
		\begin{equation}
		\abs{\phi(x)}^2-\abs{\phi(x')}^2=\left(\abs{\phi(x)}-\abs{\phi(x')}\right)^2+2\left(\abs{\phi(x)}-\abs{\phi(x')}\right)\abs{\phi(x')}\leq 2\left(\abs{\phi(x)}-\abs{\phi(x')}\right)^2+\abs{\phi(x')}^2
		\end{equation}
		So for 
		It follows that \begin{equation}
		\max_{x\in[0,R]}\abs{\phi}^2\leq 2R\int_{[0,R]}\abs{\partial \phi}^2+2\min_{x'\in[0,R]}\abs{\phi(x')}^2
		\end{equation}
		Viewing $ \Psi $ as a function of $ x_i $ we have \begin{equation}
		2\min_{\mathfrak{r}_i(x)=\abs{x_i-x_j}<R}\abs{\Psi}^2\geq \max_{\mathfrak{r}_i(x)=\abs{x_i-x_j}<R}\abs{\Psi}^2-4R\left(\int_{{\mathfrak{r}_i(x)=\abs{x_i-x_j}<R}}\abs{\partial_i \Psi}^2\right).
		\end{equation}
		Hence we find \begin{equation}
		\begin{aligned}
		&2\sum_{i<j}\int v_{ij} \abs{\Psi}^2\geq 2\sum_{i<j} \int_{B_{ij}} v_{ij} \abs{\Psi}^2 \\&\geq \left(\int v\right)\sum_{i< j}\int\left(\max_{B'_{ij}}\abs{\Psi}^2-4R\left(\int_{B'_{ij}}\abs{\partial_i\Psi}^2\diff x_i\right)\right)\diff \bar{x}^i\\
		&\geq \frac{4}{R-a}\sum_{i< j}\left(\frac{1}{2R}\int_{B_{ij}}\abs{\Psi}^2-4R\int_{B_{ij}}\abs{\partial_i\Psi}^2\right)
		\end{aligned}
		\end{equation}
		where $ B_{ij}=\{x\in \R^n \vert \mathfrak{r}_i(x)=\abs{x_i-x_j}<R \} $ and $ B'_{ij}=\{x_i\in \R \vert \mathfrak{r}_i(x)=\abs{x_i-x_j}<R \} $. Now, by \eqref{EqNormBoundBij}, we see that
		\begin{equation}
		1-\braket{\psi|\psi}\leq \text{const. } \left(R^2\sum_{i<j}\int_{B_{ij}}\abs{\partial_i \Psi}^2+R(R-a)\int\sum_{i<j} v_{ij} \abs{\Psi}^2\right)
		\end{equation}
	\end{proof}
	Choosing $ R\geq 2\abs{a} $ we have $ \braket{\psi|\psi}\geq 1- \text{const. }R^2 E $.\\
	
	
	
	The following lemma will also be useful \begin{lemma}[Dyson]\label{LemmaDyson} Let $ R>R_0=\textnormal{range}(v) $ and $ \varphi\in H^1(\R) $, then for any interval $ \mathcal{B}\ni 0 $ 
		\begin{equation}
		\int_{\mathcal{B}} \abs{\partial \varphi}^2+\frac12 v\abs{\varphi}^2\geq \int_{\mathcal{B}}\frac{2}{R-a}\left(\delta_R+\delta_{-R}\right)\varphi
		\end{equation}
		where $ a $ is the s-wave scattering length.
	\end{lemma}
	\begin{proof}
		This follows from the variational scattering problem, by comparing left-hand side to the minimizer of the scattering functional.
	\end{proof}
	This lemma will essentially allows us to replace the potential by a shell potential of range $ R $ and strength $ \frac{2}{R-a} $.\\
	\begin{lemma}\label{LemmaNormBoundEpsilon}
		Let $ \psi $ be defined as above with $R>\max\left(R_0,2\abs{a}\right) $ and let $ \epsilon\in[0,1] $, then
		\begin{equation}
		\begin{aligned}
		&\int \sum_{i}\abs{\partial_i\Psi}^2+\sum_{i\neq j} \frac{1}{2}v_{ij}\abs{\Psi}^2\geq E_{LL}^N \left(n,\tilde{\ell},\frac{2\epsilon}{R-a}\right)\braket{\psi|\psi}+ \frac{(1-\epsilon)}{R^2}\textnormal{const. }(1-\braket{\psi|\psi}).
		\end{aligned}
		\end{equation}
		where $ \tilde{\ell}=\ell-(n-1)R $.
	\end{lemma}
	\begin{proof}
		Splitting the energy functional in two parts, and using lemma \ref{LemmaNormLoss} on one term and Dyson's lemma on the other we find 
		\begin{equation}
		\begin{aligned}
		&\int \sum_{i}\abs{\partial_i\Psi}^2+\sum_{i\neq j} \frac{1}{2}v_{ij}\abs{\Psi}^2\geq\\ &\int\sum_{i}\abs{\partial_i\Psi}^2\chi_{\mathfrak{r}_i(x)>R}+\epsilon\sum_{i}\frac{2}{R-a}\delta(\mathfrak{r}_i(x)-R)\abs{\Psi}^2\\&\qquad\qquad\qquad+ (1-\epsilon)\left(\sum_{i<j}\int_{B_{ij}}\abs{\partial_i \Psi}^2+\int\sum_{i<j} v_{ij} \abs{\Psi}^2\right)
		\end{aligned}
		\end{equation}
		where $ \mathfrak{r}_i(x)=\min_{j\neq i}(\abs{x_i-x_j}) $. The nearest neighbor interaction is obtained from Dyson's lemma by dividing the integration domain into Voronoi cells, and restricting to the cell around particle $ i $.\\
		By use of lemma \ref{LemmaNormLoss} with $ R>2\abs{a} $ in the last term, and by realising that the first two terms can be obtained by using $ \psi $ as a trial state in the Lieb-Liniger model, we obtain\begin{equation}
		\int \sum_{i}\abs{\partial_i\Psi}^2+\sum_{i\neq j} \frac{1}{2}v_{ij}\abs{\Psi}^2\geq E_{LL}^N \left(n,\tilde{\ell},\frac{2\epsilon}{R-a}\right)\braket{\psi|\psi}+ \frac{(1-\epsilon)}{R^2}\text{const. }(1-\braket{\psi|\psi})
		\end{equation}
	\end{proof}
	
	
	The next lemma will bound how much mass is lost when going from the state $ \Psi $ of mass $ 1 $ to the state $ \psi $
	\begin{lemma}\label{LemmaImprovedMassBound}
		Let $ C $ denote the constant in lemma \ref{LemmaNormLoss}. For $ n(\rho R)^2\leq  \frac{3}{16\pi^2}C $, $ \rho R\ll 1 $ and $ R>2\abs{a} $ we have
		\begin{equation}\label{EqImprovedMassBound}
		\begin{aligned}
		\braket{\psi|\psi} \geq 1-\textnormal{const. }\left(n(\rho R)^3+n^{1/3}(\rho R)^2\right).
		\end{aligned}
		\end{equation}
	\end{lemma}
	\begin{proof}
		From the known upper bound, and by lemma \ref{LemmaNormBoundEpsilon} with $ \epsilon=1/2 $, it follows that 
		\begin{equation}
		n\frac{\pi^2}{3}\rho^2\left(1+2\rho a+\text{const. }(\rho R)^{3/2}\right)\geq E_{LL}^N \left(n,\tilde{\ell},\frac{1}{R-a}\right)\braket{\psi|\psi}+ \frac{C}{2R^2}(1-\braket{\psi|\psi})
		\end{equation}
		Subtracting $ E_{LL}^N \left(n,\tilde{\ell},\frac{1}{R-a}\right) $ on both sides, and using lemma \ref{LemmaLiebLinigerNeumannLowerBound} on the right-hand side we find\begin{equation}
		\begin{aligned}
		&n\frac{\pi^2}{3}\rho^2\left(1+2\rho a+\text{const. }(\rho R)^{3/2}\right)-n\frac{\pi^2}{3}\tilde{\rho}^2\left(1+4\tilde{\rho} (R-a)-\text{const. }n^{-2/3}\right)\\
		&\geq  \left(\frac{C}{2R^2}-E_{LL}^N \left(n,\tilde{\ell},\frac{1}{R-a}\right)\right)(1-\braket{\psi|\psi}),
		\end{aligned}
		\end{equation}
		with $ \tilde{\rho}=n/\tilde{\ell}=\rho/(1-(\rho-1/\ell)R)  $
		Using now the upper bound $ E^N_{LL}\left(n,\tilde{\ell},\frac{1}{R-a}\right)\leq n\frac{\pi^2}{3}\tilde{\rho}^2 $ on the left-hand side, as well as $ 2\rho \geq\tilde{\rho}\geq \rho(1+\rho R)$ we find
		\begin{equation}
		\begin{aligned}
		\text{const. }n\rho^2R^2\left(\rho R+(\rho R)^{3/2}+n^{-2/3}\right)&\geq \left(\frac{C}{2}-R^2n\frac{4\pi^2}{3}\rho^2\right)\left(1-\braket{\psi|\psi}\right)
		\end{aligned}
		\end{equation}
		It follows that we have \begin{equation}
		\braket{\psi|\psi}\geq 1-\text{const. }\left(n(\rho R)^3+n^{1/3}(\rho R)^2\right)
		\end{equation}
	\end{proof}
	\textbf{Remark:} For $ n=\mathcal{O}((\rho R)^{-9/5}) $ we find \begin{equation}
	\braket{\psi|\psi}\geq 1-\text{const. }n(\rho R)^3=1-\text{const. }(\rho R)^{6/5}
	\end{equation}
	It is now straightforward to show the result
	
	\begin{proposition}\label{PropositionLowerBoundSpecN}
		For assumptions as in lemma \ref{LemmaImprovedMassBound} we have \begin{equation}
		E^N(n,\ell)\geq n\frac{\pi^2}{3}\rho^2\left(1+2\rho a+\textnormal{const. }\left(\frac{1}{n^{2/3}}+n(\rho R)^3+n^{1/3}(\rho R)^2\right)\right)
		\end{equation}
	\end{proposition}
	\begin{proof}
		By lemma \ref{LemmaNormBoundEpsilon} with $ \epsilon=1 $, we reduce to a Lieb-Liniger model with volume $ \tilde{\ell} $, density $ \tilde{\rho} $, and coupling $ c $, and we have $ \tilde{\ell}=\ell-(n-1)R $, $ \tilde{\rho}=\frac{n}{\tilde{\ell}}\approx\rho (1+\rho R) $ and $ c=\frac{2}{R-a} $. Hence we have by Lemmas \ref{LemmaLiebLinigerNeumannLowerBound} and \ref{LemmaImprovedMassBound} \begin{equation}
		\begin{aligned}
		E^N(n,\ell)&\geq E_{LL}^N(n,\tilde{\ell},c)\braket{\psi|\psi}\\&\geq
		n\frac{\pi^2}{3}\rho^2\left(1+2\rho a-\text{const. }\frac{1}{n^{2/3}}\right)\left(1-\text{const. }\left(n(\rho R)^3+n^{1/3}(\rho R)^2\right)\right)
		\end{aligned}
		\end{equation}
	\end{proof}
	\begin{corollary} \label{CorollaryLowerBoundSpecN}
		For $ n=\textnormal{const. } (\rho R)^{-9/5} $ we have 
		\begin{equation}
		E^N(n,\ell)\geq n\frac{\pi^2}{3}\rho^2\left(1+2\rho a-\textnormal{const. }\left((\rho R)^{6/5}+(\rho R)^{7/5}\right)\right).
		\end{equation}
	\end{corollary}
	\subsection{Lower bound of the dilute Bose gas for general particle number}
	So far, we have shown the desired lower bound only for the case where the number of particles are of the order $ (\rho R)^{-9/5} $. In this subsection, we generalize this to any number of particles. We do this, by performing a Legendre transformation in the particle number, \ie going to the grand canonical ensemble. First we justify that only particle numbers of orders less than or equal to $ (\rho R)^{-9/5} $ are relevant for a certain choice of $ \mu $.
		\begin{lemma}\label{LemmaLocalizationFbound}
			Let $ \Xi\geq 4 $ be fixed and let $ n=m\Xi \rho \ell+n_0 $ with $ n_0\in[0,\Xi\rho \ell) $ for some $ m\in\mathbb{N} $ with $ n^{\ast}:=\rho\ell=\mathcal{O}(\rho R)^{-9/5} $. Furhermore, assume that $ \rho R\ll 1 $ and let $ \mu=\pi^2\rho^2\left(1+\frac{8}{3}\rho a\right) $, then \begin{equation}
			E^{N}(n,\ell)-\mu n \geq E^{N}(n_0,\ell)-\mu n_0.
			\end{equation}
		\end{lemma}
		\begin{proof}
			By corollary \ref{CorollaryLowerBoundSpecN} we have \begin{equation}
			E^{N}(\Xi\rho\ell,\ell)\geq\frac{\pi^2}{3}\Xi^3\ell\rho^3\left(1+2\Xi\rho a-\text{const. }(\rho R)^{6/5}\right).
			\end{equation}
			By superadditivity (positive potential) we have \begin{equation}
			E^N(n,\ell)-\mu n\geq m\left(E^N(\Xi\rho\ell,\ell)-\mu\Xi\rho\ell \right)+E^N(n_0,\ell)-\mu n_0.
			\end{equation}
			Thus the result follows from the fact that \begin{equation}
			\frac{\pi^2}{3}\Xi^3\ell\rho^3\left(1+2\Xi\rho a-\text{const. }(\rho R)^{6/5}\right)\geq \pi^2\rho^2\left(1+\frac{8}{3}\rho a\right) \Xi\rho\ell
			\end{equation}
		\end{proof}
	We are then ready to prove the lower bound for general particle numbers
	
	
		\begin{theorem}[Lower bound] Let $ E^N(N,L) $ denote the ground state energy of $ \mathcal{E} $ with Neumann boundary conditions. Then for $ \rho R \ll 1 $
			\begin{equation}
			E^N(N,L)\geq N\frac{\pi^2}{3}\rho^2\left(1+2\rho a-\mathcal{O}\left((\rho R)^{6/5}\right)\right)
			\end{equation}
		\end{theorem}
		\begin{proof}
			Notice that \begin{equation}
			E^N(N,L)\geq F^N(\mu,L)+\mu N
			\end{equation}
			where $ F^N(\mu,L)=\inf_{N'}\left(E^N(N',L)-\mu N'\right) $. Clearly we have \begin{equation}
			F^N(\mu,L)\geq M F^N(\mu,\ell)\label{EqLocalizationF}
			\end{equation}
			with $ \ell=L/M $ and $ M\in \mathbb{N}_+ $. 
%			This can be seen by the following argument. Split the box of volume $ L $ in $ M $ boxes of volume $ \ell $. Ignoring the interactions between boxes and using the ground state of the full Hamiltonian as a trial state of the box-divided Hamiltonian we find\begin{equation}
%			E^N(N,L)-\mu N\geq \min_{\{c_n\}}\left\{M\sum_{n=0}^{N}c_n E^N(n,\ell)\right\}-\mu N
%			\end{equation}
%			with $ \{c_n\} $ denoting a distribution of the particles in the boxes, where $ c_n $ is the fraction of the $ M $ boxes that contain exactly $ n $ particles, such that $ \sum_{n=0}^{N}c_n=1 $. Hence we have \begin{equation}
%			\begin{aligned}
%			E^N(N,L)-\mu N\geq\min_{\{c_n\}}\left\{M\sum_{n=0}^{N}c_n \left(E^N(n,\ell)-\mu n\right)\right\}\\
%			\geq MF^{N}(\mu,\ell),
%			\end{aligned}
%			\end{equation}
%			from which \eqref{EqLocalizationF} follows. 
			Now we choose $ M $ such that $ n^*:=\rho\ell=\mathcal{O}\left(\left(\rho R\right)^{-9/5}\right) $ and $ \mu=\pi^2\rho^2\left(1+\frac{8}{3}\rho a\right) $ (notice that $ \mu=\frac{\diff}{\diff \rho}(\frac{\pi^2}{3}\rho^3(1+2\rho a))$). By lemma \ref{LemmaLocalizationFbound} we have that \begin{equation}
			F^N(\mu,\ell):=\inf_{n}\left(E^N(n,\ell)-\mu n\right)=\inf_{n<\Xi n^*}\left(E^N(n,\ell)-\mu n\right).
			\end{equation}
			Now it is known from proposition \ref{PropositionLowerBoundSpecN} that for $ n<\Xi n^* $ we have \begin{equation}
			\begin{aligned}
			E^{N}(n,\ell)&\geq n\frac{\pi^2}{3}\bar{\rho}^2\left(1+2\bar{\rho} a-\textnormal{const. }\left(\frac{1}{n^{2/3}}+n(\bar{\rho} R)^3+n^{1/3}(\bar{\rho} R)^2\right)\right)\\
			&\geq \frac{\pi^2}{3}n\bar{\rho}^2\left(1+2\bar{\rho}a\right)-n^*\rho^2\mathcal{O}\left((\rho R)^{6/5}\right)
			\end{aligned}
			\end{equation}
			where $ \bar{\rho}=n/\ell $ (notice that now $ \rho=N/L=n^\ast/\ell\neq n/\ell $) and where we used $ \bar{\rho}<\Xi\rho$.
			Thus we have \begin{equation}
			F^{N}(\mu,\ell)\geq \inf_{\bar{\rho}<\Xi\rho}(g(\bar{\rho})-\mu\bar{\rho})\ell-n^\ast \rho^2 \mathcal{O}\left((\rho R)^{6/5}\right)
			\end{equation}
			where $
			g(\bar{\rho})=
			\frac{\pi^2}{3}\bar{\rho}^3\left(1+2\bar{\rho}a\right)
			$ for $ \bar{\rho}<\Xi\rho $. $ g $ is a convex, $ C^{1} $ function with invertible derivative for $ \Xi\rho a\ll 1  $. Hence we have \begin{equation}
			\begin{aligned}
			E^{N}(N,L)\geq M(F^{N}(\mu,\ell)+\mu n^*)\geq Mn^\ast\frac{\pi^2}{3} \rho^2 \left(1+2\rho a-\mathcal{O}\left((\rho R)^{6/5}\right)\right)\\
			=\frac{\pi^2}{3} N\rho^2 \left(1+2\rho a-\mathcal{O}\left((\rho R)^{6/5}\right)\right)
			\end{aligned}
			\end{equation}
			where the second inequality follows from the specific choice of $ \mu=g'(\rho) $.
		\end{proof}
	 
	\bibliographystyle{amsplain}
	\bibliography{bibliography}
	\end{document}
