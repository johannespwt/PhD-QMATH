\documentclass[a4paper,11pt]{article}
\usepackage[utf8]{inputenc}
\usepackage[margin=1in]{geometry}
\usepackage{pdfpages}
\usepackage{mathrsfs}
\usepackage{amsfonts}
\usepackage{amsmath}
\usepackage{mathtools}
\DeclareMathOperator\arctanh{arctanh}
\usepackage{amssymb}
\usepackage{bbm}
\usepackage{amsthm}
\usepackage{graphicx}
\usepackage{centernot}
\usepackage{caption}
\usepackage{subcaption}
\usepackage{braket}
\usepackage{lastpage}
\usepackage{enumitem}
\usepackage{setspace}
\usepackage{xcolor}
\usepackage[english]{babel} 

\usepackage[square,sort,comma,numbers]{natbib}
\usepackage[colorlinks=true,linkcolor=blue]{hyperref}

\usepackage{fancyhdr}
\newcommand{\euler}[1]{\text{e}^{#1}}
\newcommand{\Real}{\text{Re}}
\newcommand{\Imag}{\text{Im}}
\newcommand{\supp}{\text{supp}}
\newcommand{\norm}[1]{\left\lVert #1 \right\rVert}
\newcommand{\abs}[1]{\left\lvert #1 \right\rvert}
\newcommand{\floor}[1]{\left\lfloor #1 \right\rfloor}
\newcommand{\Span}[1]{\text{span}\left(#1\right)}
\newcommand{\dom}[1]{\mathscr D\left(#1\right)}
\newcommand{\Ran}[1]{\text{Ran}\left(#1\right)}
\newcommand{\conv}[1]{\text{co}\left\{#1\right\}}
\newcommand{\Ext}[1]{\text{Ext}\left\{#1\right\}}
\newcommand{\vin}{\rotatebox[origin=c]{-90}{$\in$}}
\newcommand{\interior}[1]{%
	{\kern0pt#1}^{\mathrm{o}}%
}
\renewcommand{\braket}[1]{\left\langle#1\right\rangle}
\newcommand*\diff{\mathop{}\!\mathrm{d}}
\newcommand{\ie}{\emph{i.e.} }
\newcommand{\eg}{\emph{e.g.} }
\newcommand{\dd}{\partial }
\newcommand{\R}{\mathbb{R}}
\newcommand{\C}{\mathbb{C}}
\newcommand{\w}{\mathsf{w}}
\newcommand{\rr}{\mathcal{R}}


\newcommand{\Gliminf}{\Gamma\text{-}\liminf}
\newcommand{\Glimsup}{\Gamma\text{-}\limsup}
\newcommand{\Glim}{\Gamma\text{-}\lim}

\newtheorem{theorem}{Theorem}
\newtheorem{definition}{Definition}
\newtheorem{proposition}{Proposition}
\newtheorem{lemma}{Lemma}
\newtheorem{corollary}{Corollary}
\newtheorem{remark}{Remark}

\numberwithin{equation}{section}
\linespread{1.3}

\pagestyle{fancy}
\fancyhf{}
\rhead{Notes on 1D anyons}
\lhead{}
\rfoot{\thepage}
\lfoot{Dated: \today}
\author{}
\date{Dated: \today}
\title{Notes on 1D anyons}
\begin{document}
	\maketitle
	Let $ \kappa\in[0,\pi] $ and define for each permuation $ \sigma=(\sigma_1,...,\sigma_N) $ the sector $ \Sigma_\sigma=\{x_{\sigma_1}<x_{\sigma_2}<...<x_{\sigma_N}\}\subset \R^N $, and consider the operator\begin{equation}
	H_N=-\sum_{i=1}^{N}\partial_{x_i}^2,\quad\text{on }\R^N\setminus\bigcup_{i<j}\{x_i=x_j\}
	\end{equation}
	with domian \begin{equation}
	\begin{aligned}
	\mathcal{D}(H_N)&=\left\{\varphi=\euler{-i\frac{\kappa}{2}\Lambda(x)}f(x)\ \Bigg\vert\ f\in \left((\oplus_{\text{sym}})_{\sigma\in S_N } C^\infty(\overline{\Sigma_{\sigma}})\right)\cap C_0(\R^N) ,\right.\\&\qquad \left.\ (\partial_i-\partial_j)\varphi\rvert^{ij}_+-(\partial_i-\partial_j)\varphi\rvert^{ij}_-=2c\ \euler{-i\frac{\kappa}{2}\Lambda(x)} f\rvert^{ij}_0 \text{ for all }i\neq j \right\}
	\end{aligned}
	\end{equation}
	where $ \Lambda(x)= \sum_{i<j}\epsilon(x_i-x_j) $  with $ \epsilon(x)=\begin{cases}
	1&\text{for }x>0\\
	-1&\text{for }x<0\\
	0&\text{for }x=0
	\end{cases} $ and $ \vert^{ij}_{\pm,0} $ means the function evaluated at $ x_i=x_{j \pm,0} $. Then the following proposition holds \begin{proposition}
		$ H_N $ is symmetric with corresponding quadratic form \begin{equation}
		\mathcal{E}(\varphi)=\sum_{i=1}^{N}\int_{{\R^N\setminus\bigcup_{i<j}\{x_i=x_j\}}} \abs{\partial_{x_i}\varphi(x)}^2+\frac{2c}{\cos(\kappa/2)}\sum_{i<j} \delta(x_i-x_j)\abs{\varphi(x)}^2\diff^{N}x
		\end{equation}
	\end{proposition}
	\begin{proof}
		Let $ \varphi,\vartheta\in \mathcal{D}(H_N) $, then by partial integration we have \begin{equation}
		\begin{aligned}
		\braket{\vartheta\vert H_N \varphi}&=-\sum_{i=1}^{N}\int_{\R^N\setminus\bigcup_{i<j}\{x_i=x_j\}}\overline{\vartheta} \partial_{x_i}^2\varphi\\&=\sum_{i=1}^{N}\int_{\R^N\setminus\bigcup_{i<j}\{x_i=x_j\}}\overline{\partial_{x_i}\vartheta}\partial_{x_i}\varphi-\int_{\R^{N-1}\setminus\bigcup_{i<j}\{x_i=x_j\}}\sum_{i\neq j}\left(\overline{\vartheta}\partial_{x_i}\varphi\vert^{ij}_--\overline{\vartheta}\partial_{x_i}\varphi\vert^{ij}_+\right)\\
		&=\sum_{i=1}^{N}\int_{\R^N\setminus\bigcup_{i<j}\{x_i=x_j\}}\overline{\partial_{x_i}\vartheta}\partial_{x_i}\varphi+\int_{\R^{N-1}\setminus\bigcup_{i<j}\{x_i=x_j\}}\sum_{i< j}\left(\overline{\vartheta}(\partial_{x_i}-\partial_{x_j})\varphi\vert^{ij}_+-\overline{\vartheta}(\partial_{x_i}-\partial_{x_j})\varphi\vert^{ij}_-\right).
		\end{aligned}
		\end{equation}
		Let $ f,g\in C^\infty_0(\R^N) $ be the functions such that $ \varphi=\euler{-i\frac{\kappa}{2}\Lambda}f $ and $ \vartheta=\euler{-i\frac{\kappa}{2}\Lambda}g $. Then we have
		
		\begin{equation}
		\begin{aligned}
		\braket{\vartheta\vert H_N \varphi}&=\sum_{i=1}^{N}\int_{\R^N\setminus\bigcup_{i<j}\{x_i=x_j\}}\overline{\partial_{x_i}\vartheta}\partial_{x_i}\varphi+\int_{\R^{N-1}\setminus\bigcup_{i<j}\{x_i=x_j\}}\sum_{i< j}\left(\overline{g}(\partial_{x_i}-\partial_{x_j})f\vert^{ij}_+-\overline{g}(\partial_{x_i}-\partial_{x_j})f\vert^{ij}_-\right)\\
		&=\sum_{i=1}^{N}\int_{\R^N\setminus\bigcup_{i<j}\{x_i=x_j\}}\overline{\partial_{x_i}\vartheta}\partial_{x_i}\varphi+\int_{\R^{N-1}\setminus\bigcup_{i<j}\{x_i=x_j\}}2\sum_{i< j}\left(\overline{g}(\partial_{x_i}-\partial_{x_j})f\vert^{ij}_+\right)
		\end{aligned}
		\end{equation}
		where the last equality follows from symmetry of $ f $. Notice that by the boundary condtion on $ \mathcal{D}(H_N) $ we have \begin{equation}
		(\partial_i-\partial_j)\varphi\rvert^{ij}_+-(\partial_i-\partial_j)\varphi\rvert^{ij}_-=\euler{-i\frac{\kappa}{2}\left(-1+S\right)}(\partial_i-\partial_j)f\rvert^{ij}_+-\euler{-i\frac{\kappa}{2}\left(1+S\right)}(\partial_i-\partial_j)f\rvert^{ij}_-=2c \varphi\rvert^{ij}_0=\euler{-i\frac{\kappa}{2}S}2c f\rvert^{ij}_0
		\end{equation}
		where $ S=\Lambda-\epsilon(x_i-x_j) $. By symmetry of $ f $ it follows that \begin{equation}
		\begin{aligned}
		\euler{-i\frac{\kappa}{2}\left(-1+S\right)}(\partial_i-\partial_j)f\rvert^{ij}_+-\euler{-i\frac{\kappa}{2}\left(1+S\right)}(\partial_i-\partial_j)f\rvert^{ij}_-
		&=\euler{-i\frac{\kappa}{2}\left(-1+S\right)}(\partial_i-\partial_j)f\rvert^{ij}_++\euler{-i\frac{\kappa}{2}\left(1+S\right)}(\partial_i-\partial_j)f\rvert^{ij}_+\\
		&=\euler{-i\frac{\kappa}{2}S}2\cos(\kappa/2)(\partial_i-\partial_j)f\rvert^{ij}_+\\
		&=\euler{-i\frac{\kappa}{2}S}2c f\rvert^{ij}_0.
		\end{aligned}
		\end{equation}
		so that \begin{equation}
		2(\partial_i-\partial_j)f\rvert^{ij}_+=\frac{2c}{\cos(\kappa/2)}f\rvert^{ij}_0.
		\end{equation}
		Hence it follows that \begin{equation}\label{EqQuadraticFormDerivation}
		\braket{\vartheta\vert H_N \varphi}=\sum_{i=1}^{N}\int_{{\R^N\setminus\bigcup_{i<j}\{x_i=x_j\}}}\overline{\partial_{x_i}\vartheta} \partial_{x_i}\varphi(x)+\frac{2c}{\cos(\kappa/2)}\sum_{i<j} \delta(x_i-x_j)\overline{\vartheta(x)}\varphi(x)\diff^{N}x.
		\end{equation}
		Now it is clear that starting from $ \braket{H_N\vartheta\vert \phi} $, we can by the same steps arrive at \eqref{EqQuadraticFormDerivation}, proving that $ H_N $ is symmetric. 	
		\end{proof}
		\begin{remark}
			 Since $ \mathcal{E}\geq0 $, $ H_N $ has a self-adjoint Friedrichs extension, $ \tilde{H}_N $, which we regard as the Hamiltonian for the one dimensional anyon gas with statistical parameter, $ \kappa $, and a zero-range interaction of strength, $ c $.
		\end{remark}
		\begin{remark}
			By the quadratic form formulation, and the fact that the phase-factor is not contributing to the value of the quadratic form, it follows that $ \tilde{H}_N $ is unitarily equivalent to the Lieb-Liniger Hamiltonian $ H_{LL}(N,\frac{c}{\cos(\kappa/2)}) $, with $ N $ particles and coupling $ c/\cos(\kappa/2) $.
		\end{remark}
	
\end{document}