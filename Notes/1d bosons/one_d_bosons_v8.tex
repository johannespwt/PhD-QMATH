\documentclass[a4paper,11pt]{article}
\usepackage[utf8]{inputenc}
\usepackage[margin=1in]{geometry}
\usepackage{pdfpages}
\usepackage{mathrsfs}
\usepackage{amsfonts}
\usepackage{amsmath}
\usepackage{mathtools}
\DeclareMathOperator\arctanh{arctanh}
\usepackage{amssymb}
\usepackage{bbm}
\usepackage{amsthm}
\usepackage{graphicx}
\usepackage{centernot}
\usepackage{caption}
\usepackage{subcaption}
\usepackage{braket}
\usepackage{lastpage}
\usepackage{enumitem}
\usepackage{setspace}
\usepackage{xcolor}
\usepackage[english]{babel} 

\usepackage[square,sort,comma,numbers]{natbib}
\usepackage[colorlinks=true,linkcolor=blue]{hyperref}

\usepackage{fancyhdr}
\newcommand{\euler}[1]{\text{e}^{#1}}
\newcommand{\Real}{\text{Re}}
\newcommand{\Imag}{\text{Im}}
\newcommand{\supp}{\text{supp}}
\newcommand{\norm}[1]{\left\lVert #1 \right\rVert}
\newcommand{\abs}[1]{\left\lvert #1 \right\rvert}
\newcommand{\floor}[1]{\left\lfloor #1 \right\rfloor}
\newcommand{\Span}[1]{\text{span}\left(#1\right)}
\newcommand{\dom}[1]{\mathscr D\left(#1\right)}
\newcommand{\Ran}[1]{\text{Ran}\left(#1\right)}
\newcommand{\conv}[1]{\text{co}\left\{#1\right\}}
\newcommand{\Ext}[1]{\text{Ext}\left\{#1\right\}}
\newcommand{\vin}{\rotatebox[origin=c]{-90}{$\in$}}
\newcommand{\interior}[1]{%
	{\kern0pt#1}^{\mathrm{o}}%
}
\renewcommand{\braket}[1]{\left\langle#1\right\rangle}
\newcommand*\diff{\mathop{}\!\mathrm{d}}
\newcommand{\ie}{\emph{i.e.} }
\newcommand{\eg}{\emph{e.g.} }
\newcommand{\dd}{\partial }
\newcommand{\R}{\mathbb{R}}
\newcommand{\C}{\mathbb{C}}
\newcommand{\w}{\mathsf{w}}
\newcommand{\rr}{\mathcal{R}}


\newcommand{\Gliminf}{\Gamma\text{-}\liminf}
\newcommand{\Glimsup}{\Gamma\text{-}\limsup}
\newcommand{\Glim}{\Gamma\text{-}\lim}

\newtheorem{theorem}{Theorem}
\newtheorem{definition}{Definition}
\newtheorem{proposition}{Proposition}
\newtheorem{lemma}{Lemma}
\newtheorem{corollary}{Corollary}
\newtheorem{remark}{Remark}

\numberwithin{equation}{section}
\linespread{1.3}

\pagestyle{fancy}
\fancyhf{}
\rhead{Notes on 1D bosons}
\lhead{}
\rfoot{\thepage}
\lfoot{Dated: \today}
\author{}
\date{Dated: \today}
\title{Notes on 1D bosons}
\begin{document}
	\maketitle
	In this paper, we analyze the ground state energy of the quadratic form\begin{equation}
	\mathcal{E}(\Psi)=\int \sum_{i=1}^{N}\abs{\nabla_i\Psi}^2+\sum_{i<j}^{N}v_{ij}\abs{\Psi}^2.
	\end{equation}
	We assume that $ v_{ij}:=v(\abs{x_i-x_j}) $ is a symmetric and translation invariant measure. Furthermore, we assume that $ v $ is of the form $ v=v_{\text{h.c.}}+v_{\text{reg}} $, where $ v_{\text{reg}} $ is a finite measure, and $ v_{\text{h.c.}} $ is a sum of hard core potentials.
	\section{Upper bound}
	\subsection{The trial state}
	We provide the upper bound for the ground state energy, by using the variational principle with a suitable trial state. Since we are interested in an upper bound, we consider Dirichlet boundary conditions.
	For $ b>R_0 $, consider the trial state	\begin{equation}
	\Psi(x)=\begin{cases}
	\omega(\rr(x))\frac{\tilde{\Psi}_F(x)}{\rr(x)}& \text{if }\rr(x)<b,\\
	\tilde{\Psi}_F(x)&\text{if }\rr(x)\geq b,
	\end{cases}
	\end{equation}
	where $ \omega $ is the suitably normalized solution to the two-body scattering equation, \ie $ \omega(x)=f(x)\frac{b}{f(b)} $ where $ f $ is any solution of the two-body scattering equation,  $ \tilde{\Psi}_F(x)=\abs{\Psi_F} $ is the absolute value of the free fermionic ground state, and $ \rr(x)=\min_{i<j}(\abs{x_i-x_j}) $ is uniquely defined a.e. Notice that defining $ B=\{x\in\R^N\ \vert\ \mathcal{R}(x)<b \} $ we have that $ \Psi=\tilde{\Psi}_F $ on $ B^\complement $.\\
	The energy of this trial state is then\begin{equation}
	\mathcal{E}(\Psi)=\int \sum_{i=1}^{N}\abs{\nabla_i\Psi}^2+\sum_{i<j}^{N}v_{ij}\abs{\Psi}^2,
	\end{equation}
	where $ v_{ij}(x)=v(x_i-x_j) $. In the following we define some useful quantities:\\
	We define the sets $ B_{12}\coloneqq\{x\in\R^N \vert \rr(x)<b,\ \rr(x)=\abs{x_1-x_2} \}\subset A_{12}\coloneqq\{x\in\R^N\vert \abs{x_1-x_2}<b\} $. Up to a set of measure zero, $ B_{12} $ is the intersection of $ B $ with the set $ \{\text{"1 and 2 are closest"}\} $. Define also the quantities
	
	\begin{equation}
	\begin{aligned}
	E_1&\coloneqq\binom{N}{2}\int_{A_{12}} \sum_{i=1}^{N}\abs{\nabla_i\tilde{\Psi}}^2+\sum_{i<j}^{N}(v_{\text{reg}})_{ij}\abs{\tilde{\Psi}}^2-\sum_{i=1}^{N}\abs{\nabla_i\Psi_F}^2, \\
	E_2^{(1)}&\coloneqq\binom{N}{2}2N\int_{A_{12}\cap A_{13}}\sum_{i=1}^{N}\abs{\nabla_i\tilde{\Psi}_F}^2,\\ E_2^{(2)}&\coloneqq\binom{N}{2}\binom{N-2}{2}\int_{A_{12}\cap A_{34}}\sum_{i=1}^{N}\abs{\nabla_i\tilde{\Psi}_F}^2.
	\end{aligned}
	\end{equation}
	We then have \begin{lemma}
		Let $ v\geq 0 $, then \begin{equation}\label{EqBound1}
		\mathcal{E}(\Psi)\leq E_1+E_2^{(1)}+E_2^{(2)}.
		\end{equation}
	\end{lemma}
	\begin{proof}
		Since $ v $ is supported in $ B_b $ and $ \Psi=\tilde{\Psi}_F $ except in the region $ B=\{x\in\R^N \vert \rr(x)<b \} $, we may rewrite this, using the diamagnetic inequality, as \begin{equation}
		\mathcal{E}(\Psi)\leq E_0+\int_B \sum_{i=1}^{N}\abs{\nabla_i\Psi}^2+\sum_{i<j}^{N}v_{ij}\abs{\Psi}^2-\sum_{i=1}^{N}\abs{\nabla_i\Psi_F}^2,
		\end{equation}
		with $ E_0=N\frac{\pi^2}{3}\rho^2(1+\mathcal{O}(1/N)) $ the ground state energy of the free Fermi gas. Using that $ v\geq0 $, symmetry under exchange of particles, and using the diamagnetic inequality, we find \begin{equation}
		\begin{aligned}
		\mathcal{E}(\Psi)&\leq E_0+\binom{N}{2}\int_{B_{12}} \sum_{i=1}^{N}\abs{\nabla_i\Psi}^2+\sum_{i<j}^{N}v_{ij}\abs{\Psi}^2-\sum_{i=1}^{N}\abs{\nabla_i\Psi_F}^2\\&
		\leq E_0+\binom{N}{2}\int_{B_{12}} \sum_{i=1}^{N}\abs{\nabla_i\tilde{\Psi}}^2+\sum_{i<j}^{N}(v_{\text{reg}})_{ij}\abs{\tilde{\Psi}}^2-\sum_{i=1}^{N}\abs{\nabla_i\Psi_F}^2
		\end{aligned}
		\end{equation}
		where we have defined $
		\tilde{\Psi}=
		\omega(x_1-x_2)\frac{\Psi_F(x)}{(x_1-x_2)} $ on $ A_{12} $.
		and used that $ \Psi=0 $ on the support of $ (v_{\text{h.c.}})_{ij} $ for all $ i,j $.
		\begin{equation}
		\begin{aligned}
		\mathcal{E}(\Psi)&\leq E_0+\binom{N}{2}\int_{A_{12}} \sum_{i=1}^{N}\abs{\nabla_i\tilde{\Psi}}^2+\sum_{i<j}^{N}(v_{\text{reg}})_{ij}\abs{\tilde{\Psi}}^2-\sum_{i=1}^{N}\abs{\nabla_i\Psi_F}^2\\&\qquad
		-\binom{N}{2}\int_{A_{12}\setminus B_{12}} \sum_{i=1}^{N}\abs{\nabla_i\tilde{\Psi}}^2+\sum_{i<j}^{N}(v_{\text{reg}})_{ij}\abs{\tilde{\Psi}}^2-\sum_{i=1}^{N}\abs{\nabla_i\Psi_F}^2\\&
		\leq E_0+E_1+\binom{N}{2}\int_{A_{12}\setminus B_{12}}\sum_{i=1}^{N}\abs{\nabla_i\Psi_F}^2
		\end{aligned}
		\end{equation}	
		We may, by an inclusion-exclusion argument, estimate\begin{equation}
		\begin{aligned}
		\binom{N}{2}\int_{A_{12}\setminus B_{12}}\sum_{i=1}^{N}\abs{\nabla_i\Psi_F}^2&\leq \binom{N}{2}\left(2N\left[\int_{A_{12}\cap A_{13}}\sum_{i=1}^{N}\abs{\nabla_i\Psi_F}^2-\int_{B_{12}\cap A_{13}}\sum_{i=1}^{N}\abs{\nabla_i\Psi_F}^2\right]\right.\\
		&\qquad\qquad\left.+\binom{N-2}{2}\left[\int_{A_{12}\cap A_{34}}\sum_{i=1}^{N}\abs{\nabla_i\Psi_F}^2-\int_{B_{12}\cap A_{34}}\sum_{i=1}^{N}\abs{\nabla_i\Psi_F}^2\right]\right)\\
		&\leq \binom{N}{2}\left[2N\int_{A_{12}\cap A_{13}}\sum_{i=1}^{N}\abs{\nabla_i\Psi_F}^2+\binom{N-2}{2}\int_{A_{12}\cap A_{34}}\sum_{i=1}^{N}\abs{\nabla_i\Psi_F}^2\right]
		\end{aligned}
		\end{equation}
		Thus we find $
		\mathcal{E}(\Psi)\leq E_0+E_1+E_2^{(1)}+E_2^{(2)}$ as desired.
	\end{proof}
	%	Since $ v $ is supported in $ B_b $ and $ \Psi=\tilde{\Psi}_F $ except in the region $ B=\{x\in\R^N \vert \rr(x)<b \} $, we may rewrite this, using the diamagnetic inequality, as \begin{equation}
	%	\mathcal{E}(\Psi)=E_0+\int_B \sum_{i=1}^{N}\abs{\nabla_i\Psi}^2+\sum_{i<j}^{N}v_{ij}\abs{\Psi}^2-\sum_{i=1}^{N}\abs{\nabla_i\Psi_F}^2,
	%	\end{equation}
	%	with $ E_0=N\frac{\pi^2}{3}\rho^2(1+\mathcal{O}(1/N))\norm{\Psi}^2 $ the ground state energy of the free Fermi gas. Using that $ v\geq0 $, symmetry under exchange of particles, and defining the set $ B_{12}=\{x\in\R^N \vert \rr(x)<b,\ \rr(x)=\abs{x_1-x_2} \}\subset A_{12}=\{x\in\R^N\vert \abs{x_1-x_2}<b\} $ which up to a set of measure zero is the intersection of $ B $ and the set $ \{\text{"1 and 2 are closest"}\} $, and using the diamagnetic inequality, we find \begin{equation}
	%	\begin{aligned}
	%	\mathcal{E}(\Psi)&\leq E_0+\binom{N}{2}\int_{B_{12}} \sum_{i=1}^{N}\abs{\nabla_i\Psi}^2+\sum_{i<j}^{N}v_{ij}\abs{\Psi}^2-\sum_{i=1}^{N}\abs{\nabla_i\Psi_F}^2\\&
	%	\leq E_0+\binom{N}{2}\int_{B_{12}} \sum_{i=1}^{N}\abs{\nabla_i\tilde{\Psi}}^2+\sum_{i<j}^{N}v_{ij}\abs{\tilde{\Psi}}^2-\sum_{i=1}^{N}\abs{\nabla_i\Psi_F}^2\\&
	%	=E_0+\binom{N}{2}\int_{A_{12}} \sum_{i=1}^{N}\abs{\nabla_i\tilde{\Psi}}^2+\sum_{i<j}^{N}v_{ij}\abs{\tilde{\Psi}}^2-\sum_{i=1}^{N}\abs{\nabla_i\Psi_F}^2\\&\qquad
	%	-\binom{N}{2}\int_{A_{12}\setminus B_{12}} \sum_{i=1}^{N}\abs{\nabla_i\tilde{\Psi}}^2+\sum_{i<j}^{N}v_{ij}\abs{\tilde{\Psi}}^2-\sum_{i=1}^{N}\abs{\nabla_i\Psi_F}^2\\&
	%	\leq E_0+E_1+\binom{N}{2}\int_{A_{12}\setminus B_{12}}\sum_{i=1}^{N}\abs{\nabla_i\Psi_F}^2
	%	\end{aligned}
	%	\end{equation}
	%	where we have defined \begin{equation*}
	%		\tilde{\Psi}=\begin{cases}
	%		\omega(x_1-x_2)\frac{\Psi_F(x)}{(x_1-x_2)}& \text{if }\abs{x_1-x_2}<b,\\
	%		\text{sgn}(x_1-x_2)\Psi_F(x)&\text{if }\abs{x_1-x_2}\geq b,
	%		\end{cases}
	%	\end{equation*} and $ E_1=\binom{N}{2}\int_{A_{12}} \sum_{i=1}^{N}\abs{\nabla_i\tilde{\Psi}}^2+\sum_{i<j}^{N}v_{ij}\abs{\tilde{\Psi}}^2-\sum_{i=1}^{N}\abs{\nabla_i\Psi_F}^2 $.\\
	%	We may, by an inclusion-exclusion argument, estimate\begin{equation}
	%	\begin{aligned}
	%	\binom{N}{2}\int_{A_{12}\setminus B_{12}}\sum_{i=1}^{N}\abs{\nabla_i\Psi_F}^2&=\binom{N}{2}\left(2N\left[\int_{A_{12}\cap A_{13}}\sum_{i=1}^{N}\abs{\nabla_i\Psi_F}^2-\int_{B_{12}\cap A_{13}}\sum_{i=1}^{N}\abs{\nabla_i\Psi_F}^2\right]\right.\\
	%	&\qquad\qquad\left.+\binom{N-2}{2}\left[\int_{A_{12}\cap A_{34}}\sum_{i=1}^{N}\abs{\nabla_i\Psi_F}^2-\int_{B_{12}\cap A_{34}}\sum_{i=1}^{N}\abs{\nabla_i\Psi_F}^2\right]\right)\\
	%	&\leq \binom{N}{2}\left[2N\int_{A_{12}\cap A_{13}}\sum_{i=1}^{N}\abs{\nabla_i\Psi_F}^2+\binom{N-2}{2}\int_{A_{12}\cap A_{34}}\sum_{i=1}^{N}\abs{\nabla_i\Psi_F}^2\right]
	%	\end{aligned}
	%	\end{equation}
	%	Thus we find \begin{equation}\label{EqBound1}
	%	\mathcal{E}(\Psi)\leq E_0+E_1+E_2^{(1)}+E_2^{(2)}
	%	\end{equation}
	%	with $ E_2^{(1)}=\binom{N}{2}2N\int_{A_{12}\cap A_{13}}\sum_{i=1}^{N}\abs{\nabla_i\tilde{\Psi}_F}^2 $ and $ E_2^{(2)}=\binom{N}{2}\binom{N-2}{2}\int_{A_{12}\cap A_{34}}\sum_{i=1}^{N}\abs{\nabla_i\tilde{\Psi}_F}^2 $.\\
	%	We notice that since $ \tilde{\Psi}_F=\abs{\Psi_F} $ so by the diamagnetic inequality we have $ \abs{\nabla_i\tilde{\Psi}_F}^2\leq \abs{\nabla_i\Psi_F}^2 $, which implies that $ \tilde{\Psi}_F $ is in $ H^{1}(\Lambda_L) $. Furthermore, $ \Psi_F $ is $ C^{1}(\Lambda_L) $ with a zero set $ \{\Psi_F=0\} $ of measure zero, so $ \abs{\nabla_i\tilde{\Psi}_F}^2 $ and $ \abs{\nabla_i\Psi_F}^2 $ are equal a.e. But then $ \abs{\nabla_i\tilde{\Psi}_F}=\abs{\nabla_i\Psi_F} $ as $ L^{2}(\Lambda_L) $ functions. Hence we may replace $ \tilde{\Psi}_F $ with $ \Psi_F $ in all integrals above.
	\subsection{The free Fermi ground state}
	We now construct the free Fermi ground state. The Dirichlet eigenstates of the Laplacian are $ \phi_j(x)=\sqrt{2/L}\sin(\pi j x/L) $. Thus the free Fermi ground state is \begin{equation}
	\Psi_F(x)=\det\left(\phi_j(x_i)\right)_{i,j=1}^{N}=\sqrt{\frac{2}{L}}^N\left(\frac{1}{2i}\right)^N\begin{vmatrix}
	\euler{iy_1}-\euler{-iy_1}&\euler{i2y_1}-\euler{-i2y_1}&\ldots&\euler{iNy_1}-\euler{-iNy_1}\\
	\euler{iy_2}-\euler{-iy_2}&\euler{i2y_2}-\euler{-i2y_2}&\ldots&\euler{iNy_2}-\euler{-iNy_2}\\
	\vdots&\vdots&\ddots&\vdots\\
	\euler{iy_N}-\euler{-iy_N}&\euler{i2y_N}-\euler{-i2y_N}&\ldots&\euler{iNy_N}-\euler{-iNy_N}
	\end{vmatrix},
	\end{equation}
	where we defined $ y_i=\frac{\pi}{L}x_i $. Defining $ z=\euler{iy} $ and using the relation $ (x^n-y^n)/(x-y)=\sum_{k=0}^{n-1}x^ky^{n-1-k} $ we find\begin{equation}
	\Psi_F(x)=\sqrt{\frac{2}{L}}^N\left(\frac{1}{2i}\right)^N\prod_{i=1}^{N}(z_i-z_i^{-1})\begin{vmatrix}
	1&z_1+z_1^{-1}&\ldots&\sum_{k=0}^{N-1}z_1^{2k-N+1}\\
	1&z_2+z_2^{-1}&\ldots&\sum_{k=0}^{N-1}z_2^{2k-N+1}\\
	\vdots&\vdots&\ddots&\vdots\\
	1&z_N+z_N^{-1}&\ldots&\sum_{k=0}^{N-1}z_N^{2k-N+1}\\
	\end{vmatrix}.
	\end{equation}
	Notice now that $ (z+z^{-1})^n=\sum_{k=0}^{n}\binom{n}{k}z^{2k-n} $.
	Now for $ i $ from $ 1 $ to $ N-1 $ we add $ \left(\binom{N-1}{i}-\binom{N-1}{i-1}\right) $ times column $ N-i $ to column $ N $. This of course does not change the determinant, and we find \begin{equation}
	\Psi_F(x)=\sqrt{\frac{2}{L}}^N\left(\frac{1}{2i}\right)^N\prod_{i=1}^{N}(z_i-z_i^{-1})\begin{vmatrix}
	1&z_1+z_1^{-1}&\ldots&\sum_{k=0}^{N-2}z_1^{2k-N+1}&(z_1+z_1^{-1})^{N-1}\\
	1&z_2+z_2^{-1}&\ldots&\sum_{k=0}^{N-2}z_2^{2k-N+1}&(z_2+z_2^{-1})^{N-1}\\
	\vdots&\vdots&\ddots&\vdots&\vdots\\
	1&z_N+z_N^{-1}&\ldots&\sum_{k=0}^{N-2}z_N^{2k-N+1}&(z_N+z_N^{-1})^{N-1}\\
	\end{vmatrix}.
	\end{equation}
	Now for $ i=1 $ to $ N-2 $ we add $ \left(\binom{N-2}{i}-\binom{N-2}{i-1}\right) $ times column $ N-1-i $ to column $ N-1 $, continue this process, \ie for $ j=3 $ to $ N $: for $ i=1 $ to $ N-j $ add  $ \left(\binom{N-j}{i}-\binom{N-j}{i-1}\right) $ times column $ N-1-i $ to column $ N-j+1 $. Then we obtain \begin{equation}
	\Psi_F(x)=\sqrt{\frac{2}{L}}^N\left(\frac{1}{2i}\right)^N\prod_{i=1}^{N}(z_i-z_i^{-1})\begin{vmatrix}
	1&z_1+z_1^{-1}&(z_1+z_1^{-1})^2&\ldots&(z_1+z_1^{-1})^{N-1}\\
	1&z_2+z_2^{-1}&(z_2+z_2^{-1})^2&\ldots&(z_2+z_2^{-1})^{N-1}\\
	\vdots&\vdots&\vdots&\ddots&\vdots\\
	1&z_N+z_N^{-1}&(z_N+z_N^{-1})^2&\ldots&(z_N+z_N^{-1})^{N-1}\\
	\end{vmatrix}.
	\end{equation}
	The determinant is recognized as a Vandermonde determinant and thus we have \begin{equation}
	\begin{aligned}
	\Psi_F(x)&=\sqrt{\frac{2}{L}}^N\left(\frac{1}{2i}\right)^N\prod_{k=1}^{N}(z_k-z_k^{-1})\prod_{i<j}^{N}\left((z_i+z_i^{-1})-(z_j+z_j^{-1})\right)\\
	&=2^{\binom{N}{2}}\sqrt{\frac{2}{L}}^N\prod_{k=1}^{N}\sin\left(\frac{\pi}{L}x_k\right)\prod_{i<j}^{N}\left[\cos\left(\frac{\pi}{L}x_i\right)-\cos\left(\frac{\pi}{L}x_j\right)\right]\\
	&=-2^{\binom{N}{2}+1}\sqrt{\frac{2}{L}}^N\prod_{k=1}^{N}\sin\left(\frac{\pi}{L}x_k\right)\prod_{i<j}^{N}\sin\left(\frac{\pi(x_i-x_j)}{2L}\right)\sin\left(\frac{\pi(x_i+x_j)}{2L}\right)
	.
	\end{aligned}
	\end{equation}
	
	\subsubsection{Reduced density matrices}
	We compute the one-particle reduced density matrix of the free Fermi ground state with Dirichlet b.c. in the usual way\begin{equation}
	\begin{aligned}
	&\gamma^{(1)}(x,y)=\frac{2}{L}\sum_{j=1}^{N}\sin\left(\frac{\pi}{L}jx\right)\sin\left(\frac{\pi}{L} jy\right)=\frac{\sin\left(\pi\left(\rho+\frac{1}{2L}\right)(x-y)\right)}{2L\sin\left(\frac{\pi}{2L}(x-y)\right)}-\frac{\sin\left(\pi\left(\rho+\frac{1}{2L}\right)(x+y)\right)}{2L\sin\left(\frac{\pi}{2L}(x+y)\right)}.
	\end{aligned}
	\end{equation}
	Of course Wick's theorem applies to comupte a general $ n $-particle reduced matrix.
	\subsubsection{Taylor's theorem}
	For later use, we define the one particle reduced density matrix $ \gamma^{(1)}(x,y) $ as well as the translation invariant part $ \tilde{\gamma}^{(1)}(x,y) $ \begin{equation}
	\begin{aligned}
	\gamma^{(1)}(x,y)&=\frac{\pi}{L}\left(D_{N}\left(\pi\frac{x-y}{L}\right)-D_{N}\left(\pi\frac{x+y}{L}\right)\right),\\
	\tilde{\gamma}^{(1)}(x,y)&\coloneqq \frac{\pi}{L}D_{N}\left(\pi \frac{x-y}{L}\right)
	%	\tilde{\gamma}(x,y)&\coloneqq \begin{cases}
	%	\gamma^{(1)}_{\text{per}}(x,y)=\frac{2\pi}{L}D_{(N-1)/2}\left(2\pi\frac{x-y}{L}\right),\qquad &N\text{ odd}\\
	%	\gamma^{(1)}_{\text{anti-per}}(x,y)=\frac{2\pi}{L}\left(\euler{i\pi(x-y)}D_{N/2-1}\left(2\pi\frac{x-y}{L}\right)+\frac{1}{2\pi}\euler{-i\pi N(x-y)}\right),\qquad &N\text{ even}
	%	\end{cases}
	\end{aligned}
	\end{equation}
	where $ D_n(x)=\frac{1}{2\pi}\sum_{k=-n}^{n}\euler{ikx}=\frac{\sin((n+1/2)x)}{2\pi\sin(x/2)} $ is the Dirichlet kernel. One obvious consequence is that $ \abs{\partial_{x}^{k_1}\partial_{y}^{k_2}\gamma^{(1)}(x,y)}\leq \frac{1}{\pi}(2N)^{k_1+k_2+1}\left(\frac{\pi}{L}\right)^{k_1+k_2+1}=\pi^{k_1+k_2}(2\rho)^{k_1+k_2+1} $. This bound will allow us to Taylor expand any $ \gamma^{(k)} $, as all derivatives are uniformly bounded by a constant times some power of $ \rho $. In fact the relevant power of $ \rho $ can be directly obtained from dimensional analysis. Alternatively Taylor expanding may be thought of a using the mean value theorem mulitple times.
	\subsection{Some useful bounds}
	\begin{lemma}\label{Lemma rho2 bound}
		$ \rho^{(2)}(x_1,x_2)=\left(\frac{\pi^2}{3}\rho^4+f(x_2)\right)(x_1-x_2)^2+\mathcal{O}(\rho^6(x_1-x_2)^4) $ with $ \int \abs{f(x_2)}\diff x_2\leq \textnormal{ const. }\rho^3\log(N) $.
	\end{lemma}
	\begin{proof}
		Notice that with periodic/anti-periodic b.c. we have by translation invariance $ \tilde{\gamma}^{(1)}(x,y)-(\rho+1/(2L))=\frac{\pi^2}{6}(\rho^4+\rho^3\mathcal{O}(1/L))(x_1-x_2)^2+\mathcal{O}(\rho^4(x_1-x_2)^4) $. Furhtermore, we have $ \gamma^{(1)}(x_1,x_2)-\rho^{(1)}\left((x_1+x_2)/2\right)=\tilde{\gamma}^{(1)}(x_1,x_2)-(\rho+1/(2L)) $. Now by Wick's theorem we find \begin{equation}
		\rho^{(2)}(x_1,x_2)=\rho^{(1)}(x_1)\rho^{(1)}(x_2)-\gamma^{(1)}(x_1,x_2)\gamma^{(1)}(x_2,x_1).
		\end{equation}
		Using that $ \gamma^{(1)} $ is symmetric, and that \begin{equation}
		\begin{aligned}
		\rho^{(1)}(x_1)=\rho^{(1)}((x_1+x_2)/2)&+\rho^{(1)\prime}((x_1+x_2)/2)\frac{x_1-x_2}{2}\\&+\frac{1}{2}\rho^{(1)\prime\prime}((x_1+x_2)/2)\left(\frac{x_1-x_2}{2}\right)^2+\mathcal{O}(\rho^4(x_1-x_2)^3),
		\end{aligned}
		\end{equation}
		\begin{equation}
		\begin{aligned}
		\rho^{(1)}(x_2)=\rho^{(1)}((x_1+x_2)/2)&+\rho^{(1)\prime}((x_1+x_2)/2)\frac{x_2-x_1}{2}\\&+\frac{1}{2}\rho^{(1)\prime\prime}((x_1+x_2)/2)\left(\frac{x_1-x_2}{2}\right)^2+\mathcal{O}(\rho^4(x_1-x_2)^3),
		\end{aligned}
		\end{equation}
		where both expressions can be expanded further if needed, we see that \begin{equation}
		\begin{aligned}
		\rho^{(2)}(x_1,x_2)=\rho^{(1)}((x_1+x_2)/2)^2-\gamma^{(1)}(x_1,x_2)^2-\left[\rho^{(1)\prime}((x_1+x_2)/2)\right]^2\left(\frac{x_1-x_2}{2}\right)^2\\+\rho^{(1)}((x_1+x_2)/2)\rho^{(1)\prime\prime}((x_1+x_2)/2)\left(\frac{x_1-x_2}{2}\right)^2+\mathcal{O}(\rho^6(x_1-x_2)^4)
		\end{aligned}
		\end{equation}
		Notice that $ \mathcal{O}(\rho^5(x_1-x_2)^3) $ terms must cancel due to symmetry.\\
		Now use the fact that $ 0\leq\rho^{(1)}\leq 2\rho $, and that $ \rho^{(1)\prime}:[0,L]\to \R $, and that $\int_{[0,L]}\abs{\rho^{(1)\prime\prime}}\leq \text{const. }\rho^2\log(N) $ and $\int_{[0,L]}\abs{\rho^{(1)\prime}}\leq \text{const. }\rho\log(N) $, which follows from the bound on Dirichlet's kernel $ \norm{D_N^{(k)}}_{L^1([0,2\pi])}\leq \text{const. }N^{k}\log(N) $, to conclude that
		\begin{equation}
		\begin{aligned}
		\rho^{(2)}(x_1,x_2)=\rho^{(1)}((x_1+x_2)/2)^2-\gamma^{(1)}(x_1,x_2)^2+g_1(x_1+x_2)(x_1-x_2)^2+\mathcal{O}(\rho^6(x_1-x_2)^4),
		\end{aligned}
		\end{equation}
		for some function $ g_1 $ satisfying $ \int_{[0,L]}\abs{g_1}\leq \text{const. }\rho^3\log(N)$.
		Furthermore, notice that 
		\begin{equation}
		\begin{aligned}
		&\rho^{(1)}((x_1+x_2)/2)^2-\gamma^{(1)}(x_1,x_2)^2\\
		&\hspace{1cm}=(\rho^{(1)}((x_1+x_2)/2)-\gamma^{(1)}(x_1,x_2))(\rho^{(1)}((x_1+x_2)/2)+\gamma^{(1)}(x_1,x_2))\\&\hspace{1cm}
		=\left[\rho+1/(2L)-\tilde{\gamma}^{(1)}(x_1,x_2)\right]\left[-\rho-1/(2L)+\tilde{\gamma}^{(1)}(x_1,x_2)+2\rho^{(1)}((x_1+x_2)/2)\right]\\&\hspace{1cm}
		=-\left[\rho+1/(2L)-\tilde{\gamma}^{(1)}(x_1,x_2)\right]^2+2\left[\rho+1/(2L)-\tilde{\gamma}^{(1)}(x_1,x_2)\right]\rho^{(1)}((x_1+x_2)/2)\\&\hspace{1cm}
		= 2\left(\frac{\pi^2}{6}(\rho+1/(2L))^3(x_1-x_2)^2+\mathcal{O}(\rho^5(x_1-x_2)^4)\right)\left(\rho+\frac{1}{2L}-\frac{\pi}{L}D_{N}((x_1+x_2)/(2L))\right)\\&\hspace{1cm}
		=\frac{\pi^2}{3}\rho^4(x_1-x_2)^2+g_2(x_1-x_2)(x_1-x_2)^2+\mathcal{O}(\rho^6(x_1-x_2)^4),
		\end{aligned}
		\end{equation}
		where we have choosen $ g_2(x)=\frac{\pi^2}{3}\rho^3\left(\frac{\text{const.}}{2L}+\abs{\frac{\pi}{L}D_N(x/(2L))} \right) $ which clearly satifies $  \int_{[0,L]} g_2\leq \text{const. } \rho^3 \log(N) $.
		Thus we conclude that \begin{equation}
		\rho^{(2)}(x_1,x_2)=\left(\frac{\pi^2}{3}\rho^4+f(x_2)\right)(x_1-x_2)^2+\mathcal{O}(\rho^6(x_1-x_2)^4)
		\end{equation}
		with $ f=g_1+g_2 $, satifying $ \int_{[0,L]} \abs{f}\leq \text{const. } \rho^3 \log(N) $
	\end{proof}
	\begin{lemma}\label{LemmaDensityBounds}
		We have the following bounds\begin{equation}
		\begin{aligned}
		\rho^{(3)}(x_1,x_2,x_3)&\leq \textnormal{const. }\rho^9(x_1-x_2)^2(x_2-x_3)^2(x_1-x_3)^2,\\
		\rho^{(4)}(x_1,x_2,x_3,x_4)&\leq \textnormal{const. }\rho^8(x_1-x_2)^2(x_3-x_4)^2\\
		\sum_{i=1}^{2}\partial_{y_i}^2\gamma^{(2)}(x_1,x_2,y_1,y_2)\rvert_{y=x}&\leq \textnormal{const. } \rho^{6}(x_1-x_2)^2,\\
		\abs{\partial_{y_1}^2\left(\frac{\gamma^{(2)}(x_1,x_2,y_1,y_2)}{y_1-y_2}\right)\Bigg\rvert_{y=x}}&\leq \textnormal{const. } \rho^{6}\abs{x_1-x_2},\\
		\sum_{i=1}^{2}(-1)^{i-1}\partial_{y_i}\left(\frac{\gamma^{(2)}(x_1,x_2,y_1,y_2)}{y_1-y_2}\right)\Bigg\rvert_{y=x}&\leq \textnormal{const. } \rho^{6}(x_1-x_2)^2.
		\end{aligned}
		\end{equation}
	\end{lemma}
	\begin{proof}
		The bounds follows straightforwardly from Taylor's theorem and symmetries of the left-hand sides. As an example, consider $ \sum_{i=1}^{2}\partial_{y_i}^2\gamma^{(2)}(x_1,x_2,y_1,y_2)\rvert_{y=x} $. Notice first that $ \sum_{i=1}^{2}\partial_{y_i}^2\gamma^{(2)}(x_1,x_2,y_1,y_2) $ is anti-symmetric in $ (x_1,x_2) $ and in $ (y_1,y_2) $. As we have previously argued that all derivatives of $ \gamma^{(n)} $ are bounded by a constant times $ \rho^k $ for some $ k\in\mathbb{N} $, we can clearly Taylor expand $ \gamma^{(2)} $. Taylor expanding $ x_1 $ around $ x_2 $ and similarly $ y_1 $ around $ y_2 $ we see by the anti-symmetry that $ \sum_{i=1}^{2}\partial_{y_i}^2\gamma^{(2)}(x_1,x_2,y_1,y_2)\leq \text{const. }\rho^6(x_1-x_2)(y_1-y_2) $, where the power of $ \rho $ can be found by simple dimensional analysis.
	\end{proof}
	\begin{lemma} \label{LemmaDensityBounds2}
		We have the following bounds
		\begin{equation}
		\begin{aligned}
		\sum_{i=1}^{3}\left(\partial_{x_i}\partial_{y_i}\gamma^{(3)}(x_1,x_2,x_3;y_1,y_2,y_3)\right)\Bigg\vert_{y=x}&\leq \textnormal{const. }\rho^9(x_2-x_3)^2(x_1-x_2)^2,\\
		\abs{\sum_{i=1}^{3}\left(\partial_{y_i}^2\gamma^{(3)}(x_1,x_2,x_3;y_1,y_2,y_3)\right)\Bigg\vert_{y=x}}&\leq\textnormal{const. }\rho^9(x_1-x_2)^2(x_2-x_3)^2,\\
		\abs{\left[\partial_{y}\gamma^{(4)}(x_1,x_2,x_3,x_4;y,x_2,x_3,x_4)\bigg\vert_{y=x_1}\right]_{x_1=x_2-b}^{x_1=x_2+b}}&\leq \textnormal{const. }\rho^8b(x_3-x_4)^2.
		\end{aligned}
		\end{equation}
	\end{lemma}
	\begin{proof}
		The proof follows straigthforwardly from Taylor's theorem and symmetries of the left-hand sides. 
	\end{proof}
	\begin{remark}
		We do not claim that the lemmas \ref{LemmaDensityBounds} and \ref{LemmaDensityBounds2} are in any way optimal bounds, and many of them can indeed easily be improved by same proof technique using more symmetries. However, they are sufficient for our needs.
	\end{remark}
	\subsection{Estimating $ E_1 $}
		Recall the definition \begin{equation}
		E_1\coloneqq\binom{N}{2}\int_{A_{12}} \sum_{i=1}^{N}\abs{\nabla_i\tilde{\Psi}}^2+\sum_{i<j}^{N}(v_{\text{reg}})_{ij}\abs{\tilde{\Psi}}^2-\sum_{i=1}^{N}\abs{\nabla_i\tilde{\Psi}_F}^2
		\end{equation}
		We now prove the result \begin{lemma}\label{LemmaE1Bound}
			\begin{equation}
			E_1\leq E_0 \left(2\rho a\frac{b}{b-a}+ \textnormal{const.}\ N(\rho b)^3\left[ 1+ \rho b^2\int v_{\textnormal{reg}}\right]\right)
			\end{equation}
		\end{lemma}
		\begin{proof}
		We estimate $ E_1 $ by splitting it in four terms $ E_1=E_1^{(1)}+E_1^{(2)}+E_1^{(3)}+E_1^{(4)} $. First we have \begin{equation}
		\begin{aligned}
		E_1^{(1)}&=2\binom{N}{2}\int_{A_{12}}\abs{\nabla_1\tilde{\Psi}}^2\\&
		=2\binom{N}{2}\int_{A_{12}}\overline{\tilde{\Psi}}\left( -\Delta_1 \tilde{\Psi} \right)+2\binom{N}{2}\int\left[\overline{\tilde{\Psi}}\nabla_1\tilde{\Psi}\right]_{x_1=x_2-b}^{x_1=x_2+b}\diff \bar{x}^1.
		\end{aligned}
		\end{equation}
		The boundary term can be explicitly calculated and we find \begin{equation}
		\begin{aligned}
		2\binom{N}{2}\int\left[\overline{\tilde{\Psi}}\nabla_1\tilde{\Psi}\right]_{x_1=x_2-b}^{x_1=x_2+b}\diff \bar{x}^1=\int\left[\frac{\omega(x_1-x_2)}{\abs{x_1-x_2}}\partial_{x_1}\left(\frac{\omega(x_1-x_2)}{\abs{x_1-x_2}}\right)\rho^{(2)}(x_1,x_2)\right]_{x_2-b}^{x_2+b}\diff x_2\\+\int\left[\left(\frac{\omega(x_1-x_2)}{\abs{x_1-x_2}}\right)^2\partial_{x_1}\left(\gamma^{(2)}(x_1,x_2;y,x_2)\right)\bigg\vert_{y=x_1}\right]_{x_2-b}^{x_2+b}\diff x_2.
		\end{aligned}
		\end{equation}
		Since the continuously differentiable function $ \frac{\omega(x)}{\abs{x_1-x_2}}=\frac{\abs{x_1-x_2}-a}{b-a}\frac{b}{\abs{x_1-x_2}} $ for $ \abs{x_1-x_2}>b $, we see that \begin{equation}
		\partial_{x_1}\left(\frac{\omega(x_1-x_2)}{\abs{x_1-x_2}}\right)\bigg\vert_{x=x_2\pm b}=\pm \frac{\frac{b}{b-a}-1}{ b}=\pm\frac{a}{b(b-a)}.
		\end{equation}
		Using lemma \ref{Lemma rho2 bound}, we find \begin{equation}
		\int\left[\frac{\omega(x_1-x_2)}{\abs{x_1-x_2}}\partial_{x_1}\left(\frac{\omega(x_1-x_2)}{\abs{x_1-x_2}}\right)\rho^{(2)}(x_1,x_2)\right]_{x_2-b}^{x_2+b}\diff x_2\leq 2a\frac{b}{b-a} N\frac{\pi^2}{3}\rho^3\left(1+\text{const. }\frac{\log(N)}{N}\right)
		\end{equation}
		
		Furthermore, we denote \begin{equation}\label{EqGammaDeriv2.}
		\begin{aligned}
		&\int\left[\left(\frac{\omega(x_1-x_2)}{\abs{x_1-x_2}}\right)^2\partial_{x_1}\left(\gamma^{(2)}(x_1,x_2;y,x_2)\right)\bigg\vert_{y=x_1}\right]_{x_2-b}^{x_2+b}\diff x_2\\
		&\quad=\int\left[\partial_{x_1}\left(\gamma^{(2)}(x_1,x_2;y,x_2)\right)\bigg\vert_{y=x_1}\right]_{x_2-b}^{x_2+b}\diff x_2=:\kappa_1
		\end{aligned}
		\end{equation}
		Thus we have \begin{equation}
		E_1^{(1)}=\frac{\pi^2}{3}N\rho^3 (2a)\frac{b}{b-a}+\kappa_1+2\binom{N}{2}\int_{A_{12}}\overline{\tilde{\Psi}}(-\Delta_1\tilde{\Psi})
		\end{equation}
		Another contribution to $ E_1 $ is \begin{equation}
		\begin{aligned}
		E_1^{(2)}&=-\binom{N}{2}\int_{A_{12}}\left(2\abs{\nabla_1\Psi_F}^2+\sum_{i=3}^{N}\abs{\nabla_i\Psi_F}^2\right)\\&=-\binom{N}{2}\int_{A_{12}}\sum_{i=1}^{N}\overline{\Psi_F}(-\Delta_i\Psi_F)-2\binom{N}{2}\int\left[\overline{\Psi_F}\nabla_1\Psi_F\right]_{x_1=x_2-b}^{x_1=x_2+b}\\
		&=-E_0\binom{N}{2}\int_{A_{12}}\abs{\Psi_F}^2-\underbrace{\int\left[\partial_y\gamma^{(2)}(x_1,x_2;y,x_2)\vert_{y=x_1}\right]_{x_2-b}^{x_2+b} \diff x_2}_{\kappa_1}
		\end{aligned}
		\end{equation}
		Again using lemma \ref{Lemma rho2 bound} we find \begin{equation}
		E_1^{(2)}=-\text{const. }E_0 N\rho^3b^3-\kappa_1.
		\end{equation}
		The last contributions are\\ $ E^{(3)}_1=\binom{N}{2}\int_{A_{12}} \sum_{i<j}^{N}(v_{\text{reg}})_{ij}\abs{\tilde{\Psi}}^2=\binom{N}{2}\int_{A_{12}}v_{12}\abs{\tilde{\Psi}}^2+2\binom{N}{2}\int_{A_{12}} \sum_{2\leq i<j}^{N}(v_{\text{reg}})_{ij}\abs{\tilde{\Psi}}^2 $ and\\ $ E_1^{(4)}=\int_{A_{12}}\sum_{i=3}^{N}\abs{\nabla_i\tilde{\Psi}}^2 $.
		First we notice that \begin{equation}
		\begin{aligned}
		&\binom{N}{2}\int_{A_{12}} \sum_{2\leq i<j}^{N}(v_{\text{reg}})_{ij}\abs{\tilde{\Psi}}^2\\&\quad\leq \text{const. }b^2 \left(\int_{\{\abs{x_1-x_2}<b\}\cap\supp((v_{\text{reg}})_{34})}v_{\text{reg}}(\abs{x_3-x_4})\frac{1}{(x_1-x_2)^2}\rho^{(4)}(x_1,x_2,x_3,x_4)\right.\\
		&\qquad\qquad\qquad\left.+\int_{\{\abs{x_1-x_2}<b\}\cap\supp((v_{\text{reg}})_{23})}v_{\text{reg}}(\abs{x_2-x_3})\frac{1}{(x_1-x_2)^2}\rho^{(3)}(x_1,x_2,x_3)\right).
		\end{aligned}
		\end{equation}
		By lemma \ref{LemmaDensityBounds} we have
		\begin{equation}
		\begin{aligned}
		&\binom{N}{2}\int_{A_{12}} \sum_{2\leq i<j}^{N}(v_{\text{reg}})_{ij}\abs{\tilde{\Psi}}^2\\&\quad\leq \text{const. } \left(N^2(\rho b)^3\rho^3\int x^2 v_{\text{reg}}(x)\diff x+N(\rho b)^3 \rho^5 \int x^4 v_{\text{reg}}(x)\diff x+N(\rho b)^4\rho^4 \int x^3 v_{\text{reg}}(x)\diff x\right.\\
		&\qquad \qquad \qquad \qquad\hspace{6cm}\left.+N(\rho b)^5 \rho^3 \int x^2 v_{\text{reg}}(x)\diff x\right)\\
		&\quad \leq \text{const. } N^2(\rho b)^5\rho \int v_{\text{reg}}=\text{const. }E_0 N (\rho b)^3 \left(\rho b^2\int v_{\text{reg}}\right),
		\end{aligned}
		\end{equation}
		and then we find that \begin{equation}
		\begin{aligned}
		E_1&=E_1^{(1)}+E_1^{(2)}+E_1^{(3)}+E_1^{(4)}\\&\leq \frac{2\pi^2}{3}N\rho^3 a\frac{b}{b-a}+2\binom{N}{2}\int_{A_{12}}\left(\overline{\tilde{\Psi}}(-\Delta_1)\tilde{\Psi}+\frac{1}{2}\sum_{i=3}^{N}\abs{\nabla_i\tilde{\Psi}}^2+\frac{1}{2}v_{12}\abs{\tilde{\Psi}}^2\right)\\&\qquad \qquad +E_0N(\rho b)^3\text{const. }\left(1+\rho b^2 \int v_{\text{reg}}\right).
		\end{aligned}
		\end{equation}
		Using the two body scattering equation this implies \begin{equation}
		\begin{aligned}
		E_1&\leq \frac{2\pi^2}{3}N\rho^3 a\frac{b}{b-a}+2\binom{N}{2}\int_{A_{12}}\frac{\overline{\Psi_F}}{(x_1-x_2)}\omega^2(-\Delta_1)\frac{\Psi_F}{(x_1-x_2)}\\&\quad+2\binom{N}{2}\int_{A_{12}}\frac{\overline{\Psi_F}}{(x_1-x_2)}\omega(\nabla_1\omega)\nabla_1\frac{\Psi_F}{(x_1-x_2)}\\
		&\quad +\binom{N}{2}\int_{A_{12}}\sum_{i=3}^{N} \frac{\overline{\Psi_F}}{(x_1-x_2)}\frac{\omega^2}{(x_1-x_2)}(-\Delta_i)\Psi_F
		\\&\quad+\text{const. }E_0 N (\rho b)^3 \left(1+\rho b^2\int v_{\text{reg}}\right).
		\end{aligned}
		\end{equation}
		Furhtermore we have \begin{equation}
		\begin{aligned}
		&\binom{N}{2}\int_{A_{12}}\sum_{i=3}^{N} \overline{\tilde{\Psi}}\frac{\omega}{(x_1-x_2)}(-\Delta_i)\Psi_F\\&\quad=E_0\binom{N}{2}\int_{A_{12}}\left\lvert\frac{\omega}{(x_1-x_2)}\Psi_F\right\rvert^2-2\binom{N}{2}\int_{A_{12}} \overline{\tilde{\Psi}}\frac{\omega}{(x_1-x_2)}(-\Delta_1)\Psi_F,
		\end{aligned}
		\end{equation}
		so by lemma \ref{Lemma rho2 bound} it follows that
		\begin{equation}
		\binom{N}{2}\int_{A_{12}}\left\lvert\frac{\omega}{(x_1-x_2)}\Psi_F\right\rvert^2\leq b^2\int_{\{\abs{x_1-x_2}<b\}} \frac{\rho^{(2)}(x_1,x_2)}{\abs{x_1-x_2}^2}\diff x_1\diff x_2\leq  \text{const. } b^2\rho^4  L b=\text{const. }N\rho^3 b^3,
		\end{equation}
		and by lemma \ref{LemmaDensityBounds} it follows that \begin{equation}
		\begin{aligned}
		2\binom{N}{2}\int_{A_{12}} \overline{\tilde{\Psi}}\frac{\omega}{(x_1-x_2)}(-\Delta_1)\Psi_F&=\frac12\sum_{i=1}^{2}\int_{A_{12}}\abs{\frac{\omega}{x_1-x_2}}^2\left[\partial^2_{y_i}\gamma^{(2)}(x_1,x_2,y_1,y_2)\right]\Big\rvert_{y=x}\\&\leq \text{const. } N\rho^2(\rho b)^3,
		\end{aligned}
		\end{equation}
		and we find \begin{equation}
		\binom{N}{2}\int_{A_{12}}\sum_{i=3}^{N} \overline{\tilde{\Psi}}\frac{\omega}{(x_1-x_2)}(-\Delta_i)\Psi_F\leq \text{const. } E_0 N(\rho b)^3.
		\end{equation}
		Finally, again by lemma \ref{LemmaDensityBounds} \begin{equation}
		\begin{aligned}
		2\binom{N}{2}\int_{A_{12}}\overline{\tilde{\Psi}}\omega(-\Delta_1)\frac{\Psi_F}{(x_1-x_2)}&=\int_{A_{12}}\abs{\frac{\omega^2}{x_1-x_2}}\left[\partial^2_{y_1}\left(\frac{\gamma^{(2)}(x_1,x_2,y_1,y_2)}{(y_1-y_2)}\right)\right]\Big\rvert_{y=x}\\&\leq\text{const. }N\rho^2 (\rho b)^3,
		\end{aligned}
		\end{equation}
		and by using $ \Delta\omega=\frac{1}{2}v\omega\geq0 $ which implies $ 0\leq\omega'(x)\leq \omega'(b)=\frac{b}{b-a} $ for $ \abs{x}<b $, we find that \begin{equation}
		\begin{aligned}
		2\binom{N}{2}\int_{A_{12}}\overline{\tilde{\Psi}}(\nabla_1\omega)\nabla_1\left(\frac{\Psi_F}{(x_1-x_2)}\right)&\leq\frac12\sum_{i=1}^{2}\int_{A_{12}}\abs{\frac{\omega}{x_1-x_2}}(-1)^{i-1}\omega'(x_1-x_2)\partial_{y_i}\left(\frac{\gamma^{(2)}(x_1,x_2,y_1,y_2)}{y_1-y_2}\right)\\
		&\leq \text{const. }\frac{b}{b-a}N\rho^2(\rho b)^3.
		\end{aligned}
		\end{equation}
		Combining everything we get \begin{equation}
		E_1\leq E_0 \left(2\rho a\frac{b}{b-a}+ \text{const.}\ N(\rho b)^3\left[ 1+ \rho b^2\int v_{\text{reg}}\right]\right)
		\end{equation}
	\end{proof}
%		\subsubsection{A remark about the hard core potential}
%		Notice that it appears that we cannot deal with the hard core case. However, in the above calculation we threw away the term $ \int_{A_{12}\setminus B_{12}} \sum_{2\leq i<j}^{N}v_{ij}\abs{\Psi}^2 $. Adding this back in, we get the error $ \binom{N}{2}\int_{B_{12}} \sum_{2\leq i<j}^{N}v_{ij}\abs{\tilde{\Psi}}^2 $ instead of $ \binom{N}{2}\int_{A_{12}} \sum_{2\leq i<j}^{N}v_{ij}\abs{\tilde{\Psi}}^2 $. In doing so, we immediately see that in the presence of a hard core potential wall, $ \tilde{\Psi} $ is zero, whenever two coordinates are within the hard core. Thus we may replace $ v_{ij} $ by $ \tilde{v}_{ij} $ which is zero whenever $ \abs{x_i-x_j} $ is within the range of the hard core. Thus our result, generalizes to the case of a hard core, plus an integrable potential.
		\subsection{Estimating $ E_2 $}
		Recall that $ E_2=E_2^{(1)}+E_2^{(2)} $ with \begin{equation}
		\begin{aligned}
		E_2^{(1)}&=\binom{N}{2}2N\int_{A_{12}\cap A_{13}}\sum_{i=1}^{N}\abs{\nabla_i\Psi_F}^2\\ E_2^{(2)}&=\binom{N}{2}\binom{N-2}{2}\int_{A_{12}\cap A_{34}}\sum_{i=1}^{N}\abs{\nabla_i\Psi_F}^2.
		\end{aligned}
		\end{equation}
		We now prove the result
		\begin{lemma}\label{LemmaE2Bound}
			\begin{equation}
			E_2\leq E_0(N(\rho b)^4+N^2(\rho b)^6)
			\end{equation}
		\end{lemma}
		\begin{proof}
		To estimate $ E_2^{(1)} $ and $ E_2^{(2)} $, we first split them in two terms each and use partial integration. Consider first $ E_2^{(1)} $: 
		\begin{equation}
		\begin{aligned}
		E_2^{(1)}&=\binom{N}{2}2N\int_{A_{12}\cap A_{13}}\sum_{i=1}^{N}\abs{\nabla_i\Psi_F}^2\\
		&=\binom{N}{2}2N\left(\int_{A_{12}\cap A_{13}}\abs{\nabla_1\Psi_F}^2+2\int_{A_{12}\cap A_{13}}\abs{\nabla_2\Psi_F}^2\right)+\binom{N}{2}2N\int_{A_{12}\cap A_{13}}\sum_{i=4}^{N}\abs{\nabla_i\Psi_F}^2
		\end{aligned}
		\end{equation}
		For the second term, we can perform partial integration directly, in order to obtain \begin{equation}
		\begin{aligned}
		\binom{N}{2}&2N\int_{A_{12}\cap A_{13}}\sum_{i=4}^{N}\abs{\nabla_i\Psi_F}^2=\binom{N}{2}2N\int_{A_{12}\cap A_{13}}\sum_{i=4}^{N}\overline{\Psi_F}(-\Delta_i\Psi_F)\\
		&\leq E_0 N^3\int_{A_{12}\cap A_{23}}\abs{\Psi_F}^2-N^3\int_{A_{12}\cap A_{23}}\sum_{i=1}^{3}\overline{\Psi_F}(-\Delta_i\Psi_F)\\&\leq 2E_0\int_{[0,L]}\int_{[x_2-b,x_2+b]}\int_{[x_2-b,x_2+b]}\rho^{(3)}(x_1,x_2,x_3)\diff x_3\diff x_1\diff x_2-N^3\int_{A_{12}\cap A_{23}}\sum_{i=1}^{3}\overline{\Psi_F}(-\Delta_i\Psi_F)
		\end{aligned}
		\end{equation}
		Using lemma \ref{LemmaDensityBounds} we find \begin{equation}
		\begin{aligned}
		2E_0\int_{[0,L]}\int_{[x_2-b,x_2+b]}\int_{[x_2-b,x_2+b]}\rho^{(3)}(x_1,x_2,x_3)\diff x_3\diff x_1\diff x_2\leq NE_0(\rho b)^6
		\end{aligned}
		\end{equation}
		Furthermore, we find by lemma \ref{LemmaDensityBounds2} \begin{equation}
		\binom{N}{2}2N\int_{A_{12}\cap A_{13}}\sum_{i=1}^{3}\left(\abs{\nabla_i\Psi_F}^2-\overline{\Psi_F}(-\Delta_i\Psi_F)\right)\leq \text{const. }\rho^9 L b^6=\text{const. }E_0 (b\rho)^6.
		\end{equation}
		Collecting everything we find \begin{equation}
		E_2^{(1)}\leq \text{const. }NE_0(\rho b)^6.
		\end{equation}
		To estimate $ E_2^{(2)} $, we use integration by parts\begin{equation}
		\begin{aligned}
		E_2^{(2)}&=\binom{N}{2}\binom{N-2}{2}\int_{A_{12}\cap A_{34}} \left(4\abs{\nabla_1\Psi_F}^2+\sum_{i=5}^{N}\abs{\nabla_i\Psi_F}^2\right)\\
		&=\binom{N}{2}\binom{N-2}{2}\left(4\int_{\abs{x_3-x_4}<b}\left[\overline{\Psi_F}\nabla_1\Psi_F\right]_{x_1=x_2-b}^{x_1=x_2+b} +\int_{A_{12}\cap A_{34}} \sum_{i=1}^{N}\overline{\Psi_F}(-\Delta_i\Psi_F)\right)\\
		&=4\int_{x_2\in[0,L]}\int_{\abs{x_3-x_4}<b}\left[\partial_{y_1}\gamma^{(4)}(x_1,x_2,x_3,x_4;y_1,x_2,x_3,x_4)\bigg\vert_{y_1=x_1}\right]_{x_1=x_2-b}^{x_1=x_2+b}+E_0\int_{A_{12}\cap A_{34}}\rho^{(4)}(x_1,..,x_4).
		\end{aligned}
		\end{equation}
		By lemma \ref{LemmaDensityBounds2} we get \begin{equation}
		4\int_{x_2\in[0,L]}\int_{\abs{x_3-x_4}<b}\left[\partial_{y_1}\gamma^{(4)}(x_1,x_2,x_3,x_4;y_1,y_2,y_3,y_4)\bigg\vert_{y_1=x_1}\right]_{x_1=x_2-b}^{x_1=x_2+b}=\text{const. }E_0 N (\rho b)^4
		\end{equation}
		Furthermore, by lemma \ref{LemmaDensityBounds2} again, it follows that \begin{equation}
		E_0\int_{A_{12}\cap A_{34}}\rho^{(4)}(x_1,..,x_4)\leq \text{const. } E_0 N^2(\rho b)^6.
		\end{equation}
	\end{proof}
		
	\subsection{Localization with Dirichlet b.c.}
	We will in this section localize in smaller boxes, in order to have better control on the error. The localization is straigthforward with Dirichlet boundary conditions, as gluing the wavefunctions for each box together is simple, since the wavefunctions vanish at the boundaries. Thus we consider the state $ \Psi_{\text{full}}=\prod_{i=1}^{M}\Psi_{\ell}(x^i_1,...,x^i_{\tilde{N}}) $, where $ (x_1^i,...,x_{\tilde{N}}^i) $ are the particles in box $ i $ and $ \ell $ is the length of each box. Of course $ \cup_{i=1}^{M}\{x_1^i,...,x_{\tilde{N}}^i\}=\{x_1,...,x_N\} $ and $ \{x_1^i,...,x_{\tilde{N}}^i\}\cap\{x_1^j,...,x_{\tilde{N}}^j\}=\emptyset $ for $ i\neq j $, such that $ M\tilde{N}=N $. The boxes are of length $ \ell=L/M-b $, and are equally spaced through out $ [0,L] $ such that they are a distance of $ b $ from each other. This is to make sure that no particle interact between boxes. Combining lemmas \ref{LemmaE1Bound} and \ref{LemmaE2Bound}, the full energy is then bounded by \begin{equation}
	E\leq M e_0\left(1+2\tilde{\rho} a\frac{b}{b-a} + \text{const. } \tilde{N} (b\tilde{\rho})^3\left(1+\rho b^2\int v_{\text{reg}}\right)\right)/\norm{\Psi}^2
	\end{equation}
	with $ e_0=\frac{\pi^2}{3}\tilde{N}\tilde{\rho}^2(1+\text{const. }\frac{1}{\tilde{N}}) $ and $ \tilde{\rho}=\tilde{N}/\ell=\rho/(1-\frac{bM}{L})\leq \rho(1+2bM/L) $ for $ bM/L\leq 1/2 $. Clearly we have $ \norm{\Psi}^2\geq 1-\int_B\abs{\Psi_F}^2\geq 1-\int_{\abs{x_1-x_2}<b}\rho^{(2)}(x_1,x_2)\geq 1-\text{const. }\tilde{N}(\rho b)^3 $, where the last inequality follows from lemma \ref{Lemma rho2 bound}.
	Thus, choosing $ M $ such that $ bM/L\ll 1 $ we have \begin{equation}
	E\leq N\frac{\pi^2}{3}\rho^2\frac{\left(1+\frac{2\rho ab}{b-a}+\text{const. }\frac{M}{N}+\text{const. }2\rho abM/L+\text{const. }\tilde{N}(b\rho)^3\left(1+\rho b^2\int v_{\text{reg}}\right)\right)}{1-\tilde{N}(\tilde{\rho} b)^3}.
	\end{equation}
	Now in fact, we would choose $ \tilde{N}=N/M=\rho L/M\gg 1 $, \ie $ M/L\ll \rho $. Setting $ x=M/N $ we see that the error is \begin{equation}
	\text{const. }\left[(1+2\rho^2 ab^2/(b-a))x+x^{-1}(b\rho)^3\left(1+\rho b^2\int v_{\text{reg}}\right)\right],
	\end{equation}
	here we used that it will turn out to be the cases that $ \tilde{N}(\rho b)^3\leq 1/2 $ so that we have\\ $ 1/(1-\tilde{N}(\rho b)^3)\leq 1+2\tilde{N}(\rho b)^3 $.
	Optmizing in $ x $ we find $ x=M/N=\frac{(b\rho)^{3/2}\left(1+\rho b^2\int v_{\text{reg}}\right)^{1/2}}{1+2\rho^2 a b}\simeq(b\rho)^{3/2}\left(1+\rho b^2\int v_{\text{reg}}\right)^{1/2} $, which gives the error \begin{equation}
	\text{const. }(b\rho)^{3/2}\left(1+\rho b^2 \int v_{\textnormal{reg}}\right)^{1/2}.
	\end{equation}
	At last we choose $ b=R\coloneqq\max(\rho^{-1/5}\abs{a}^{4/5},R_0) $. Then for $(\rho \abs{a})^{1/5}\leq 1/2$ $$ \frac{b}{b-a}\leq1+2a/b\leq  1+2(\rho\abs{a})^{1/5}. $$
	Thus we arrive at the following result
	\begin{theorem}[Upper bound]\label{TheoremUpperBound}
		Let the two-body potential $ v\in L^{1}([0,L])+\textnormal{h.c.p.} $ be fixed, with two-body s-wave scattering length $ a $. Then bosonic $ N $-body ground state energy satisfies the upper bound\begin{equation}
		E\leq E_0\left(1+2\rho a + \mathcal{O}\left((R\rho)^{3/2}\left(1+\rho R^2 \int v_{\textnormal{reg}}\right)^{1/2}\right)\right),
		\end{equation}
		where $ E_0 $ is the free fermionic ground state energy.
	\end{theorem}
	here $ \textnormal{h.c.p} $ denotes the space of hard core potentials. Notice that $$ (R\rho)^{3/2}= \max\left((\rho \abs{a})^{6/5},(\rho R_0)^{3/2}\right)\leq  (\rho \abs{a})^{6/5}+(\rho R_0)^{3/2}. $$\\
	Notice also that the result \begin{equation}
	E\leq N\frac{\pi^2}{3}\left(1+2\rho a + \mathcal{O}\left((R\rho)^{3/2}\left(1+\rho R^2 \int v_{\textnormal{reg}}\right)^{1/2}\right)\right)
	\end{equation}
	holds only for $ N\geq (\rho b)^{-3/2} $, but since the free Fermi, $ E_0 $ energy also contains a correction of order $ \frac{1}{N} $, \ie $ E_0=N\frac{\pi^2}{3}\rho^2(1+\text{const. }1/N) $, the result remains true for $ N<(\rho b)^{-3/2}$.

	\section{Lower bound}
	We will in this section provide a lower bound for the one dimensional dilute Bose gas. The proof is based on a reduction to a Lieb Liniger model, and thus we will first recall some known features about this model
	\subsection{The Lieb Liniger model}
	Recall that the energy in thermodynamic limit of the the Lieb Liniger model (with periodic boundary conditions), is determined by the sytem of equation ((3.3) and (3.18)--(3.20) in \cite{PhysRev.130.1605})
	\begin{align}
	\rho^3 e(\gamma)&=\lim\limits_{\ell\to\infty,\ n/\ell\text{ fixed}}\frac{E_{LL}(n,\ell,c=\gamma n/\ell)}{\ell},\label{Eq1}\\
	e(\gamma)&=\frac{\gamma^3}{\lambda^3}\int_{-1}^{1}g(x)x^2\diff x,\label{Eq2}\\
	2\pi g(y)&=1+2\lambda\int_{-1}^{1}\frac{g(x)\diff x}{\lambda^2+(x-y)^2},\label{Eq3}\\
	\lambda&=\gamma\int_{-1}^{1}g(x)\diff x.\label{Eq4}
	\end{align}
	The first lemma provides a rigorous lower bound for the thermodynamic Lieb Liniger energy density. 
	\begin{lemma}[Lieb Liniger lower bound] \label{LemmaLL-LowerBound}
		Let $ \gamma>0 $, then
		\begin{equation}
		e(\gamma)\geq \frac{\pi^2}{3}\left(\frac{\gamma}{\gamma+2}\right)^2\geq \frac{\pi^2}{3}\left(1-\frac{4}{\gamma}\right).
		\end{equation}
	\end{lemma}
	\begin{proof}
		Neglecting $ (x-y)^2 $ in the denominator of \eqref{Eq3}, we see that $ g\leq \frac{1}{2\pi}+2\frac{1}{\lambda}\int_{-1}^{1}g(x)\diff x $. On the other hand $ \eqref{Eq4} $ shows that $ e(\gamma)=\frac{\int_{-1}^{1}g(x)x^2\diff x}{\left(\int_{-1}^{1}g(x)\diff x\right)^3} $. Hence we denote $ \int_{-1}^{1}g(x)\diff x=M $, and notice that we have $ g\leq \frac{1}{2\pi}\left(1+\frac{2M}{\lambda}\right) $. It is now easily verified that, $ \int_{-1}^{1}g(x)x^2\diff x $ with $ M $ fixed and $ g\leq \frac{1}{2\pi}\left(1+\frac{2M}{\lambda}\right)=\frac{1}{2\pi}\left(1+2\gamma^{-1}\right) $ is mininmized by $ g=K\chi_{[-M/(2K),M/(2K)]} $, with $ K=\frac{1}{2\pi}\left(1+\frac{2}{\gamma}\right) $. This gives us $ \int_{-1}^{1}g(x)x^2\diff x=\frac{M^3}{3K^2}$ so that we have $ e(\gamma)\geq \frac{1}{3K^2}=\frac{\pi^2}{3}\left(\frac{\gamma}{\gamma+2}\right)^2\geq \frac{\pi^2}{3}(1-\frac{4}{\gamma})$ for $ \gamma>0 $.
	\end{proof}
	The next result concerns finite volume correction to the thermodynamic limit. Since we are interested in a lower bound, we consider the Neumann boundary conditions case denoted by a superscript "$ N $". Dirichlet boundary conditions are denoted by superscript "$ D $"\\
	Now in order to compare energies with different boundary conditions we introduce a cutoff function $ h $, with the following properties:\begin{enumerate}
		\item $ h $ is real, symmetric, and continuously differentiable.
		\item $ h(x)=0 $ for $ \abs{x}>L/2+b $
		\item $ h(x)=1 $ for $ \abs{x}<L/2-b $
		\item $ h(L/2-x)^2+h(L/2+x)^2=1 $ for $ 0<x<b $.
		\item $ \abs{\frac{\diff h}{\diff x}}^2\leq \frac{1}{b^2} $, and $ h^2\leq 1 $
	\end{enumerate}
	Furthermore, we say that a symmetric, translation invariant measure, $ v $, is decreasing if $ v\circ \mathfrak{c}\geq v $ for any contraction $ \mathfrak{c} $
	\begin{lemma}[Robinson's bound]\label{LemmaRobinson}
		Let $ v$ be symmetric and decreasing, then for any $ b>0 $ we have \begin{equation}\label{EqRobinsonBound}
		E^D_{\Lambda_{L+2b}}\leq E^N_{\Lambda_L}+\frac{2n}{b^2}
		\end{equation}
	\end{lemma}
	\begin{proof}
		The idea of the prove is given in \cite{robinson2014thermodynamic} page 66, but we shall give a more explicit proof here. Let $ f\in \mathcal{D}(\mathcal{E}^N_{\Lambda_L}) $. Define $ \tilde{f} $ by extending $ f $ to $ \Lambda_{3L} $ by reflecting $ f $ across each face of its domain in $ \Lambda_{3L} $. Define then $ V:L^2(\Lambda_L)\to L^2(\Lambda_{L+2b})  $ by $ Vf(x):=\tilde{f}(x)\prod_{i=1}^{n}h(x_i) $. It is not hard to show that $ V $ is an isometry, this is shown in \cite{robinson2014thermodynamic} lemma 2.1.12. Also we clearly have $ Vf\in \mathcal{D}(\mathcal{E}^D_{\Lambda_{L+2b}})  $.  Let $ \psi $ be the ground state for $ \mathcal{E}^N_{\Lambda_L} $. Then we define the trial state $ \psi_{\text{trial}}=V\psi $. Without the potential, the bound \eqref{EqRobinsonBound} is obtained in lemma 2.1.13 of \cite{robinson2014thermodynamic}. Thus we need only prove that no energy is gained by by the potential in the trial state. To see this, define $ \tilde{\psi} $ to be $ \psi $ extended by reflection as above and notice that for $ \abs{x_2}<L/2-b $ we have \begin{equation}
		\begin{aligned}
		&\int_{-L/2-b}^{L/2+b}v(\abs{x_1-x_2})\abs{\tilde{\psi}(x)}^2h(x_1)^2h(x_2)^2\diff x_1\leq\\&\quad  \int_{-L/2+b}^{L/2-b}v(\abs{x_1-x_2})\abs{\tilde{\psi}(x)}^2\diff x_1+\sum_{s\in\{-1,1\}}s\int_{s(L/2-b)}^{s(L/2)}v(\abs{x_1-x_2})\abs{\tilde{\psi}(x)}^2(h(x)^2+h(L-x)^2)\diff x_1\\
		&\quad =\int_{-L/2}^{L/2}v(\abs{x_1-x_2})\abs{\tilde{\psi}(x)}^2\diff x_1
		\end{aligned}
		\end{equation}
		where we used that $ v $ is symmetric decreasing in the first inequality and that $ h(x)^2+h(L-x)^2=1 $ for $ L/2-b\leq x\leq L/2 $ which is just property 4.
		\begin{equation}
		\begin{aligned}
		&\sum_{(s_1,s_2)\in\{-1,1\}^2}s_1s_2\int_{L/2-s_1b}^{L/2}\int_{L/2-s_2b}^{L/2}v(\abs{x_1-x_2})\abs{\tilde{\psi}(x)}^2h(x_1)^2h(x_2)^2\diff x_2\diff x_1\\
		&\quad\quad =\sum_{(s_1,s_2)\in\{-1,1\}^2}\int_{0}^{b}\int_{0}^{b}v(\abs{s_1y_1-s_2y_2})\abs{\tilde{\psi}(L/2-s_1 y_1,L/2-s_2 y_2,\bar{x}^{1,2})}^2\\&\hspace{5cm}\times h(L/2-s_1 y_1)^2h(L/2-s_2 y_2)^2\diff y_2\diff y_1\\
		&\quad\quad\leq \int_{0}^{b}\int_{0}^{b}v(\abs{y_1-y_2})\abs{\tilde{\psi}(L/2-y_1,L/2- y_2,\bar{x}^{1,2})}^2\\&\hspace{5cm}\times\sum_{(s_1,s_2)\in\{-1,1\}^2}h(L/2-s_1 y_1)^2h(L/2-s_2 y_2)^2\diff y_2\diff y_1\\
		&\quad\quad=\int_{0}^{b}\int_{0}^{b}v(\abs{y_1-y_2})\abs{\tilde{\psi}(L/2-y_1,L/2- y_2,\bar{x}^{1,2})}^2\diff y_2\diff y_1
		\end{aligned}
		\end{equation}
		In the third line we use the definition of $ \tilde{\psi} $ as well as the fact that $ \abs{s_1y_1-s_2y_2}\geq \abs{y_1-y_2} $ for $ y_1,y_2\geq 0 $. In the last line we used property 4. of $ h $.
		By combining the two bounds above, we clearly have 
		\begin{equation}
		\begin{aligned}
		&\int_{-L/2-b}^{L/2+b}\int_{-L/2-b}^{L/2+b}v(\abs{x_1-x_2})\abs{\tilde{\psi}(x)}^2h(x_1)^2h(x_2)^2\diff x_1\diff x_2\\&\qquad\qquad\qquad\qquad \leq \int_{-L/2}^{L/2}\int_{-L/2}^{L/2}v(\abs{x_1-x_2})\abs{\tilde{\psi}(x)}^2\diff x_1\diff x_2
		\end{aligned}
		\end{equation}
		 The result now follows from the fact that $ V $ is an isometry.
	\end{proof}
	\begin{lemma}[Finite volume corrections]\label{LemmaLiebLinigerNeumannLowerBound}
		\begin{equation}
		E_{LL}^{N}(n,\ell,c)\geq \frac{\pi^2}{3}n\rho^2\left(1-4\rho/c-\textnormal{const. }\frac{1}{n^{2/3}}\right)
		\end{equation}
	\end{lemma}
	\begin{proof}
		By Robinsons bound \ref{LemmaRobinson}, we have for any $ b>0 $ \begin{equation}
		E_{LL}^{N}(n,\ell,c)\geq E_{LL}^D(n,\ell+b,c)-\text{const. }\frac{n}{b^2}.
		\end{equation}
		Since the range of the interaction in the Lieb-Liniger model is zero, we see that $ e^D_{LL}(2^mn,2^m\ell)=\frac{1}{2^m\ell}E_{LL}^{D}(2^mn,2^m\ell) $ is a decreasing sequence. To see this, simply split the box of size $ 2^m\ell $ in two boxes of size $ 2^{m-1}\ell $, now by ignoring interactions between the boxes and using the the product state of the two $ 2^{m-1}n $-particle ground states in each box as a trial state, we see that $ E^D_{LL}(2^{m}n,2^m\ell)\leq 2E^D_{LL}(2^{m-1}n,2^{m-1}\ell)  $. Since we also have $ e^D_{LL}(2^mn,2^m\ell)\geq e_{LL}(2^mn,2^m\ell)\to e_{LL}(n/\ell) $ as $ m\to\infty $ \cite{PhysRev.130.1605}, we see that \begin{equation}
		\begin{aligned}
		E_{LL}^{N}(n,\ell,c)\geq e_{LL}(n/(\ell+b),c)(\ell+b)-\text{const. }\frac{n}{b^2}\\\geq \frac{\pi^2}{3}n\rho^2\left(1-4\rho/c-\text{const. }\left(3b/\ell-\frac{1}{\rho^2b^2}\right)\right),
		\end{aligned}
		\end{equation}
		with $ \rho=n/\ell $, where the second inequality follows from lemma \ref{LemmaLL-LowerBound}. Optmizing in $ b $ we find \begin{equation}
		E_{LL}^{N}(n,\ell,c)\geq \frac{\pi^2}{3}n\rho^2\left(1-4\rho/c-\text{const. }\frac{1}{n^{2/3}}\right).
		\end{equation}
	\end{proof}
	\subsection{Lieb Liniger reduction}
	We will in this subsection lower bound the dilute bose gas by a Lieb Liniger energy. The reduction is optained by constructing a trial state for a Lieb Liniger model i a smaller volume from the true ground state of the Bose gas.\\
		Let $ \Psi $ be the ground state of $ \mathcal{E} $, we then define $ \psi\in L^2([0,\ell-(n-1)R]^n) $ by $ \psi(x_1,x_2,...,x_n)=\Psi(x_1,R+x_2,...,(n-1)R+x_n) $ for $ x_1\leq x_2\leq...\leq x_n $ and symmetrically extended. 
	\begin{lemma}
		For any function $ \psi\in H^1(\R) $ such that $ \psi(0)=0 $ then we have\begin{equation}\label{EqSobolevIneq}
		\int_{[0,R]}\abs{\partial\psi}^2\geq \max_{[0,R]}\abs{\psi}^2/R
		\end{equation}
	\end{lemma}
	\begin{proof}
		write $ \psi(x)=\int_{0}^{x}\psi'(t)\diff t $, and find that \begin{equation}
		\abs{\psi(x)}\leq \int_{0}^{x}\abs{\psi'(t)}\diff t.
		\end{equation}
		Hence $ \max_{x\in[0,R]}\abs{\psi(x)}\leq \int_{0}^{R}\abs{\psi'(t)}\diff t\leq \sqrt{R}\left(\int\abs{\psi'(t)}^2\diff t\right)^{1/2} $
	\end{proof}
	We can estimate the norm loss in the following way
	\begin{equation}\label{EqNormBoundBij}
	\begin{aligned}
	\braket{\psi|\psi}=1-\int_{B}\abs{\Psi}^2\geq 1-\sum_{i<j}\int_{B_{ij}}\abs{\Psi}^2
	\end{aligned}
	\end{equation}
	where $ B=\{x\in\R^n\vert \min_{i,j}\abs{x_i-x_j}<R \} $, and $ B_{ij}=\{x\in\R^n \vert \mathfrak{r}_i(x)=\abs{x_i-x_j}<R \} $. To give a good bound on the right-hand side, we need the following lemma
	\begin{lemma}\label{LemmaNormLoss}
		Let $ \psi $ be defined as above, then \begin{equation}
		1-\braket{\psi|\psi}\leq\textnormal{const. } \left(R^2\sum_{i<j}\int_{B_{ij}}\abs{\partial_i \Psi}^2+R(R-a)\sum_{i<j}\int v_{ij} \abs{\Psi}^2\right)
		\end{equation}
	\end{lemma}
	\begin{proof}
		Notice that by \eqref{EqSobolevIneq} we have for any $ \phi\in H^1 $, \begin{equation}
		\abs{\abs{\phi(x)}-\abs{\phi(x')}}^2\leq\abs{\phi(x)-\phi(x')}^2\leq R\left(\int_{[0,R]}\abs{\partial \phi}^2\right),
		\end{equation}
		for $ x,x'\in[0,R] $. Furhtermore, 
		\begin{equation}
		\abs{\phi(x)}^2-\abs{\phi(x')}^2=\left(\abs{\phi(x)}-\abs{\phi(x')}\right)^2+2\left(\abs{\phi(x)}-\abs{\phi(x')}\right)\abs{\phi(x')}\leq 2\left(\abs{\phi(x)}-\abs{\phi(x')}\right)^2+\abs{\phi(x')}^2
		\end{equation}
		So for 
		It follows that \begin{equation}
		\max_{x\in[0,R]}\abs{\phi}^2\leq 2R\int_{[0,R]}\abs{\partial \phi}^2+2\min_{x'\in[0,R]}\abs{\phi(x')}^2
		\end{equation}
		Viewing $ \Psi $ as a function of $ x_i $ we have \begin{equation}
		2\min_{\mathfrak{r}_i(x)=\abs{x_i-x_j}<R}\abs{\Psi}^2\geq \max_{\mathfrak{r}_i(x)=\abs{x_i-x_j}<R}\abs{\Psi}^2-4R\left(\int_{{\mathfrak{r}_i(x)=\abs{x_i-x_j}<R}}\abs{\partial_i \Psi}^2\right).
		\end{equation}
		Hence we find \begin{equation}
		\begin{aligned}
		&2\sum_{i<j}\int v_{ij} \abs{\Psi}^2\geq 2\sum_{i<j} \int_{B_{ij}} v_{ij} \abs{\Psi}^2 \\&\geq \left(\int v\right)\sum_{i< j}\int\left(\max_{B'_{ij}}\abs{\Psi}^2-4R\left(\int_{B'_{ij}}\abs{\partial_i\Psi}^2\diff x_i\right)\right)\diff \bar{x}^i\\
		&\geq \frac{4}{R-a}\sum_{i< j}\left(\frac{1}{2R}\int_{B_{ij}}\abs{\Psi}^2-4R\int_{B_{ij}}\abs{\partial_i\Psi}^2\right)
		\end{aligned}
		\end{equation}
		where $ B_{ij}=\{x\in \R^n \vert \mathfrak{r}_i(x)=\abs{x_i-x_j}<R \} $ and $ B'_{ij}=\{x_i\in \R \vert \mathfrak{r}_i(x)=\abs{x_i-x_j}<R \} $. Now, by \eqref{EqNormBoundBij}, we see that
		\begin{equation}
		1-\braket{\psi|\psi}\leq \text{const. } \left(R^2\sum_{i<j}\int_{B_{ij}}\abs{\partial_i \Psi}^2+R(R-a)\int\sum_{i<j} v_{ij} \abs{\Psi}^2\right)
		\end{equation}
	\end{proof}
	Choosing $ R\geq 2\abs{a} $ we have $ \braket{\psi|\psi}\geq 1- \text{const. }R^2 E $.\\
	
	
	
	The following lemma will also be useful \begin{lemma}[Dyson]\label{LemmaDyson} Let $ R>R_0=\textnormal{range}(v) $ and $ \varphi\in H^1(\R) $, then for any interval $ \mathcal{B}\ni 0 $ 
		\begin{equation}
		\int_{\mathcal{B}} \abs{\partial \varphi}^2+\frac12 v\abs{\varphi}^2\geq \int_{\mathcal{B}}\frac{2}{R-a}\left(\delta_R+\delta_{-R}\right)\varphi
		\end{equation}
		where $ a $ is the s-wave scattering length.
	\end{lemma}
	\begin{proof}
		This follows from the variational scattering problem, by comparing left-hand side to the minimizer of the scattering functional.
	\end{proof}
	This lemma will essentially allows us to replace the potential by a shell potential of range $ R $ and strength $ \frac{2}{R-a} $.\\
	\begin{lemma}\label{LemmaNormBoundEpsilon}
		Let $ \psi $ be defined as above with $R>\max\left(R_0,2\abs{a}\right) $ and let $ \epsilon\in[0,1] $, then
		\begin{equation}
		\begin{aligned}
		&\int \sum_{i}\abs{\partial_i\Psi}^2+\sum_{i\neq j} \frac{1}{2}v_{ij}\abs{\Psi}^2\geq E_{LL}^N \left(n,\tilde{\ell},\frac{2\epsilon}{R-a}\right)\braket{\psi|\psi}+ \frac{(1-\epsilon)}{R^2}\textnormal{const. }(1-\braket{\psi|\psi}).
		\end{aligned}
		\end{equation}
		where $ \tilde{\ell}=\ell-(n-1)R $.
	\end{lemma}
	\begin{proof}
		Splitting the energy functional in two parts, and using lemma \ref{LemmaNormLoss} on one term and Dyson's lemma on the other we find 
		\begin{equation}
		\begin{aligned}
		&\int \sum_{i}\abs{\partial_i\Psi}^2+\sum_{i\neq j} \frac{1}{2}v_{ij}\abs{\Psi}^2\geq\\ &\int\sum_{i}\abs{\partial_i\Psi}^2\chi_{\mathfrak{r}_i(x)>R}+\epsilon\sum_{i}\frac{2}{R-a}\delta(\mathfrak{r}_i(x)-R)\abs{\Psi}^2\\&\qquad\qquad\qquad+ (1-\epsilon)\left(\sum_{i<j}\int_{B_{ij}}\abs{\partial_i \Psi}^2+\int\sum_{i<j} v_{ij} \abs{\Psi}^2\right)
		\end{aligned}
		\end{equation}
		where $ \mathfrak{r}_i(x)=\min_{j\neq i}(\abs{x_i-x_j}) $. The nearest neighbor interaction is obtained from Dyson's lemma by dividing the integration domain into Voronoi cells, and restricting to the cell around particle $ i $.\\
		By use of lemma \ref{LemmaNormLoss} with $ R>2\abs{a} $ in the last term, and by realising that the first two terms can be obtained by using $ \psi $ as a trial state in the Lieb-Liniger model, we obtain\begin{equation}
		\int \sum_{i}\abs{\partial_i\Psi}^2+\sum_{i\neq j} \frac{1}{2}v_{ij}\abs{\Psi}^2\geq E_{LL}^N \left(n,\tilde{\ell},\frac{2\epsilon}{R-a}\right)\braket{\psi|\psi}+ \frac{(1-\epsilon)}{R^2}\text{const. }(1-\braket{\psi|\psi})
		\end{equation}
	\end{proof}
	
	
	The next lemma will bound how much mass is lost when going from the state $ \Psi $ of mass $ 1 $ to the state $ \psi $
	\begin{lemma}\label{LemmaImprovedMassBound}
		Let $ C $ denote the constant in lemma \ref{LemmaNormLoss}. For $ n(\rho R)^2\leq  \frac{3}{16\pi^2}C $, $ \rho R\ll 1 $ and $ R>2\abs{a} $ we have
		\begin{equation}\label{EqImprovedMassBound}
		\begin{aligned}
		\braket{\psi|\psi} \geq 1-\textnormal{const. }\left(n(\rho R)^3+n^{1/3}(\rho R)^2\right).
		\end{aligned}
		\end{equation}
	\end{lemma}
	\begin{proof}
		From the known upper bound, and by lemma \ref{LemmaNormBoundEpsilon} with $ \epsilon=1/2 $, it follows that 
		\begin{equation}
		n\frac{\pi^2}{3}\rho^2\left(1+2\rho a+\text{const. }(\rho R)^{3/2}\right)\geq E_{LL}^N \left(n,\tilde{\ell},\frac{1}{R-a}\right)\braket{\psi|\psi}+ \frac{C}{2R^2}(1-\braket{\psi|\psi})
		\end{equation}
		Subtracting $ E_{LL}^N \left(n,\tilde{\ell},\frac{1}{R-a}\right) $ on both sides, and using lemma \ref{LemmaLiebLinigerNeumannLowerBound} on the right-hand side we find\begin{equation}
		\begin{aligned}
		&n\frac{\pi^2}{3}\rho^2\left(1+2\rho a+\text{const. }(\rho R)^{3/2}\right)-n\frac{\pi^2}{3}\tilde{\rho}^2\left(1+4\tilde{\rho} (R-a)-\text{const. }n^{-2/3}\right)\\
		&\geq  \left(\frac{C}{2R^2}-E_{LL}^N \left(n,\tilde{\ell},\frac{1}{R-a}\right)\right)(1-\braket{\psi|\psi}),
		\end{aligned}
		\end{equation}
		with $ \tilde{\rho}=n/\tilde{\ell}=\rho/(1-(\rho-1/\ell)R)  $
		Using now the upper bound $ E^N_{LL}\left(n,\tilde{\ell},\frac{1}{R-a}\right)\leq n\frac{\pi^2}{3}\tilde{\rho}^2 $ on the left-hand side, as well as $ 2\rho \geq\tilde{\rho}\geq \rho(1+\rho R)$ we find
		\begin{equation}
		\begin{aligned}
		\text{const. }n\rho^2R^2\left(\rho R+(\rho R)^{3/2}+n^{-2/3}\right)&\geq \left(\frac{C}{2}-R^2n\frac{4\pi^2}{3}\rho^2\right)\left(1-\braket{\psi|\psi}\right)
		\end{aligned}
		\end{equation}
		It follows that we have \begin{equation}
		\braket{\psi|\psi}\geq 1-\text{const. }\left(n(\rho R)^3+n^{1/3}(\rho R)^2\right)
		\end{equation}
	\end{proof}
	\textbf{Remark:} For $ n=\mathcal{O}((\rho R)^{-9/5}) $ we find \begin{equation}
	\braket{\psi|\psi}\geq 1-\text{const. }n(\rho R)^3=1-\text{const. }(\rho R)^{6/5}
	\end{equation}
	It is now straightforward to show the result
	
	\begin{proposition}\label{PropositionLowerBoundSpecN}
		For assumptions as in lemma \ref{LemmaImprovedMassBound} we have \begin{equation}
		E^N(n,\ell)\geq n\frac{\pi^2}{3}\rho^2\left(1+2\rho a+\textnormal{const. }\left(\frac{1}{n^{2/3}}+n(\rho R)^3+n^{1/3}(\rho R)^2\right)\right)
		\end{equation}
	\end{proposition}
	\begin{proof}
		By lemma \ref{LemmaNormBoundEpsilon} with $ \epsilon=1 $, we reduce to a Lieb-Liniger model with volume $ \tilde{\ell} $, density $ \tilde{\rho} $, and coupling $ c $, and we have $ \tilde{\ell}=\ell-(n-1)R $, $ \tilde{\rho}=\frac{n}{\tilde{\ell}}\approx\rho (1+\rho R) $ and $ c=\frac{2}{R-a} $. Hence we have by Lemmas \ref{LemmaLiebLinigerNeumannLowerBound} and \ref{LemmaImprovedMassBound} \begin{equation}
		\begin{aligned}
		E^N(n,\ell)&\geq E_{LL}^N(n,\tilde{\ell},c)\braket{\psi|\psi}\\&\geq
		n\frac{\pi^2}{3}\rho^2\left(1+2\rho a-\text{const. }\frac{1}{n^{2/3}}\right)\left(1-\text{const. }\left(n(\rho R)^3+n^{1/3}(\rho R)^2\right)\right)
		\end{aligned}
		\end{equation}
	\end{proof}
	\begin{corollary} \label{CorollaryLowerBoundSpecN}
		For $ n=\textnormal{const. } (\rho R)^{-9/5} $ we have 
		\begin{equation}
		E^N(n,\ell)\geq n\frac{\pi^2}{3}\rho^2\left(1+2\rho a-\textnormal{const. }\left((\rho R)^{6/5}+(\rho R)^{7/5}\right)\right).
		\end{equation}
	\end{corollary}
	\subsection{Lower bound of the dilute Bose gas for general particle number}
	So far, we have shown the desired lower bound only for the case where the number of particles are of the order $ (\rho R)^{-9/5} $. In this subsection, we generalize this to any number of particles. We do this, by performing a Legendre transformation in the particle number, \ie going to the grand canonical ensemble. First we justify that only particle numbers of orders less than or equal to $ (\rho R)^{-9/5} $ are relevant for a certain choice of $ \mu $.
		\begin{lemma}\label{LemmaLocalizationFbound}
			Let $ \Xi\geq 4 $ be fixed and let $ n=m\Xi \rho \ell+n_0 $ with $ n_0\in[0,\Xi\rho \ell) $ for some $ m\in\mathbb{N} $ with $ n^{\ast}:=\rho\ell=\mathcal{O}(\rho R)^{-9/5} $. Furhermore, assume that $ \rho R\ll 1 $ and let $ \mu=\pi^2\rho^2\left(1+\frac{8}{3}\rho a\right) $, then \begin{equation}
			E^{N}(n,\ell)-\mu n \geq E^{N}(n_0,\ell)-\mu n_0.
			\end{equation}
		\end{lemma}
		\begin{proof}
			By corollary \ref{CorollaryLowerBoundSpecN} we have \begin{equation}
			E^{N}(\Xi\rho\ell,\ell)\geq\frac{\pi^2}{3}\Xi^3\ell\rho^3\left(1+2\Xi\rho a-\text{const. }(\rho R)^{6/5}\right).
			\end{equation}
			By superadditivity (positive potential) we have \begin{equation}
			E^N(n,\ell)-\mu n\geq m\left(E^N(\Xi\rho\ell,\ell)-\mu\Xi\rho\ell \right)+E^N(n_0,\ell)-\mu n_0.
			\end{equation}
			Thus the result follows from the fact that \begin{equation}
			\frac{\pi^2}{3}\Xi^3\ell\rho^3\left(1+2\Xi\rho a-\text{const. }(\rho R)^{6/5}\right)\geq \pi^2\rho^2\left(1+\frac{8}{3}\rho a\right) \Xi\rho\ell
			\end{equation}
		\end{proof}
	We are then ready to prove the lower bound for general particle numbers.
	
	
		\begin{theorem}[Lower bound]\label{TheoremLowerBound} Let $ E^N(N,L) $ denote the ground state energy of $ \mathcal{E} $ with Neumann boundary conditions. Then for $ \rho R \ll 1 $
			\begin{equation}
			E^N(N,L)\geq N\frac{\pi^2}{3}\rho^2\left(1+2\rho a-\mathcal{O}\left((\rho R)^{6/5}\right)\right)
			\end{equation}
		\end{theorem}
		\begin{proof}
			Notice that \begin{equation}
			E^N(N,L)\geq F^N(\mu,L)+\mu N
			\end{equation}
			where $ F^N(\mu,L)=\inf_{N'}\left(E^N(N',L)-\mu N'\right) $. Clearly we have \begin{equation}
			F^N(\mu,L)\geq M F^N(\mu,\ell)\label{EqLocalizationF}
			\end{equation}
			with $ \ell=L/M $ and $ M\in \mathbb{N}_+ $. 
%			This can be seen by the following argument. Split the box of volume $ L $ in $ M $ boxes of volume $ \ell $. Ignoring the interactions between boxes and using the ground state of the full Hamiltonian as a trial state of the box-divided Hamiltonian we find\begin{equation}
%			E^N(N,L)-\mu N\geq \min_{\{c_n\}}\left\{M\sum_{n=0}^{N}c_n E^N(n,\ell)\right\}-\mu N
%			\end{equation}
%			with $ \{c_n\} $ denoting a distribution of the particles in the boxes, where $ c_n $ is the fraction of the $ M $ boxes that contain exactly $ n $ particles, such that $ \sum_{n=0}^{N}c_n=1 $. Hence we have \begin{equation}
%			\begin{aligned}
%			E^N(N,L)-\mu N\geq\min_{\{c_n\}}\left\{M\sum_{n=0}^{N}c_n \left(E^N(n,\ell)-\mu n\right)\right\}\\
%			\geq MF^{N}(\mu,\ell),
%			\end{aligned}
%			\end{equation}
%			from which \eqref{EqLocalizationF} follows. 
			Now we choose $ M $ such that $ n^*:=\rho\ell=\mathcal{O}\left(\left(\rho R\right)^{-9/5}\right) $ and $ \mu=\pi^2\rho^2\left(1+\frac{8}{3}\rho a\right) $ (notice that $ \mu=\frac{\diff}{\diff \rho}(\frac{\pi^2}{3}\rho^3(1+2\rho a))$). By lemma \ref{LemmaLocalizationFbound} we have that \begin{equation}
			F^N(\mu,\ell):=\inf_{n}\left(E^N(n,\ell)-\mu n\right)=\inf_{n<\Xi n^*}\left(E^N(n,\ell)-\mu n\right).
			\end{equation}
			Now it is known from proposition \ref{PropositionLowerBoundSpecN} that for $ n<\Xi n^* $ we have \begin{equation}
			\begin{aligned}
			E^{N}(n,\ell)&\geq n\frac{\pi^2}{3}\bar{\rho}^2\left(1+2\bar{\rho} a-\textnormal{const. }\left(\frac{1}{n^{2/3}}+n(\bar{\rho} R)^3+n^{1/3}(\bar{\rho} R)^2\right)\right)\\
			&\geq \frac{\pi^2}{3}n\bar{\rho}^2\left(1+2\bar{\rho}a\right)-n^*\rho^2\mathcal{O}\left((\rho R)^{6/5}\right)
			\end{aligned}
			\end{equation}
			where $ \bar{\rho}=n/\ell $ (notice that now $ \rho=N/L=n^\ast/\ell\neq n/\ell $) and where we used $ \bar{\rho}<\Xi\rho$.
			Thus we have \begin{equation}
			F^{N}(\mu,\ell)\geq \inf_{\bar{\rho}<\Xi\rho}(g(\bar{\rho})-\mu\bar{\rho})\ell-n^\ast \rho^2 \mathcal{O}\left((\rho R)^{6/5}\right)
			\end{equation}
			where $
			g(\bar{\rho})=
			\frac{\pi^2}{3}\bar{\rho}^3\left(1+2\bar{\rho}a\right)
			$ for $ \bar{\rho}<\Xi\rho $. $ g $ is a convex, $ C^{1} $ function with invertible derivative for $ \Xi\rho a\ll 1  $. Hence we have \begin{equation}
			\begin{aligned}
			E^{N}(N,L)\geq M(F^{N}(\mu,\ell)+\mu n^*)\geq Mn^\ast\frac{\pi^2}{3} \rho^2 \left(1+2\rho a-\mathcal{O}\left((\rho R)^{6/5}\right)\right)\\
			=\frac{\pi^2}{3} N\rho^2 \left(1+2\rho a-\mathcal{O}\left((\rho R)^{6/5}\right)\right)
			\end{aligned}
			\end{equation}
			where the second inequality follows from the specific choice of $ \mu=g'(\rho) $.
		\end{proof}
		\section{A remark on anyons}
			Let $ \kappa\in[0,\pi] $ and define for each permuation $ \sigma=(\sigma_1,...,\sigma_N) $ the sector $ \Sigma_\sigma=\{x_{\sigma_1}<x_{\sigma_2}<...<x_{\sigma_N}\}\subset \R^N $, and consider the operator\begin{equation}
			H_N=-\sum_{i=1}^{N}\partial_{x_i}^2,\quad\text{on }\R^N\setminus\bigcup_{i<j}\{x_i=x_j\}
			\end{equation}
			with domian \begin{equation}
			\begin{aligned}
			\mathcal{D}(H_N)&=\bigg\{\varphi=\euler{-i\frac{\kappa}{2}\Lambda(x)}f(x)\ \bigg\vert\ f\in \left((\oplus_{\text{sym}})_{\sigma\in S_N } C^\infty(\overline{\Sigma_{\sigma}})\right)\cap C_0(\R^N) ,\\&\qquad \ (\partial_i-\partial_j)\varphi\rvert^{ij}_+-(\partial_i-\partial_j)\varphi\rvert^{ij}_-=2c\ \euler{-i\frac{\kappa}{2}\Lambda(x)} f\rvert^{ij}_0 \text{ for all }i\neq j \bigg\}
			\end{aligned}
			\end{equation}
			where $ c\geq0 $ and $ \Lambda(x)= \sum_{i<j}\epsilon(x_i-x_j) $  with $ \epsilon(x)=\begin{cases}
			1&\text{for }x>0\\
			-1&\text{for }x<0\\
			0&\text{for }x=0
			\end{cases} $ and $ \vert^{ij}_{\pm,0} $ means the function evaluated at $ x_i=x_{j \pm,0} $. Then the following proposition holds \begin{proposition}\label{PropositionAnyonQuadraticForm}
				$ H_N $ is symmetric with corresponding quadratic form \begin{equation}
				\mathcal{E}(\varphi)=\sum_{i=1}^{N}\int_{{\R^N\setminus\bigcup_{i<j}\{x_i=x_j\}}} \abs{\partial_{x_i}\varphi(x)}^2+\frac{2c}{\cos(\kappa/2)}\sum_{i<j} \delta(x_i-x_j)\abs{\varphi(x)}^2\diff^{N}x
				\end{equation}
			\end{proposition}
			\begin{proof}
				Let $ \varphi,\vartheta\in \mathcal{D}(H_N) $, then by partial integration we have \begin{equation}
				\begin{aligned}
				\braket{\vartheta\vert H_N \varphi}&=-\sum_{i=1}^{N}\int_{\R^N\setminus\bigcup_{i<j}\{x_i=x_j\}}\overline{\vartheta} \partial_{x_i}^2\varphi\\&=\sum_{i=1}^{N}\int_{\R^N\setminus\bigcup_{i<j}\{x_i=x_j\}}\overline{\partial_{x_i}\vartheta}\partial_{x_i}\varphi-\int_{\R^{N-1}\setminus\bigcup_{i<j}\{x_i=x_j\}}\sum_{i\neq j}\left(\overline{\vartheta}\partial_{x_i}\varphi\vert^{ij}_--\overline{\vartheta}\partial_{x_i}\varphi\vert^{ij}_+\right)\\
				&=\sum_{i=1}^{N}\int_{\R^N\setminus\bigcup_{i<j}\{x_i=x_j\}}\overline{\partial_{x_i}\vartheta}\partial_{x_i}\varphi+\int_{\R^{N-1}\setminus\bigcup_{i<j}\{x_i=x_j\}}\sum_{i< j}\left(\overline{\vartheta}(\partial_{x_i}-\partial_{x_j})\varphi\vert^{ij}_+-\overline{\vartheta}(\partial_{x_i}-\partial_{x_j})\varphi\vert^{ij}_-\right).
				\end{aligned}
				\end{equation}
				Let $ f,g\in C^\infty_0(\R^N) $ be the functions such that $ \varphi=\euler{-i\frac{\kappa}{2}\Lambda}f $ and $ \vartheta=\euler{-i\frac{\kappa}{2}\Lambda}g $. Then we have
				
				\begin{equation}
				\begin{aligned}
				\braket{\vartheta\vert H_N \varphi}&=\sum_{i=1}^{N}\int_{\R^N\setminus\bigcup_{i<j}\{x_i=x_j\}}\overline{\partial_{x_i}\vartheta}\partial_{x_i}\varphi+\int_{\R^{N-1}\setminus\bigcup_{i<j}\{x_i=x_j\}}\sum_{i< j}\left(\overline{g}(\partial_{x_i}-\partial_{x_j})f\vert^{ij}_+-\overline{g}(\partial_{x_i}-\partial_{x_j})f\vert^{ij}_-\right)\\
				&=\sum_{i=1}^{N}\int_{\R^N\setminus\bigcup_{i<j}\{x_i=x_j\}}\overline{\partial_{x_i}\vartheta}\partial_{x_i}\varphi+\int_{\R^{N-1}\setminus\bigcup_{i<j}\{x_i=x_j\}}2\sum_{i< j}\left(\overline{g}(\partial_{x_i}-\partial_{x_j})f\vert^{ij}_+\right)
				\end{aligned}
				\end{equation}
				where the last equality follows from symmetry of $ f $. Notice that by the boundary condtion on $ \mathcal{D}(H_N) $ we have \begin{equation}
				(\partial_i-\partial_j)\varphi\rvert^{ij}_+-(\partial_i-\partial_j)\varphi\rvert^{ij}_-=\euler{-i\frac{\kappa}{2}\left(-1+S\right)}(\partial_i-\partial_j)f\rvert^{ij}_+-\euler{-i\frac{\kappa}{2}\left(1+S\right)}(\partial_i-\partial_j)f\rvert^{ij}_-=2c \varphi\rvert^{ij}_0=\euler{-i\frac{\kappa}{2}S}2c f\rvert^{ij}_0
				\end{equation}
				where $ S=\Lambda-\epsilon(x_i-x_j) $. By symmetry of $ f $ it follows that \begin{equation}
				\begin{aligned}
				\euler{-i\frac{\kappa}{2}\left(-1+S\right)}(\partial_i-\partial_j)f\rvert^{ij}_+-\euler{-i\frac{\kappa}{2}\left(1+S\right)}(\partial_i-\partial_j)f\rvert^{ij}_-
				&=\euler{-i\frac{\kappa}{2}\left(-1+S\right)}(\partial_i-\partial_j)f\rvert^{ij}_++\euler{-i\frac{\kappa}{2}\left(1+S\right)}(\partial_i-\partial_j)f\rvert^{ij}_+\\
				&=\euler{-i\frac{\kappa}{2}S}2\cos(\kappa/2)(\partial_i-\partial_j)f\rvert^{ij}_+\\
				&=\euler{-i\frac{\kappa}{2}S}2c f\rvert^{ij}_0.
				\end{aligned}
				\end{equation}
				so that \begin{equation}
				2(\partial_i-\partial_j)f\rvert^{ij}_+=\frac{2c}{\cos(\kappa/2)}f\rvert^{ij}_0.,
				\end{equation}
				where we by $ \kappa=\pi $ mean $ f\rvert^{ij}_0=0 $.
				Hence it follows that \begin{equation}\label{EqQuadraticFormDerivation}
				\braket{\vartheta\vert H_N \varphi}=\sum_{i=1}^{N}\int_{{\R^N\setminus\bigcup_{i<j}\{x_i=x_j\}}}\overline{\partial_{x_i}\vartheta} \partial_{x_i}\varphi(x)+\frac{2c}{\cos(\kappa/2)}\sum_{i<j} \delta(x_i-x_j)\overline{\vartheta(x)}\varphi(x)\diff^{N}x,
				\end{equation}
				where again $ \kappa=\pi $ refers to $ \braket{\vartheta\vert H_N \varphi}=\sum_{i=1}^{N}\int_{{\R^N\setminus\bigcup_{i<j}\{x_i=x_j\}}}\overline{\partial_{x_i}\vartheta} \partial_{x_i}\varphi(x) $ with Dirichlet boundary conditions on the planes of intersection.
				Now it is clear that starting from $ \braket{H_N\vartheta\vert \phi} $, we can by the same steps arrive at \eqref{EqQuadraticFormDerivation}, proving that $ H_N $ is symmetric. 	
			\end{proof}
			\begin{remark}
				Since $ \mathcal{E}\geq0 $, $ H_N $ has a self-adjoint Friedrichs extension, $ \tilde{H}_N $, which we regard as the Hamiltonian for the one dimensional anyon gas with statistical parameter, $ \kappa $, and a zero-range interaction of strength, $ c $.
			\end{remark}
			\begin{remark}
				By the quadratic form formulation, and the fact that the phase-factor is not contributing to the value of the quadratic form, it follows that $ \tilde{H}_N $ is unitarily equivalent to the Lieb-Liniger Hamiltonian $ H_{LL}(N,\frac{c}{\cos(\kappa/2)}) $, with $ N $ particles and coupling $ c/\cos(\kappa/2) $.
			\end{remark}
			We regard in the following the potential $ 2c\delta_0 $ with $ c=\infty $ as a hard of zero range, in the sense, that it imposes a Dirichlet boundary condition on the planes of intersection.
			In the above sense, we see that our result applies to a general anyonic gas, in the following form
			\begin{theorem}
				Let $ E_{(\kappa,c)}(N,L) $ be the ground state energy of anyon gas with $ v=v_{\text{h.c.}}+v_{\text{reg}}+2c\delta_0 $, with $ (\kappa,c)\in [0,2\pi]\times [0,\infty] $. \begin{equation}
				E_{(\kappa,c)}(N,L)=N\frac{\pi^2}{3}\rho^2\left(1+2\rho a_{\kappa}+\mathcal{O}\left(\rho R\right)^{6/5}\right),
				\end{equation}
				where $ a_\kappa $ is defined to be scattering length associated with potential $ v=v_{\text{h.c.}}+v_{\text{reg}}+\frac{2c}{\cos(\kappa/2)}\delta_0 $.
			\end{theorem}
			\begin{proof}
				This follows from Theorems \ref{TheoremUpperBound} and \ref{TheoremLowerBound} as well as Proposition \ref{PropositionAnyonQuadraticForm} with the observation that the quadratic form is independent of the phase-factors.
			\end{proof}
		 \begin{corollary}
		 	Let $ E_F(N,L) $ denote the fermionic ground state energy, then $ E_F(N,L)=E_{(\pi,c)}(N,L) $ for any $ c>0 $ and hence \begin{equation}
		 	E_F(N,L)=N\frac{\pi^2}{3}\rho^2\left(1+2\rho a_{\pi}+\mathcal{O}\left(\rho R\right)^{6/5}\right).
		 	\end{equation}
		 	where $ a_{\pi} $ is the odd wave scattering length.
		 \end{corollary}
		 It is easily verified that $ a_\pi\geq 0 $ for any $ v\geq 0 $, and hence we see that the first order correction to the Fermi ground state energy is positive as expected.
	\bibliographystyle{amsplain}
	\bibliography{bibliography}
	\end{document}
