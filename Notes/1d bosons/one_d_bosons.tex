\documentclass[a4paper,11pt]{article}
\usepackage[utf8]{inputenc}
\usepackage[margin=1in]{geometry}
\usepackage{pdfpages}
\usepackage{mathrsfs}
\usepackage{amsfonts}
\usepackage{amsmath}
\DeclareMathOperator\arctanh{arctanh}
\usepackage{amssymb}
\usepackage{bbm}
\usepackage{amsthm}
\usepackage{graphicx}
\usepackage{centernot}
\usepackage{caption}
\usepackage{subcaption}
\usepackage{braket}
\usepackage{pgfplots}
\usepackage{lastpage}
\usepackage{enumitem}
\usepackage{setspace}
\usepackage{xcolor}
\usepackage[english]{babel} 

\usepackage[square,sort,comma,numbers]{natbib}
\usepackage[colorlinks=true,linkcolor=blue]{hyperref}

\usepackage{fancyhdr}
\newcommand{\euler}[1]{\text{e}^{#1}}
\newcommand{\Real}{\text{Re}}
\newcommand{\Imag}{\text{Im}}
\newcommand{\supp}{\text{supp}}
\newcommand{\norm}[1]{\left\lVert #1 \right\rVert}
\newcommand{\abs}[1]{\left\lvert #1 \right\rvert}
\newcommand{\floor}[1]{\left\lfloor #1 \right\rfloor}
\newcommand{\Span}[1]{\text{span}\left(#1\right)}
\newcommand{\dom}[1]{\mathscr D\left(#1\right)}
\newcommand{\Ran}[1]{\text{Ran}\left(#1\right)}
\newcommand{\conv}[1]{\text{co}\left\{#1\right\}}
\newcommand{\Ext}[1]{\text{Ext}\left\{#1\right\}}
\newcommand{\vin}{\rotatebox[origin=c]{-90}{$\in$}}
\newcommand{\interior}[1]{%
	{\kern0pt#1}^{\mathrm{o}}%
}
\renewcommand{\braket}[1]{\left\langle#1\right\rangle}
\newcommand*\diff{\mathop{}\!\mathrm{d}}
\newcommand{\ie}{\emph{i.e.} }
\newcommand{\eg}{\emph{e.g.} }
\newcommand{\dd}{\partial }
\newcommand{\R}{\mathbb{R}}
\newcommand{\C}{\mathbb{C}}
\newcommand{\w}{\mathsf{w}}
\newcommand{\rr}{\mathcal{R}}


\newcommand{\Gliminf}{\Gamma\text{-}\liminf}
\newcommand{\Glimsup}{\Gamma\text{-}\limsup}
\newcommand{\Glim}{\Gamma\text{-}\lim}

\newtheorem{theorem}{Theorem}
\newtheorem{definition}{Definition}
\newtheorem{proposition}{Proposition}
\newtheorem{lemma}{Lemma}
\newtheorem{corollary}{Corollary}

\numberwithin{equation}{section}
\linespread{1.3}

\pagestyle{fancy}
\fancyhf{}
\rhead{Notes on 1D bosons}
\lhead{Johannes Agerskov}
\rfoot{\thepage}
\lfoot{Dated: \today}
\author{Johannes Agerskov}
\date{Dated: \today}
\title{Notes on 1D bosons}
\begin{document}
	\maketitle
	We consider the dilute bose gas in one dimension, where we seek to prove the formula for the ground state energy \begin{equation}\label{EqGroundstateE}
	\frac{E}{L}=\frac{\pi^2}{3}\rho^3\left(1+2\rho a+ \mathcal{O}\left((\rho a)^2\right)\right).
	\end{equation} 
	We assume that the interaction potential $ v $ has compact support, say in the ball of radius $ b $, $ B_b $.
	\section{Upper bound}
	We provide the upper bound for \eqref{EqGroundstateE}, by using the variational principle with a suitable trial state. Consider the trial state\begin{equation}
	\Psi(x)=\begin{cases}
	\omega(\rr(x))\frac{\Psi_F(x)}{\sin\left(\frac{\pi}{L}\rr(x)\right)}& \text{if }\rr(x)<b,\\
	\Psi_F(x)&\text{if }\rr(x)\geq b,
	\end{cases}
	\end{equation}
	where $ \omega $ is the suitably normalized solution to the two-body scattering equation, \ie $ \omega(x)=f(x)\frac{\sin\left(\frac{\pi}{L}b\right)}{f(b)} $ where $ f $ is any solution of the two-body scattering equation.  $ \Psi_F(x)=\mathcal{N}^{1/2}\prod_{i<j}^{N}\sin\left(\frac{\pi}{L}(x_i-x_j)\right) $ is the free fermionic ground state, and $ \rr(x)=\min_{i<j}(\abs{x_i-x_j}) $.\\
	The energy of this trial state is then\begin{equation}
	\mathcal{E}(\Psi)=\int \sum_{i=1}^{N}\abs{\nabla_i\Psi}^2+\sum_{i<j}^{N}v_{ij}\abs{\Psi}^2,
	\end{equation}
	where $ v_{ij}(x)=v(x_i-x_j) $. Since $ v $ is supported in $ B_b $ and $ \Psi=\Psi_F $ except in the region $ B=\{x\in\R^N \vert \rr(x)<b \} $, we may rewrite this as \begin{equation}
	\mathcal{E}(\Psi)=E_0+\int_B \sum_{i=1}^{N}\abs{\nabla_i\Psi}^2+\sum_{i<j}^{N}v_{ij}\abs{\Psi}^2-\sum_{i=1}^{N}\abs{\nabla_i\Psi_F}^2,
	\end{equation}
	where $ E_0=N\frac{\pi^2}{3}\rho^2 $ is the ground state energy of the free Fermi gas.
\end{document}