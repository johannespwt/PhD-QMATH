\documentclass[a4paper,11pt]{article}
\usepackage[utf8]{inputenc}
\usepackage[margin=1in]{geometry}
\usepackage{pdfpages}
\usepackage{mathrsfs}
\usepackage{amsfonts}
\usepackage{amsmath}
\usepackage{amssymb}
\usepackage{bbm}
\usepackage{amsthm}
\usepackage{graphicx}
\usepackage{centernot}
\usepackage{caption}
\usepackage{subcaption}
\usepackage{braket}
\usepackage{pgfplots}
\usepackage{lastpage}
\usepackage{enumitem}
\usepackage{setspace}
\usepackage[english]{babel} 

\usepackage[square,sort,comma,numbers]{natbib}

\usepackage{fancyhdr}
\newcommand{\euler}[1]{\text{e}^{#1}}
\newcommand{\Real}{\text{Re}}
\newcommand{\Imag}{\text{Im}}
\newcommand{\floor}[1]{\left\lfloor #1 \right\rfloor}
\newcommand{\Span}[1]{\text{span}\left(#1\right)}
\newcommand{\dom}[1]{\mathscr D\left(#1\right)}
\newcommand{\Ran}[1]{\text{Ran}\left(#1\right)}
\newcommand{\conv}[1]{\text{co}\left\{#1\right\}}
\newcommand{\Ext}[1]{\text{Ext}\left\{#1\right\}}
\newcommand{\vin}{\rotatebox[origin=c]{-90}{$\in$}}
\newcommand{\interior}[1]{%
	{\kern0pt#1}^{\mathrm{o}}%
}
\newcommand{\ie}{\emph{i.e.} }
\newcommand{\eg}{\emph{e.g.} }
\newcommand{\dd}{\partial }
\newcommand{\R}{\mathbb{R}}
\newcommand{\C}{\mathbb{C}}
\newcommand{\w}{\mathsf{w}}


\newtheorem{theorem}{Theorem}
\newtheorem{definition}{Definition}
\newtheorem{proposition}{Proposition}
\newtheorem{lemma}{Lemma}

\pagestyle{fancy}
\fancyhf{}
\rhead{Notes on stability of the Fermi gas with point interactions}
\lhead{Johannes Agerskov}
\rfoot{Page \thepage \ of \pageref{LastPage}}
\lfoot{Dated: \today}
\author{Johannes Agerskov}
\date{Dated: \today}
\title{Notes on stability of the Fermi gas with point interactions}
\begin{document}

	\maketitle
	\tableofcontents
\vspace{1cm}
We study in these notes the Fermi gas, \ie a many-body system of spin-$ \frac{1}{2} $ fermions or more generally just two species of fermions. The specific gas we study is interacting \emph{via.} point interactions \emph{or} zero-range interactions. We will restrict to the case where the two species can have different mass, but all fermions in one species have equal mass. The relevant quantity in this case is the relative mass of the two. Thus by setting the mass of one species to 1 and the mass of the other to $ m $ we have en mass-ratio of $ m $. Formally the system we are studying can thus be described by the Hamiltonian \begin{equation}\label{Formal Hamiltonian}
H=-\frac{1}{2m}\sum_{j=1}^{M}\Delta_{y_j}-\frac{1}{2}\sum_{i=1}^{N}\Delta_{x_i}+\gamma\sum_{i=1}^{N}\sum_{j=1}^{M}\delta(x_i-y_j),
\end{equation}
where $ x_i\in\mathbb{R}^3 $ for all $ i\in\{1,...,N\} $ and $ y_j\in\mathbb{R}^3 $ for all $ j\in\{1,...,M\} $. Notice that we also restrict to the case of equal coupling between all particles. These formally defined Hamiltonians are clearly ill-defined as the $ \delta $-function is a  temperate distribution and thus is only defined on the Schwartz functions. However, restricting the domain to the Schwartz functions will not make the Laplacians self-adjoint, furthermore the codomain of $ \delta $ is not in $ L_2(\mathbb{R}^3) $. Thus no self-adjoint operator of this form exists.
As a quadratic form $ \braket{\psi|H\psi} $ might make sense. Since the $ \delta $-function only makes sense, at least on continuous functions, there exist no sensible domain of this quadratic form such that it is closed. If such a domain existed both the Laplacian would be closed on it, however, this is only true for $ H_1(\mathbb{R}^3) $ which contains non-continuous functions (defined a.e.)

One way of rigorously studying such formal Hamiltonians is to consider self-adjoint extensions of more well defined Hamiltonians. While this approach is very successful in the $ N=M=1 $ case, it becomes increasingly difficult as the number of self-adjoint extension become infinite already at the $ N=2 $ and $ M=1 $ case.  
In \cite{FINCO2012131} quadratic forms where developed in order to describe systems of the form \eqref{Formal Hamiltonian}. These quadratic forms are generally more well defined, however their origin and connection the the formal Hamiltonian might be obscured as they need regularization and renormalization procedures to make sense of the point interactions. We will in these notes aim to construct the quadratic form corresponding to the formal expression in \eqref{Formal Hamiltonian}, and show that they can be reaches by considering a sequence of rank one perturbations. We aim at showing that operators corresponding to these rank one perturbations actually converge to the operator of the quadratic form given by \cite{FINCO2012131}. Furthermore, it is our hope that this will shed light on the stability of these systems which has only been shown the cases of  $ (N,M)=(N,1) $ and $ (N,M)=(2,2) $.
We start out by considering the simpler case which is $ (N,M)=(N,1) $ also denoted the $ N+1 $ case.
\section{Formal Hamiltonian for the $ N+1 $ case}
The formal Hamiltonian of \eqref{Formal Hamiltonian} can be rewritten in the $ N+1 $ case by separating the centre of mass. Notice that this indeed already restricts the set of possible self-adjoint Hamiltonians mimicking \eqref{Formal Hamiltonian} as this asserts the translational invariance of the Hamiltonian. Thus this separation of the centre of mass restricts to couplings that are independent of the centre of mass coordinate. Defining the centre of mass and the relative coordinates by\begin{equation}
X=\frac{my+\sum_{i=1}^{N}x_i}{m+N}, \quad \tilde{x}_i=x_i-y,
\end{equation}
we obtain that \begin{equation}\begin{aligned}
\Delta_{x_i}=\sum_{j=1}^{3}\dd_{x^j_i}\dd_{x^j_i}=\sum_{j=1}^{3}\left(\frac{\dd X^j}{\dd x^j_{i}}\dd_{X^j}+\frac{\dd \tilde{x_i}^j}{\dd x_i^j}\dd_{\tilde{x}_i^j}\right)\left(\frac{\dd X^j}{\dd x^j_{i}}\dd_{X^j}+\frac{\dd \tilde{x_i}^j}{\dd x_i^j}\dd_{\tilde{x}_i^j}\right)\\
=\frac{1}{(m+N)^2}\Delta_{X}+\Delta_{\tilde{x}_i}+\frac{2}{m+N}\nabla_X\cdot\nabla_{\tilde{x}_i},
\end{aligned}
\end{equation}
\begin{equation}\begin{aligned}
\Delta_{y}=\sum_{j=1}^{3}\dd_{y^j}\dd_{y^j}=\sum_{j=1}^{3}\left(\frac{\dd X^j}{\dd y^j}\dd_{X^j}+\sum_{i=1}^{N}\frac{\dd \tilde{x_i}^j}{\dd y^j}\dd_{\tilde{x}_i^j}\right)\left(\frac{\dd X^j}{\dd y^j}\dd_{X^j}+\sum_{i=1}^{N}\frac{\dd \tilde{x_i}^j}{\dd y^j}\dd_{\tilde{x}_i^j}\right)\\
=\frac{m^2}{(m+N)^2}\Delta_{X}+\sum_{i=1}^{N}\Delta_{\tilde{x}_i}+2\sum_{\substack{(i,j)=(1,1)\\i<j}}^{(N,N)}\nabla_{\tilde{x}_i}\cdot\nabla_{\tilde{x}_j}-\frac{2m}{m+N}\sum_{i=1}^{N}\nabla_X\cdot\nabla_{\tilde{x}_i}.
\end{aligned}
\end{equation}
Thus we get the Hamiltonian \begin{equation}
H=-\frac{1}{2(m+N)}\Delta_X-\frac{m+1}{2m}\sum_{i=1}^{N}\Delta_{\tilde{x}_i}+\frac{2}{2m}\sum_{\substack{(i,j)=(1,1)\\i<j}}^{(N,N)}\nabla_{\tilde{x}_i}\cdot\nabla_{\tilde{x}_j}+\gamma\sum_{i=1}^{N}\delta(\tilde{x}_i),
\end{equation}
which can be recast as \begin{equation}
H=H_{\text{CM}}+\frac{m+1}{2m}H_{\text{rel}},
\end{equation}
with $ H_{\text{CM}}=-\frac{1}{2(m+N)}\Delta_X $ the free centre of mass and the relative Hamiltonian given by \begin{equation}\label{Relative Hamiltonian}
H_{\text{rel}}=\sum_{i=1}^{N}\Delta_{\tilde{x}_i}+\frac{2}{m+1}\sum_{\substack{(i,j)=(1,1)\\i<j}}^{(N,N)}\nabla_{\tilde{x}_i}\cdot\nabla_{\tilde{x}_j}+\tilde{\gamma}\sum_{i=1}^{N}\delta(\tilde{x}_i),
\end{equation}
where $ \tilde{\gamma}=\frac{2m}{m+1}\gamma $. Notice that the problem has now been split in two independent parts and thus we recognize the centre of mass part as the free particle which is solved by the Laplacian being essentially self adjoint on $ C^\infty_c(\mathbb{R}^3) $ functions with self-adjoint extension $ \Delta $ on $ H_2(\mathbb{R}^3) $ where $ \Delta $ acts in the distributional sense. The relative Hamiltonian on the other hand will be the main focus in the first part of these notes.
\section{The $ 1+1 $ case}
We are now going to study different ways of rigorously defining the relative Hamiltonian \eqref{Relative Hamiltonian} in the case of $ N=1 $. The first method is easily implemented for $ N=1 $ but is hard to generalize.
\subsection{Self-adjoint extension}
The first method we are going to study is that of self-adjoint extension. We thus restrict the formal Hamiltonian to a domain in which it is well defined. This could for example be $ C_c^\infty(\mathbb{R}^3\setminus\{0\}) $. Notice since we have removed $ \{0\} $ the $ \delta $-function has no support on this space and thus vanish. Therefore, we have the relative Hamiltonian \begin{equation}
H_{\text{rel}}=-\Delta\rvert_{C_c^\infty(\mathbb{R}^3\setminus\{0\})}.
\end{equation}
We now seek to extend this operator to a self-adjoint operator on a larger domain. This is possible since $ H_{\text{rel}} $ is symmetric and its closure, denoted $ \dot{H}_{\text{rel}} $ have deficiency indices $ K_+=K_-=1 $, with $ K_\pm=\text{Ran}(H_{\text{rel}}\pm iI)^\perp=\ker(H_{\text{rel}}^*\mp iI) $ where $ H_{\text{rel}}^* $ denotes the adjoint of $ H_{\text{rel}} $. \\
By definition on the adjoint we have that $ \dom{H_{\text{rel}}^*}=\{f\in L_2(\R^3)\rvert \braket{f|H_{\text{rel}}\cdot} \text{ is bounded on }\dom{H_{\text{rel}}}\} $, where the adjoint of the Laplacian acts as the Laplacian in the distributional sense. We determine first the closure of $ H_{\text{rel}} $. This can be done by taking the adjoint twice. Notice that the domain of the adjoint is $ \dom{H_{\text{rel}}^*}=\{f\in L_2(\R^3)\rvert \braket{f|\Delta\cdot} \text{ is bounded on }C_c^\infty(\R^3\setminus\{0\})\} $. This can be directly calculated to be \begin{equation}
\dom{H_{\text{rel}}^*}=\{f\in H_{2,\text{loc}}(\R^3\setminus\{0\})\cap L_2(\R^3)\rvert \Delta f \in L_2(\R^3)\}
\end{equation}
We emphasise that all elements in $ \dom{H_{\text{rel}}^*} $ should be viewed as distributions in $ H_{2,\text{loc}}(\R^3\setminus\{0\}) $. Therefore the requirement $ \Delta f\in L_2(\R^3) $ does not simply restrict the domain to be $ H_2(\R^3) $ as elements or their derivative (up to second order) can have singular behaviour at $ 0 $, e.g. $ \delta $-functions.
 Notice that $ C_c^\infty(\mathbb{R}^3\setminus\{0\}) $ is dense in $ L_2(\mathbb{R}^3\setminus\{0\})=L_2(\R^3) $ (only defined a.e).  The domain of the double adjoint is then given by \begin{equation}\begin{aligned}
\dom{H_{\text{rel}}^{**}}&=\{f\in L_2(\R^3)\rvert \braket{H_{\text{rel}}^*\cdot|f} \text{ is bounded on }\dom{H_{\text{rel}}^*}\}=H^0_2(\R^3\setminus\{0\}),
\end{aligned}
\end{equation} 
where $ \Delta $ acts in the distributional sense and we have defined\begin{equation}
H_2^0(\R^3\setminus\{0\})=\left\{u\in L_2(\R^3)\ \big\rvert\ \Delta u\in L_2(\R^3)\text{ and } u(x)\to0\wedge\nabla u(x)\to0 \text{ for } |x|\to0\vee |x|\to\infty\right\}.
\end{equation} Thus we have \begin{equation}
\dot{H}_{\text{rel}}=-\Delta,\qquad \dom{\dot{H}_{\text{rel}}}=H^0_2(\R^3\setminus\{0\}).
\end{equation}
The adjoint of $ \dot{H}_{\text{rel}} $ is simply given by $ \dot{H}_{\text{rel}}^*=H_{\text{rel}}^* $, as the adjoint is already closed. Thus we are ready to find all self-adjoint extensions of $ H_{\text{rel}} $. By the Krein theorem there exist self-adjoint extension if and only if $ \dim\left(\Ran{H_{\text{rel}}-iI}^\perp\right)=\dim\left(\Ran{H_{\text{rel}}+iI}^\perp\right) $ or equivalently $ \dim\left(\ker{\left(H^*_{\text{rel}}+iI\right)}\right)=\dim\left(\ker{\left(H^*_{\text{rel}}-iI\right)}\right) $ thus we seek solutions of the equation \begin{equation}
H_{\text{rel}}^*\psi_\pm=\pm i \psi_\pm,\quad \psi_\pm\in\dom{H_{\text{rel}}^*}.
\end{equation}
The equation $ -\Delta \psi_\pm=\pm i \psi_\pm $ has the unique solution \begin{equation}
\psi(x)_\pm=\frac{\euler{ i\sqrt{\pm i}|x-y|}}{|x-y|},\qquad x\in\R^3\setminus\{y\}.
\end{equation}
In order for this function to be in the domain of $ H_{\text{rel}}^*$ we need to choose $ y=0 $.
Thus we see that $ \dim\left(\ker{\left(H^*_{\text{rel}}+iI\right)}\right)=\dim\left(\ker{\left(H^*_{\text{rel}}-iI\right)}\right)=1 $. By Krein's extension theorem for symmetric operators we have that there exist a one-parameter family of self-adjoint extensions of $ H_{\text{rel}} $. Parametrizing the family by a complex phase we have the extensions \begin{equation}
\dom{H_{\text{rel},\theta}}=\left\{ h+c(\xi_++\euler{i\theta}\xi_-)\Big\rvert h\in\dom{\dot{H}_{\text{rel}}},\ c\in\C \right\},
\end{equation}
where $ \theta\in[0,2\pi) $, $ \xi_+\in \ker\left(H_{\text{rel}}^*+iI\right),\xi_-\in \ker\left(H_{\text{rel}}^*-iI\right) $ are fixed with $ ||\xi_+||=||\xi_-|| $, and where \begin{equation}
H_{\text{rel},\theta}(h+\xi_++\euler{i\theta}\xi_-)=H_{\text{rel}}^*(h+\xi_++\euler{i\theta}\xi_-)=h+i(\euler{i\theta}\xi_--\xi_+)
\end{equation}
Following the methods of \cite{albeverio2012solvable}, we however now that there is another characterization of these extensions. By decomposing the Hilbert space into spherical coordinates we obtain the decomposition \begin{equation}
L_2(\R^3,d^3x)=L_2((0,\infty),r^2dr)\otimes L_2(S^2,d\Omega)
\end{equation}
Furthermore by decomposing into spherical harmonics we have \begin{equation}
L_2(\R^3,d^3x)=\bigoplus_{l=0}^{\infty}L_2((0,\infty),r^2dr)\otimes \braket{Y_l^{-l},Y_l^{-l+1},...,Y_l^0,...,.Y_l^l}
\end{equation}
Now using the unitary transformation $ U:L_2((0,\infty),r^2dr)\to L_2((0,\infty),dr) $, defined by\\ $ Uf(r)=rf(r) $
\begin{equation}
L_2(\R^3,d^3x)=\bigoplus_{l=1}^{\infty}U^{-1}L_2((0,\infty),r^2dr)\otimes \braket{Y_l^{-l},Y_l^{-l+1},...,Y_l^0,...,.Y_l^l}
\end{equation}
where $ \braket{...} $ denotes the span. Using the Laplacian in spherical coordinates \begin{equation}
\Delta\phi=\frac{1}{\sqrt{g}}\partial_i(\sqrt{g}g^{ij}\partial_j(\phi))=\frac{1}{r}\frac{\partial^2}{\partial r^2}(r\phi)+\frac{1}{r^2\sin\varphi}\frac{\partial}{\partial\varphi}(\sin\varphi\frac{\partial\phi}{\partial\varphi})+\frac{1}{r^2\sin^2\varphi}\frac{\partial^2\phi}{\partial^2\theta},
\end{equation} with the usual notation $ g_{ij} $ the metric, $ g^{ij} $ the inverse metric, $ g=\det(g_{ij}) $ and where $ \theta $ denotes the azimuthal angle and $ \varphi $ the zenith angle, it is straightforward to show that\begin{equation}
\dot{H}_{\text{rel}}=\bigoplus_{l=0}^{\infty}U^{-1}h_lU\otimes\text{Id}_{\ell}
\end{equation}
with $ \text{Id}_\ell $ being the identity on $ \braket{Y_l^{-l},Y_l^{-l+1},...,Y_l^0,...,.Y_l^l} $. Here we have defined \begin{equation}
h_l=-\frac{d^2}{dr^2}+\frac{l(l+1)}{r^2}
\end{equation}
with the domains\begin{equation}
\begin{aligned}\dom{h_0}=\left\{u\in L_2((0,\infty),dr)|u,u'\in\text{AC}_{\text{loc}}(0,\infty),u''\in L_2((0,\infty),dr), u(0_+)=0, u'(0_+)=0\right\}
\end{aligned}
\end{equation}
\begin{equation}
\begin{aligned}
\dom{h_l}=\left\{u\in L_2((0,\infty),dr)|u,u'\in\text{AC}_{\text{loc}}(0,\infty),-u''+l(l+1)r^{-2}u\in L_2((0,\infty),dr)\right\},\quad l\geq 1
\end{aligned}
\end{equation}
Here $ \text{AC}(0,\infty) $ denotes the absolutely continuous functions on $ (0,\infty) $, and $ \text{AC}_{\text{loc}}(0,\infty) $ denotes the locally absolutely continuous functions, \ie AC on all compact intervals. Notice that for $ l\geq1 $ the boundary conditions $ u(0_+)=0, u'(0_+)=0 $ are automatically satisfied by the requirement $ -u''+l(l+1)r^{-2}u\in L_2((0,\infty),dr) $ and continuity of $ u $.
According to \cite{albeverio2012solvable}, it is a standard result that $ h_l $ is self-adjoint for $ l\geq1 $ However, it is not hard to see that $ h_0 $ has deficiency indices (1,1) and thus admits a one-parameter family of self-adjoint extensions. These extensions can all be characterized in terms of their self-adjoint boundary condition and are given by \begin{equation}
h_{0,\alpha}=-\frac{d^2}{dr^2},
\end{equation}
with domain \begin{equation}
\begin{aligned}
\dom{h_0}=\left\{u\in L_2((0,\infty),dr)|u,u'\in\text{AC}_{\text{loc}}(0,\infty),u''\in L_2((0,\infty),dr), -4\pi \alpha u(0_+)+u'(0_+)=0\right\},
\end{aligned}
\end{equation}
with $ \alpha\in (-\infty,\infty]$. The case $ \alpha=\infty $ simply corresponds to the boundary condition $ u(0+)=0 $, which simply implies $\lim\limits_{|x|\to 0}|x|\psi(x)=0$ for all $ \psi\in\dom{H_{\text{rel}}^\infty} $. This is the usual Friedrich extension i.e. $ H_{\text{rel}}^\infty=-\Delta $ with $ \dom{H_{\text{rel}}^\infty}=H_2(\R^3) $, \ie the free particle. 
$ \alpha $ can be related to $ \theta $ from before by a simple computation:
Let $ f=h+c(\xi_++\euler{i\theta}\xi_-)\in \dom{H_{\text{rel},\theta}} $ with $ \xi_\pm=\frac{\euler{i\sqrt{\pm i}|x|}}{4\pi |x|}= $ then
 \begin{equation}
 \lim\limits_{|x|\to0}|x|f(x)=\frac{c}{4\pi}(1+\euler{i\theta}),\qquad \lim\limits_{|x|\to0}\frac{d}{d|x|}(|x|f(x))=\frac{ic}{4\pi}(\sqrt{i}+\euler{i\theta}\sqrt{-i})
\end{equation}
Thus we have \begin{equation}
4\pi\alpha(1+\euler{i\theta})=i(\sqrt{i}+\euler{i\theta}\sqrt{-i})=\sqrt{i}(i-\euler{i\theta})=(\euler{i\frac{3}{4}}-\euler{i(\theta+\frac{1}{4})})
\end{equation}
from which it follows that \begin{equation}
\begin{aligned}
\alpha=\frac{(\euler{i\frac{3}{4}}-\euler{i(\theta+\frac{1}{4})})}{4\pi(1+\euler{i\theta})}=\frac{1}{4\pi}\frac{\euler{i(\theta+1)/2}(\euler{-i(\theta-\frac{1}{2})/2}-\euler{i(\theta-\frac{1}{2})/2})}{\euler{i\theta/2}(\euler{-i\theta/2}+\euler{i\theta/2})}=\frac{1}{4\pi}\frac{i(\euler{-i(\theta-\frac{1}{2})/2}-\euler{i(\theta-\frac{1}{2})/2})}{(\euler{-i\theta/2}+\euler{i\theta/2})}\\=\frac{1}{4\pi}\frac{\sin((\theta-\frac{1}{2})/2)}{\cos(\theta/2)}=\frac{1}{4\pi}\frac{-\cos(\theta/2)\sin(\frac{1}{4})+\sin(\theta/2)\cos(\frac{1}{4})}{\cos(\theta/2)}=\frac{1}{4\sqrt{2}\pi}\left(\tan(\theta/2)-1\right)\\
\end{aligned}
\end{equation}
Thereby we see that $ \alpha\in \R $ for $ \theta\in[0,\pi)\cup(\pi,2\pi) $ and that $ \alpha\to\infty $ when $ \theta\uparrow\pi $. Now we study the the resolvent of these extensions i.e. $ H_{\text{rel}}^{\alpha}=U^{-1}h_{0,\alpha}U\otimes\text{Id}_{0}\oplus\left(\bigoplus_{l=1}^{\infty}U^{-1}h_lU\otimes\text{Id}_{\ell} \right) $. To do this let us briefly summarize Krein's formula. In the following $ \rho(O) $ denotes the resolvent set of the operator $ O $.
\begin{theorem}[Krein's formula, A.2 in \cite{albeverio2012solvable}]
	\label{Krein's formula}
	Let $ B $ and $ C $ be self-adjoint extensions of the densely defined, closed, and symmetric operator $ A $ on the Hilbert space $ H $ with deficiency indices $ (1,1) $. Then their resolvant are related by:\begin{equation}
	(B-z)^{-1}-(C-z)^{-1}=\lambda(z)\braket{\phi(\bar{z}),\cdot}\phi(z),\qquad z\in\C\setminus\R,
	\end{equation}
	where $ \lambda(z)\neq0 $ for $ z\in\rho(B)\cap\rho(C) $ and $\lambda, \phi $ may be  chosen to be analytic functions in $ z\in\rho(B)\cap\rho(C) $. In fact, $ \phi $ may be taken as \begin{equation}
	\label{phi relation}
	\phi(z)=\phi(z_0)+(z-z_0)(C-z)^{-1}\phi(z_0),\quad z\in \rho(C)
	\end{equation}
	with $ \phi(z_0) $, $ z_0\in \C\setminus\R $ being a solution of \begin{equation}
	A^*\phi(z_0)=z_0\phi(z_0),
	\end{equation}
	Choosing this $ \phi $, we furthermore have $ \lambda $ satisfying the equation\begin{equation}
	\label{lambda relation}
	\lambda(z)^{-1}=\lambda(z')^{-1}-(z-z')\braket{\phi(\bar{z}),\phi(z')},\qquad z,z'\in\rho(B)\cap\rho(C).
	\end{equation}
\end{theorem}
\begin{proof}
	In order to prove this we remember the Krein extension theorem for densely defined, closed, symmetric operators. We have that the $ B $ and $ C $ are both of the form\begin{equation}\begin{aligned}
	\dom{B}=\{h+c(\xi_z+\euler{i\theta}\xi_{\bar{z}})\ |\ h\in\dom{A},c\in\C\},\qquad\qquad\qquad\qquad\\
	B(h+c(\xi_z+\euler{i\theta}\xi_{\bar{z}}))=Ah+c(z\xi_z+\euler{i\theta}\bar{z}\xi_{\bar{z}}),\qquad\qquad\qquad\Imag(z)\neq0,\quad
	\end{aligned}
	\end{equation}
	and 
	\begin{equation}\begin{aligned}
	\dom{C}=\{h+c(\xi_z+\euler{i\omega}\xi_{\bar{z}})\ |\ h\in\dom{A},\ c\in\C\},\qquad\qquad\qquad\qquad\\
	C(h+c(\xi_z+\euler{i\omega}\xi_{\bar{z}}))=Ah+c(z\xi_z+\euler{i\omega}\bar{z}\xi_{\bar{z}}).\qquad\qquad\qquad\Imag(z)\neq0\quad
	\end{aligned}
	\end{equation}
	where $  \xi_z\in\ker(A^*-z), \xi_{\bar{z}}\in\ker(A^*-\bar{z}), \omega\in[0,2\pi) $ and $ \theta\in[0,2\pi) $ are fixed with $ ||\xi_z||=||\xi_{\bar{z}}||=1 $,
	Now for $ z\in\rho(C) $, we know that $ (C-z) $ has full range and thus for $ x\in H $ we write $ x=(C-z)y $. Assuming that $ \Imag(z)\neq0 $ we can write $ y=h+\xi_z+\euler{i\omega}\xi_{\bar{z}}\in\dom{C} $ where we have absorbed the $ c\in\C $ into the $ \xi_z $ and $ \xi_{\bar{z}} $. Consider now \begin{equation}
	\left((B-z)^{-1}-(C-z)^{-1}\right)x=(B-z)^{-1}(C-z)(h+\xi_z+\euler{i\omega}\xi_{\bar{z}})-(h+\xi_z+\euler{i\omega}\xi_{\bar{z}}).
	\end{equation}
	Clearly $ (B-z)^{-1}(C-z)h=h $ since $ h\in\dom{A} $, and $ C $ and $ B $ are both extension of $ A $. Thus we obtain\begin{equation}
		\left((B-z)^{-1}-(C-z)^{-1}\right)x=(B-z)^{-1}((\bar{z}-z)\euler{i\omega}\xi_{\bar{z}})-(\xi_z+\euler{i\omega}\xi_{\bar{z}}).
	\end{equation}
	Since we have that $ (B-z)(\xi_z+\euler{i\theta}\xi_{\bar{z}})=\euler{i\theta}(\bar{z}-z)\xi_{\bar{z}} $, with $ \xi_z+\euler{i\theta}\xi_{\bar{z}}\in\dom{B} $, we find that \begin{equation}
	\left((B-z)^{-1}-(C-z)^{-1}\right)x=(\euler{i(\omega-\theta)}-1)\xi_{z},\qquad x=(C-z)(h+\xi_z+\euler{i\omega}\xi_{\bar{z}})=(C-z)h+(\bar{z}-z)\euler{i\omega}\xi_{\bar{z}},
	\end{equation}
	and we conclude \begin{equation}
		\left((B-z)^{-1}-(C-z)^{-1}\right)x=\frac{(\euler{-i\theta}-\euler{-i\omega})}{\bar{z}-z}\frac{\braket{\xi_{\bar{z}},x}}{||\xi_{\bar{z}}||^2}\xi_{z}=\lambda(z,\bar{z})\braket{\phi(\bar{z}),x}\phi(z)
	\end{equation}
	where we have used that $ (C-z)h=(A-z)h\in\Ran{A-z}\subset\ker(A^*-\bar{z})^\perp $, and the fact that by defining $\phi(z)=\phi(z_0)+(z-z_0)(C-z)^{-1}\phi(z_0) $, with $ \phi(z_0) $ being a solution of $ A^*\phi(z_0)=z_0\phi(z_0) $, we clearly have $ \phi(z)\in\ker(A^*-z) $ such that $ \phi(z)\parallel\xi_z $.\\
%	In order to calculate $ \lambda(z,\bar{z}) $ notice that \begin{equation}
%	\xi_z=\frac{\braket{\phi(z),\xi_z}}{||\phi(z)||^2}\phi(z)
%	\end{equation}
%	and that \begin{equation}
%	\begin{aligned}
%	\braket{\phi(z),\xi_z}&=\braket{\phi(z_0),\xi_z}+(\bar{z}-\bar{z}_0)\braket{\phi(z_0),(C-\bar{z})^{-1}\xi_z}\\
%	&=\braket{\phi(z_0),\xi_z}+\frac{\bar{z}-\bar{z}_0}{z-\bar{z}}\braket{\phi(z_0),\xi_z+\euler{i\omega}\xi_{\bar{z}}}
%	\end{aligned}
%	\end{equation}
%	Thus we have \begin{equation}
%	\begin{aligned}
%	\braket{\phi(z_0),\phi(z)}=\braket{\phi(z_0),\phi(z_0)}+(z-z_0)\braket{\phi(z_0),(C-z)^{-1}\phi(z_0)}
%	\end{aligned}
%	\end{equation}
%	such that \begin{equation}
%	\begin{aligned}
%	\braket{\phi(z),\phi(z)}=\braket{\phi(z_0),\phi(z)}+(\bar{z}-\bar{z}_0)\braket{(C-z)^{-1}\phi(z_0),\phi(z)}\\
%	=\braket{\phi(z_0),\phi(z)}+(\bar{z}-\bar{z}_0)\braket{\phi(z_0),(C-\bar{z})^{-1}\phi(z)}\\
%	=\braket{\phi(z_0),\phi(z)}+\frac{(\bar{z}-\bar{z}_0)}{z-\bar{z}}\braket{\phi(z_0),\phi(z)+\euler{i\omega}\phi(\bar{z})}
%		\end{aligned}
%	\end{equation}
%	So\begin{equation}
%	\begin{aligned}
%	(\bar{z}-z)\braket{\phi(z),\phi(z)}=(\bar{z}-z)\braket{\phi(z_0),\phi(z)}-(\bar{z}-\bar{z}_0)\left(\braket{\phi(z_0),\phi(z)}+\euler{i\omega}\braket{\phi(z_0),\phi(\bar{z})}\right)\\
%	=(\bar{z}_0-z)\braket{\phi(z_0),\phi(z)}+(\bar{z}_0-\bar{z})\braket{\phi(z_0),\phi(\bar{z})}
%	\end{aligned}
%	\end{equation}
%	\\
	We have that $ \lambda $ is given by the formula \begin{equation}
	\lambda(z,\bar{z})=\frac{(\euler{-i\theta}-\euler{-i\omega})}{\bar{z}-z}\frac{\braket{\phi(z)\xi_{z}}}{\braket{\phi(\bar{z}),\xi_{\bar{z}}}||\phi(z)||^2}
	\end{equation}
Notice that the above calculation is for fixed $ z $. Thus if we want to vary $ z $, we get that $ \theta $ and $ \omega $ might depend on $ z $ as we may choose $ \xi_z $ and $ \xi_{\bar{z}} $ differently at each $ z $ making a fixed $ \theta $ or $ \omega $ correspond to different extensions for  each $ z $.
 The fact that $ \lambda $ is analytic stems from the fact that all matrix elements of the resolvents are analytic in their resolvent sets and that we have chosen $ \phi(z) $ such that $ \braket{\phi(\bar{z}),x} $ is analytic for all $ x\in H $. To show that $ \lambda $ satisfies \eqref{lambda relation} is simply a long computation and we refer to appendix \ref{Lambda calculation} for the computation. 
%	Thus \begin{equation}\begin{aligned}
%	\lambda(z,\bar{z})^{-1}-\lambda(z',\bar{z}')^{-1}=\frac{1}{(\euler{-i\theta}-\euler{-i\omega})}\left((\bar{z}-z)\braket{\phi(z),\phi(z)}-(\bar{z'}-z')\braket{\phi(z'),\phi(z')}\right)\\
%	=\frac{1}{(\euler{-i\theta}-\euler{-i\omega})}\left((\bar{z}-z)[\braket{\phi(z),\phi(z)}-\braket{\phi(\bar{z}),\phi(z')}+\braket{\phi(\bar{z}),\phi(z')}]-(\bar{z'}-z')\braket{\phi(z'),\phi(z')}\right)
%	\end{aligned}
%	\end{equation}
\end{proof}
Now we have two self-adjoint extension of $ \dot{H}_{\text{rel}} $, namely $ H_{\text{rel}}^\infty $ and $ H_{\text{rel}}^{\alpha} $. It is easily verified that by imposing $ \alpha=\infty $ we obtain the Friedrich extension, given by (see above)\begin{equation}
\dom{H_{\text{rel}}^\infty}=H_2(\R^3),\qquad H_{\text{rel}}^\infty=-\Delta.
\end{equation}  We have already found the solution of $ H_{\text{rel}}^*\phi(z)=z\phi(z) $ (although we only found it for $ z=\pm i $) namely\begin{equation}
\phi(z)(x)=\frac{\euler{i\sqrt{z}|x|}}{4\pi|x|},\qquad \Imag(\sqrt{z})>0,
\end{equation}
Furthermore it is a straightforward generalization of this result that the Green function of $ (H_{\text{rel}}^\infty-z) $, \ie the integral kernel of the resolvent $ (H_{\text{rel}}^\infty-z)^{-1} $, is then \begin{equation}
G_z(x,x')=\frac{\euler{i\sqrt{z}|x-x'|}}{4\pi|x-x'|}.
\end{equation}
We immediately see that then\begin{equation}
\braket{\phi(\bar{z}),\phi(z')}=\frac{1}{4\pi}\int_{(0,\infty)}dr\euler{i(\sqrt{z'}-\overline{\sqrt{\bar{z}}})r}=\frac{1}{4\pi}\frac{-i}{\sqrt{z}-\sqrt{z'}},\qquad \Imag(\sqrt{z'}),\Imag(\sqrt{\bar{z}})>0
\end{equation}
Remember that $ \overline{\sqrt{\bar{z}}}\rvert_{\Imag(\sqrt{\bar{z}})>0}=\sqrt{z}|_{\Imag(\sqrt{z})<0}=-\sqrt{z}|_{\Imag(\sqrt{z})>0} $
so we have \begin{equation}
\lambda(z)^{-1}-\lambda(z')^{-1}=\frac{i}{4\pi}\frac{z'-z}{\sqrt{z}+\sqrt{z'}}=\frac{i}{4\pi}(\sqrt{z'}-\sqrt{z}),\qquad \Imag(z),\Imag(z')>0
\end{equation}
From which it follows that $ \lambda(z)=(\kappa-\frac{i}{4\pi}\sqrt{z})^{-1} $ Furthermore we have from Krein's formula (Theorem \ref{Krein's formula}) that \begin{equation}
(H_{\text{rel}}^\alpha-z)^{-1}=(H_{\text{rel}}^\infty-z)^{-1}+(\kappa-\frac{i}{4\pi}\sqrt{z})^{-1}\braket{\phi(\bar{z}),\cdot}\phi(z),
\end{equation}
where we notice that by \eqref{phi relation} we have\begin{equation}
\begin{aligned}
(\dot{H}_{\text{rel}}^*-z)\phi(z)=(\dot{H}_{\text{rel}}^*-z)\phi(z_0)+(z-z_0)(\dot{H}_{\text{rel}}^*-z)(H_{\text{rel}}^\infty-z)^{-1}\phi(z_0)&\\
=(z_0-z)\phi(z_0)+(z-z_0)\phi(z_0)=0&\qquad z\in\rho(H_{\text{rel}}^\infty)
\end{aligned}
\end{equation}
where we used that $ H_{\text{rel}}^\infty $ is a restriction of $ \dot{H}_{\text{rel}}^* $ such that $ (\dot{H}_{\text{rel}}^*-z)(H_{\text{rel}}^\infty-z)^{-1}=\text{Id} $. However from this we conclude that $ \phi(z)=G_{z}(x,0)=\frac{\euler{i\sqrt{z}|x|}}{|x|}, $ $ \Imag(\sqrt{z})>0 $. Thereby we have \begin{equation}
(H_{\text{rel}}^\alpha-z)^{-1}=(H_{\text{rel}}^\infty-z)^{-1}+(\kappa+i\sqrt{z})^{-1}\braket{G_{\bar{z}}(\ast,0),\cdot(\ast)}G_z(\cdot,0),\qquad z\in\rho(H_{\text{rel}}^\alpha)\cap\rho(H_{\text{rel}}^\infty),
\end{equation}
where the $ \ast $ refers to the integrated variable in the inner product.\\
In order to determine $ \kappa $, we perform a simple calculation. Let $ u\in\dom{h^\alpha_0} $ Then $ \frac{1}{r}uY_0^0\in\dom{H_\text{rel}^\alpha} $ and we have \begin{equation}
(H_\text{rel}^\alpha-z)\frac{1}{r}uY_0^0=\left(-\frac{1}{r}\frac{d^2u(r)}{dr^2}-z\frac{1}{r}u(r)\right)Y^0_0
\end{equation}
Thus we have \begin{equation}
\begin{aligned}
&u(0)=\lim\limits_{r\to0}r\left((H_\text{rel}^\alpha-z)^{-1}(H_\text{rel}^\alpha-z)\frac{1}{r}u\right)(r)\\&=\lim\limits_{r\to0}r4\pi\left(\int_{(0,\infty)}drr \left(-\frac{d^2u}{dr^2}-zu\right)G_z(r,0)+(\kappa-\frac{i}{4\pi}\sqrt{z})^{-1}G_z(r,0)\int dr r\overline{G_{\bar{z}}(r,0)}\left(-\frac{d^2u}{dr^2}-zu\right)\right)
\end{aligned}
\end{equation}
Notice that $ \overline{G_{\bar{z}}(r,0)}=G_z(r,0) $ and that $ r4\pi G_z(r,0)=\euler{i\sqrt{z}r} $. By partial integration twice we have
\begin{equation}
\int_{(0,\infty)}dr \left(-\euler{i\sqrt{z}r}\frac{d^2}{dr^2}u+u\frac{d^2}{dr^2}\euler{i\sqrt{z}r}\right)=\frac{du}{dr}(0+)-i\sqrt{z}u(0+),
\end{equation}
from which we get
\begin{equation}
u(0)=(\kappa-\frac{i}{4\pi}\sqrt{z})^{-1}\frac{1}{4\pi}\left(\frac{du}{dr}(0+)-i\sqrt{z}u(0+)\right).
\end{equation}
By imposing the boundary condition on $ u $ at $ 0 $ we obtain the equation for $ \kappa $\begin{equation}
1=(\kappa-\frac{i}{4\pi}\sqrt{z})^{-1}\frac{1}{4\pi}\left(4\pi\alpha-i\sqrt{z}\right),
\end{equation}
Thus that $ \kappa=\alpha $ and we have the resolvent \begin{equation}
(H_{\text{rel}}^\alpha-z)^{-1}=(H_{\text{rel}}^\infty-z)^{-1}+(\alpha-\frac{i}{4\pi}\sqrt{z})^{-1}\braket{G_{\bar{z}}(\ast,0),\cdot(\ast)}G_z(\cdot,0),
\end{equation}
We are now ready to study the spectrum of the the operators $ (H_{\text{rel}}^\alpha)_{\{\alpha\in(-\infty,\infty]\}} $. Clearly $ z\in\sigma(H_{\text{rel}}^\alpha) $ if $ z\in\sigma(-\Delta|_{C_c^\infty(\R^3)})=[0,\infty) $. On the other hand we see that if $ \alpha<0 $ then $ z=-(4\pi\alpha)^2\in\sigma(H_{\text{rel}}^\alpha) $. Therefore the spectrum can be characterized as\begin{equation}
\sigma(H_{\text{rel}}^\alpha)=\begin{cases}
[0,\infty)&\text{if }\alpha\geq0,\\
\{-(4\pi\alpha)^2\}\cup[0,\infty)&\text{if }\alpha<0.
\end{cases}
\end{equation}
It is of course an exercise to show that no other points are in the spectrum. We refer to \cite{albeverio2012solvable} for a short proof, and further classification of different parts of the spectrum, \ie point-, singular continuous-, and absolute continuous spectrum. 
We note that for $ \alpha<0 $ the point in the spectrum $ \{-(4\pi\alpha)^2\} $ is an eigenvalue (\ie a part of the point spectrum). Furthermore, we can actually, in the $ \alpha<0 $ case, determine the eigenfunction corresponding to the eigenvalue $ -(4\pi\alpha^2) $. To do this, notice that the domain of $ H_{\text{rel}}^\alpha $ can be written as\\ $ \dom{H_{\text{rel}}^\alpha}=\{w(x)+(\alpha-\frac{i}{4\pi}k)^{-1}w(0)G_{k^2}(x,0)\ |\ w\in H_2(\R^3),\ k^2\in\rho(H_{\text{rel}}^\alpha)\} $ for $ k\in\rho(H_{\text{rel}}^\alpha)\cap(H_{\text{rel}}^\alpha) $. This follows by the fact that \begin{equation}
\dom{H_{\text{rel}}^\alpha}=(H_{\text{rel}}^\alpha-k^2)^{-1}(H_{\text{rel}}^\infty-k^2)\dom{H_{\text{rel}}^\infty},
\end{equation}
where we have used that $ \braket{G_{\bar{z}}(x,0),(H_{\text{rel}}^\infty-z)w}=\braket{(H_{\text{rel}}^\infty-\bar{z})G_{\bar{z}}(x,0),w}=\braket{\delta_0,w}=w(0) $. Notice that $ w\in H_2(\R^3) $ is continuous, so $ w(0) $ makes sense. We thus have the action of $ H_{\text{rel}}^\alpha $\begin{equation}
(H_{\text{rel}}^\alpha-k^2)(w(x)+(\alpha-\frac{i}{4\pi}k)^{-1}w(0)G_{k^2}(x,0))=(H_{\text{rel}}^\infty-k^2)w(x)=(-\Delta-k^2)w(x).
\end{equation}
Now notice that if we fix $ w\in H_2(\R^3) $ such that $ w(0)=1 $ and we define $ (x_n)_{(n\geq1)} $ such that $ x_n\to0 $ as $ n\to\infty $, then $ x_nw\to0 $ in $ L_2(\R^3) $. Furthermore let $ k_n^2\to-(4\pi\alpha)^2 $ such that $ \left(\alpha-ik_n/(4\pi)\right)\frac{1}{x_n}=1 $ for all $ n\geq1 $, then we have \begin{equation}
(H_{\text{rel}}^\alpha-k_n^2)(x_nw+(\alpha-\frac{i}{4\pi}k_n)^{-1}x_nG_{k_n^2})=x_n(-\Delta-k^2)w\to 0\qquad \text{as $ n\to\infty $},
\end{equation}
with the notation $ G_{k^2} $ for $ G_{k^2}(x,0) $. Thus we have that \begin{equation}
\begin{aligned}
\lim_{n\to\infty}\left(H_{\text{rel}}^\alpha(x_nw+(\alpha-\frac{i}{4\pi}k_n)^{-1}x_nG_{k_n^2})\right)=\lim\limits_{n\to\infty}\left(k_n^2(x_nw+(\alpha-\frac{i}{4\pi}k_n)^{-1}x_nG_{k_n^2})\right)\\= -(4\pi\alpha)^2G_{-(4\pi\alpha)^2}.
\end{aligned}
\end{equation}
Where we have used that $ k_n^2I\to-(4\pi\alpha)^2I $ in operator norm as $ n\to \infty $ and that $ (k_n^2I)_{(n\geq1)} $ is uniformly bounded. Furthermore, we used that $ G_{k_n^2}\to G_{-(4\pi\alpha)^2} $ in $ L_2(\R^3) $ for $ \alpha<0 $. Thus we conclude that defining $ \chi_n=x_nw+(\alpha-\frac{i}{4\pi}k_n)^{-1}x_nG_{k_n^2} $ we have that $ \left(\chi_n\right)_{(n\geq1)}\subset\dom{H_{\text{rel}}^\alpha} $ converges in $ L_2(\R^3) $ to $ G_{-(4\pi\alpha)^2} $ and that $ \left(H_{\text{rel}}^\alpha\chi_n\right)_{(n\geq1)} $ converges in $ L_2(\R^3) $ to $ -(4\pi\alpha)^2G_{-(4\pi\alpha)^2} $. By closedness of the self-adjoint operator $ H_{\text{rel}}^\alpha $ we conclude that $ G_{-(4\pi\alpha)^2}\in\dom{H_{\text{rel}}^\alpha} $ and that\begin{equation}
H_{\text{rel}}^\alpha G_{-(4\pi\alpha)^2}=-(4\pi\alpha)^2G_{-(4\pi\alpha)^2}.
\end{equation}
Thus $ G_{-(4\pi\alpha)^2} $ is the eigenfunction corresponding to the bound state with energy $ -(4\pi\alpha)^2 $. We have thus constructed the Hamiltonian of the point-interaction in $ 3d $. We have seen that the parameter $ \alpha $, in some sense, controls the strength of the interaction, \ie $ \alpha=\infty $ is the free particle, and $ \alpha<0 $ is the attractive point-interaction, since it has a bound state. By doing a analysis of the scattering theory of $ H_{\text{rel}}^\alpha $ one finds that $ -4\pi\alpha=\frac{1}{a} $, where $ a $ denotes the scattering length of the interaction. 
\subsection{Quadratic form}
An alternative way of studying the point interaction is by the means of quadratic forms. In \cite{FINCO2012131}, the quadratic form, $ F_\alpha $ describing a gas of point interacting fermions was obtained. It is a well-known result that if such a quadratic form is closed and bounded from below, then the corresponding operator is a bounded from below, self-adjoint operator. On the other hand it was proven in \cite{FINCO2012131} that if the quadratic form $ F_\alpha $ is not bounded from below, then the corresponding operator is not bounded from below and self-adjoint. The quadratic form was initially introduced by means of renormalization. On the other hand it is more clearly introduced as a rank-one perturbation of the free quadratic form. We write the rank-one perturbation as $ \gamma\braket{\phi,\cdot}\phi $. Thus we imagine perturbing the Hamiltonian as $ H=H_0-\gamma\braket{\phi,\cdot}\phi $. For the point interaction we do this by simply projecting onto a ball $ B_R(0) $, \ie the ball of radius $ R $ centred at $ 0 $, in momentum space \ie\begin{equation}
\widehat{Hu}=\widehat{H_0u}-\mathbbm{1}_{B_R(0)}\frac{\gamma}{(2\pi)^3}\int_{B_R(0)}\!\!\!\!\!\!\!\!d^3p\ \hat{u}(p)=H_0u-\mathbbm{1}_{B_R(0)}\frac{\gamma}{(2\pi)^3}\int_{B_R(-k)}\!\!\!\!\!\!\!\!\!\!\!\! d^3p\ \hat{u}(k+p).
\end{equation}
Thus we obtain the quadratic form\begin{equation}
F^R_\gamma(\hat{u})=\int_{\R^3} d^3k\ \left(\bar{\hat{u}}(k)(k^2)\hat{u}(k)-\frac{\gamma}{(2\pi)^3}\int_{B_R(-k)}\!\!\!\!\!\!\!\!\!\!\!\! d^3p\ \bar{\hat{u}}(k) \hat{u}(k+p)\right).
\end{equation}
This can be rewritten in the following form\begin{equation}
	\begin{aligned}
	F_\gamma^R(u)=\int_{\R^3} d^3k\ \left(k^2+\mu\right)|\hat{u}(k)-\widehat{G\rho^R}(k)|^2-\mu||u||^2_{L_2(\R^3)}-\int_{\R^3} d^3k\ (k^2+\mu)|\widehat{G\rho^R}(k)|^2\\
	+2\Real \int_{\R^3} d^3k\ \bar{\hat{u}}(k)(k^2+\mu)\widehat{G\rho^R}(k)-\int_{\R^3} d^3k\ \bar{\hat{u}}(k)\hat{\rho}^R(k),
	\end{aligned}
\end{equation}
where $ \mu>0 $ and we have defined
\begin{equation}
\begin{aligned}
\hat{G}(k)=\frac{1}{k^2+\mu},\qquad \hat{\rho}^R(k)=\gamma_R\mathbbm{1}_{B_R(0)}(k)\int_{B_R(-k)}\!\!\!\!\!\!\!\!\!d^3p\ \hat{u}(k+p) =\mathbbm{1}_{B_R(0)}(k)\xi_R,\\
\xi_R=\gamma_R\int_{B_R(0)} d^3p\ \hat{u}(p),\qquad\qquad\qquad\qquad\qquad\qquad
\end{aligned}
\end{equation}
furthermore, $ \widehat{G\rho^R}(k)=\hat{G}(k)\hat{\rho^R}(k) $ and $ \gamma_R=\frac{\gamma}{(2\pi)^3} $, which we have allowed to depend on $ R $, since it will need to be renormalized eventually. Now straightforward calculation shows that \begin{equation}
\begin{aligned}
\overline{\int_{\R^3} d^3k\ \bar{\hat{u}}(k)(k^2+\mu)\widehat{G\rho^R}(k)}=\int_{\R^3} d^3k\ \overline{ \bar{\hat{u}}(k)\hat{\rho}^R(k)}=\gamma_R\int_{B_R(0)} d^3k\ \hat{u}(k) \int_{B_R(0)} d^3p\ \bar{\hat{u(p)}} \\
=\gamma_R\int_{B_R(0)} d^3k\ \bar{\hat{u}}(k) \int_{B_R(0)} d^3p\ \hat{u(p)}=\int_{\R^3} d^3k\ \bar{\hat{u}}(k)\hat{\rho}^R(k)=\int_{\R^3} d^3k\ \bar{\hat{u}}(k)(k^2+\mu)\widehat{G\rho^R}(k),
\end{aligned}
\end{equation}
such that $ 2\Real \int_{\R^3} d^3k\ \bar{\hat{u}}(k)(k^2+\mu)\widehat{G\rho^R}(k)=2\int_{\R^3} d^3k\ \bar{\hat{u}}(k)\hat{\rho}^R(k) $. Thereby we find the quadratic form\begin{equation}
\begin{aligned}
F_\gamma^R(u)&=\int_{\R^3} d^3k\ \left(k^2+\mu\right)|\hat{u}(k)-\widehat{G\rho^R}(k)|^2-\mu||u||^2_{L_2(\R^3)}-\int_{\R^3} d^3k\ (k^2+\mu)|\widehat{G\rho^R}(k)|^2\\
&\qquad\qquad\qquad\qquad\qquad\qquad\qquad\qquad\qquad\qquad\qquad\qquad+ \int_{\R^3} d^3k\ \bar{\hat{u}}(k)\hat{\rho}^R(k)\\
&=\int_{\R^3} d^3k\ \left(k^2+\mu\right)|\hat{u}(k)-\widehat{G\rho^R}(k)|^2-\mu||u||^2_{L_2(\R^3)}-|\xi_R|^2\int_{B_R(0)} d^3k \hat{G}(k)
+ \gamma_R^{-1}|\xi_R|^2.
\end{aligned}
\end{equation}
Now by computing the \begin{equation}
\begin{aligned}
\int_{B_R(0)}\hat{G}=4\pi\int_{0}^{R}dr\ \frac{r^2}{r^2+\mu}=4\pi\sqrt{\mu}\int_{0}^{R/\sqrt{\mu}}dq\ \frac{q^2}{q^2+1}=4\pi\sqrt{\mu}\left(\frac{R}{\sqrt{\mu}}-\int_{0}^{R/\sqrt{\mu}}dq\ \frac{1}{q^2+1}\right)\\=4\pi\left(R-\sqrt{\mu}\arctan\left(\frac{R}{\sqrt{\mu}}\right)\right).
\end{aligned}
\end{equation}
Thereby we have the quadratic form\begin{equation}
\begin{aligned}
F_\gamma^R(u)=\int_{\R^3} d^3k\ \left(k^2+\mu\right)|\hat{u}(k)-\widehat{G\rho^R}(k)|^2-\mu||u||^2_{L_2(\R^3)}\\-|\xi_R|^2\left(4\pi R-4\pi\sqrt{\mu}\arctan\left(\frac{R}{\sqrt{\mu}}\right)-\gamma_R^{-1}\right).
\end{aligned}
\end{equation}
Since we are interested in the limit $ R\to\infty $ (corresponding to localizing the interaction to a point) we choose $ \gamma_R $ such that the divergence in $ R $ disappears. Choosing the coupling $ \gamma_R^{-1}=4\pi R+\alpha $ we obtain the final quadratic form\begin{equation}
\begin{aligned}
F_\alpha^R(u)=\int_{\R^3} d^3k\ \left(k^2+\mu\right)|\hat{w}_R(k)|^2-\mu||u||^2_{L_2(\R^3)}+|\xi_R|^2\left(\alpha+4\pi\sqrt{\mu}\arctan\left(\frac{R}{\sqrt{\mu}}\right)\right),
\end{aligned}
\end{equation}
with $ \hat{w}_R=\hat{u}-\widehat{G\rho^R} $. Notice, that the domain of this quadratic form is \begin{equation}
\dom{F_\alpha^R}=\left\{u\in L_2(\R^3)\ |\ w_R\in H_1(\R^3) \right\}
\end{equation} Heuristically, we can take the limit $ R\to\infty $. This is done by noticing that for $ w\in H_1(\R^3) $ we have that $ \hat{w}(k)=\frac{\hat{f}(k)}{(|k|^2+1)^{\frac{1}{2}}} $ for some $ f\in L_2(\R^3) $. By H\"older's inequality $ (2,2) $ we thus see \begin{equation}
\int_{B_R(0)} \hat{w}\leq\left(\int_{B_R(0)}\left\lvert\frac{1}{|k|^2+1}\right\rvert^2 \right)^{\frac{1}{2}}||f||_2\lesssim\sqrt{R}
\end{equation}
where by $ f(x)\lesssim g(x)$ we mean that there exist some constant $ C\in\R $ such that $ f(x)\leq Cg(x) $. Thus we see that the equation for $ \xi_R $ \begin{equation}
\xi_R=\frac{1}{(4\pi R+\alpha)}\int_{B_R(0)}\left(\hat{w}+\widehat{G\rho^R}\right)=\frac{1}{(4\pi R+\alpha)}\int_{B_R(0)}\left(\hat{w}+\xi_R\hat{G}\right),
\end{equation} becomes in the limit $ R\to\infty $ the equation $ \xi_R=\xi_R $. Thus any choice of $ \xi_R $ is consistent and simply let it be a free parameter $ \xi_R=\xi\in\C $. We also see that in the limit $ R\to\infty $ we have that $ \arctan\left(R/\sqrt{\mu}\right)\to\frac{\pi}{2} $. Thus we get the quadratic form\begin{equation}
F_\alpha(u)=\int_{\R^3} d^3k\ \left(k^2+\mu\right)|\hat{w}(k)|^2-\mu||u||^2_{L_2(\R^3)}+|\xi|^2\left(\alpha+2\pi^2\sqrt{\mu}\right),
\end{equation}
with domain\begin{equation}
\dom{F_\alpha(u)}=\left\{u\in L_2(\R^3) \ |\ \hat{u}=\hat{w}+\xi\hat{G},\ w\in H_1(\R^3),\ \xi\in\C \right\},
\end{equation}
which matches the expression of \cite{Moser_2017} in the $ N=M=1 $ case.
\subsection{Hamiltonian from quadratic form}
We are in this subsection going to construct the Hamiltonian for the point interactions from the quadratic form. This will serve both as an example of how obtain the Hamiltonian given its quadratic form, but also as a motivation that the quadratic form given in the previous section is indeed equivalent to the the self-adjoint extension $ H_{\text{rel}}^\alpha $.\\
First we need to define a few properties of quadratic forms.\begin{definition}
	We say a quadratic form on some Banach space $ q: \dom{q}\to\R $ is bounded from below if $ q(v)\geq-c||v||^2 $ for some $ c>0 $. 
\end{definition}
\begin{definition}
	We say that a quadratic form, $ q:\dom{q}\to\R $, which is bounded from below, $ q(v,v)\geq-c||v||^2 $, is closed if its domain, $ \dom{q} $, is a Banach space when equipped with the norm $ ||v||^2_q=q(v,v)+C||v||^2 $, where $ C>c $.
\end{definition}
Notice that given a quadratic form $ q:\dom{q}\to\R $ we can always construct a symmetric sesquilinear from by \begin{equation}
\begin{aligned}
\Real (q(u,v))=\frac{1}{2}(q(u-v)-q(u)-q(v)),\\ \Imag(q(u,v))=\frac{1}{2i}(q(u+iv)-q(u)-q(v)).
\end{aligned}
\end{equation}
where we abuse notation and use the symbol $ q $ for both the quadratic form and the sesquilinear form.
This motivates the following proposition \begin{proposition}
	A quadratic form on some Hilbert space $ H $, $ q:\dom{q}\to\R $, which is bounded from below, $ q(v)\geq-c||v||^2 $, is closed if and only if its domain, $ \dom{q} $, is a Hilbert space when equipped with the inner product $ \braket{u,v}_q=q(u,v)+C\braket{u,v} $, where $ C>c $.
\end{proposition}
\begin{proof}
	Since the norm $ ||\cdot||_q$ is generated by $ \braket{\cdot,\cdot}_q $ it is clear that this follows if we can show that $ \braket{u,v}_q $ is in fact an inner product. Sesquilinearity is obvious by construction. Furthermore $ \braket{v,v}_q=||v||_q^2\geq0 $ since $ q $ is bounded from below, $ q(v)\geq-c||v||^2 $, and $ \braket{v,v}_q=0 $ if and only if $ v=0 $ follows from the fact that $ \braket{v,v}_q\geq(C-c)||v||^2 $.
\end{proof}
Notice that it follows from the fact that $ \braket{\cdot,\cdot}_q $ is an inner product that $ ||\cdot||_q $ is in fact a norm.
We start out by the quadratic form from then previous section
\begin{equation}
F_\alpha(u)=\int_{\R^3} d^3k\ \left(k^2+\mu\right)|\hat{w}(k)|^2-\mu||u||^2_{L_2(\R^3)}+|\xi|^2\left(\alpha+2\pi^2\sqrt{\mu}\right),
\end{equation}
with domain\begin{equation}
\dom{F_\alpha(u)}=\left\{u\in L_2(\R^3) \ |\ \hat{u}=\hat{w}+\xi\hat{G},\ w\in H_1(\R^3),\ \xi\in\C \right\}.
\end{equation}
This quadratic for is closed and bounded from below, it is also clear that this quadratic form has a corresponding symmetric sesquilinear form, with domain\\ $ \dom{F_\alpha(\cdot,\cdot)}=\dom{F_\alpha(\cdot)}\times\dom{F_\alpha(\cdot
	)} $, given by\begin{equation}
F_\alpha(u,v)=\int_{\R^3} d^3k\ \left(k^2+\mu\right)\overline{\hat{w}(k)}\hat{h}(k)-\mu\braket{u,v}_{L_2(\R^3)}+\bar{\xi}\chi\left(\alpha+2\pi^2\sqrt{\mu}\right),
\end{equation}
where $ u=w+\xi G $ and $ v=h+\chi G $. The domain of the corresponding operator $ H_\alpha $ is defined by \begin{equation}
\dom{H_\alpha}=\{u\in \dom{F_\alpha}\ |\ F_\alpha(u,\cdot) \text{ is an $ L_2(\R^3) $ bounded linear functional on }\dom{F_\alpha} \}.
\end{equation}
By density of $ \dom{F_\alpha} $ in $ L_2(\R^3) $ and Riez representation theorem, we known that if $ h\in\dom{H_\alpha} $ then $ F_\alpha(u,\cdot)=\braket{x,\cdot} $ and we define $ H_\alpha u=x $. Clearly $ H_\alpha $ is linear and symmetric by the very construction\begin{equation}
\braket{H_\alpha u,v}=F_\alpha(u,v)=\overline{F_\alpha(v,u)}=\overline{\braket{H_\alpha v, u}}=\braket{u,H_\alpha v}
\end{equation} 
For $ u,v \in\dom{H_\alpha} $. Notice that \begin{equation}
\begin{aligned}
\dom{H_\alpha^*}&=\{v\in \dom{F_\alpha}\ |\ \braket{H_\alpha \cdot, v}\text{ is bounded on } \dom{H_\alpha} \}\\&=\{v\in L_2(\R^3)\ |\ F_\alpha(\cdot,v) \text{ is bounded on } \dom{H_\alpha} \}.
\end{aligned}
\end{equation}
Assuming that $ \dom{H_\alpha} $ is dense in $ \dom{F_\alpha} $ we then have $ \dom{H_\alpha^*}=\dom{H_\alpha} $ and the operator, $ H_\alpha $, is self-adjoint. It is a general fact that $ \dom{H_\alpha} $ is dense whenever $ F_\alpha $ is closed. Thus we are now ready to calculate the Hamiltonian of the quadratic form $ F_\alpha $. Notice that by the definition of $ \dom{H_\alpha} $ we must have that for $ u\in\dom{H_\alpha} $ and $  (v_n)_{n\geq1}\subset\dom{F_\alpha} $  such that $ v_n\to0 $ in $ L_2(\R^3) $ it holds that  $ F_\alpha(v,u_n)\to 0 $. Thus by writing $ u=w+\xi G $ with $ w\in H_1(\R^3) $ and $ \xi\in\C $ and $ v_n=h_n+\chi G\in\dom{F_\alpha} $, with $ h_n\in H_1(\R^3) $ such that $ h_n\to-\chi G $ in $ L_2(\R^3) $ we have \begin{equation}
F(u,v_n)=\int_{\R^3} d^3k\ \left(k^2+\mu\right)\overline{\hat{w}(k)}\hat{h}_n(k)-\mu\braket{u,v_n}_{L_2(\R^3)}+\bar{\xi}\chi\left(\alpha+2\pi^2\sqrt{\mu}\right).
\end{equation}
We immediately see that for the first term to be $ L_2(\R^3) $ bounded in $ h_n $ we must have that $ w\in H_2(\R^3) $. Secondly since the first term is $ L_2(\R^3) $ bounded we must have \begin{equation}
\int_{\R^3} d^3k\ \left(k^2+\mu\right)\overline{\hat{w}(k)}\hat{h}_n(k)\to-\int_{\R^3} d^3k\ \left(k^2+\mu\right)\overline{\hat{w}(k)}\chi \hat{G}(k)=-\chi\int_{\R^3} d^3k\ \overline{\hat{w}(k)}\quad \text{as }n\to\infty.
\end{equation} 
The second term obviously goes to zero by continuity of the inner product. Thus we need to estimate $ \int_{\R^3}\hat{h} $. This is done in the following lemma.
\begin{lemma}
	Let $ w\in H_2(\R^3) $, then $ w $ is continuous and $ \frac{1}{(2\pi)^{3/2}}\int_{\R^3}d^3k\ \hat{w}(k)e^{ik\cdot x}=w(x) $.
\end{lemma}
\begin{proof}
	First that $ w $ is continuous follows from Sobolev's embedding theorem. Next notice that since $ w\in H_2(\R^3) $ we must have that \begin{equation}
	\hat{w}(k)=\frac{\hat{f}(k)}{|k|^2+1},
	\end{equation}
	for some $ \hat{f}\in L_2(\R^3) $. Thus we also see that $ \hat{w} $ is clearly in $ L_1(\R^3) $ by H\"older's inequality and the fact that $ \frac{1}{|k|^2+1}\in L_2(\R^3) $ and therefore $ \check{\hat{w}} $ is bounded and continuous. By Fourier's inversion theorem we have that $ \check{\hat{w}}=w $ a.e, and by continuity of $ w $, we conclude that $ \check{\hat{w}}=w $. Since $ \hat{w}\in L_1(\R^3) $ this amounts to \begin{equation}
	\frac{1}{(2\pi)^{3/2}}\int_{\R^3}d^3k\ \hat{w}(k)e^{ik\cdot x}=w(x) 
	\end{equation}
\end{proof}
Now given the above lemma we clearly see that \begin{equation}
\chi\int_{\R^3} d^3k\ \overline{\hat{w}(k)}=(2\pi)^{3/2}\chi\overline{w(0)}.
\end{equation}
Thereby we find the condition on the domain\begin{equation}
-(2\pi)^{3/2}\chi\overline{w(0)}+\bar{\xi}\chi(\alpha+2\pi^2\sqrt{\mu})=0
\end{equation}
correpsponding to the boundary condition $ w(0)=(2\pi)^{-3/2}\xi(\alpha+2\pi^2\sqrt{\mu}) $. Now turning to the action of the operator $ H_\alpha $ it is easier to consider $ H_\alpha+\mu $ since we then have \begin{equation}
\begin{aligned}
\braket{(H_\alpha+\mu)u,v}=F_\alpha(u,v)+\mu\braket{u,v}&=\int_{\R^3} d^3k\ \left(k^2+\mu\right)\overline{\hat{w}(k)}\hat{h}(k)+\bar{\xi}\chi\left(\alpha+2\pi^2\sqrt{\mu}\right)\\
&=\int_{\R^3} d^3k\ \left(k^2+\mu\right)\overline{\hat{w}(k)}\hat{h}(k)+\chi\int_{\R^3}d^3k \overline{\hat{w}(k)}\\
&=\int_{\R^3} d^3k\ \left(k^2+\mu\right)\overline{\hat{w}(k)}\hat{h}(k)+\chi\int_{\R^3}d^3k(k^2+\mu) \overline{\hat{w}(k)}\hat{G}(k)\\
&=\braket{(-\Delta+\mu)w,v},
\end{aligned}
\end{equation}
with $ u=w+\xi G $ and $ v=h+\chi G $ and where we used the boundary condition we found above in line $ 2 $.\\
 We can therefore write down the Hamiltonian in the following manner:\begin{equation}
\begin{aligned}
\dom{H_\alpha}=\left\{ u\in L_2(\R^3)\ |\ u=w+\xi G,\ w\in H_2(\R^3),\ w(0)=(2\pi)^{-3/2}\xi(\alpha+2\pi^2\sqrt{\mu}),\ \xi\in\C \right\}\\
(H_\alpha+\mu)u=(-\Delta+\mu)w\qquad\qquad\qquad\qquad\qquad\qquad\qquad
\end{aligned}
\end{equation}
This concludes how to obtain the Hamiltonian given the quadratic form. Notice that this expression also matches the one we found by self-adjoint extension. To see this, we have to notice that there is a bit a mismatch between the normalizations of the Green functions used in the two methods. We see this by computing \begin{equation}
\begin{aligned}
\hat{G}_{-\mu}(k)=\frac{1}{(2\pi)^{3/2}}\int_{\R^3} d^3x G_{-\mu}(x,0)\euler{-ik\cdot x}=\frac{2\pi}{4\pi(2\pi)^{3/2}}\int_{-1}^{1} d[\cos(\varphi)]\int_{0}^{\infty}dr r \euler{-\sqrt{\mu}r}\euler{-i|k|r\cos(\varphi)}\\
=\frac{1}{(2\pi)^{3/2}}\int_{0}^{\infty}dr r=\frac{1}{(2\pi)^{3/2}} \frac{\euler{-\sqrt{\mu}r}\sin(|k|r)}{|k|r}
=\frac{1}{(2\pi)^{3/2}}\frac{1}{2i|k|}\left[\frac{1}{i(i\sqrt{\mu}-|k|)}-\frac{1}{i(i\sqrt{\mu}+|k|)}\right]\\
=\frac{1}{(2\pi)^{3/2}}\frac{1}{2|k|}\left[\frac{i\sqrt{\mu}-|k|-i\sqrt{\mu}-|k|}{-\mu-|p^2|}\right]=\frac{1}{(2\pi)^{3/2}}\frac{1}{\mu+|k|^2}=\frac{1}{(2\pi)^{3/2}}\frac{1}{\mu+|k|^2}
\end{aligned}
\end{equation}
Thus we see that $ G_{-\mu}=\frac{1}{(2\pi)^{3/2}}G $, and we get
\begin{equation}
\begin{aligned}
\dom{H_\alpha}=\left\{ u\in L_2(\R^3)\ |\ u=w+\xi G,\ w\in H_2(\R^3),\ w(0)=(2\pi)^{-3/2}\xi(\alpha+2\pi^2\sqrt{\mu}),\ \xi\in\C \right\}\\
=\left\{ u\in L_2(\R^3)\ |\ u=w+(2\pi)^{3/2}(\alpha+2\pi^2\sqrt{\mu})^{-1}(2\pi)^{3/2}w(0) G_{-\mu},\ w\in H_2(\R^3)\right\}\\
=\left\{ u\in L_2(\R^3)\ |\ u=w+\left(\frac{\alpha}{(2\pi)^3}+\frac{\sqrt{\mu}}{4\pi}\right)^{-1}w(0) G_{-\mu},\ w\in H_2(\R^3)\right\}
\end{aligned}
\end{equation}
We see that this exactly equal to the domain found in the previous section except for the fact that $ \alpha $ has been replaced by $ \alpha/(2\pi)^{3} $\begin{equation}
\dom{H_{\text{rel}}^\alpha}=\left\{ u\in L_2(\R^3)\ |\ u=w+\left(\alpha-i\frac{i\sqrt{\mu}}{4\pi}\right)^{-1}w(0) G_{-\mu},\ w\in H_2(\R^3)\right\}.
\end{equation}
Therefore we conclude that $ H_\alpha=H_{\text{rel}}^{\frac{\alpha}{(2\pi)^3}} $.
\section{$ \Gamma $-convergence}
\appendix
\section{$ \lambda $ relation in Krein formula \eqref{lambda relation}}
\label{Lambda calculation}
To show the relation of $ \lambda(z,\bar{z}) $ we use the following properties established in the proof in the main text. We have that \begin{equation}\label{Krein relation 1}
(B-z)^{-1}-(C-z)^{-1}=\lambda(z)\braket{\phi(\bar{z}),\cdot}\phi(z),
\end{equation}
where we have already used that $ \lambda(z,\bar{z})=\lambda(z) $, and where we have defined \begin{equation}\label{Krein relation 2}
\phi(z)=\phi(z_0)+(z-z_0)(C-z)^{-1}\phi(z_0),
\end{equation}
for some $ \phi(z_0) $ satisfying $ A^*\phi(z_0)=z_0\phi(z_0) $. \\
In the following we will switch to bra-ket notation to simplify the calculations and ease the notation. Now let us consider the relation \begin{equation}
(B-z)^{-1}=(C-z)^{-1}+\lambda(z)\ket{\phi(z)}\bra{\phi(\bar{z})}.
\end{equation}
By multiplying this relation with itself, but with $ z' $ instead of $ z $ we get\begin{equation}
\begin{aligned}
(B-z)^{-1}(B-z')^{-1}=(C-z)^{-1}(C-z')^{-1}+\lambda(z)\ket{\phi(z)}\bra{\phi(\bar{z})}(C-z')^{-1}\\+\lambda(z')(C-z)^{-1}\ket{\phi(z')}\bra{\phi(\bar{z}')}+\lambda(z)\lambda(z')\ket{\phi(z)}\braket{\phi(\bar{z}),\phi(z')}\bra{\phi(\bar{z}')}.
\end{aligned}
\end{equation}
Now using that $ (B-z)^{-1}-(B-z')^{-1}=(z-z')(B-z)^{-1}(B-z')^{-1} $ and the same relation for $ C $ we get, by multiplying through with $ (z-z') $ that \begin{equation}
\begin{aligned}
(B-z)^{-1}-(B-z')^{-1}-(C-z)^{-1}+(C-z')^{-1}=\qquad\qquad\qquad\qquad\qquad\qquad\\(z-z')\lambda(z)\ket{\phi(z)}\bra{\phi(\bar{z})}(C-z')^{-1}+(z-z')\lambda(z')(C-z)^{-1}\ket{\phi(z')}\bra{\phi(\bar{z}')}\\+(z-z')\lambda(z)\lambda(z')\ket{\phi(z)}\braket{\phi(\bar{z}),\phi(z)'}\bra{\phi(\bar{z}')}.
\end{aligned}
\end{equation}
Using again the relation \eqref{Krein relation 1} we obtain\begin{equation}
\begin{aligned}
\lambda(z)\ket{\phi(z)}\bra{\phi(\bar{z})}-\lambda(z')\ket{\phi(z')}\bra{\phi(\bar{z}')}=\qquad\qquad\qquad\qquad\qquad\quad\qquad\qquad\qquad\\(z-z')\lambda(z)\ket{\phi(z)}\bra{\phi(\bar{z})}(C-z')^{-1}+(z-z')\lambda(z')(C-z)^{-1}\ket{\phi(z')}\bra{\phi(\bar{z}')}\\+(z-z')\lambda(z)\lambda(z')\ket{\phi(z)}\braket{\phi(\bar{z}),\phi(z)'}\bra{\phi(\bar{z}')}
\end{aligned}
\end{equation}
from which we obtain by simple rearrangement \begin{equation}
\begin{aligned}
\lambda(z)\ket{\phi(z)}\bra{\phi(\bar{z})}\left(I-(z-z')(C-z')^{-1}\right)-\lambda(z')\left(I-(z'-z)(C-z)^{-1}\right)\ket{\phi(z')}\bra{\phi(\bar{z}')}=\\(z-z')\lambda(z)\lambda(z')\ket{\phi(z)}\braket{\phi(\bar{z}),\phi(z)'}\bra{\phi(\bar{z}')}
\end{aligned}
\end{equation}
Notice now that \begin{equation}
\left(I-(z-z')(C-z')^{-1}\right)=(C-z')^{-1}(C-z),\quad\text{and}\quad \left(I-(z'-z)(C-z)^{-1}\right)=(C-z')(C-z)^{-1}.
\end{equation}
Now clearly by \eqref{Krein relation 2} we have \begin{equation}
(C-z')(C-z)^{-1}\phi(z')=(C-z')(C-z)^{-1}\phi(z_0)+(z'-z_0)(C-z)^{-1}\phi(z_0)
\end{equation}
where we have used that $ (C-z')(C-z)^{-1}h=(C-z)^{-1}(C-z')h $ whenever $ h\in\dom{(C-z')} $. By using that  $ (C-z')(C-z)^{-1}=I+(z-z')(C-z)^{-1} $ we obtain\begin{equation}
(C-z')(C-z)^{-1}\phi(z')=\phi(z_0)+(z-z_0)(C-z)^{-1}\phi(z_0)=\phi(z).
\end{equation}
By a similar computation we have \begin{equation}
(C-z)(C-z')^{-1}\phi(z)=\phi(z_0)+(z'-z_0)(C-z')^{-1}\phi(z_0)=\phi(z'),
\end{equation}
and thus we obtain\begin{equation}
\begin{aligned}
\lambda(z)\ket{\phi(z)}\bra{\phi(\bar{z})}\left(I-(z-z')(C-z')^{-1}\right)-\lambda(z')\left(I-(z'-z)(C-z)^{-1}\right)\ket{\phi(z')}\bra{\phi(\bar{z}')}=\\
\lambda(z)\ket{\phi(z)}\bra{\left(I-(\bar{z}-\bar{z}')(C-\bar{z}')^{-1}\right)\phi(\bar{z})}-\lambda(z')\ket{\left(I-(z'-z)(C-z)^{-1}\right)\phi(z')}\bra{\phi(\bar{z}')}=\\
\lambda(z)\ket{\phi(z)}\bra{\phi(\bar{z}')}-\lambda(z')\ket{\phi(z)}\bra{\phi(\bar{z}')}=(z-z')\lambda(z)\lambda(z')\ket{\phi(z)}\braket{\phi(\bar{z}),\phi(z)'}\bra{\phi(\bar{z}')}.
\end{aligned}
\end{equation}
Observing the last line in the above calculation we observe that we have the relation\begin{equation}
\lambda(z)-\lambda(z')=\lambda(z)\lambda(z')(z-z')\braket{\phi(\bar{z}),\phi(z')},
\end{equation}
which is equivalent to the relation\begin{equation}
\lambda(z)^{-1}-\lambda(z')^{-1}=-(z-z')\braket{\phi(\bar{z}),\phi(z')}.
\end{equation}
This proves equation \eqref{lambda relation}
\bibliographystyle{amsplain}
\bibliography{bibliography}
\end{document}