\documentclass[a4paper,11pt]{article}
\usepackage[utf8]{inputenc}
\usepackage[margin=1in]{geometry}
\usepackage{pdfpages}
\usepackage{mathrsfs}
\usepackage{amsfonts}
\usepackage{amsmath}
\DeclareMathOperator\arctanh{arctanh}
\usepackage{amssymb}
\usepackage{bbm}
\usepackage{amsthm}
\usepackage{graphicx}
\usepackage{centernot}
\usepackage{caption}
\usepackage{subcaption}
\usepackage{braket}
\usepackage{lastpage}
\usepackage{enumitem}
\usepackage{setspace}
\usepackage{xcolor}
\usepackage[english]{babel} 

\usepackage[square,sort,comma,numbers]{natbib}
\usepackage[colorlinks=true,linkcolor=blue]{hyperref}

\usepackage{fancyhdr}
\newcommand{\euler}[1]{\text{e}^{#1}}
\newcommand{\Real}{\text{Re}}
\newcommand{\Imag}{\text{Im}}
\newcommand{\supp}{\text{supp}}
\newcommand{\norm}[1]{\left\lVert #1 \right\rVert}
\newcommand{\abs}[1]{\left\lvert #1 \right\rvert}
\newcommand{\floor}[1]{\left\lfloor #1 \right\rfloor}
\newcommand{\Span}[1]{\text{span}\left(#1\right)}
\newcommand{\dom}[1]{\mathscr D\left(#1\right)}
\newcommand{\Ran}[1]{\text{Ran}\left(#1\right)}
\newcommand{\conv}[1]{\text{co}\left\{#1\right\}}
\newcommand{\Ext}[1]{\text{Ext}\left\{#1\right\}}
\newcommand{\vin}{\rotatebox[origin=c]{-90}{$\in$}}
\newcommand{\interior}[1]{%
	{\kern0pt#1}^{\mathrm{o}}%
}
\renewcommand{\braket}[1]{\left\langle#1\right\rangle}
\newcommand*\diff{\mathop{}\!\mathrm{d}}
\newcommand{\ie}{\emph{i.e.} }
\newcommand{\eg}{\emph{e.g.} }
\newcommand{\dd}{\partial }
\newcommand{\R}{\mathbb{R}}
\newcommand{\C}{\mathbb{C}}
\newcommand{\w}{\mathsf{w}}

\newcommand{\Gliminf}{\Gamma\text{-}\liminf}
\newcommand{\Glimsup}{\Gamma\text{-}\limsup}
\newcommand{\Glim}{\Gamma\text{-}\lim}

\newcommand{\pipe}{\hspace{.5mm}|\hspace{.5mm}}

\newtheorem{theorem}{Theorem}
\newtheorem{definition}{Definition}
\newtheorem{proposition}{Proposition}
\newtheorem{lemma}{Lemma}
\newtheorem{corollary}{Corollary}
\title{spectral flow}
\author{ }
\date{June 2021}

\begin{document}

\maketitle

\section{Introduction}
\section{Special case}
For each $ t\in\R $ let $ A_t $ be a self-adjoint operator on the Hilbert space $ H $ with domain $ D(A_t)=W $ independent of $ t $. We assume that $ W $ is compactly embedded and dense in $ H $. Assume furthermore, that $ t\mapsto A_t $ is continuously differentiable in the weak operator topology (WOT), \ie a differentiable map, with derivative that is continuous in the WOT. In the following, we shall assume that $ A_t $ takes the form $ A_t=\sum_{j} \lambda_j(t)\ket{u_j}\bra{u_j} $. Define the operator $ D_A=\frac{\diff}{\diff t}-A_t $, with domain \begin{equation*}
D(D_A)=\left\{\sum_{j}\alpha_j(t)\ket{u_j}\Bigg\vert \int_{\R}\sum_{j}\abs{\alpha_j(t)}^2\diff t<\infty,\ \int_{\R}\sum_{j}\abs{\alpha_j'(t)-\lambda_j(t)\alpha_j(t)}^2\diff t<\infty\right\}.
\end{equation*}
Then we have the following result \begin{theorem}
	Let $ D_A $ be as described above, and assume that $ A_t $ converges to invertible operators $ A_\pm $ as $ t\to\pm\infty $. Then $ D_A $ is Fredholm, and its index is given by the spectral flow of the family $ (A_t)_{t\in\R} $.
\end{theorem}
In order to show this theorem, we show first that to each element in the kernel of $ D_A $ or the kernel of $ D_A^* $, we may associate an eigenvalue of $A_t $, $ \lambda_j(t) $, crossing zero and odd number of times. More precisely we show that each element in $ \ker(D_A) $ can be associated to eigenvalues $ \lambda_j(t) $ with the properties $ \lim\limits_{t\to-\infty}\lambda_j(t)>0 $ and $ \lim\limits_{t\to\infty}\lambda_j(t)<0 $. Similarly each element in $ \ker(D_A^*) $ can be associated to eigenvalues $ \lambda_j(t) $ with the properties $ \lim\limits_{t\to-\infty}\lambda_j(t)<0 $ and $ \lim\limits_{t\to\infty}\lambda_j(t)>0 $.
\begin{proposition}\label{Prop1}
	Let $ D_A $ be as above, then $ \ker(D_A) $ is spanned by $ \{\beta_j(t)\ket{u_j}\}_{j\in M} $, where $ M=\{j\pipe \text{$ \lim\limits_{t\to-\infty}\lambda_j(t)>0 $ and $ \lim\limits_{t\to\infty}\lambda_j(t)<0 $} \} $ and $ \beta_j(t)=\euler{\int_{0}^{t}\lambda_j(s)\diff s} $.
\end{proposition}
\begin{proof}
	It is clear that $ \beta_j(t)\ket{u_j} $ is in the kernel under the assumptions. On the contrary if $ \sum_{j}\alpha_j(t)\ket{u_j}\in\ker(D_A) $, then $ \alpha_j'(t)-\lambda_j(t)\alpha_j(t)=0 $ for all $ \alpha_j $. Therefore $ \alpha_j(t)=\alpha_j(0)\euler{\int_{0}^{t}\lambda_j(s)\diff s} $, for all $ j $. But then $ \sum_{j}\alpha_j(t)\ket{u_j}\in D(D_A) $, only if $ \lim\limits_{t\to-\infty}\lambda_j(t)>0 $ and $ \lim\limits_{t\to\infty}\lambda_j(t)<0 $. Furthermore, we show below that only finitely many eigenvalues cross zero, and therefore, we may conclude that $ \sum_{j}\alpha_j(t)\ket{u_j} $ is a finite sum, and hence $ \sum_{j}\alpha_j(t)\ket{u_j}\in D(D_A) $ if $ \lim\limits_{t\to-\infty}\lambda_j(t)>0 $ and $ \lim\limits_{t\to\infty}\lambda_j(t)<0 $.
\end{proof}
\begin{proposition}
	Let $ D_A $ be as above, then $ \ker(D_A^*) $ is spanned by $ \{\beta_j(t)\ket{u_j}\}_{j\in M} $, where $ M=\{j\pipe \text{$ \lim\limits_{t\to-\infty}\lambda_j(t)<0 $ and $ \lim\limits_{t\to\infty}\lambda_j(t)>0 $} \} $ and $ \beta_j(t)=\euler{-\int_{0}^{t}\lambda_j(s)\diff s} $.
\end{proposition}
\begin{proof}
	The proof is similar to the one for Proposition \ref{Prop1}
\end{proof}
We now show that the eigenvalues of $ A_t $, can cross zero only finitely many times. We first need to establish that all eigenvalues must cross zero within some compact interval. This is a consequence of the following lemma 
\begin{lemma}\label{LemmaCrossingsBound}
	Let $ (A_t)_{t\in\R} $ be a family of self-adjoint operators with $ t $-independent domain $ W$. Assume furthermore, $ (A_t)_{t\in\R}$ converge to invertible operators $A^\pm $ in the norm-topology on $ \mathcal{L}(W,H) $ and that $ t\mapsto A_t $ is countinuously differentiable in the WOT. Then there exist $ t_1,t_2\in\R $ and $ c>0 $ such that $ |A_t|>c>0 $ for $ t<t_1 $ or $ t>t_2 $.
\end{lemma}
\begin{proof}
	It is direct consequence of invertibility of $ A^\pm $, that there exist $ d>0 $ such that $ \abs{A^\pm}>d $. Notice now that for invertible operators $ A $ and $ B $ we have $ \frac{1}{A}-\frac{1}{B}=\frac{1}{A}(A-B)\frac{1}{B} $. Since $ (A_t+i) $ and $ A\pm+i $ are invertible it follows that \begin{equation}
	\norm{\frac{1}{A_t+i}-\frac{1}{A^\pm+i}}_{\mathcal{L}(H,H)}\leq\norm{\frac{1}{A_t+i}}_{\mathcal{L}(H,H)}\norm{A_t-A^\pm}_{\mathcal{L}(W,H)}\norm{\frac{1}{A^\pm+i}}_{\mathcal{L}(H,W)}\leq \norm{A_t-A}_{\mathcal{L}(W,H)}.
	\end{equation}
	Thus we conclude that $ A_t $ converges to $ A^\pm $ as $ t\to\pm \infty $ in the norm resolvent sense. Thus for any $ \epsilon>0 $ we have that there exist $ t_1 $ and $ t_2 $ such that  \begin{equation}
	\norm{\frac{1}{A_t+i}}_{\mathcal{L}(H,H)}\leq \norm{A_t-A^\pm}_{\mathcal{L}(W,H)}+\norm{\frac{1}{A^\pm+i}}_{\mathcal{L}(H,H)}\leq \epsilon+\norm{\frac{1}{A^\pm+i}}_{\mathcal{L}(H,H)}
	\end{equation}
	for $ t<t_1 $ or $ t>t_2 $
	from which it follows that \begin{equation}
	\sup_{i}\frac{1}{\left(\abs{\lambda_i}^2+1\right)^{1/2}}\leq \epsilon + \sup_i\frac{1}{\left(\abs{\lambda^\pm_i}^2+1\right)^{1/2}}.
	\end{equation}
	Equivalently we have $ \inf_{i}\left(\abs{\lambda_i}^2+1\right)^{1/2}=\left(\inf_i\abs{\lambda_i}^2+1\right)^{1/2}\geq\left( \epsilon+\frac{1}{\left(\inf_i\abs{\lambda^\pm_i}^2+1\right)^{1/2}}\right)^{-1} $,
	and we see that \begin{equation}
	\left(\inf_i\abs{\lambda_i}^2+1\right)^{1/2}\geq \left( \epsilon+\frac{1}{\left(d^2+1\right)^{1/2}}\right)^{-1}
	\end{equation}
	$ \inf_i\abs{\lambda_i}\geq\left(\left( \epsilon+\frac{1}{\left(d^2+1\right)^{1/2}}\right)^{-2}-1\right)^{1/2}  $, and the result follows by choosing $ \epsilon<1-\frac{1}{\left(d^2+1\right)^{1/2}} $
\end{proof}
We are now ready to show the result
\begin{proposition}
	Let $ (A_t)_{t\in\R} $ be a family of self-adjoint operators with $ t $-independent domain $ W$. Assume furthermore, $ (A_t)_{t\in\R}$ converge to invertible operators $A^\pm $ in the norm-topology on $ \mathcal{L}(W,H) $ and that $ t\mapsto A_t $ is countinuously differentiable in the WOT. Then only finitely many eigenvalues of $ A_t $ cross zero.
\end{proposition}
\begin{proof}
	It is known that $ A_t $ has discrete spectrum for all $ t\in\R $. Now assume that infinitely many eigenvalues cross zero. By lemma \ref{LemmaCrossingsBound}, there exist $ t_1,t_2 $ such that all crossing happen in the interval $ [t_1,t_2] $. Letting the crossing points define a sequence, it is clear by the Bolzano-Weierstrass theorem, that there exist a point, $ t^* $, such that any interval $ I $ with $ t^*\in \interior{I} $ contains infinitely many crossings. It is then clear that $ I_\epsilon=[t^*-\epsilon,t^*+\epsilon] $ contains infinitely many crossings for every $ \epsilon>0 $. Since $ \dot{A}_t=\frac{\diff A_t}{\diff t} $ is continuous in the WOT, it holds that $ f_x:\R\to H $ defines by $ f_x(t)=\dot{A}_tx $ is continuous, when $ H $ is equipped with the weak topology. Therefore, $ f_x([t^*-\epsilon,t^*+\epsilon]) $ is weakly compact and hence norm bounded. We conclude that $ \sup_{t\in I_\epsilon}\left\{\norm{\dot{A}_t x}_H\right\}<\infty $ for all $ x\in W $. By the uniform boundedness principle, it follows that $ \sup_{t\in I_\epsilon}\left\{\norm{\dot{A}_t}_{\mathcal{L}(W,H)}\right\}<\infty $. Thus we conclude that there exist $ C>0 $ such that $ \abs{\lambda_j'(t)}\leq C\left(\sqrt{\abs{\lambda_j(t)}^2+1}\right)\leq C(\abs{\lambda_j(t)}+1)  $ for all $ t\in I_\epsilon $. Letting $ M_j= \max_{t\in I_\epsilon}\abs{\lambda_j(t)} $, we see that by the mean value theorem, there exist a point $ t'\in I_\epsilon $ where $ \abs{\lambda_j'(t)}\geq \frac{M}{2\epsilon} $, and therefore $ \frac{M}{2\epsilon}\leq C(M+1) $ or equivalently $ M\leq \frac{2\epsilon C}{1-2\epsilon C}   $. This clearly contradicts the fact, that $ A_t^* $ has dicrete spectrum, as the eigenvalues of $ A_{t^*} $ accumulate at $ 0 $. Thereby, we conclude that the number of crossing eigenvalues must be finite.
\end{proof}


\end{document}