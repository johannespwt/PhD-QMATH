\section{Uge 2}
\begin{opg}[Den svære] 
Betragt nedenstående følge
$$ \seq{a}, \QUAD a_n = \sin\pare{n} $$
For hvilke $x\in\mR$ findes en delfølge $\cbrac{a_{n_k}}_{k\in\mN}$, som konvergerer mod $x$?
\end{opg}

	\begin{opg}
		Definér følgen $(a_n)_{n\in\N}$ rekursivt ved $a_1 = 1$ og 
		$$
		a_{n+1} = \frac{1}{1 + a_n}.
		$$
		Vis, at
			$$
			\abs{a_{n+1} - a_n} < \frac{1}{2}\abs{a_n - a_{n-1}},
			$$
			for $n \geq 2$, og brug det til at vise, at $(a_n)_{n\in\N}$ er en Cauchyfølge. Følgen er således konvergent. Find grænseværdien og giv mening til udsagnet
			$$
			\frac{1 + \sqrt{5}}{2} = 1 + \frac{1}{1 + \frac{1}{1 + \dots}}.
			$$
			(Denne identitet kaldes \emph{kædebrøksfremstillingen for det gyldne snit}.)
	\end{opg}
	Løsning: For $n \geq 2$ har vi
	\begin{align*}
	\abs{a_{n+1} - a_n} &= \abs{\frac{1}{1 + a_n} - a_n} 
	= \abs{\frac{1 - a_n(1 + a_n)}{1 + a_n}} 
	= \frac{1}{1 + a_n}\abs{1 - \frac{1}{1 + a_{n-1}}(1 + a_n)} 
	\\&= \frac{1}{1 + a_n}\abs{\frac{1 + a_{n-1} - (1 + a_n)}{1 + a_{n-1}}} = \frac{1}{(1 + a_n)(1 + a_{n-1})}\abs{a_n - a_{n-1}}.
	\end{align*}
	Da
	$$
	(1 + a_n)(1 + a_{n-1}) = (1 + \frac{1}{1 + a_{n-1}})(1 + a_{n-1}) = 1 + a_{n-1} + 1 > 2,
	$$
	har vi altså
	$$
	\abs{a_{n+1} - a_n} < \frac{1}{2}\abs{a_n - a_{n-1}}.
	$$
	Ved at bruge dette gentagne gang får vi for $n \geq 2$
	$$
	\abs{a_{n+1} - a_n} < \frac{1}{2^{n-1}}\abs{a_2 - a_1} = \frac{1}{2^n}.
	$$
	Nu fås for $m > n \geq 2$
	$$
	\abs{a_m - a_n} \leq \sum_{k = n}^{m-1} \abs{a_{k+1} - a_k} < \sum_{k = n}^{m-1} \frac{1}{2^k},
	$$
	og da rækken $\sum_{k = 1}^{\infty} 2^{-k}$ er konvergent, fås heraf, at $(a_n)_{n\in\N}$ er en Cauchyfølge.
	
	\section{Uge 3}
	\begin{opg}
		Vis, at 
		$$
		\sum_{n = 1}^{N} \frac{1}{n} - \log N = \sum_{n = 1}^{N - 1} \int_{n}^{n+1} \frac{1}{n} - \frac{1}{t}\,dt + \frac{1}{N},
		$$
		og brug det til at vise, at udtrykket har en grænseværdi for $N \to \infty$. (Grænseværdien, der betegnes $\gamma$, kaldes \emph{Euler-Mascheroni-konstanten}, og $\gamma \approx 0.57721$.)
	\end{opg}
	Løsning: Ved at bruge, at $\log N = \int_{1}^{N} 1/t\,dt$ og $1/n = \int_{n}^{n + 1} 1/n\,dt$, fås
	$$
	\sum_{n = 1}^{N} \frac{1}{n} - \log N = \sum_{n = 1}^{N-1}\int_{n}^{n + 1} \frac{1}{n}\,dt + \frac{1}{N} - \int_1^N \frac{1}{t}\,dt = \sum_{n = 1}^{N-1}\int_{n}^{n + 1} \frac{1}{n} - \frac{1}{t}\,dt + \frac{1}{N}.
	$$
	Ved at bruge, at $-1/x$ er en stamfunktion til $1/x^2$ fås
	$$
	\abs{\int_{n}^{n + 1} \frac{1}{n} - \frac{1}{t}\,dt} = \abs{\int_{n}^{n + 1}\int_{n}^{t}\frac{1}{x^2}\,dxdt} \leq \int_{n}^{n+1}\int_{n}^{n+1} \frac{1}{x^2} \,dxdt \leq \frac{1}{n^2},
	$$
	så
	$$
	\sum_{n = 1}^{\infty} \int_{n}^{n+1} \frac{1}{n} - \frac{1}{t}\,dt
	$$
	er absolut konvergent. Med andre ord har udtrykket
	$$
	\sum_{n = 1}^{N - 1} \int_{n}^{n+1} \frac{1}{n} - \frac{1}{t}\,dt
	$$
	en grænseværdi for $N \to \infty$. Da $1/N \to 0$ for $N \to \infty$, følger det, at
	$$
	\sum_{n = 1}^{N} \frac{1}{n} - \log N = \sum_{n = 1}^{N - 1} \int_{n}^{n+1} \frac{1}{n} - \frac{1}{t}\,dt + \frac{1}{N}
	$$
	har en grænseværdi for $N \to \infty$.