\documentclass{article}

\usepackage{ANpreamble}
%\usepackage{enumerate}

\newif\ifanswers

\answerstrue %udkommenter denne linje for at skjule løsninger


\title{Analyse 1 2020/2021 - Rettevejledning til Hjemmeopgave 1}
\author{Johannes Agerskov}
\date{\today}
\setcounter{section}{1}
\begin{document}
	
	\maketitle
Følgende rettevejledning er, som det fremgår af navnet, \emph{vejledende}. Det er derfor op til den enkelte instruktor at afgøre i hvilken grad, de angivne forslag i denne vejledning passende til den enkelte besvarelse. Vejledningen er på ingen måde dækkende over alle typer af besvarelser, og ofte vil der i retteprocessen også indgå et helhedsintryk når der gives point. Det er derimod vejledningens formål, at give et overordnet billede af, hvor hårdt der slåes ned på nogle udvalgte (måske typiske) fejl eller mangler der kan optræde i besvarelserne. Det er forventeligt, at fejl og mangler, som er gennemgået i denne vejledning, vil optræde i forskellige grader eller variationer i besvarelserne, og i så fald, må det vurderes hvorvidt der skal trækkes færre eller flere point i disse tilfælde. Det er håbet, at fejl, der ikke måtte være gennemgået i denne vejledning, kan relateres til eller sammenlignes med fejl gennemgået i vejledningen, og at pointgivningen for disse besvarelse dermed kan ekstra-/interpoleres fra vejledningen.
\vspace*{0.5cm}\\
Som beskrevet i "Regler og vejledning for aflevering a Hjemmeopgave 1" på sidste side i opgavesættet, lægges der vægt på klar og og præcis formulering. Det er dermed essentielt, at opgaverne er letlæselige, og at det er klart, hvad den studerende mener. For at der kan gives fuld point for en opgave, skal argumentet være fuldkommen, og der bør ikke være brug for at færdiggøre argumenterne, for at forstå dem.
\vspace*{0.2cm}
\\
Hvis der skulle være brug for let tolkning af en forklaring, dvs. at det er tydeligt, hvad der er menes, men formuleringen er delvis utilstrækkelig, fratrækkes der som udgangs punkt få point, altså $ 1 $ eller $ 2 $ point afhængig af graden af utilstrækkelighed.
\vspace*{0.2cm}\\
Hvis der er brug for større tolkning af en forklaring, eller brug for færdiggørelse af argumentet for at verificere dets korrekthed, anses denne del af opgaven for værende ikke løst, og der kan trækkes point svarende til den del af opgaven.
\vspace*{0.2cm}\\
Hvis der er givet et argument et forkert sted i besvarelsen, f.eks. arguementet står i 2.a) men er først nødvendigt (og brugbart) i 2.b, da fratrækkes 1 point, med mindre selvfølglig, at der i opgave 2.b henvises til det tidligere argument.
\vspace*{0.2cm}\\
Vi retter efter et oppefra og ned princip, hvilket vil sige, at enhver besvarelse som udgangspunkt har 100/100 point. Der fratrækkes så point for fejl, mangler og ander utilstrækkeligheder i besvarelsen.
\vspace*{0.2cm}\\
I følgende tilfælde trækkes der \textbf{ikke} point:
\begin{itemize}
	\item Der trækkes ikke point for ligegyldig tekst.
	\item Der trækkes ikke point for et forkert argument, \textbf{hvis} der efterfølgende gives et korrekt argument for samme resultat.
	\item Der fratrækkes ikke point for følgefejl, altså fejl som skyldes fejl i tidligere opgaver.
\end{itemize}

En forløbig rettevejledning for Hjemmeopgave 1 ser således ud:

\begin{opg}[40 \emph{point}]\hfill
	\begin{enumerate}
		\item (10 \emph{point})\begin{enumerate}[label=(\roman*)]
			\item Regnefejl ($ -2 $ \textit{point})
			\item Glemmer at nævne $ a>0 $ ved brug af $ \arctan $ eller lign. ($ -2 $\textit{ point})
			\item Regner polarformer forkert ($ -3 $ \emph{point})
			\item Glemmer at vise at $ z^6 $ er reel ($ -1 $ \emph{point})
		\end{enumerate}
		\item (10 \emph{point}) Jeg har ikke kunne komme på nogen typiske fejl.
		\item (10 \emph{point})\begin{enumerate}[label=(\roman*)]
			\item Viser ikke, at $ \seq
			 a $ divergerer for $ \abs{b}>\frac{6}{5} $ ($ -3 $ \emph{point})
			\item Viser ikke, at $ \seq a $ divergerer for $ \abs{b}=\frac{6}{5} $ ($ -2 $ \emph{point})
			\item Finder forkert konvergens-interval (-3 \emph{point})
		\end{enumerate}
		\item (10 \emph{point})\begin{enumerate}[label=(\roman*)]
			\item Viser ikke, at $ \seq a $ \emph{har} en konvergent delfølge for $ \abs{b}=\frac{6}{5} $ ($ -4 $ \emph{point})
			\item Glemmer at argumenterer for/nævne, at $ \seq a $ har en konvergent delfølge, når $ \seq a $ konvergerer, altså når $ \abs{b}<\frac{6}{5} $ ($ -3 $ \emph{point})
		\end{enumerate}
	\end{enumerate}	
\end{opg}
\begin{opg}[30 \emph{point}]\hfill
	\begin{enumerate}
		\item (10 $ \emph{point} $)
		\begin{enumerate}[label=(\roman*)]
			\item Henviser ikke til Sætning 1.39, eller noget ækvivalent ($ -4 $ \emph{point}) 
		\end{enumerate}
		\item (10 \emph{point}) 
		\begin{enumerate}[label=(\roman*)]
			\item Omskriver $ \seq a $ forkert ($ -3 $ point)
			\item Forsøger at bruge Sætnig 1.39 direkte på et $ \infty/\infty $-udtryk ($ -5 $ point)
		\end{enumerate}
		\item (10 \emph{point}) Bemærk, at denne opgave, som addition til løsningsforslaget i den vejledende besvarelse, kan løses ved at omskrive til summen af to konvergente rækker. I så fald, skal der argumenteres for de to rækkers konvergens, samt til resultat for konvergens af sum af to konvergent følger (Sætning 1.39).
		\begin{enumerate}[label=(\roman*)]
			\item I tilfælde af ovenstående løsningsmetode: Viser ikke, at $ \sum_{n=1}^{N}\frac{1}{n^2} $ konvergerer. (f.eks. skriver bare at $ \sum_{n=1}^{\infty}\frac{1}{n^2}=\frac{\pi^2}{6} $ uden reference, \texttt{maple} er ikke en reference). ($ -5 $ \emph{point})
		\end{enumerate}
	\end{enumerate}
\end{opg}
\begin{opg}[30 \emph{point}]\hfill
	\begin{enumerate}
		\item (10 \emph{point}) \begin{enumerate}[label=(\roman*)]
			\item Glemmer $ a=0 $ tilfældet ($ -4 $ point)
			\item Omskriver ikke før brug af Sætning 1.39 (eller ækvivalent), altså forsøger at bruge Sætning 1.39 på $ \infty/\infty $-udtryk. ($ -4 $ \emph{point})
		\end{enumerate}
		\item (10 \emph{point}) Denne opgave kan løses på flere måder, bland andet med Observation 1.42 \begin{enumerate}[label=(\roman*)]
			\item Nævner ikke kontinuitet af funktion før brug at Sætnign 1.43 ($ -4 $ \emph{point})
			\item Bruger L'H\^opital's regel forkert eller lignende ($ -3 $ \emph{point})
			\item Regnefejl ($ -2 $ \emph{point})
			
		\end{enumerate}
		\item (10 \emph{point}) I denne opgave er der en række store fejl eller mangler man kan have. Hvis én af disse optræder trækkes det angivne antal point fra. Men det giver ikke mening at trække det angivne antal point ved flere fejl, da man risikerer at give $ 0 $ point (eller endda negative point) selvom opgaven ikke er fuldkommen forkert, hvilket selvfølgelig er absurd. Derfor foreslår jeg:
		\begin{enumerate}[label=(\Roman*)]
			\item Ved én af nedenstående fejl: $ +0 $ \emph{point}
			\item Ved to af nedenstående fejl: $ +1 $ \emph{point}
			\item Ved tre af nedenstående fejl: $ +2 $ \emph{point}
		\end{enumerate}
		\begin{enumerate}[label=(\roman*)]
			\item Viser ikke explicit at $ z_n $ er en middelsum for et Riemann integrale. Dvs, konstruerer (ækvidistant) inddeling \emph{og} finder korrekt interval som inddelingen løber over. ($ -3 $ \emph{point}) 
			\item Argumenterer ikke for at $ 1/x $ er Riemann integrabel på $ [1,2] $ ($ -4 $ \emph{point})
			\item Viser ikke konvergens korrekt (ud fra definition af Riemann integrabilitet), f.eks. kommenterer ikke på findheden af inddelingen ($ -4 $ \emph{point})
		\end{enumerate}
	\end{enumerate}
\end{opg}
\end{document}