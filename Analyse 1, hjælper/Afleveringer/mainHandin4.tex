\documentclass{article}

\usepackage{ANpreamble}
%\usepackage{enumerate}

\newif\ifanswers
\newif\ifrules

\answerstrue 	%udkommenter denne linje for at skjule løsninger
\rulestrue 	%udkommenter denne linje for at skjule regler

\title{Analyse 1 2020/2021 - Hjemmeopgave 4}
\author{}
\date{\vspace{-1cm}Afleveres senest kl 13:00 på Absalon, 25. juni 2021}

\begin{document}
	
	\maketitle
	
	\noindent
	
	\setcounter{section}{4}

\begin{opg}\hfill \\
	Betragt funktionen $ f:[-\pi,\pi]\to \R $ givet ved $ f(x)=x\sin(x) $.
	\begin{enumerate}
		\item Redegør for at $ f $ kan udvides til en lige og stykkevis $ C^1 $ funktion i $ \text{PCN}_{2\pi} $. Beregn $ c_{-1}(f) $,$ c_0(f) $ og $ c_1(f) $.
		\ifanswers
		\begin{proof}[Løsning]
			Det observeres, at $ f(-x)=f(x) $ for alle $ x\in[-\pi,\pi] $, da $ x $ og $ \sin(x) $ hver især er ulige. Definér nu $ \tilde{f}:\R\to\R $ ved $ \tilde{f}(x)=f(x+2\pi k) $ hvis $ k\in\mathbb{Z} $ og $ x+2\pi k\in[-\pi,\pi) $. At $ \tilde{f} $ er veldefineret ses ved, at der for ethvert $ x\in\R $ eksisterer et unikt $ k\in\mathbb{Z} $, således at $ x+2\pi k\in [-\pi,\pi) $.
			Det gælder trivielt, at $ \tilde{f}(x)=f(x) $ for alle $ x\in[-\pi,\pi) $. Desuden ses det, at $ f(-\pi)=f(\pi)=0 $, hvoraf det følger at $ \tilde{f}(\pi)=f(-\pi)=f(\pi) $. Derfor, er $ \tilde{f} $ en udvidelse af $ f $. Det ses desuden, at $ \tilde{f}(-x)=f(-x+2\pi k)=f(x-2\pi k)=\tilde{f}(x) $, hvor $ k\in\mathbb{Z} $ opfylder at $ -x+2\pi k\in[-\pi,\pi) $, således at $ x-2\pi k\in(-\pi,\pi] $ (hvis $ x-2\pi k=\pi $ så gælder $ f(x-2\pi k)=f(\pi)=f(-\pi)=f(x-2\pi(k+1))=\tilde{f(x)} $). Det gælder tydeligt, at $ \tilde{f(x)}=\tilde{f(x+2\pi)} $ for ethvert $ x\in\R $, så $ \tilde{f} $ er $ 2\pi $-periodisk. Det er derfor nok at tjekke kontinuitet af $ \tilde{f} $ på intervallet $ (-\pi,2\pi) $. Kontinuitet følger direkte af definitionen på $ (-\pi,\pi) $ samt $ (\pi,2\pi) $. Derudover gælder $ \lim\limits_{x\to\pi_-}\tilde{f}(x)=\lim\limits_{x\to\pi_-}f(x)=f(\pi)=0 $, og $ \lim\limits_{x\to\pi_+}\tilde{f}(x)=\lim\limits_{x\to\pi_+}f(x-2\pi)=f(-\pi)=0 $, hvoraf, det følger at $ \tilde{f} $ er kontinuert. Betragt nu $ d_0=-\pi $, $ d_1=\pi $. Da gælder, at restriktionen $ \tilde{f}:[-\pi,\pi]\to\R $ er $ C_1 $, da $ \tilde{f}(x)=f(x) $ for $ x\in[-\pi,\pi] $ samt at $ f'(x)=\sin(x)+x\cos(x) $ er kontinuert på $ (-\pi,\pi) $ samt at $ \lim\limits_{x\to-\pi_+}f'(x)=-\pi=f'(-\pi) $ og $ \lim\limits_{x\to\pi_-}f'(x)=\pi=f'(\pi) $. Dermed er det vist, at $ f $ kan udvides til en lige stykkevist $ C^1 $ funktion i $ \text{PCN}_{2\pi} $.
			Koefficienterne $ c_{-1}(f) $,$ c_0(f) $ og $ c_1(f) $ beregnes nu ved \begin{equation*}
			c_{\pm 1}(f)=\frac{1}{2\pi}\int_{-\pi}^{\pi}f(x)e^{\mp ix} \diff x=\frac{1}{2\pi}\int_{-\pi}^{\pi}x\sin(x)\cos(x) \diff x=\frac{1}{2\pi}\left[\frac{x}{2}\sin^2(x)\right]_{-\pi}^{\pi}-\frac{1}{4\pi}\int_{-\pi}^{\pi}\sin^2(x)\diff x=-\frac{1}{4},
			\end{equation*}
			hvor vi har brugt partiel integration, og 
			$$
			c_0(f)=\frac{1}{2\pi}\int_{-\pi}^{\pi}x\sin(x)\diff x=\frac{1}{2\pi}\left[x(-\cos(x))\right]_{-\pi}^{\pi}+\frac{1}{2\pi}\int_{-\pi}^{\pi}\cos(x)\diff x=1,
			$$
			hvor vi igen har brugt partiel integration.
		\end{proof}
		\fi
		\item Beregn $ c_n(f) $ for alle øvrige $ n $, og opstil Fourierrækken for $ f $. 
		\ifanswers
		\begin{proof}[Løsning]
			Bemærk at $ \sin(x)=\frac{e^{ix}-e^{-ix}}{2i} $. Derfor har (Ved vi fra Opgave 6.6.c.i uge 6??) vi\begin{equation*}
			\begin{aligned}
			c_n(f)=\frac{1}{2\pi}\int_{-\pi}^{\pi}f(x)e^{-inx}\diff x=\frac{1}{4\pi i} \left(\int_{-\pi}^{\pi}xe^{-i(n-1)x}\diff x-\int_{-\pi}^{\pi}xe^{-i(n+1)x}\diff x\right)\\
			=\frac{1}{4\pi i} \left(\left[\frac{i}{n-1}xe^{-i(n-1)x}\right]_{-\pi}^{\pi}-\left[\frac{i}{n+1}xe^{-i(n+1)x}\right]_{-\pi}^{\pi}\right)\\
			=\frac{1}{4\pi}\left(\frac{1}{n-1}(-1)^{n-1}2\pi-\frac{1}{n+1}(-1)^{n+1}2\pi\right)\\
			=-\frac{(-1)^n}{n^2-1}.
			\end{aligned}
			\end{equation*}
			Dermed ses, at vi har Fourierrækken for $ f $
			$$
			\sum_{n\in \mathbb{Z}}c_n(f)e^{inx}=1-\frac{1}{4}(e^{ix}+e^{-ix})-\sum_{n=2}^{\infty}\frac{(-1)^n}{n^2-1}\left(e^{inx}+e^{-inx}\right)
			$$
		\end{proof}
		\fi
		\item Bevis, at Fourierrækken konvergerer uniformt mod $ f $.
		\ifanswers
		\begin{proof}[Løsning]
			Det følger direkte af sætning 5.46, samt af resultaterne fra a), at Fourierrækken konvergerer uniformt mod $ \tilde{f} $ og dermed uniform mod $ f $ på $ [-\pi,\pi] $.
		\end{proof}
		\fi
		\item Opstil cosinus-rækken for $ f $. Indsæt heri $ x=\pi $ og udled formlen
		$$
		\sum_{n=2}^\infty 1/(n^2-1)=3/4
		$$
		\ifanswers
		\begin{proof}[Løsning]
			Det findes direkte fra resultatet i c), at $ f(x)=1-\frac{1}{2}\cos(x)-2\sum_{n=2}^{\infty}\frac{(-1)^n}{n^2+1}\cos(nx) $. For $ x=\pi $ finder vi $ 0=1+1/2-2\sum_{n=2}^{\infty}\frac{1}{n^2+1} $, hvorfra det følger, at $ \sum_{n=2}^{\infty}\frac{1}{n^2+1}=\frac{3}{4} $.
		\end{proof}
		\fi
	\end{enumerate}
\end{opg}
	
	\ifrules
	\newpage
	\noindent
	{\LARGE Regler og vejledning for aflevering af Hjemmeopgave 4}
	
	\noindent\hrulefill \\
	
	\noindent
	Besvarelsen skal udarbejdes individuelt, og afskrift behandles efter universitetets regler om eksamenssnyd. Besvarelsen vil blive bedømt på en skala fra 0 til 100. Denne bedømmelse indgår med en vægt på omtrent en fjerdedel af den endelige karakter. På tværs af de fire Hjemmeopgaver skal man have mindst 50 point i gennemsnit for at bestå.
	
	Ved bedømmelsen lægges vægt på klar og præcis formulering og på argumentation på grundlag af og med henvisning til relevante resultater i pensum, herunder opgaver regnet ved øvelserne. I kan bruge følgende som rettesnor for henvisninger.
	\begin{itemize}
		\item Tænk på henvisninger som en hjælp til at forklare sig. Hvis det er klart af fra konteksten, hvilke resultater man bruger, så er det ikke nødvendigt at henvise.
		
		\item I må henvise til resultater fra noterne [MC], øvelsesopgaverne og forelæsningsslides, samt til alle l\ae{}reb\o{}ger brugt p\aa{} andre f\o{}rste\aa{}rskurser p\aa{} matematikstudierne. Hvis I citerer fra materiale n\ae{}vnt i litteraturlisten i [MC] kan I uden videre benytte forkortelserne brugt her (fx [EHM], [Li]). Det er tilladt at henvise til pensum, som endnu ikke er gennemgået. Det er ikke nødvendigt at angive sidetal på henvisninger. 
		
		\item Det er næsten aldrig relevant at henvise til definitioner.
	\end{itemize}
	%	I skal argumentere med en grundighed svarende til de vejledende besvarelser af de tidligere hjemmeopgaver.
	
%	\bigskip  
%	\noindent
%	Det er ikke n�dvendigt at bruge computeralgebrasystemer til l�sningen af opgaverne, Alle estimater skal som altid begrundes, og der gives ikke fuld point, s�fremt der er brugt CAS uden at dette er forklareti opgaven.".
	
	\bigskip  
	\noindent
	Besvarelsen må udfærdiges i hånden eller med \LaTeX\ eller lignende, men skal være ensartet og letlæselig. Billeder, plots og lignende må gerne udfærdiges i andre programmer. \textbf{Det er et krav}, at håndskrevne besvarelser ikke fylder mere end 7 sider,
	og at besvarelser udarbejdet elektronisk ikke fylder mere end 5 sider. Billeder, plots og lignende tæller med i sideantallet. Håndskrevne besvarelser skal være tydeligt læsbare.
	Overskrides denne begrænsning, vil der blive tildelt halvt pointtal for første overskredne side og kvart pointtal for anden overskredne side. Fra tredje overskredne side og videre, vil besvarelsen ikke blive rettet.
	
	\bigskip 
	\noindent
	P\aa{} hver side af den afleverede l\o{}sning skal I skrive jeres navn og KU-id (``svenske nummerplade''). Vi anbefaler at I ogs\aa{} skriver ``side $x$ ud af $y$'' p\aa{} hver side.
	
	\bigskip 
	\noindent
	Det er kun tilladt at aflevere gennem Absalon, og man skal uploade sin besvarelse som én .pdf. Vi opfordrer jer til hvis muligt at aflevere i god tid for at undgå at Absalon går ned, fordi alle afleverer samtidig.
	
	\bigskip
	\noindent
	Bedømmelsen og en .pdf med et udvalg af forklarende kommentarer vil blive uploadet til Absalon inden for en uge.
	\fi
\end{document}